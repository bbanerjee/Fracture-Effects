\chapter{Isotropic metal plasticity}
\begin{NoteBox}
The deformation gradient ($\BF$) can be decomposed into a rotation tensor ($\BR$)
and a stretch tensor ($\BU$) with the polar decomposition $\BF = \BR\cdot\BU$.
In the isotropic metal plasticity model implemented in \Vaango, $\BR$ is used 
to rotate the stress ($\Bsig$) and the rate of deformation ($\BdT$) 
into the unrotated configuration before the updated stress is computed:
\Beq
  \widehat{\Bsig} = \BR^T\cdot\Bsig\cdot\BR ~;~~
  \dot{\BVeps} = \BR^T\cdot\BdT\cdot\BR 
\Eeq
where $\dot{\BVeps}$ is a ``natural'' strain rate.
After the stress has been updated, it is rotated back using
\Beq
  \Bsig = \BR\cdot\widehat{\Bsig}\cdot\BR^T \,.
\Eeq
In the following discussion, all equations should be treated as referring
to the hatted quantities even though we drop the hats for convenience.
\end{NoteBox}

\section{The model}
The metal plasticity model assumes that we know the total strain rate ($\dot{\BVeps}^t$) and
that this strain rate can be decomposed into a mechanical component ($\dot{\BVeps}$) and a 
thermal expansion component ($\dot{\BVeps}^\alpha$): 
\Beq
  \dot{\BVeps}^t = \dot{\BVeps} + \dot{\BVeps}^\alpha \,.
\Eeq
The thermal expansion component is assumed to be of the rate form
\Beq
  \dot{\BVeps}^\alpha  = \Partial{\BVeps^\alpha}{T} \dot{T} = \alpha \dot{T}
\Eeq
where $T$ is the temperature and $\alpha$ is a coefficient of thermal expansion.
Then the mechanical strain rate can be expressed as
\Beq
  \dot{\BVeps} = \dot{\BVeps}^t - \alpha\dot{T} \,.
\Eeq
The primary function of the metal plasticity model is to compute the stress when 
a mechanical strain rate 
$\dot{\BVeps}$ is given, which we assume can be additively decomposed into 
elastic ($\dot{\BVeps}^e$) and plastic ($\dot{\BVeps}^p$) parts:
\Beq
  \dot{\BVeps} = \dot{\BVeps}^e + \dot{\BVeps}^p \,.
\Eeq
The Cauchy stress ($\Bsig$) is decomposed into volumetric and deviatoric parts:
\Beq \label{eq:stress_decomp}
  \Bsig = p~\BI + \BsT \quad \text{where} \quad  
  p = \Third~\Tr(\Bsig) \quad \Tand \quad
  \BsT = \Dev(\Bsig) = \Bsig - \Third\Tr(\Bsig) \,.
\Eeq
In the above $p = \sigma_m$ is the mean stress and $\BsT$ is the deviatoric stress.
An alternative decomposition that can be used for the isotropic metal plasticity models
implemented in \Vaango is
\Beq \label{eq:stress_decomp_iso}
  \Bsig = \sigma_p \hat{\BI} + \sigma_s \hat{\BsT}~,~~\hat{\BI} = \frac{1}{\sqrt{3}}\BI ~,~~
  \hat{\BsT} = \frac{\BsT}{\Norm{\BsT}{}}~,~~\sigma_p = \sqrt{3} p ~,~~\sigma_s = \Norm{\BsT}{} \,.
\Eeq
This decomposition is useful because $\hat{\BI}$ and $\hat{\BsT}$ for a basis that can be
used to express several other quantities in the metal plasticity models implemented
in \Vaango.  The time derivative of stress can then be expressed as
\Beq
  \dot{\Bsig} = \dot{p}~\BI + \dot{\BsT} = \dot{\sigma}_p~\hat{\BI} + \dot{\sigma}_s~\hat{\BsT} \,.
\Eeq
The isotropy of the material allows us to compute the mean stress using an
equation of state if desired. The deviatoric stress is computed using a rate-form
stress-strain relation.  For convenience, we assume that rate-form relations are used for
both the mean stress and the deviatoric stress.  

\subsection{Purely elastic loading/unloading}
The elastic constitutive relation is assumed to be of the form
\Beq \label{eq:sig_dot_e}
  \dot{\Bsig}^e = \Partial{\Bsig}{\BVeps^e} : \dot{\BVeps}^e
                = \SfC^e : \dot{\BVeps}^e ~,~~
  \SfC^e =  \left(\kappa - \tfrac{2}{3}\mu\right) \BI\otimes\BI + 2\mu\, \Tsym(\SfI)
\Eeq
or,
\Beq \label{eq:elastic_stress_rate}
  \dot{\Bsig}^e = \left(\kappa - \tfrac{2}{3}\mu\right) \Tr(\dot{\BVeps}^e) \BI + 
     2\mu\, \dot{\BVeps}^e\,.
\Eeq
In the above, $\mu(\rho, p, T, \phi, D)$ is the shear modulus, $\kappa(\rho, p,T, \phi, D)$ is the tangent 
bulk modulus, $\BI$ is the second-order identity tensor and $\SfI$ is the fourth-order
identity tensor.  Also, $\rho$ is the mass density, $p$ is the pressure, $T$ is the current 
temperature, $\phi$ is the current porosity and $D$ is a scalar damage parameter.  

The inverse relationship is
\Beq
  \dot{\BVeps}^e = \SfS^e : \dot{\Bsig}^e ~,~~
  \SfS^e = \frac{1}{3}\left(\frac{1}{3\kappa}-\frac{1}{2\mu}\right)\BI\otimes\BI + \frac{1}{2\mu} \Tsym(\SfI)
\Eeq
Using the decomposition~\eqref{eq:stress_decomp_iso}, we can write
\Beq \label{eq:elastic_strain_rate}
  \dot{\BVeps}^e = \SfS^e : \left(\dot{\sigma}_p^e~\hat{\BI} + \dot{\sigma}_s^e~\hat{\BsT}^e \right)
    = \frac{\dot{\sigma}_p^e}{3\kappa}\hat{\BI} + 
      \frac{\dot{\sigma}_s^e}{2\mu} \hat{\BsT}^e \,.
\Eeq

\subsection{Yield condition}
The isotropic metal plasticity yield conditions implemented in \Vaango have the form
\Beq
  f(\Bsig_\beta, \Veps^\Teq_p, \dot{\Veps}^\Teq_p, \phi, D, T, \Edot{\Teq}, \dots) = 0 ~,~~
  \Bsig_\beta := \Bsig-\Bbeta \,.
\Eeq
The quantity $\Bsig_\beta$ is further decomposed into isotropic and deviatoric parts:
\Beq
  \Bsig_\beta = p_\beta \BI + \Bxi
\Eeq
where $\Bxi = \Dev(\Bsig_\beta)$ and $p_\beta = \Tr(\Bsig_\beta)/3$.  Most of the metal yield
conditions in \Vaango use this notation.
Derivatives of $f$ with respect to the stress ($\Bsig$) can therefore be expressed as
\Beq
  \Partial{f}{\Bsig} = \Partial{f}{\Bsig_\beta}:\Partial{\Bsig_\beta}{\Bsig} = \Partial{f}{\Bsig_\beta}
    = \Partial{f}{p_\beta}\Partial{p_\beta}{\Bsig_\beta} + \Partial{f}{\Bxi}:\Partial{\Bxi}{\Bsig_\beta} 
    = \tfrac{1}{3} \Partial{f}{p_\beta} \BI + \Partial{f}{\Bxi} - \tfrac{1}{3} \Tr\left(\Partial{f}{\Bxi}\right)\BI\,.
\Eeq
The isotropic metal yield conditions in \Vaango are expressed in terms of
$\sigma_\Teff^\xi = \sqrt{3 J_2^\xi} = \sqrt{\tfrac{3}{2} \Bxi:\Bxi}$.
Therefore,
\Beq
  \Partial{f}{\Bsig} = 
    \tfrac{1}{3} \Partial{f}{p_\beta} \BI + \Partial{f}{\sigma_\Teff^\xi} \Partial{\sigma_\Teff^\xi}{\Bxi}
   = \tfrac{1}{3} \Partial{f}{p_\beta} \BI + 
     \sqrt{\tfrac{3}{2}} \Partial{f}{\sigma_\Teff^\xi} \frac{\Bxi}{\Norm{\Bxi}{}}\,.
\Eeq
If we express both $\BI$ and $\Bxi$ in terms of the basis $\hat{\BI}$ and $\hat{\BsT}$, we have
\Beq
  \BI = \sqrt{3} \hat{\BI} ~,~~ \Bxi = \xi_s \hat{\BsT}~,~~\xi_s = \Bxi:\hat{\BsT}
  \quad \Tand \quad \Norm{\Bxi}{} = \xi_s
  \quad \implies \quad \frac{\Bxi}{\Norm{\Bxi}{}} = \hat{\BsT} \,.
\Eeq
Therefore,
\Beq \label{eq:N_df_dsig}
  \BN = \Partial{f}{\Bsig} = 
    \frac{1}{\sqrt{3}} \Partial{f}{p_\beta} \hat{\BI} + 
    \sqrt{\tfrac{3}{2}} \Partial{f}{\sigma_\Teff^\xi} \hat{\BsT} ~,~~
  \Norm{\BN}{} = \sqrt{\tfrac{1}{3} \left(\Partial{f}{p_\beta}\right)^2 + 
                       \tfrac{3}{2} \left(\Partial{f}{\sigma_\Teff^\xi}\right)^2} ~,~~
  \hat{\BN} = \frac{\BN}{\Norm{\BN}{}} \,.
\Eeq

The Kuhn-Tucker loading-unloading conditions are
\Beq
  \dot{\lambda} \ge 0 ~;~~  f \le 0 ~;~~ \dot{\lambda}~f = 0
\Eeq
and the consistency condition is $\dot{\lambda}\dot{f} = 0$.

\subsection{Flow rule}
We assume that the plastic rate of deformation is given by the flow rule
\Beq
  \dot{\BVeps}^p = \dot{\lambda}~\hat{\BM} \,.
\Eeq
For the isotropic metal plasticity models in \Vaango, we assume associated plasticity:
\Beq
  \dot{\BVeps}^p = \dot{\lambda}~\hat{\BN}
    = \frac{\dot{\lambda}}{\Norm{\BN}{}}
    \left[\tfrac{1}{\sqrt{3}} \Partial{f}{p_\beta} \hat{\BI} + 
    \sqrt{\tfrac{3}{2}} \Partial{f}{\sigma_\Teff^\xi} \hat{\BsT}\right]\,.
\Eeq

\subsection{Isotropic and kinematic hardening/softening rules}
The equivalent plastic strain ($\Veps_p^\Teq$) evolves according to the relation
\Beq
  \dot{\Veps}^\Teq_p = \dot{\lambda}~h^{\Veps_p} \,.
\Eeq
The back stress ($\Bbeta$) evolves according to the relation
\Beq
  \dot{\Bbeta} = \dot{\lambda}~\Bh^{\beta} \,.
\Eeq
The porosity ($\phi$) is assumed to evolve according to the relation
\Beq
  \dot{\phi} = \dot{\lambda}~h^{\phi} ~.
\Eeq
The damage parameter ($D$) evolves as
\Beq
  \dot{D} = \dot{\lambda}~h^D \,.
\Eeq
The temperature ($T_p$) due to plastic dissipation evolves as
\Beq
  \dot{T}_p = \dot{\lambda}~h^T \,.
\Eeq

\subsection{Elastic-plastic loading/unloading}
During purely elastic loading and unloading
\Beq
  \dot{\lambda} = 0 ~,~~ \dot{\BVeps}^p = \Bzero ~,~~ \dot{\BVeps} = \dot{\BVeps}^e \,.
\Eeq
In that situation, the stress is updated using \eqref{eq:sig_dot_e}.

However, during elastic-plastic deformation, $\dot{\lambda} > 0$, and we have
\Beq
  \Bal
  \dot{\Bsig} & = \Partial{\Bsig}{\BVeps^e} : \dot{\BVeps}^e + 
                  \Partial{\Bsig}{\Bbeta} : \dot{\Bbeta} + 
                  \Partial{\Bsig}{\Veps^\Teq_p}  \dot{\Veps}^\Teq_p + 
                  \Partial{\Bsig}{\phi}  \dot{\phi} + 
                  \Partial{\Bsig}{D}  \dot{D} + 
                  \Partial{\Bsig}{T_p}  \dot{T}_p \\
   & = \SfC^e : \dot{\BVeps}^e + 
       \dot{\lambda}\left[\Partial{\Bsig}{\Bbeta} : \Bh^\beta + 
       \Partial{\Bsig}{\Veps^\Teq_p}  h^{\Veps_p} + 
       \Partial{\Bsig}{\phi} h^\phi + 
       \Partial{\Bsig}{D} h^D + 
       \Partial{\Bsig}{T_p} h^T\right] \\
   & = \SfC^e : \dot{\BVeps} -
       \dot{\lambda}\left[\SfC^e : \hat{\BM} - \Partial{\Bsig}{\Bbeta} : \Bh^\beta -   
       \Partial{\Bsig}{\Veps^\Teq_p}  h^{\Veps_p} -         
       \Partial{\Bsig}{\phi} h^\phi -       
       \Partial{\Bsig}{D} h^D -        
       \Partial{\Bsig}{T_p} h^T\right]
  \Eal
\Eeq
Define
\Beq \label{eq:P_tensor}
  \BP := \SfC^e : \hat{\BM} - \Partial{\Bsig}{\Bbeta} : \Bh^\beta -   
         \Partial{\Bsig}{\Veps^\Teq_p}  h^{\Veps_p} -         
         \Partial{\Bsig}{\phi} h^\phi -       
         \Partial{\Bsig}{D} h^D -        
         \Partial{\Bsig}{T_p} h^T \,.
\Eeq
Then, 
\Beq \label{eq:sig_trial}
  \dot{\Bsig} = \SfC^e : \dot{\BVeps} - \dot{\lambda} \BP 
              = \dot{\Bsig}^\Trial - \dot{\lambda} \BP 
  \quad \text{where} \quad
  \dot{\Bsig}^\Trial := \SfC^e : \dot{\BVeps} \,.
\Eeq
In \Vaango, we assume that the coupling terms $\partial\Bsig/\partial\Bbeta$ are zero
and $T = T_p$ for elastic-plastic coupling.  From~\eqref{eq:sig_dot_e} we have
\Beq
  \dot{\Bsig}^e = \left(\kappa - \tfrac{2}{3}\mu\right) \Tr(\dot{\BVeps}^e) \BI + 
                  2\mu\, \dot{\BVeps}^e
\Eeq
We can use this relation to estimate the coupling terms for the internal variables
$\eta \in \{\Veps^\Teq_p, \phi, D, T\}$: 
\Beq \label{eq:dsig_deta}
  \Partial{\Bsig}{\eta} =          
     \left(\Partial{\kappa}{\eta} - \tfrac{2}{3}\Partial{\mu}{\eta}\right) 
     \Tr(\BVeps^e) \BI + 2\Partial{\mu}{\eta}\, \BVeps^e 
\Eeq
Similarly, from \eqref{eq:elastic_strain_rate}, choosing the basis to be $\hat{\BI}$ and 
$\hat{\BsT}^\Trial = \Dev(\Bsig^\Trial)/\Norm{\Dev(\Bsig^\Trial)}{}$, 
\Beq 
  \BVeps^e 
    = \frac{\sigma_p^e}{3\kappa}\hat{\BI} + 
      \frac{\sigma_s^e \sigma_{ss}}{2\mu} \hat{\BsT}^\Trial \quad \Tand \quad
  \Tr(\BVeps^e) =  \frac{\sigma_p^e}{\sqrt{3}\kappa}
\Eeq
where $\sigma_{ss} = \hat{\BsT}^e:\hat{\BsT}^\Trial$.
Substitution into \eqref{eq:dsig_deta} leads to
\Beq
  \Partial{\Bsig}{\eta} =          
     \frac{1}{\kappa} \Partial{\kappa}{\eta} \sigma_p^e \hat{\BI} + 
     \frac{1}{\mu} \Partial{\mu}{\eta} \sigma_s^e \sigma_{ss} \hat{\BsT}^\Trial \,.
\Eeq
Also, for associated plasticity and using \eqref{eq:N_df_dsig}, 
\Beq \label{eq:CM_iso}
  \SfC^e : \hat{\BM} = \frac{1}{\Norm{\BN}{}}\left[\sqrt{3}\kappa \Partial{f}{p_\beta}\hat{\BI} + 
   2\sqrt{\tfrac{3}{2}}\mu \Partial{f}{\sigma_\Teff^\xi}\hat{\BsT}\right] = 
  \frac{1}{\Norm{\BN}{}}\left[\sqrt{3}\kappa \Partial{f}{p_\beta}\hat{\BI} + \sqrt{6}\mu\Partial{f}{\sigma_\Teff^\xi}\sigma_{ss}\hat{\BsT}^\Trial\right]
\Eeq
Therefore,
\Beq \label{eq:P_iso}
  \BP = \left[
          \frac{\sqrt{3}\kappa}{\Norm{\BN}{}} \Partial{f}{p_\beta} 
          - \frac{1}{\kappa} \sum_\eta \Partial{\kappa}{\eta} 
            \sigma_p ^e \right] \hat{\BI} + 
        \left[
          \frac{\sqrt{6}\mu}{\Norm{\BN}{}}\Partial{f}{\sigma_\Teff^\xi}
          - \frac{1}{\mu}
            \sum_\eta \Partial{\mu}{\eta} \sigma_s^e\right] \sigma_{ss} \hat{\BsT}^\Trial
\Eeq
For an elastic-plastic load step, we can compute the plastic strain rate  
using \eqref{eq:elastic_strain_rate}:
\Beq \label{eq:plastic_strain_rate}
  \dot{\BVeps}^p = \dot{\BVeps} - 
    \frac{\dot{\sigma}_p^e}{3\kappa} \hat{\BI} - 
    \frac{\dot{\sigma}_s^e}{2\mu} \hat{\BsT}^\Trial \,.
\Eeq

\subsection{Consistency condition}
The consistency condition requires that, when $\dot{\lambda} > 0$,
\Beq
  \dot{f}(\Bsig, \Bbeta, \Veps_p^\Teq, \dot{\Veps}_p^\Teq, \phi, D, T, \Edot{\Teq}, \dots) = 0 ~.
\Eeq
For rate-independent plasticity, from the chain rule,
\Beq
  \dot{f} = \Partial{f}{\Bsig}:\dot{\Bsig} + \Partial{f}{\Bbeta}:\dot{\Bbeta} + 
    \Partial{f}{\Veps^\Teq_p}~\dot{\Veps}^\Teq_p + \Partial{f}{\phi}~\dot{\phi} +
    \Partial{f}{D}~\dot{D} + \Partial{f}{T_p}~\dot{T}_p = 0~.
\Eeq
Using the hardening/softening rules, 
\Beq
  \Partial{f}{\Bsig}:\dot{\Bsig} + \dot{\lambda}\left[\Partial{f}{\Bbeta}:\Bh^\beta + 
    \Partial{f}{\Veps^\Teq_p}~h^{\Veps_p} + \Partial{f}{\phi}~h^\phi +
    \Partial{f}{D}~h^D + \Partial{f}{T_p}~h^T\right] = 0
\Eeq
or
\Beq
  \Partial{f}{\Bsig}:\dot{\Bsig} + \dot{\lambda} H = 0 \,.
\Eeq
Define,
\Beq \label{eq:def_N_H}
  \BN := \Partial{f}{\Bsig} ~,~~ \hat{\BN} := \frac{\BN}{\Norm{\BN}{}} ~,~~
  \hat{H} := \frac{H}{\Norm{\BN}{}} \,.
\Eeq
Then,
\Beq \label{eq:consistency}
  \hat{\BN}:\dot{\Bsig} + \dot{\lambda} \hat{H} = 0 \,.
\Eeq
Combining the stress-rate equation \eqref{eq:sig_trial} with the consistency equation
\eqref{eq:consistency}, we have
\Beq
  \hat{\BN}:\dot{\Bsig}^\Trial = \hat{\BN} : \SfC^e : \dot{\BVeps} = 
    \dot{\lambda} (\hat{\BN}:\BP - \hat{H})  \,.
\Eeq
Therefore, 
\Beq \label{eq:dot_lambda}
  \dot{\lambda} = \frac{\hat{\BN}:\dot{\Bsig}^\Trial}{\hat{\BN}:\BP - \hat{H}} 
                = \frac{\hat{\BN} : \SfC^e : \dot{\BVeps}}{\hat{\BN}:\BP - \hat{H}} 
\Eeq
Substituting this expression to \eqref{eq:sig_trial}, we have
\Beq 
  \dot{\Bsig} = \dot{\Bsig}^\Trial - \frac{\hat{\BN}:\dot{\Bsig}^\Trial}{\hat{\BN}:\BP - \hat{H}} \BP 
     = \dot{\Bsig}^\Trial - \frac{\BP\otimes\hat{\BN}}{\hat{\BN}:\BP - \hat{H}}:\dot{\Bsig}^\Trial 
\Eeq
or,
\Beq
  \dot{\Bsig} = \SfC^e : \dot{\BVeps}  
      - \frac{(\BP\otimes\hat{\BN}):\SfC^e}{\hat{\BN}:\BP - \hat{H}} : \dot{\BVeps}  
     = \SfC^{ep} : \dot{\BVeps} \,.
\Eeq
The quantity $\SfC^{ep}$ is the continuum elastic-plastic tangent modulus. 

\section{Stress update}
The first step in the stress update procedure is to compute a trial stress state from
\Beq 
  \dot{\Bsig}^\Trial = \SfC^e : \dot{\BVeps} \,.
\Eeq
We assume that
\Beq
  \Bsig^\Trial = \Bsig_n + \Delta t (\SfC^e_n : \dot{\BVeps}_{n+1}) 
\Eeq
where $\Bsig_n$ is the stress at the end of time $t_n$, $\SfC^e_n$ is the elastic modulus
at that time, $\dot{\BVeps}_{n+1}$ is the strain rate computed from the symmetric
part of the unrotated velocity gradient, and $\Delta t = t_{n+1} - t_n$ is the timestep size.

The trial state contains the vector
\Beq
  \Beta^\Trial = 
    \left[\Bsig^\Trial, \Bbeta_n, (\Veps^\Teq_p)_n, (\dot{\Veps}^\Teq_p)_n,
           \phi_n, D_n, T_n, \Edot{\Teq}_n, \kappa_n, \mu_n, \dots\right].
\Eeq
where the subscript ($n$) indicates the state at the end of time $t_n$.

The trial state is used to compute the yield function
\Beq
  f_y = f[\Bsig^\Trial, \Bbeta_n, (\Veps^\Teq_p)_n, 
          (\dot{\Veps}^\Teq_p)_n, \phi_n, D_n, T_n, \Edot{\Teq}_n, \kappa_n, \mu_n, \dots]
\Eeq
If $f_y \le 0$, the trial state is in the elastic regime and we update the stress using
\Beq
  \Bal
  &\Bsig_{n+1} = \Bsig^\Trial~,~~\Bbeta_{n+1} = \Bbeta_n ~,~~(\Veps^\Teq_p)_{n+1} = (\Veps^\Teq_p)_n
  ~,~~ (\dot{\Veps}^\Teq_p)_{n+1} = (\dot{\Veps}^\Teq_p)_n \\
  &\phi_{n+1} = \phi_n
  ~,~~ D_{n+1} = D_n ~,~~ T_{n+1} = T_n \\
  &\kappa_{n+1} = \kappa(p_{n+1}, T_n) ~,~~
  \mu_{n+1} = \mu(p_{n+1}, T_n) \,.
  \Eal
\Eeq
If $f_y > 0$, the trial state is outside the yield surface in the elastic-plastic regime.  We
can used a backward Euler algorithm to compute the updated stress state:
\Beq
  \frac{\Bsig_{n+1} - \Bsig_{n}}{\Delta t} = \frac{\Bsig^\Trial - \Bsig_n}{\Delta t} - \frac{\lambda_{n+1} - \lambda_n}{\Delta t} \BP_{n+1} 
 \quad \Tor \quad
  \Bsig_{n+1} = \Bsig^\Trial - \Delta\lambda_{n+1} \BP_{n+1} \,.
\Eeq
The plastic strain and the internal variables can similarly be updated using
\Beq
  \Bal
    \BVeps^p_{n+1} &= \BVeps^p_n &+& \Delta\lambda_{n+1}~\hat{\BM}_{n+1} \\
    (\Veps^\Teq_p)_{n+1} &= (\Veps^\Teq_p)_n &+& \Delta\lambda_{n+1}~h^{\Veps_p}_{n+1} \\
    \Bbeta_{n+1} & = \Bbeta_n &+& \Delta\lambda_{n+1}~\Bh^{\beta}_{n+1} \\
    \phi_{n+1} & = \phi_n &+& \Delta\lambda_{n+1}~h^{\phi}_{n+1} \\
    D_{n+1} & = D_n &+& \Delta\lambda_{n+1}~h^D_{n+1} \\
    (T_p)_{n+1} & = (T_p)_n &+& \Delta\lambda_{n+1}~h^T_{n+1} \,.
  \Eal
\Eeq
In addition, the stress state has to lie on the yield surface:
\Beq
  f(\Bsig^\Trial - \Delta\lambda_{n+1} \BP_{n+1}) = 0 \,.
\Eeq
Finally, the consistency condition needs to be satisfied:
\Beq
  \hat{\BN}_{n+1} : (\Bsig_{n+1} - \Bsig_n) + \Delta \lambda_{n+1} \hat{H}_{n+1}  = 0
\Eeq
or,
\Beq
  \hat{\BN}_{n+1} : (\Bsig^\Trial - \Bsig_n) = \Delta \lambda_{n+1} (\hat{\BN}_{n+1}:\BP_{n+1} - \hat{H}_{n+1})\,.
\Eeq

\subsection{Iterative solution}
\Vaango assumes associated plasticity for isotropic metals, i.e., $\hat{\BM} = \hat{\BN}$.  Therefore,
the following coupled equations, not all of which are independent, have to be solved for 
$\Gamma := \Delta\lambda_{n+1}$ and the updated state
$\left[\Bsig_{n+1}, \kappa_{n+1}, \mu_{n+1}, \BVeps^p_{n+1}, (\Veps^\Teq_p)_{n+1}, \Bbeta_{n+1}, 
       \phi_{n+1}, D_{n+1}, (T_p)_{n+1}\right]$:
\Beq \label{eq:coupled_metal}
  \Bal
  \Bsig_{n+1} & = \Bsig^\Trial - \Gamma \BP_{n+1} \\
  \BVeps^p_{n+1} &= \BVeps^p_n + \Gamma \hat{\BN}_{n+1} \\
  \Bbeta_{n+1} & = \Bbeta_n + \Gamma \Bh^{\beta}_{n+1} \\
  (\Veps^\Teq_p)_{n+1} &= (\Veps^\Teq_p)_n + \Gamma h^{\Veps_p}_{n+1} \\
  \phi_{n+1} & = \phi_n + \Gamma h^{\phi}_{n+1} \\
  D_{n+1} & = D_n + \Gamma h^D_{n+1} \\
  (T_p)_{n+1} & = (T_p)_n + \Gamma h^T_{n+1} \\
  \hat{\BN}_{n+1} &: (\Bsig^\Trial - \Bsig_n) - \Gamma (\hat{\BN}_{n+1}:\BP_{n+1} - \hat{H}_{n+1})  = 0\\
  f_{n+1} &= f(\Bsig^\Trial - \Gamma \BP_{n+1}) = 0 
  \Eal
\Eeq
where
\Beq 
  \Bal
  \BP_{n+1}   & = \SfC^e_{n+1} : \hat{\BN}_{n+1} - 
                  \left.\Partial{\Bsig}{\Bbeta}\right|_{n+1} : \Bh^\beta_{n+1} -   
                  \left.\Partial{\Bsig}{\Veps^\Teq_p}\right|_{n+1}  h^{\Veps_p}_{n+1} -         
                  \left.\Partial{\Bsig}{\phi}\right|_{n+1} h^\phi_{n+1} -       
                  \left.\Partial{\Bsig}{D}\right|_{n+1} h^D_{n+1} -        
                  \left.\Partial{\Bsig}{T_p}\right|_{n+1} h^T_{n+1} \\
  \hat{\BN}_{n+1} &= \frac{\BN_{n+1}}{\Norm{\BN_{n+1}}{}} ~,~~
  \BN_{n+1}  = \left.\Partial{f}{\Bsig}\right|_{n+1} \\
  \hat{H}_{n+1} &= \frac{H_{n+1}}{\Norm{\BN_{n+1}}{}} ~,~~
  H_{n+1}  = \left.\Partial{f}{\Bbeta}\right|_{n+1}:\Bh^\beta_{n+1} + 
                    \left.\Partial{f}{\Veps^\Teq_p}\right|_{n+1}~h^{\Veps_p}_{n+1} + 
                    \left.\Partial{f}{\phi}\right|_{n+1}~h^\phi_{n+1} +
                    \left.\Partial{f}{D}\right|_{n+1}~h^D_{n+1} + 
                    \left.\Partial{f}{T_p}\right|_{n+1}~h^T_{n+1} \,.
  \Eal
\Eeq
Let us express the stresses and strains as vectors:
\Beq
  \BSv := \left[\sigma_{11}, \sigma_{22}, \sigma_{33}, \sqrt{2}\sigma_{23}, \sqrt{2}\sigma_{31}, 
               \sqrt{2}\sigma_{12}\right] \\
  \BEv^p := \left[\Veps^p_{11}, \Veps^p_{22}, \Veps^p_{33}, \sqrt{2}\Veps^p_{23}, \sqrt{2}\Veps^p_{31}, 
               \sqrt{2}\Veps^p_{12}\right] \,.
\Eeq
We can also write
\Beq
  \BNv :=  \left[\Partial{f}{\sigma_{11}}, \Partial{f}{\sigma_{22}}, \Partial{f}{\sigma_{33}}, 
                 \sqrt{2}\Partial{f}{\sigma_{23}}, \sqrt{2}\Partial{f}{\sigma_{31}}, 
                 \sqrt{2}\Partial{f}{\sigma_{12}}\right] ~,~~ \hat{\BNv} = \frac{\BNv}{\Norm{\BNv}{}} \,.
\Eeq
Because of the isotropy of the elasticity tensor, we have
\Beq
  \Bal
  \SfC^e : \hat{\BN} = \BCv &= 
    \left[2\mu N_1 + (\kappa -\tfrac{2}{3}\mu) (N_1+N_2+N_3), 
          2\mu N_2 + (\kappa -\tfrac{2}{3}\mu) (N_1+N_2+N_3), \right.\\
    &\left. \quad      2\mu N_3 + (\kappa -\tfrac{2}{3}\mu) (N_1+N_2+N_3), 
          2\sqrt{2} \mu N_{4}, 2\sqrt{2} \mu N_{5}, 2\sqrt{2} \mu N_{6}\right] 
  \Eal
\Eeq
where $N_i$ are the components of the vector $\hat{\BNv}$.
If we ignore the elastic-plastic coupling term $\partial \Bsig/\partial \Bbeta$, and define
\Beq
  \Bal
  \BZv^{\Veps_p} &:= 
    \left[\Partial{\sigma_{11}}{\Veps_p^\Teq} h^{\Veps_p}, \Partial{\sigma_{22}}{\Veps_p^\Teq} h^{\Veps_p}, 
          \Partial{\sigma_{33}}{\Veps_p^\Teq} h^{\Veps_p}, \sqrt{2}\Partial{\sigma_{23}}{\Veps_p^\Teq} h^{\Veps_p}, 
          \sqrt{2}\Partial{\sigma_{31}}{\Veps_p^\Teq} h^{\Veps_p}, \sqrt{2}\Partial{\sigma_{12}}{\Veps_p^\Teq} h^{\Veps_p}\right] \\
  \BZv^{\phi} &:= 
    \left[\Partial{\sigma_{11}}{\phi} h^{\phi}, \Partial{\sigma_{22}}{\phi} h^{\phi}, 
          \Partial{\sigma_{33}}{\phi} h^{\phi}, \sqrt{2}\Partial{\sigma_{23}}{\phi} h^{\phi}, 
          \sqrt{2}\Partial{\sigma_{31}}{\phi} h^{\phi}, \sqrt{2}\Partial{\sigma_{12}}{\phi} h^{\phi}\right] \\
  \BZv^{D} &:= 
    \left[\Partial{\sigma_{11}}{D} h^D, \Partial{\sigma_{22}}{D} h^D, 
          \Partial{\sigma_{33}}{D} h^D, \sqrt{2}\Partial{\sigma_{23}}{D} h^D, 
          \sqrt{2}\Partial{\sigma_{31}}{D} h^D, \sqrt{2}\Partial{\sigma_{12}}{D} h^D\right] \\
  \BZv^{T_p} &:= 
    \left[\Partial{\sigma_{11}}{T_p} h^T, \Partial{\sigma_{22}}{T_p} h^T, 
          \Partial{\sigma_{33}}{T_p} h^T, \sqrt{2}\Partial{\sigma_{23}}{T_p} h^T, 
          \sqrt{2}\Partial{\sigma_{31}}{T_p} h^T, \sqrt{2}\Partial{\sigma_{12}}{T_p} h^T\right] 
  \Eal
\Eeq
we have
\Beq
  \BPv = \BCv - \BZv^{\Veps_p} - \BZv^{\phi} - \BZv^D - \BZv^{T_p} \,.
\Eeq
Noting that $\Bbeta$ is required to be symmetric for the conservation of angular momentum, we
can express the internal variables as a vector such that
\Beq
  \Bal
  \BQv &:= \left[\beta_{11}, \beta_{22}, \beta_{33}, \sqrt{2}\beta_{23}, \sqrt{2}\beta_{31}, \sqrt{2}\beta_{12},
                \Veps^\Teq_p, \phi, D, T_p\right] \\
  \BHv &:= \left[h^\beta_{11}, h^\beta_{22}, h^\beta_{33}, \sqrt{2}h^\beta_{23}, \sqrt{2}h^\beta_{31}, 
                    \sqrt{2}h^\beta_{12}, h^{\Veps_p}, h^\phi, h^D, h^T\right] \\
  \Partial{f}{\BQv} &:= \left[\Partial{f}{\beta_{11}}, \Partial{f}{\beta_{22}}, \Partial{f}{\beta_{33}}, 
                    \sqrt{2}\Partial{f}{\beta_{23}}, \sqrt{2}\Partial{f}{\beta_{31}}, 
                    \sqrt{2}\Partial{f}{\beta_{12}}, \Partial{f}{\Veps_p^\Teq}, \Partial{f}{\phi}, 
                    \Partial{f}{D}, \Partial{f}{T_p}\right] \\
  \Eal
\Eeq
Therefore,
\Beq \label{eq:H_def_vec}
  H = \Partial{f}{\BQv} \cdot \BHv ~,~~ \hat{H} = \frac{H}{\Norm{\BNv}{}}\,.
\Eeq
Then \eqref{eq:coupled_metal}, can be written in vector form as
\Beq \label{eq:Gamma_direct}
  \Bal
  \BSv_{n+1} & = \BSv^\Trial - \Gamma \BPv_{n+1} \\
  \BEv^p_{n+1} &= \BEv^p_n + \Gamma \hat{\BNv}_{n+1} \\
  \BQv_{n+1} & = \BQv_n + \Gamma \BHv_{n+1} \\
  \hat{\BNv}_{n+1} \cdot & (\BSv^\Trial - \BSv_n) - \Gamma (\hat{\BNv}_{n+1}\cdot\BPv_{n+1} - \hat{H}_{n+1})  = 0 \quad \implies \quad
  \Gamma = \frac{\hat{\BNv}_{n+1}\cdot (\BSv^\Trial - \BSv_n)}  
                 {\left(\hat{\BNv}_{n+1}\cdot\BPv_{n+1} - \hat{H}_{n+1}\right)} \\
  f_{n+1} &= f(\BSv^\Trial - \Gamma \BPv_{n+1}) = 0 
  \Eal
\Eeq
Since $\BPv = \BPv(\BSv, \BEv^p, \BQv)$, $\BNv = \BNv(\BSv, \BEv^p, \BQv)$, $\BHv = \BHv(\BSv, \BEv^p, \BQv)$, 
and $H = H(\BSv, \BEv^p, \BQv)$, we can write the equations above as residuals:
\Beq
  \Bal
  \Brv_S(\Gamma,\BSv,\BEv^p,\BQv) & := -\BSv_{n+1} + \BSv^\Trial - \Gamma \BPv_{n+1}(\BSv,\BEv^p, \BQv) = \Bzero  \\
  \Brv_E(\Gamma,\BSv,\BEv^p,\BQv) & := -\BEv^p_{n+1} + \BEv^p_n + \Gamma \hat{\BNv}_{n+1}(\BSv,\BEv^p, \BQv) = \Bzero \\
  \Brv_q(\Gamma,\BSv,\BEv^p,\BQv) & := -\BQv_{n+1} + \BQv_n + \Gamma \BHv_{n+1} (\BSv,\BEv^p, \BQv) = \Bzero \\
  r_f(\Gamma,\BSv,\BEv^p,\BQv) &:= f(\BSv_{n+1},\BEv^p_{n+1},\BQv_{n+1}) = 0 
  \Eal
\Eeq
First-order Taylor series expansions of these functions at $(\Gamma,\BSv_n,\BEv^p_n,\BQv_n)$, give
\Beq
  \Bal
  \Brv_S(\Gamma,\BSv,\BEv^p,\BQv) & \approx \Brv_S(\Gamma,\BSv_n,\BEv^p_n,\BQv_n) + \\
                        &\quad       \left.\Partial{\Brv_S}{\Gamma}\right|_n (\Gamma - \Gamma_n) + 
                                     \left.\Partial{\Brv_S}{\BSv}\right|_n\cdot(\BSv - \BSv_n) + 
                                     \left.\Partial{\Brv_S}{\BEv^p}\right|_n\cdot(\BEv^p - \BEv^p_n) + 
                                     \left.\Partial{\Brv_S}{\BQv}\right|_n\cdot(\BQv - \BQv_n) \\
  \Brv_E(\Gamma,\BSv,\BEv^p,\BQv) & \approx \Brv_E(\Gamma,\BSv_n,\BEv^p_n,\BQv_n) + \\
                        &\quad       \left.\Partial{\Brv_Q}{\Gamma}\right|_n (\Gamma - \Gamma_n) + 
                                     \left.\Partial{\Brv_Q}{\BSv}\right|_n\cdot(\BSv - \BSv_n) + 
                                     \left.\Partial{\Brv_Q}{\BEv^p}\right|_n\cdot(\BEv^p - \BEv^p_n) + 
                                     \left.\Partial{\Brv_Q}{\BQv}\right|_n\cdot(\BQv - \BQv_n) \\
  \Brv_Q(\Gamma,\BSv,\BEv^p,\BQv) & \approx \Brv_Q(\Gamma,\BSv_n,\BEv^p_n,\BQv_n) + \\
                        &\quad       \left.\Partial{\Brv_E}{\Gamma}\right|_n (\Gamma - \Gamma_n) + 
                                     \left.\Partial{\Brv_E}{\BSv}\right|_n\cdot(\BSv - \BSv_n) + 
                                     \left.\Partial{\Brv_E}{\BEv^p}\right|_n\cdot(\BEv^p - \BEv^p_n) + 
                                     \left.\Partial{\Brv_E}{\BQv}\right|_n\cdot(\BQv - \BQv_n) \\
  r_f(\Gamma,\BSv,\BEv^p,\BQv)    & \approx r_f(\Gamma,\BSv_n,\BEv^p_n,\BQv_n) + \\
                        &\quad       \left.\Partial{r_f}{\Gamma}\right|_n (\Gamma - \Gamma_n) + 
                                     \left.\Partial{r_f}{\BSv}\right|_n\cdot(\BSv - \BSv_n) + 
                                     \left.\Partial{r_f}{\BEv^p}\right|_n\cdot(\BEv^p - \BEv^p_n) + 
                                     \left.\Partial{r_f}{\BQv}\right|_n\cdot(\BQv - \BQv_n) 
  \Eal
\Eeq
Since the residuals are required to be zero at the end of the timestep, we get the following 
rule for the $k$-th iteration, 
\Beq
  \Bal
    &\left.\Partial{\Brv_S}{\Gamma}\right|_k \Delta\Gamma + 
    \left.\Partial{\Brv_S}{\BSv}\right|_{k}\cdot \Delta\BSv + 
    \left.\Partial{\Brv_S}{\BEv^p}\right|_{k}\cdot \Delta\BEv^p + 
    \left.\Partial{\Brv_S}{\BQv}\right|_{k}\cdot \Delta\BQv = -\Brv_S(\Gamma_k,\BSv_k,\BEv^p_k,\BQv_k) \\
    &\left.\Partial{\Brv_E}{\Gamma}\right|_k \Delta\Gamma + 
    \left.\Partial{\Brv_E}{\BSv}\right|_{k}\cdot \Delta\BSv + 
    \left.\Partial{\Brv_E}{\BEv^p}\right|_{k}\cdot \Delta\BEv^p + 
    \left.\Partial{\Brv_E}{\BQv}\right|_{k}\cdot \Delta\BQv = -\Brv_E(\Gamma_k,\BSv_k,\BEv^p_k,\BQv_k) \\
    &\left.\Partial{\Brv_Q}{\Gamma}\right|_k  \Delta\Gamma + 
    \left.\Partial{\Brv_Q}{\BSv}\right|_{k}\cdot \Delta\BSv + 
    \left.\Partial{\Brv_Q}{\BEv^p}\right|_{k}\cdot \Delta\BEv^p + 
    \left.\Partial{\Brv_Q}{\BQv}\right|_{k}\cdot \Delta\BQv = -\Brv_Q(\Gamma_k,\BSv_k,\BEv^p_k,\BQv_k) \\
    &\left.\Partial{r_f}{\Gamma}\right|_k  \Delta\Gamma +
    \left.\Partial{r_f}{\BSv}\right|_{k}\cdot \Delta\BSv +
    \left.\Partial{r_f}{\BEv^p}\right|_{k}\cdot \Delta\BEv^p +
    \left.\Partial{r_f}{\BQv}\right|_{k}\cdot \Delta\BQv = -r_f(\Gamma_k,\BSv_k,\BEv^p_k,\BQv_k) 
  \Eal
\Eeq
where
\Beq
  \Delta\Gamma = \Gamma_{k+1} - \Gamma_k~,~~
  \Delta\BSv = \BSv_{k+1} - \BSv_k~,~~
  \Delta\BEv^p = \BEv^p_{k+1} - \BEv^p_k~,~~
  \Delta\BQv = \BQv_{k+1} - \BQv_k \,.
\Eeq
The derivatives of the residuals are (dropping subscripts $n+1$ for convenience), 
\Beq
  \Bal
  &\Partial{\Brv_S}{\Gamma} = -\BPv ~,~~
  \Partial{\Brv_E}{\Gamma} = \hat{\BNv} ~,~~
  \Partial{\Brv_Q}{\Gamma} = \BHv ~,~~
  \Partial{r_f}{\Gamma} = 0  \\
  &\Partial{\Brv_S}{\BSv} = -\MI - \Gamma \Partial{\BPv}{\BSv} ~,~~
  \Partial{\Brv_E}{\BSv} = \Gamma \Partial{\hat{\BNv}}{\BSv} ~,~~
  \Partial{\Brv_Q}{\BSv} = \Gamma \Partial{\BHv}{\BSv} ~,~~
  \Partial{r_f}{\BSv} = \Partial{f}{\BSv}  \\
  &\Partial{\Brv_S}{\BEv^p} = - \Gamma \Partial{\BPv}{\BEv^p} ~,~~
  \Partial{\Brv_E}{\BEv^p} = -\MI + \Gamma \Partial{\hat{\BNv}}{\BEv^p} ~,~~
  \Partial{\Brv_Q}{\BEv^p} = \Gamma \Partial{\BHv}{\BEv^p} ~,~~
  \Partial{r_f}{\BEv^p} = \Partial{f}{\BEv^p}  \\
  &\Partial{\Brv_S}{\BQv} = - \Gamma \Partial{\BPv}{\BQv} ~,~~
  \Partial{\Brv_E}{\BQv} = \Gamma \Partial{\hat{\BNv}}{\BQv} ~,~~
  \Partial{\Brv_Q}{\BQv} = -\MI + \Gamma \Partial{\BHv}{\BQv} ~,~~
  \Partial{r_f}{\BQv} = \Partial{f}{\BQv} 
  \Eal
\Eeq
Therefore, using $\BNv = \partial f/\partial \BSv$, 
\Beq
  \Bal
    &-\BPv_k \Delta\Gamma  
     -\left(\MI + \Gamma_k \left.\Partial{\BPv}{\BSv}\right|_k\right)\cdot \Delta\BSv 
     -\Gamma_k \left.\Partial{\BPv}{\BEv^p}\right|_k \cdot \Delta\BEv^p  
     -\Gamma_k \left.\Partial{\BPv}{\BQv}\right|_k\cdot \Delta\BQv = -\Brv_S(\Gamma_k,\BSv_k,\BEv^p_k,\BQv_k) \\
    & \hat{\BNv}_k \Delta\Gamma  
     +\Gamma_k \left.\Partial{\hat{\BNv}}{\BSv}\right|_k\cdot \Delta\BSv  
     -\left(\MI - \Gamma_k \left.\Partial{\hat{\BNv}}{\BEv^p}\right|_k\right)\cdot \Delta\BEv^p 
     +\Gamma_k \left.\Partial{\hat{\BNv}}{\BQv}\right|_k\cdot \Delta\BQv = -\Brv_E(\Gamma_k,\BSv_k,\BEv^p_k,\BQv_k) \\
    & \BHv_k  \Delta\Gamma  
     +\Gamma_k \left.\Partial{\BHv}{\BSv}\right|_{k}\cdot \Delta\BSv  
     +\Gamma_k \left.\Partial{\BHv}{\BEv^p}\right|_k\cdot \Delta\BEv^p  
     -\left(\MI - \Gamma_k \left.\Partial{\BHv}{\BQv}\right|_k\right)\cdot \Delta\BQv = -\Brv_Q(\Gamma_k,\BSv_k,\BEv^p_k,\BQv_k) \\
    & \BNv_k \cdot \Delta\BSv 
     +\left.\Partial{f}{\BEv^p}\right|_k\cdot \Delta\BEv^p 
     +\left.\Partial{f}{\BQv}\right|_k\cdot \Delta\BQv = -r_f(\Gamma_k,\BSv_k,\BEv^p_k,\BQv_k)
  \Eal
\Eeq
Because the derivatives of $\hat{\BNv}$, $\BPv$, $\BHv$ with respect 
to $\BSv$, $\BEv^p$, $\BQv$ may be difficult to calculate, it is more convenient
to use a semi-implicit scheme where the quantities $\hat{\BNv}$, $\BPv$, $\BHv$ are evaluated 
at $t_n$.  In that case we have
\Beq
  \Bal
    &-\BPv_k \Delta\Gamma  
     -\Delta\BSv 
     = -\Brv_S(\Gamma_k,\BSv_k,\BEv^p_k,\BQv_k) \\
    & \hat{\BNv}_k \Delta\Gamma  
     -\Delta\BEv^p 
     = -\Brv_E(\Gamma_k,\BSv_k,\BEv^p_k,\BQv_k) \\
    & \BHv_k  \Delta\Gamma  
     -\Delta\BQv 
     = -\Brv_Q(\Gamma_k,\BSv_k,\BEv^p_k,\BQv_k) \\
    &\BNv_k \cdot \Delta\BSv 
     +\left.\Partial{f}{\BEv^p}\right|_k\cdot \Delta\BEv^p 
     +\left.\Partial{f}{\BQv}\right|_k\cdot \Delta\BQv = -r_f(\Gamma_k,\BSv_k,\BEv^p_k,\BQv_k)
  \Eal
\Eeq
We now force $\Brv_S$, $\Brv_E$, and $\Brv_Q$ to be zero at all times, leading
to the expressions
\Beq
  \Bal
    &\Delta\BSv = -\BPv_k \Delta\Gamma  \\
    &\Delta\BEv^p = \hat{\BNv}_k \Delta\Gamma  \\
    &\Delta\BQv = \BHv_k  \Delta\Gamma  \\
    &r_f(\Gamma_k,\BSv_k,\BEv^p_k,\BQv_k)
     +\BNv_k \cdot \Delta\BSv 
     +\left.\Partial{f}{\BEv^p}\right|_k\cdot \Delta\BEv^p 
     +\left.\Partial{f}{\BQv}\right|_k\cdot \Delta\BQv = 0
  \Eal
\Eeq
Plugging the expressions for $\Delta\BSv$, $\Delta\BEv^p$, $\Delta\BQv$ from the 
first three equations into the fourth gives us
\Beq
  r_f(\Gamma_k,\BSv_k,\BEv^p_k,\BQv_k)
     -\BNv_k \cdot \BPv_k  \Delta\Gamma
     +\left.\Partial{f}{\BEv^p}\right|_k\cdot \hat{\BNv}_k\Delta\Gamma
     +\left.\Partial{f}{\BQv}\right|_k\cdot \BHv_k\Delta\Gamma = 0
\Eeq
or
\Beq
  \Delta\Gamma = \frac{f(\BSv_k,\BEv^p_k,\BQv_k)}
                      {\BNv_k \cdot \BPv_k
                       -\left.\Partial{f}{\BEv^p}\right|_k\cdot \hat{\BNv}_k
                       -\left.\Partial{f}{\BQv}\right|_k\cdot \BHv_k}
\Eeq
In the metal plasticity models implemented in \Vaango, there is no direct
dependence of $f$ on $\BEv^p$. Therefore, using \eqref{eq:H_def_vec}, 
\Beq \label{eq:Gamma_iter}
  \Gamma_{k+1} = \Gamma_k + \frac{f(\BSv_k,\BQv_k)}
                                 {\BNv_k \cdot \BPv_k - H_k} \,.
\Eeq
All quantities on the right hand side of the above equation are known, and we can
compute $\Gamma_{k+1}$.  The other variables can now be updated using
\Beq \label{eq:Stress_iter}
  \BSv_{k+1} = \BSv_k -\BPv_k \Delta\Gamma ~,~~
  \BEv^p_{k+1} = \BEv^p_k + \hat{\BNv}_k \Delta\Gamma ~,~~
  \BQv_{k+1} = \BQv_k + \BHv_k \Delta\Gamma 
\Eeq
The iterative process can be stopped when $r_f$ is close to 0 and $\Gamma_{k+1}$ is
close to the value required to satisfy consistency given in \eqref{eq:Gamma_direct}.

\subsection{Stress update in reduced stress space}
The isotropic metal yield functions in \Vaango depend only on two invariants of stress.
Therefore, the return mapping can be carried out in the reduced stress space spanned
by $\hat{\BI}$ and $\hat{\BsT}^\Trial$ where
\Beq
  \hat{\BI} = \frac{\BI}{\Norm{\BI}{}} = \frac{\BI}{\sqrt{3}} 
  \quad \Tand \quad
  \hat{\BsT}^\Trial = \frac{\BsT^\Trial}{\Norm{\BsT^\Trial}{}} =
                      \frac{\Dev(\Bsig^\Trial)}{\Norm{\Dev(\Bsig^\Trial)}{}} \,.
\Eeq
The stress can be expressed in terms of this basis as,
\Beq
  \Bsig_{n+1} = (\sigma_p)_{n+1} \hat{\BI} + (\sigma_s)_{n+1} \hat{\BsT}^\Trial
  \quad \text{where} \quad
  (\sigma_p)_{n+1} = \Bsig_{n+1} : \hat{\BI} = \frac{\Tr(\Bsig_{n+1})}{\sqrt{3}}
  ~,~~
  (\sigma_s)_{n+1} = \Bsig_{n+1} : \hat{\BsT}^\Trial \,.
\Eeq
As before,
\Beq
  \Bsig^\Trial = \Bsig_n + \Delta t (\SfC^e_n : \dot{\BVeps}_{n+1}) 
     = \Bsig_n + 
       \Delta t \left[\left(\kappa_n - \tfrac{2}{3}\mu_n\right) \Tr(\dot{\BVeps}_{n+1})\BI +
                      2\mu_n\dot{\BVeps}_{n+1}\right] 
\Eeq
The trial stress is then decomposed into
\Beq
  \Bsig^\Trial = \sigma_p^\Trial \hat{\BI} + \sigma_s^\Trial \hat{\BsT}^\Trial
  \quad \text{where} \quad
  \sigma_p^\Trial = \frac{\Tr(\Bsig^\Trial)}{\sqrt{3}}
  ~,~~
  \sigma_s^\Trial = \Bsig^\Trial : \hat{\BsT}^\Trial = \Norm{\BsT^\Trial} \,.
\Eeq
The yield function is computed using
\Beq
  f_y = f[\Bsig_\beta^\Trial, (\Veps^\Teq_p)_n, 
          (\dot{\Veps}^\Teq_p)_n, \phi_n, D_n, T_n, \Edot{\Teq}_n, \kappa_n, \mu_n, \dots]
  ~,~~ \Bsig_\beta^\Trial = \Bsig^\Trial - \Bbeta_n \,.
\Eeq
If $f_y \le 0$, the state is updated using
\Beq
  \Bal
  &\Bsig_{n+1} = \Bsig^\Trial~,~~\Bbeta_{n+1} = \Bbeta_n ~,~~(\Veps^\Teq_p)_{n+1} = (\Veps^\Teq_p)_n
  ~,~~ (\dot{\Veps}^\Teq_p)_{n+1} = (\dot{\Veps}^\Teq_p)_n \\
  &\phi_{n+1} = \phi_n
  ~,~~ D_{n+1} = D_n ~,~~ T_{n+1} = T_n \\
  &\kappa_{n+1} = \kappa(p_{n+1}, T_n) ~,~~
  \mu_{n+1} = \mu(p_{n+1}, T_n) \,.
  \Eal
\Eeq
If $f_y > 0$, integration of the stress rate by backward Euler leads to
\Beq
  \Bsig_{n+1} = \Bsig^\Trial - \Delta\lambda_{n+1} \BP_{n+1} \,.
\Eeq
Expressed in terms of the trial basis using \eqref{eq:P_iso}, and noting that
$\Bsig_{n+1} = \Bsig^e_{n+1}$ and $(\sigma_{ss})_{n+1} = \hat{\BsT}_{n+1}:\hat{\BsT}^\Trial$,
\Beq \label{eq:P_iso_upd}
  \Bal
  \BP_{n+1} & = \left[\frac{\sqrt{3}\kappa_{n+1}}{\Norm{\BN_{n+1}}{}} \Partial{f_{n+1}}{p_\beta} 
          - \frac{1}{\kappa_{n+1}}
            \sum_\eta \Partial{\kappa_{n+1}}{\eta} 
            (\sigma_p)_{n+1} \right] \hat{\BI} \\
        & + \left[\frac{\sqrt{6}\mu_{n+1}}{\Norm{\BN_{n+1}}{}}\Partial{f_{n+1}}{\sigma_\Teff^\xi}
          - \frac{1}{\mu_{n+1}}
            \sum_\eta \Partial{\mu_{n+1}}{\eta} (\sigma_s)_{n+1}\right] 
            (\sigma_{ss})_{n+1} \hat{\BsT}^\Trial
  \Eal
\Eeq
Therefore,
\Beq
  \Bal
  (\sigma_p)_{n+1} &= \sigma_p^\Trial  
    - \Delta\lambda_{n+1}\left[
        \frac{\sqrt{3}\kappa_{n+1}}{\Norm{\BN_{n+1}}{}} \Partial{f_{n+1}}{p_\beta} 
      - \frac{1}{\kappa_{n+1}}
        \sum_\eta\Partial{\kappa_{n+1}}{\eta}(\sigma_p)_{n+1}\right] \\
  (\sigma_s)_{n+1} &= \sigma_s^\Trial  
    - \Delta\lambda_{n+1}\left[
        \frac{\sqrt{6} \mu_{n+1}}{\Norm{\BN_{n+1}}{}}\Partial{f_{n+1}}{\sigma_\Teff^\xi}
      - \frac{1}{\mu_{n+1}} 
        \sum_\eta\Partial{\mu_{n+1}}{\eta} (\sigma_s)_{n+1}\right](\sigma_{ss})_{n+1}
  \Eal
\Eeq
The plastic strain can be updated using \eqref{eq:plastic_strain_rate}:
\Beq \label{eq:plastic_strain_update}
  \BVeps^p_{n+1} = \BVeps^p_{n} + \Delta t \dot{\BVeps}_{n+1} - 
    \left(\frac{(\sigma_p)_{n+1}-(\sigma_p)_n}{3\kappa_{n+1}}\right)\hat{\BI} - 
    \left(\frac{(\sigma_s)_{n+1}-(\sigma_s)_n}{2\mu_{n+1}}\right) \hat{\BsT}^\Trial \,.
\Eeq
The internal variables can be updated using
\Beq
  \Beta_{n+1} = \Beta_n + \Delta\lambda_{n+1}~\Bh^{\eta}_{n+1} 
\Eeq
Also, as before, the stress state has to lie on the yield surface:
\Beq
  f(\Bsig^\Trial - \Delta\lambda_{n+1} \BP_{n+1}) = 0 
\Eeq
and the consistency condition needs to be satisfied:
\Beq
  \hat{\BN}_{n+1} : (\Bsig^\Trial - \Bsig_n) = \Delta \lambda_{n+1} (\hat{\BN}_{n+1}:\BP_{n+1} - \hat{H}_{n+1})
\Eeq
where
\Beq \label{eq:N_iso_upd}
  \hat{\BN}_{n+1} = \frac{\BN_{n+1}}{\Norm{\BN_{n+1}}{}} ~,~~
  \BN_{n+1} = \tfrac{1}{\sqrt{3}}\Partial{f_{n+1}}{p_\beta} \hat{\BI}+ 
              \sqrt{\tfrac{3}{2}}\Partial{f_{n+1}}{\sigma_\Teff^\xi}\hat{\BsT}^\Trial \,.
\Eeq
We can now attempt to express the iterative semi-implicit stress update algorithm 
given in \eqref{eq:Gamma_iter} and \eqref{eq:Stress_iter} in terms of the trial basis.
Recall that
\Beq 
  \Gamma_{k+1} = \Gamma_k + \frac{f(\BSv_k,\BQv_k)}
                                 {\BNv_k \cdot \BPv_k - H_k} ~,~~
  \BEv^p_{k+1} = \BEv^p_k + \hat{\BNv}_k \Delta\Gamma ~,~~
  \BSv_{k+1} = \BSv_k -\BPv_k \Delta\Gamma ~,~~
  \BQv_{k+1} = \BQv_k + \BHv_k \Delta\Gamma 
\Eeq
Reverting back to tensor notation, 
\Beq 
  \Gamma_{k+1} = \Gamma_k + \frac{f(\Bsig_k,\Beta_k)}
                                 {\Norm{\BN_k}{}(\hat{\BN}_k : \BP_k - \hat{H}_k)} ~,~~
  \BVeps^p_{k+1} = \BVeps^p_k + \hat{\BN}_k \Delta\Gamma ~,~~
  \Bsig_{k+1} = \Bsig_k -\BP_k \Delta\Gamma ~,~~
  \Beta_{k+1} = \Beta_k + \BH_k \Delta\Gamma 
\Eeq
Using \eqref{eq:N_iso_upd} and \eqref{eq:P_iso_upd}, we have
\Beq
  \Bal
  \Norm{\BN_k}{}(\hat{\BN}_{k}:\BP_{k}) &= 
    \tfrac{1}{\sqrt{3}}\Partial{f_k}{p_\beta} 
     \left[\sqrt{3}\kappa_k \Partial{f_k}{p_\beta} 
          - \frac{\Norm{\BN_k}{}}{\kappa_k}
            \sum_\eta \Partial{\kappa_k}{\eta} 
            (\sigma_p)_k \right]  + \\
    & \quad \sqrt{\tfrac{3}{2}}\Partial{f_k}{\sigma_\Teff^\xi}
     \left[\sqrt{6}\mu_k\Partial{f_k}{\sigma_\Teff^\xi}
          - \frac{\Norm{\BN_k}{}}{\mu_k}
            \sum_\eta \Partial{\mu_k}{\eta} (\sigma_s)_k\right] 
            (\sigma_{ss})_k
  \Eal
\Eeq
Also,
\Beq
  \Norm{\BN_k}{} \hat{H}_k = \Partial{f_k}{\Bbeta}:(\Bh^\beta)_k + 
        \Partial{f_k}{\Veps_p^\Teq} (h^{\Veps_p})_k + 
        \Partial{f_k}{\phi} (h^{\phi})_k + 
        \Partial{f_k}{D} (h^D)_k + 
        \Partial{f_k}{T_p} (h^T)_k \,.
\Eeq
At the end of the iterative process, the plastic strain tensor may also be updated
using \eqref{eq:plastic_strain_update} and a non-hardening return used to force
the computed stress state on the final yield surface.

\subsection{Algorithm 1}
Two implementations of the model described in this chapter have been implemented in
\Vaango.  The algorithm described in this section is used when the \Textbfc{elastic\_plastic\_hp}
algorithm is invoked in an input file.  This plastic return algorithm is more robust but does not
include kinematic hardening or softening due to damage and temperature changes.  Softening and
damage are treated in an uncoupled manner after the stress has been updated. Also,
elastic-plastic coupling is ignored.  For this algorithm, the timestep is divided into substeps and
the following algorithm is applied at each substep.  At the end of the last substep, the stress
state is projected back to the updated yield surface without any changes to the internal variables.
\begin{enumerate}
  \item Inputs:
  \Beq
    \Bal
    \text{Timestep size:} &\quad \Delta t \\
    \text{New strain rate:}   &\quad \dot{\BVeps}_{n+1}, \dot{\Veps}^\Teq_{n+1} \\
    \text{Old stress:}    &\quad \Bsig_n \\
    \text{Old moduli:}    &\quad \kappa_n, \mu_n \\
    \text{Old plastic strain:} & \quad \BVeps^p_n \\
    \text{Old equivalent plastic rate:} & \quad (\dot{\Veps}^\Teq_p)_n \\
    \text{Old internal variables:} & \quad (\Veps^\Teq_p)_n, \phi_n, T_n \\
    \text{Trial stress:}  &\quad  
      \Bsig^\Trial = \Bsig_n + 
           \Delta t \left[\left(\kappa_n - \tfrac{2}{3}\mu_n\right) \Tr(\dot{\BVeps}_{n+1})\BI +
           2\mu_n\dot{\BVeps}_{n+1}\right] 
    \Eal
  \Eeq
  \item Decompose trial stress
  \Beq
    \Bal
      p^\Trial & = \tfrac{1}{3} \Tr(\Bsig^\Trial) ~,~~
      \BsT^\Trial = \Bsig^\Trial - \tfrac{1}{3} \Tr(\Bsig^\Trial) \BI~,~~
      \hat{\BsT}^\Trial = \frac{\BsT^\Trial}{\Norm{\BsT^\Trial}{}}\\
      \Bsig^\Trial & = \sigma_p^\Trial \hat{\BI} + \sigma_s^\Trial \hat{\BsT}^\Trial ~,~~
      \sigma_p^\Trial = \sqrt{3} p^\Trial ~,~~
      \sigma_s^\Trial = \Norm{\BsT^\Trial}{} 
    \Eal
  \Eeq
  \item Decompose start-of-timestep stress
  \Beq
    \Bal
    \Bsig_n &= p_n\BI + \BsT_n = (\sigma_p)_n \hat{\BI} + (\sigma_s)_n \hat{\BsT}^\Trial ~,~~
    (\sigma_\Teff)_n = \sqrt{\tfrac{3}{2} \BsT_n:\BsT_n} \\
    (\sigma_p)_n &= \Bsig_n : \hat{\BI} ~,~~
    (\sigma_s)_n = \Bsig_n : \hat{\BsT}^\Trial = \BsT_n : \hat{\BsT}^\Trial
    \Eal
  \Eeq
  \item Compute $f_n$ and derivatives
  \Beq
    f_n = f\left(\BsT_n, p_n, (\Veps^\Teq_p)_n, (\dot{\Veps}^\Teq_p)_n, \phi_n, T_n, 
            \kappa_n, \mu_n, \dot{\Veps}^\Teq_{n+1}, \dots\right) ~,~~
    \Partial{f_n}{p}~,~~\Partial{f_n}{\sigma_\Teff}
  \Eeq
  \item Compute components of $\BN_n$ and $\Norm{\BN_n}{}$
  \Beq
    \Bal
    \BN_n &= (N_p)_n \hat{\BI} + (N_s)_n \hat{\BsT}^\Trial ~,~~
    \Norm{\BN_n}{} = \sqrt{(N_p)_n^2 + (N_s)_n^2} \\
    (N_p)_n &= \tfrac{1}{\sqrt{3}} \Partial{f_n}{p} ~,~~
    (N_s)_n = \sqrt{\tfrac{3}{2}} \Partial{f_n}{\sigma_\Teff}
    \Eal
  \Eeq
  \item Compute components of $\BP_n$
  \Beq
    \Bal
    \BP_n &= (P_p)_n \hat{\BI} + (P_s)_n \hat{\BsT}^\Trial \\
    (P_p)_n &= \frac{\sqrt{3}\kappa_n}{\Norm{\BN_n}{}} \Partial{f_n}{p} ~,~~
    (P_s)_n  = \frac{\sqrt{6}\mu_n}{\Norm{\BN_n}{}} \Partial{f_n}{\sigma_\Teff}
    \Eal
  \Eeq
  \item Initialize:
  \Beq
    \Bal
    k &= 0 ~,~~ \Gamma = 0\\
    \Bsig_k &= \Bsig^\Trial
    \Eal
  \Eeq
  \item \label{step:start_loop} Decompose current stress
  \Beq
    \Bal
    \Bsig_k &= p_k\BI + \BsT_k = (\sigma_p)_k \hat{\BI} + (\sigma_s)_k \hat{\BsT}^\Trial ~,~~
    (\sigma_\Teff)_k = \sqrt{\tfrac{3}{2} \BsT_k:\BsT_k} \\
    (\sigma_p)_k &= \Bsig_k : \hat{\BI} ~,~~
    (\sigma_s)_k = \Bsig_k : \hat{\BsT}^\Trial = \BsT_k : \hat{\BsT}^\Trial ~,~~ 
    (\sigma_{ss})_k = \frac{\BsT_k}{\Norm{\BsT_k}{}} : \hat{\BsT}^\Trial = 1
    \Eal
  \Eeq
  \item Compute $f_k$ and derivatives
  \Beq
    f_k = f\left(\BsT_k, p_k, (\Veps^\Teq_p)_k, (\dot{\Veps}^\Teq_p)_k, \phi_k, T_n, 
            \kappa_n, \mu_n, \dot{\Veps}^\Teq_{n+1}, \dots\right) ~,~~
    \Partial{f_k}{p}~,~~\Partial{f_k}{\sigma_\Teff}
  \Eeq
  \item Compute components of $\BN_k$ and $\Norm{\BN_k}{}$
  \Beq
    \Bal
    \BN_k &= (N_p)_k \hat{\BI} + (N_s)_k \hat{\BsT}^\Trial ~,~~
    \Norm{\BN_k}{} = \sqrt{(N_p)_k^2 + (N_s)_k^2} \\
    (N_p)_k &= \tfrac{1}{\sqrt{3}} \Partial{f_k}{p} ~,~~
    (N_s)_k = \sqrt{\tfrac{3}{2}} \Partial{f_k}{\sigma_\Teff} ~.~~
    \Eal
  \Eeq
  \item Compute $\BN_{k}:\BP_{n}$
  \Beq
    \BN_{k}:\BP_{n} = (N_p)_k (P_p)_n + (N_s)_k (P_s)_n
  \Eeq
  \item Compute updated $\Delta\Gamma$
  \Beq 
    \Gamma_{k+1} = \Gamma_k + \frac{f_k}{\BN_k : \BP_n} ~,~~
    \Delta\Gamma = \Gamma_{k+1} - \Gamma_k
  \Eeq
  \item Compute updated stress components:
  \Beq
    (\sigma_p)_{k+1} = (\sigma_p)_{k} - (P_p)_n \Delta\Gamma ~,~~
    (\sigma_s)_{k+1} = (\sigma_s)_{k} - (P_s)_n \Delta\Gamma 
  \Eeq
  \item Compute $f_{k+1}$
  \Beq
    f_{k+1} = f\left(\BsT_{k+1}, p_{k+1}, (\Veps^\Teq_p)_{k}, (\dot{\Veps}^\Teq_p)_{k+1}, 
                     \phi_{k+1}, D_{k+1}, T_{k+1}, \kappa_{k+1}, \mu_{k+1}, 
                     \dot{\Veps}^\Teq_{n+1}, \dots\right) 
  \Eeq
  \item If $|f_{k+1}| < f_\text{tolerance}$ and $|\Gamma_{k+1}-\Gamma_k| < \Gamma_\text{tolerance}$
        go to step \ref{step:update}.
  \item Set $k \leftarrow k+1$ and go to step \ref{step:start_loop}.
  \item Update the stress:
  \Beq
    \Bsig_{n+1} = (\sigma_p)_{k+1} \hat{\BI} + (\sigma_s)_{k+1} \hat{\BsT}^\Trial
  \Eeq
  \item \label{step:update} Compute internal variable hardening/softening moduli
  \Beq
    (h^{\Veps_p})_{k+1} ~,~~ (h^{\phi})_{k+1}
  \Eeq
  \item Compute updated internal variables
  \Beq
    (\Veps^\Teq_p)_{n+1} = (\Veps^\Teq_p)_n + (h^{\Veps_p})_{k+1} \Delta\Gamma ~,~~
    \phi_{n+1} = \phi_n + (h^{\phi})_{k+1} \Delta\Gamma ~,~~
  \Eeq
  \item Compute updated elastic strain
  \Beq
    \BVeps^e_{n+1} = \BVeps^e_{n} + 
    \left(\frac{(\sigma_p)_{k+1}-(\sigma_p)_n}{3\kappa_{n}}\right)\hat{\BI} - 
    \left(\frac{(\sigma_s)_{k+1}-(\sigma_s)_n}{2\mu_{n}}\right) \hat{\BsT}^\Trial \,.
  \Eeq
  \item Compute the updated plastic strain
  \Beq
    \BVeps^p_{n+1} = \BVeps^e_{n} + \BVeps^p_n + \Delta t \dot{\BVeps}_{n+1} - \BVeps^e_{n+1} ~,~~
    (\dot{\Veps}^\Teq_p)_{n+1} = \Norm{\BVeps^p_{n+1}}{}
  \Eeq
  \item Compute updated elastic moduli
  \Beq 
    \kappa_{n+1} = \kappa\left((\sigma_p)_{k+1}, \BVeps^e_{n+1}, (T_p)_{n}\right) ~,~~
    \mu_{n+1} = \mu\left((\sigma_p)_{k+1}, \BVeps^e_{n+1}, (T_p)_{n}\right) 
  \Eeq
\end{enumerate}

\subsection{Algorithm 2}
The following stress update algorithm is used for each (plastic) time step for models
that require kinematic hardening.
\begin{enumerate}
  \item Inputs:
  \Beq
    \Bal
    \text{Timestep size:} &\quad \Delta t \\
    \text{New strain rate:}   &\quad \dot{\BVeps}_{n+1}, \dot{\Veps}^\Teq_{n+1} \\
    \text{Old stress:}    &\quad \Bsig_n \\
    \text{Old moduli:}    &\quad \kappa_n, \mu_n \\
    \text{Old plastic strain:} & \quad \BVeps^p_n \\
    \text{Old equivalent plastic rate:} & \quad (\dot{\Veps}^\Teq_p)_n \\
    \text{Old internal variables:} & \quad \Bbeta_n, (\Veps^\Teq_p)_n, \phi_n, D_n, (T_p)_n \\
    \text{Trial stress:}  &\quad  
      \Bsig^\Trial = \Bsig_n + 
           \Delta t \left[\left(\kappa_n - \tfrac{2}{3}\mu_n\right) \Tr(\dot{\BVeps}_{n+1})\BI +
           2\mu_n\dot{\BVeps}_{n+1}\right] 
    \Eal
  \Eeq
  \item Decompose trial stress:
  \Beq
    \Bal
      p^\Trial & = \tfrac{1}{3} \Tr(\Bsig^\Trial) ~,~~
      \BsT^\Trial = \Bsig^\Trial - \tfrac{1}{3} \Tr(\Bsig^\Trial) \BI~,~~
      \hat{\BsT}^\Trial = \frac{\BsT^\Trial}{\Norm{\BsT^\Trial}{}}\\
      \Bsig^\Trial & = \sigma_p^\Trial \hat{\BI} + \sigma_s^\Trial \hat{\BsT}^\Trial ~,~~
      \sigma_p^\Trial = \sqrt{3} p^\Trial ~,~~
      \sigma_s^\Trial = \Norm{\BsT^\Trial}{} 
    \Eal
  \Eeq
  \item Initialize:
  \Beq
    \Bal
    k &= 0 ~,~~ \Gamma = 0\\
    \Bsig_k &= \Bsig^\Trial,~~ \kappa_k = \kappa_n,~~ \mu_k = \mu_n,~~ \BVeps^p_k = \BVeps^p_n\\ 
    (\Veps^\Teq_p)_k &= (\Veps^\Teq_p)_n,~~ \Bbeta_k = \Bbeta_n~.~~
    \phi_k = \phi_n,~~ D_k = D_n,~~ (T_p)_k = (T_p)_n\\
    (\Epdoteq)_k &= (\Epdoteq)_n,~~ \Edot{\Teq} = \Edot{\Teq}_{n+1}
    \Eal
  \Eeq
  \item \label{step:start} Compute shifted stress using backstress:
  \Beq
    (\Bsig_\beta)_k = \Bsig_k - \Bbeta_k ~,~~ (p_\beta)_k = \tfrac{1}{3}\Tr(\Bsig_\beta)_k ~,~~
    \Bxi_k = (\Bsig_\beta)_k - (p_\beta)_k \BI ~,~~
    (\sigma_\Teff^\xi)_k = \sqrt{\tfrac{3}{2} \Bxi_k:\Bxi_k}
  \Eeq
  \item Compute $f_k$ and derivatives
  \Beq
    f_k = f\left(\Bxi_k, (p_\beta)_k, (\Veps^\Teq_p)_k, (\dot{\Veps}^\Teq_p)_k, \phi_k, D_k, T_k, 
            \kappa_k, \mu_k, \dot{\Veps}^\Teq_{n+1}, \dots\right) ~,~~
    \Partial{f_k}{p_\beta}~,~~\Partial{f_k}{\sigma_\Teff^\xi}
  \Eeq
  \item Compute components of $\BN_k$ and $\Norm{\BN_k}{}$
  \Beq
    \Bal
    \BN_k &= (N_p)_k \hat{\BI} + (N_s)_k \hat{\BsT}^\Trial ~,~~
    \Norm{\BN_k}{} = \sqrt{(N_p)_k^2 + (N_s)_k^2} \\
    (N_p)_k &= \tfrac{1}{\sqrt{3}} \Partial{f_k}{p_\beta} ~,~~
    (N_s)_k = \sqrt{\tfrac{3}{2}} \Partial{f_k}{\sigma^\xi_\Teff} ~.~~
    \Eal
  \Eeq
  \item Compute derivatives of bulk and shear modulus with respect to internal variables
  \Beq
    \Partial{\kappa_k}{\Veps^\Teq_p}, \Partial{\kappa_k}{\phi}, \Partial{\kappa_k}{D_k} ,
      \Partial{\kappa_k}{T_k} ~,~~
    \Partial{\mu_k}{\Veps^\Teq_p}, \Partial{\mu_k}{\phi}, \Partial{\mu_k}{D_k} ,
      \Partial{\mu_k}{T_k}
  \Eeq
  \item Decompose current stress
  \Beq
    \Bal
    \Bsig_k &= p_k\BI + \BsT_k = (\sigma_p)_k \hat{\BI} + (\sigma_s)_k \hat{\BsT}^\Trial \\
    (\sigma_p)_k &= \Bsig_k : \hat{\BI} ~,~~
    (\sigma_s)_k = \Bsig_k : \hat{\BsT}^\Trial = \BsT_k : \hat{\BsT}^\Trial ~,~~ 
    (\sigma_{ss})_k = \frac{\BsT_k}{\Norm{\BsT_k}{}} : \hat{\BsT}^\Trial = 1
    \Eal
  \Eeq
  \item Compute components of $\BP_k$
  \Beq
    \Bal
    \BP_k &= (P_p)_k \hat{\BI} + (P_s)_k \hat{\BsT}^\Trial \\
    (P_p)_k & = 
       \left[\frac{\sqrt{3}\kappa_k}{\Norm{\BN_k}{}} \Partial{f_k}{p_\beta} 
            - \frac{1}{\kappa_k}
              \sum_\eta \Partial{\kappa_k}{\eta} 
              (\sigma_p)_k \right]  \\
    (P_s)_k &=   
       \left[\frac{\sqrt{6}\mu_k}{\Norm{\BN_k}{}} \Partial{f_k}{\sigma_\Teff^\xi}
            - \frac{1}{\mu_k}
              \sum_\eta \Partial{\mu_k}{\eta} (\sigma_s)_k\right] 
              (\sigma_{ss})_k
    \Eal
  \Eeq
  \item Compute $\Norm{\BN_k}{}(\hat{\BN}_{k}:\BP_{k})$
  \Beq
    \Norm{\BN_k}{}(\hat{\BN}_{k}:\BP_{k}) = (N_p)_k (P_p)_k + (N_s)_k (P_s)_k
  \Eeq
  \item Compute derivatives of $f_k$ with respect to internal variables
  \Beq
    \Partial{f_k}{\Bbeta} ~,~~ \Partial{f_k}{\Veps_p^\Teq}~,~~
    \Partial{f_k}{\phi}~,~~ \Partial{f_k}{D}~,~~ \Partial{f_k}{T_p}
  \Eeq
  \item Compute internal variable hardening/softening moduli
  \Beq
    (\Bh^\beta)_k ~,~~ (h^{\Veps_p})_k ~,~~ (h^{\phi})_k ~,~~ (h^D)_k ~,~~ (h^T)_k
  \Eeq
  \item Compute $\Norm{\BN_k}{} \hat{H}_k$
  \Beq
    \Norm{\BN_k}{} \hat{H}_k = \Partial{f_k}{\Bbeta}:(\Bh^\beta)_k + 
          \Partial{f_k}{\Veps_p^\Teq} (h^{\Veps_p})_k + 
          \Partial{f_k}{\phi} (h^{\phi})_k + 
          \Partial{f_k}{D} (h^D)_k + 
          \Partial{f_k}{T_p} (h^T)_k \,.
  \Eeq
  \item Compute updated $\Delta\Gamma$
  \Beq 
    \Gamma_{k+1} = \Gamma_k + \frac{f_k}
                                   {\Norm{\BN_k}{}(\hat{\BN}_k : \BP_k - \hat{H}_k)} ~,~~
    \Delta\Gamma = \Gamma_{k+1} - \Gamma_k
  \Eeq
  \item Compute updated stress components:
  \Beq
    (\sigma_p)_{k+1} = (\sigma_p)_{k} - (P_p)_k \Delta\Gamma ~,~~
    (\sigma_s)_{k+1} = (\sigma_s)_{k} - (P_s)_k \Delta\Gamma 
  \Eeq
  \item Compute updated internal variables:
  \Beq
    \Bal
    \Bbeta_{k+1} &= \Bbeta_k + \Bh^\beta \Delta\Gamma ~,~~
    (\Veps^\Teq_p)_{k+1} = (\Veps^\Teq_p)_k + h^{\Veps_p} \Delta\Gamma \\
    \phi_{k+1} &= \phi_k + h^{\phi} \Delta\Gamma ~,~~
    D_{k+1} = D_k + h^{D} \Delta\Gamma ~,~~
    (T_p)_{k+1} = (T_p)_k + h^{T} \Delta\Gamma \\
    \Eal
  \Eeq
  \item Compute updated elastic strain:
  \Beq
    \BVeps^e_{k+1} = \BVeps^e_{n} + 
    \left(\frac{(\sigma_p)_{k+1}-(\sigma_p)_n}{3\kappa_{k}}\right)\hat{\BI} - 
    \left(\frac{(\sigma_s)_{k+1}-(\sigma_s)_n}{2\mu_{k}}\right) \hat{\BsT}^\Trial \,.
  \Eeq
  \item Compute the updated plastic strain:
  \Beq
    \BVeps^p_{k+1} = \BVeps^e_{n} + \BVeps^p_n + \Delta t \dot{\BVeps}_{n+1} - \BVeps^e_{k+1} ~,~~
    (\dot{\Veps}^\Teq_p)_{k+1} = \Norm{\BVeps^p_{k+1}}{}
  \Eeq
  \item Compute updated elastic moduli
  \Beq 
    \kappa_{k+1} = \kappa\left((\sigma_p)_{k+1}, \BVeps^e_{k+1}, (T_p)_{k+1}\right) ~,~~
    \mu_{k+1} = \mu\left((\sigma_p)_{k+1}, \BVeps^e_{k+1}, (T_p)_{k+1}\right) 
  \Eeq
  \item Compute $f_{k+1}$
  \Beq
    f_{k+1} = f\left(\Bxi_{k+1}, (p_\beta)_{k+1}, (\Veps^\Teq_p)_{k+1}, (\dot{\Veps}^\Teq_p)_{k+1}, 
                     \phi_{k+1}, D_{k+1}, T_{k+1}, \kappa_{k+1}, \mu_{k+1}, 
                     \dot{\Veps}^\Teq_{n+1}, \dots\right) 
  \Eeq
  \item If $|f_{k+1}| < f_\text{tolerance}$ and $|\Gamma_{k+1}-\Gamma_k| < \Gamma_\text{tolerance}$
        go to step \ref{step:final}.
  \item Set $k \leftarrow k+1$ and go to step \ref{step:start}.
  \item \label{step:final}Update the state:
  \Beq
    \Bal
    \Bsig_{n+1} &= (\sigma_p)_{k+1} \hat{\BI} + (\sigma_s)_{k+1} \hat{\BsT}^\Trial ~,~~
    \Bbeta_{n+1} = \Bbeta_{k+1}~,~~
    (\Veps^\Teq_p)_{n+1} = (\Veps^\Teq_p)_{k+1} \\
    \phi_{n+1} &= \phi_{k+1} ~,~~
    D_{n+1} = D_{k+1} ~,~~
    (T_p)_{n+1} = (T_p)_{k+1} ~,~~
    (\dot{\Veps}^\Teq_p)_{n+1} = (\dot{\Veps}^\Teq_p)_{k+1} \\
    \kappa_{n+1} &= \kappa_k ~,~~ \mu_{n+1} = \mu_k \\
    \BVeps^e_{n+1} &= \BVeps^p_{n} + 
     \left(\frac{(\sigma_p)_{k+1}-(\sigma_p)_n}{3\kappa_{n+1}}\right)\hat{\BI} - 
     \left(\frac{(\sigma_s)_{k+1}-(\sigma_s)_n}{2\mu_{n+1}}\right) \hat{\BsT}^\Trial \\
    \BVeps^p_{n+1} &=  \BVeps^e_n + \BVeps^p_n + \Delta t \dot{\BVeps}_{n+1} - \BVeps^e_{n+1}
    \Eal
  \Eeq
\end{enumerate}

\section{Example 1: von Mises plasticity}
Consider the case of $J_2$ plasticity with the yield condition
\Beq
  f := \sqrt{\tfrac{3}{2}} \Norm{\BsT-\Dev(\Bbeta)}{} - \sigma_y(\Veps^\Teq_p, \dot{\Veps}^\Teq, \phi, T, \dots) = 
       \sqrt{\tfrac{3}{2}} \Norm{\Bxi}{} - \sigma_y(\Veps^\Teq_p, \dot{\Veps}^\Teq, \phi, T, \dots) \le 0 
\Eeq
where $\Norm{\Bxi}{} = \sqrt{\Bxi:\Bxi}$. 
The derivatives of the yield function with respect to the internal variables are
\Beq
  \Partial{f}{\Bbeta}  = -\sqrt{\tfrac{3}{2}} \frac{\Bxi}{\Norm{\Bxi}{}} ~,~~
  \Partial{f}{\Veps^\Teq_p}  = -\Partial{\sigma_y}{\Veps^\Teq_p} ~,~~
  \Partial{f}{\phi}  = -\Partial{\sigma_y}{\phi} ~,~~
  \Partial{f}{D}  = -\Partial{\sigma_y}{D} ~,~~
  \Partial{f}{T_p}  = -\Partial{\sigma_y}{T_p} 
\Eeq
Assume the associated flow rule
\Beq
  \BdT^p = \dot{\lambda}~\hat{\BN} =\dot{\lambda}\frac{\BN}{\Norm{\BN}{}} \quad \text{where} \quad
   \BN = \Partial{f}{\Bsig} = \Partial{f}{\Bxi} = \sqrt{\tfrac{3}{2}}~\frac{\Bxi}{\Norm{\Bxi}{}}~,~~
   \Norm{\BN}{} = \sqrt{\tfrac{3}{2}}
\Eeq
Then
\Beq
  \BdT^p = \dot{\lambda}~\frac{\Bxi}{\Norm{\Bxi}{}} ~;~~ \Dev(\BdT^p) = \BdT^p ~,~~
  \Tr(\BdT^p) = 0 ,~~\Norm{\BdT^p}{} = \Dot{\Veps}^\Teq_p = \dot\lambda ~.
\Eeq
The evolution of the equivalent plastic strain is given by
\Beq
  \dot{\Veps}^\Teq_p = \dot{\lambda}~h^{\Veps_p} 
  \quad \implies \quad h^{\Veps_p} = 1 \,.
\Eeq
The evolution of porosity is given by (there is no evolution of porosity)
\Beq
  \dot{\phi} = \dot{\lambda}~h^{\phi} = 0
  \quad \implies \quad h^{\phi} = 0 \,.
\Eeq
The evolution of the back stress is given by the Prager kinematic hardening rule
\Beq
  \dot{\Bbeta} = \dot{\lambda} \Bh^{\beta} = H'\BdT^p  
  \quad \implies \quad \Bh^\beta = H'\frac{\Bxi}{\Norm{\Bxi}{}}
\Eeq
where $H'$ is a hardening modulus.  For the Armstrong-Frederick kinematic hardening model, 
\Beq
  \dot{\Bbeta} = \dot{\lambda}~\Bh^{\beta} 
     = H_1~\BdT^p - H_2~\Bbeta~\Norm{\BdT^p}{} 
     = \dot{\lambda}\left[H_1~\frac{\Bxi}{\Norm{\Bxi}{}} - H_2~\Bbeta\right]
  \quad \implies \quad \Bh^\beta = H_1~\frac{\Bxi}{\Norm{\Bxi}{}} - H_2~\Bbeta
\Eeq

\section{Example 2: Gurson-type model}
Consider a Gurson-type yield condition with kinematic hardening.  In this
case the yield condition can be written as
\Beq
  f := \tfrac{3}{2} \frac{(\sigma_\Teff^\xi)^2}{\sigma_y^2} + 
     2~q_1~\phi^{\star}~\cosh\left(\tfrac{3}{2} \frac{q_2~p_\beta}{\sigma_y}\right)
     - [1 + q_3~(\phi^*)^2]
\Eeq
where $\sigma_y$ is the yield stress of the matrix material (zero-porosity), 
\Beq
 \Bal
   \sigma_\Teff^\xi &= \Bxi:\Bxi ~,~~ 
     \Bxi = \Dev(\Bsig-\Bbeta) ~,~~ p_\beta = \tfrac{1}{3} \Tr(\Bsig-\Bbeta) \\
   \phi^\star & = \begin{cases}
             \phi & \text{for}~ \phi \le \phi_c \\
             \phi_c - \frac{\phi_u^\star - \phi_c}{\phi_f - \phi_c}~(\phi - \phi_c) & 
              \text{for}~ \phi > \phi_c
           \end{cases}
  \Eal
\Eeq
and $\phi$ is the porosity. 
Final fracture occurs for $\phi = \phi_f$ or when $\phi_u^\star = 1/q_1$.  
In this case, the derivatives of $f$ are
\Beq
  \Partial{f}{\sigma_\Teff^\xi} = \frac{3\sigma_\Teff^\xi}{\sigma_y^2} ~,~~
  \Partial{f}{p_\beta} = \frac{3 q_1 q_2 \phi^\star}{\sigma_y} \sinh\left(\tfrac{3}{2} \frac{q_2~p_\beta}{\sigma_y}\right) ~,~~
  \Partial{f}{\sigma_y} = -\left[\frac{3(\sigma_\Teff^\xi)^2}{\sigma_y^3} +
      \frac{3 q_1 q_2 p_\beta \phi^\star}{\sigma_y^2} \sinh\left(\tfrac{3}{2} \frac{q_2~p_\beta}{\sigma_y}\right)\right]
\Eeq
and
\Beq
  \Partial{f}{\Bbeta} = -\left[2\Partial{f}{\sigma_\Teff^\xi} \Bxi + \Partial{f}{p_\beta} \BI\right]~,~~
  \Partial{f}{\Veps^\Teq_p}  = \Partial{f}{\sigma_y}\Partial{\sigma_y}{\Veps^\Teq_p} ~,~~
  \Partial{f}{D}  = \Partial{f}{\sigma_y} \Partial{\sigma_y}{D} ~,~~
  \Partial{f}{T_p}  = \Partial{f}{\sigma_y} \Partial{\sigma_y}{T_p} 
\Eeq
For the derivative with respect to $\phi$, 
\Beq
  \Partial{f}{\phi} = \Partial{f}{\phi^\star}\Partial{\phi^\star}{\phi} +
                      \Partial{f}{\sigma_y}\Partial{\sigma_y}{\phi} 
                    = \Partial{f}{\phi^\star}\Partial{\phi^\star}{\phi}
\Eeq
where
\Beq
  \Partial{f}{\phi^\star} = 2q_1\cosh\left(\tfrac{3}{2} \frac{q_2~p_\beta}{\sigma_y}\right)
                            -2q_3\phi^\star \quad \Tand \quad
  \Partial{\phi^\star}{\phi}  = \begin{cases}
             1 & \text{for}~ \phi \le \phi_c \\
             - \frac{\phi_u^\star - \phi_c}{\phi_f - \phi_c} & 
              \text{for}~ \phi > \phi_c
           \end{cases}
\Eeq
Using an associated flow rule, we have
\Beq
  \BdT^p = \dot{\lambda}~\hat{\BN} = \dot{\lambda} \frac{\BN}{\Norm{\BN}{}} ~,~~
  \BN = \Partial{f}{\Bsig} = 2 \Partial{f}{\sigma_\Teff^\xi}\Bxi + \Partial{f}{p_\beta} \BI ~,~~
  \Norm{\BN}{} = \sqrt{4\left(\Partial{f}{\sigma_\Teff^\xi}\Norm{\Bxi}{}\right)^2 + 
                       3\left(\Partial{f}{p_\beta}\right)^2}
\Eeq
For the evolution equation for the plastic strain we use
\Beq
  (\Bsig-\Bbeta):\BdT^p = (1 - \phi)\sigma_y\dot{\Veps}^\Teq_p
\Eeq
where $\dot{\Veps}^\Teq_p$ is the equivalent plastic strain rate in the matrix material.  Hence,
\Beq
  \dot{\Veps}^\Teq_p = \dot{\lambda}~h^{\Veps_p}
    = \dot{\lambda}~\frac{(p_\beta\BI + \Bxi):\hat{\BN}}{(1 - \phi)~\sigma_y} 
  \quad \implies \quad
  h^{\Veps_p} = 
      \frac{3p_\beta\Partial{f}{p_\beta} + 2\Partial{f}{\sigma_\Teff^\xi}\Bxi:\Bxi}{(1 - \phi)~\sigma_y\Norm{\BN}{}} 
\Eeq
The evolution equation for the porosity is assumed to be given by
\Beq
  \dot{\phi} = (1 - \phi)~\Tr(\BdT^p) + A~\dot{\Veps}^\Teq_p
\Eeq
where
\Beq
A = \cfrac{f_n}{\Veps_s \sqrt{2\pi}} \exp\left[-\tfrac{1}{2} \frac{(\Veps^\Teq_p - \Veps_m)^2}{\Veps_s^2}\right]
\Eeq
and $ f_n $ is the volume fraction of void nucleating particles, 
$ \Veps_m $ is the mean of the normal distribution of nucleation strains, and 
$ \Veps_s $ is the standard deviation of the distribution.
Therefore,
\Beq
  \dot{\phi} = \dot{\lambda}~h^{\phi} =
    \dot{\lambda}~\left[(1 - \phi)~\Tr(\hat{\BN}) + A~
    \frac{(p_\beta\BI+\Bxi):\hat{\BN}}{(1 - \phi)~\sigma_y}\right] 
  \implies 
  h^\phi = \frac{1}{\Norm{\BN}{}}
    \left[3(1-\phi)\Partial{f}{p_\beta} + A\frac{3p_\beta\Partial{f}{p_\beta} + 2\Partial{f}{\sigma_\Teff^\xi}\Bxi:\Bxi}{(1 - \phi)~\sigma_y}\right]
\Eeq
If the evolution of the backstress is given by the Prager kinematic hardening rule
\Beq
  \dot{\Bbeta} = \dot{\lambda}~\Bh^{\beta} = H'~\BdT^p 
  \quad \implies \Bh^\beta = \frac{H'}{\Norm{\BN}{}} 
    \left[2 \Partial{f}{\sigma_\Teff^\xi}\Bxi + \Partial{f}{p_\beta} \BI\right]
\Eeq
For the Armstrong-Frederick model, 
\Beq
  \dot{\Bbeta} = \dot{\lambda}~\Bh^{\beta} = H_1~\BdT^p - H_2~\Bbeta~\Norm{\BdT^p}{} 
  \quad \implies \Bh^\beta = \frac{H_1}{\Norm{\BN}{}} 
    \left(2 \Partial{f}{\sigma_\Teff^\xi}\Bxi + \Partial{f}{p_\beta} \BI\right) -
    H_2\Bbeta \,.
\Eeq

\section{Example 3: Nonlinear elasticity and isotropic hardening}
Let the flow stress be given by the Johnson-Cook model:
\Beq
  \sigma_y(\Ep,\Edot{\Teq},T) = 
  \left[A + B (\Ep)^n\right]\left[1 + C \ln(\Edot{\star})\right]
  \left[1 - (T^\star)^m\right]
\Eeq
The volumetric part of the stress in the intact metal is given by a Mie-Gr{\"u}neisen equation of state: 
\Beq
  p(J^e,T) = -\left[\frac{\rho_0~C_0^2~(1 - J^e)[1 - \Gamma_0 (1 - J^e)/2]}
                     {[1 - S_{\alpha}(1 - J^e)]^2} + \Gamma_0~E\right] ~,~~
  J^e = \det{\BF^e} ~,~~ E \approx \frac{\rho C_v (T-T_0)}{V}
\Eeq
The tangent bulk modulus of the intact metal is defined as
\Beq
  \kappa_m(p,J^e,T) = V^e\Partial{p}{V^e} = V^e\Partial{p}{J^e}\Partial{J^e}{V^e} = \frac{V^e}{V_0}\Partial{p}{J^e}
         = J^e\Partial{p}{J^e}
         \approx \cfrac{\rho_0 J^e C_0^2~[1 + (S_{\alpha}-\Gamma_0)(1 - J^e)]}
                 {[1 - S_{\alpha} (1-J^e)]^3}\,.
\Eeq
Since the rate of deformation is unrotated in the \Vaango metal plasticity implementation,
we can identify $\Tr(\BVeps^e)$ with $\ln(J^e)$~\cite{Neff2016} and use that quantity in the 
calculation.  
The deviatoric part of the stress in the intact metal is given by the 
Steinberg-Cochran-Guinan (SCG) shear modulus model:
\Beq
  \mu_m(p,J^e,T) = \mu_0 + p \Partial{\mu_m}{p} (J^e)^{1/3} + \Partial{\mu_m}{T}(T - T_0)
\Eeq
When we include porosity-dependence for the bulk and shear moduli, we have
\Beq
  \Bal
  \kappa(p, J^e, \phi, T) &= \frac{(1-\phi)\kappa_m }{1 - (1-K)\phi} ~,~~ K = \frac{3\kappa_m+4\mu_m}{4\mu_m} \\
  \mu(p, J^e, \phi, T) &= \frac{(1-\phi)\mu_m }{1 - (1-G)\phi} ~,~~ G = \frac{5(3\kappa_m+4\mu_m)}{9\kappa_m+8\mu_m}
  \Eal
\Eeq
The derivatives of $\sigma_y$, $\kappa$, and $\mu$ required for the algorithm can be calculated from these expressions.
