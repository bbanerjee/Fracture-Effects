\chapter{Isotropic metal plasticity}
\begin{NoteBox}
The deformation gradient ($\BF$) can be decomposed into a rotation tensor ($\BR$)
and a stretch tensor ($\BU$) with th epolar decomposition $\BF = \BR\cdot\BU$.
In the isotropic metal plasticity model implemented in \Vaango, $\BR$ is used 
to rotate the stress ($\Bsig$) and the rate of deformation ($\BdT$) 
into the unrotated configuration before the updated stress is computed:
\Beq
  \widehat{\Bsig} = \BR^T\cdot\Bsig\cdot\BR ~;~~
  \widehat{\BdT} = \BR^T\cdot\BdT\cdot\BR 
\Eeq
After the stress has been updated, it is rotated back using
\Beq
  \Bsig = \BR\cdot\widehat{\Bsig}\cdot\BR^T \,.
\Eeq
In the following discussion, all equations should be treated as referring
to the hatted quantities even though we drop the hats for convenience.
\end{NoteBox}

\section{The model}
The Cauchy stress ($\Bsig$) is decomposed into volumetric and deviatoric parts:
\Beq \label{eq:stress_decomp}
  \Bsig = p~\BI + \BsT \quad \text{where} \quad  
  p = \Third~\Tr(\Bsig) \quad \Tand \quad
  \BsT = \Dev(\Bsig) = \Bsig - \Third\Tr(\Bsig) \,.
\Eeq
In the above $p = \sigma_m$ is the mean stress and $\BsT$ is the deviatoric stress.
The isotropy of the material allows us to compute the mean stress using an
equation of state if desired. The deviatoric stress is computed using a rate-form
stress-strain relation.
We simplify the discussion by assuming that rate-form relations are used for
both the mean stress and the deviatoric stress.  The time derivative of 
\eqref{eq:stress_decomp} is
\Beq
  \dot{\Bsig} = \dot{p}~\BI + \dot{\BsT} \,.
\Eeq
These stress rates have to be computed when a rate of deformation $\BdT$ is given, which
we assume can be additively decomposed into elastic ($\BdT^e$) and plastic ($\BdT^p$) parts:
\Beq
  \BdT = \BdT^e + \BdT^p \,.
\Eeq
The isotropic metal plasticity model in \Vaango solves the coupled equations
\Beq
  \dot{p} = g_1(\BdT^e, \BdT^p) ~,~~ \dot{\Bs} = g_2(\BdT^e, \BdT^p) \,.
\Eeq

\subsection{Elastic relation}
The elastic constitutive relation is assumed to be of the form
\Beq
  \dot{\Bsig}^e = \left(\kappa - \tfrac{2}{3}\mu\right) \Tr(\BdT^e) \BI + 2\mu\BdT^e 
                - 3\kappa\alpha\Deriv{}{t}(T-T_0)\BI 
\Eeq
where $\mu(p, T, \phi)$ is the shear modulus, $\kappa(p,T, \phi)$ is the tangent 
bulk modulus, $\alpha$ is the coefficient of thermal expansion, $T_0$ is a reference 
temperature, $T$ is the current temperature, and $\phi$ is the current porosity.  

\subsection{Flow rule}
We assume that the plastic rate of deformation is given by the flow rule
\Beq
  \BdT^p = \dot{\lambda}~\BM
\Eeq

\subsection{Isotropic and kinematic hardening/softening rules}
The equivalent plastic strain ($\Veps_p^\Teq$) evolves according to the relation
\Beq
  \dot{\Veps}^\Teq_p = \dot{\lambda}~h_{\Veps_p} \,.
\Eeq
The back stress ($\Bbeta$) evolves according to the relation
\Beq
  \dot{\Bbeta} = \dot{\lambda}~\Bh_{\beta} \,.
\Eeq
The porosity ($\phi$) is assumed to evolve according to the relation
\Beq
  \dot{\phi} = \dot{\lambda}~h_{\phi} ~.
\Eeq
The damage parameter ($D$) evolves as
\Beq
  \dot{D} = \dot{\lambda}~h_D \,.
\Eeq
The temperature ($T$) due to plastic dissipation evolves as
\Beq
  \dot{T} = \dot{\lambda}~h_T \,.
\Eeq

\subsection{Yield condition}
The yield condition is of the form
\Beq
  f(\Bsig, \Bbeta, \Veps^\Teq_p, \phi, \dot{\Veps}^\Teq_p, D, T, \dots) =
  f(\Bxi, p_\beta, \Veps^\Teq_p, \phi, \dot{\Veps}^\Teq_p, D, T, \dots) = 0
\Eeq
where $\Bxi = \Dev(\Bsig - \Bbeta)$ and $p_\beta = \Tr(\Bsig - \Bbeta)/3$. 

The Kuhn-Tucker loading-unloading conditions are
\Beq
  \dot{\lambda} \ge 0 ~;~~  f \le 0 ~;~~ \dot{\lambda}~f = 0
\Eeq
and the consistency condition is $\dot{f} = 0$.

\subsection{Continuum elastic-plastic tangent modulus}
To determine whether the material has undergone a loss of stability we need to compute
the acosutic tensor which needs the computation of the continuum elastic-plastic tangent
modulus.

To do that recall that
\Beq
  \Bsig = p~\BI + \Bs \quad \implies \quad \dot{\Bsig} = \dot{p}~\BI ~.
\Eeq
We assume that 
\Beq
  \dot{p} = J~\Partial{p}{J}~\Tr(\Bd) \qquad \Tand \quad
  \dot{\Bs} = 2~\mu~\Beta^e ~.
\Eeq
Now, the consistency condition requires that
\Beq
  \dot{f}(\Bs, \Bbeta, \Ve^p, \phi, \dot{\Ve}, T, \dots) = 0 ~.
\Eeq
Keeping $\dot{\Ve}$ and $T$ fixed over the time interval, we can use the chain rule
to get
\Beq
  \dot{f} = \Partial{f}{\Bs}:\dot{\Bs} + \Partial{f}{\Bbeta}:\dot{\Bbeta} + 
    \Partial{f}{\Ve^p}~\dot{\Ve^p} + \Partial{f}{\phi}~\dot{\phi} = 0~.
\Eeq
The needed rate equations are
\Beq
  \Bal
    \dot{\Bs} & = 2~\mu~\Beta^e = 2~\mu~(\Beta - \Beta^p) = 
      2~\mu~[\Beta - \dot{\gamma}~\Dev(\Br)] \\
    \dot{\Bbeta} & = \dot{\gamma}~\Dev(\Bh^{\beta}) \\
    \dot{\Ve^p} & = \dot{\gamma}~h^{\alpha} \\
    \dot{\phi} & = \dot{\gamma}~h^{\phi} 
  \Eal
\Eeq
Plugging these into the expression for $\dot{f}$ gives
\Beq
  2~\mu~\Partial{f}{\Bs}:[\Beta - \dot{\gamma}~\Dev(\Br)] 
    + \dot{\gamma}~\Partial{f}{\Bbeta}:\Dev(\Bh^{\beta}) 
    + \dot{\gamma}~\Partial{f}{\Ve^p}~h^{\alpha} 
    + \dot{\gamma}~\Partial{f}{\phi}~h^{\phi}  = 0
\Eeq
or,
\Beq
  \dot{\gamma} = \cfrac{2~\mu~\Partial{f}{\Bs}:\Beta}
    {2~\mu~\Partial{f}{\Bs}:\Dev(\Br) - \Partial{f}{\Bbeta}:\Dev(\Bh^{\beta})
     - \Partial{f}{\Ve^p}~h^{\alpha} - \Partial{f}{\phi}~h^{\phi}} ~.
\Eeq
Plugging this expression for $\dot{\gamma}$ into the equation for $\dot{\Bs}$,
we get
\Beq
  \dot{\Bs} = 2~\mu~\left[\Beta - \left( 
    \cfrac{2~\mu~\Partial{f}{\Bs}:\Beta}
    {2~\mu~\Partial{f}{\Bs}:\Dev(\Br) - \Partial{f}{\Bbeta}:\Dev(\Bh^{\beta})
     - \Partial{f}{\Ve^p}~h^{\alpha} - \Partial{f}{\phi}~h^{\phi}}\right)~\Dev(\Br)\right]~.
\Eeq
At this stage, note that a symmetric $\Bsig$ implies a symmetric $\Bs$ and hence a 
symmetric $\Beta$.  Also we assume that $\Br$ is symmetric (and hence $\Dev(\Br)$), which
is true if the flow rule is associated.  Then we can write
\Beq
  \Beta = \SfI^{4s}:\Beta \qquad \Tand \qquad \Dev(\Br) = \SfI^{4s}:\Dev(\Br)
\Eeq
where $\SfI^{4s}$ is the fourth-order symmetric identity tensor.  Also note that
if $\BA$, $\BC$, $\BD$ are second order tensors and $\SfB$ is a fourth order tensor,
then
\Beq
  (\BA:\SfB:\BC)~(\SfB:\BD) \equiv A_{ij}~B_{ijkl}~C_{kl}~B_{mnpq}~D_{pq}
     = (B_{mnpq}~D_{pq})~(A_{ij}~B_{ijkl})~C_{kl} \equiv [(\SfB:\BD)\otimes(\BA:\SfB)]:\BC ~.
\Eeq
Therefore we have
\Beq
  \dot{\Bs} = 2~\mu~\left[\SfI^{4s}:\Beta - \left( 
    \cfrac{2~\mu~[\SfI^{4s}:\Dev(\Br)]\otimes[\Partial{f}{\Bs}:\SfI^{4s}]}
    {2~\mu~\Partial{f}{\Bs}:\Dev(\Br) - \Partial{f}{\Bbeta}:\Dev(\Bh^{\beta})
     - \Partial{f}{\Ve^p}~h^{\alpha} - \Partial{f}{\phi}~h^{\phi}}\right):\Beta\right]~.
\Eeq
Also,
\Beq
  \SfI^{4s}:\Dev(\Br) = \Dev(\Br) \qquad \Tand \qquad
  \Partial{f}{\Bs}:\SfI^{4s} = \Partial{f}{\Bs} ~.
\Eeq
Hence we can write
\Beq
  \dot{\Bs} = 2~\mu~\left[\SfI^{4s} - \left( 
    \cfrac{2~\mu~\Dev(\Br)\otimes\Partial{f}{\Bs}}
    {2~\mu~\Partial{f}{\Bs}:\Dev(\Br) - \Partial{f}{\Bbeta}:\Dev(\Bh^{\beta})
     - \Partial{f}{\Ve^p}~h^{\alpha} - \Partial{f}{\phi}~h^{\phi}}\right)\right]:\Beta
\Eeq
or,
\Beq
  \dot{\Bs} = \SfB^{ep}:\Beta = \SfB^{ep}:\left[\Bd - \Third~\Tr(\Bd)~\BI\right]
\Eeq
where
\Beq
  \SfB^{ep} := 
    2~\mu~\left[\SfI^{4s} - \left( 
    \cfrac{2~\mu~\Dev(\Br)\otimes\Partial{f}{\Bs}}
    {2~\mu~\Partial{f}{\Bs}:\Dev(\Br) - \Partial{f}{\Bbeta}:\Dev(\Bh^{\beta})
     - \Partial{f}{\Ve^p}~h^{\alpha} - \Partial{f}{\phi}~h^{\phi}}\right)\right] ~.
\Eeq
Adding in the volumetric component gives
\Beq
  \Bal
  \dot{\Bsig} & = \dot{p}~\BI + \dot{\Bs} \\
   & = J~\Partial{p}{J}~\Tr(\Bd)~\BI +  
       \SfB^{ep}:\left[\Bd - \Third~\Tr(\Bd)~\BI\right] \\
   & = \left[3~J~\Partial{p}{J}~\BI - \SfB^{ep}:\BI\right]~\cfrac{\Bd:\BI}{3}
       + \SfB^{ep}:\Bd \\
   & = J~\Partial{p}{J}~(\BI\otimes\BI):\Bd - \Third~[\SfB^{ep}:(\BI\otimes\BI)]:\Bd
       + \SfB^{ep}:\Bd ~.
  \Eal
\Eeq
Therefore,
\Beq
  \dot{\Bsig} = \left[J~\Partial{p}{J}~(\BI\otimes\BI) - 
      \Third~[\SfB^{ep}:(\BI\otimes\BI)] + \SfB^{ep}\right]:\Bd 
   = \SfC^{ep}:\Bd ~.
\Eeq
The quantity $\SfC^{ep}$ is the continuum elastic-plastic tangent modulus.  We also use the 
continuum elastic-plastic tangent modulus in the implicit version of the code.  However,
for improved accuracy and faster convergence, an algorithmically  consistent tangent modulus 
should be used instead.  That tangent modulus can be calculated in the usual manner and 
is left for development and implementation as an additional feature in the future.

\section{Stress update}
A standard return algorithm is used to compute the updated Cauchy stress.  Recall
that the rate equation for the deviatoric stress is given by
\Beq
  \dot{\Bs} = 2~\mu~\Beta^e ~.
\Eeq
Integrating the rate equation using a Backward Euler scheme gives
\Beq
  \Bs_{n+1} - \Bs_n = 2~\mu~\Delta~t~\Beta^e_{n+1}
    = 2~\mu~\Delta~t~(\Beta_{n+1} - \Beta^p_{n+1})
\Eeq
Now, from the flow rule, we have
\Beq
  \Beta^p = \dot{\gamma}~\left(\Br -\Third~\Tr(\Br)~\BI\right) ~.
\Eeq
Define the deviatoric part of $\Br$ as
\Beq
  \Dev(\Br) := \Br -\Third~\Tr(\Br)~\BI ~.
\Eeq
Therefore, 
\Beq
  \Bs_{n+1} - \Bs_n 
    = 2~\mu~\Delta~t~\Beta_{n+1} - 2~\mu~\Delta\gamma_{n+1}~\Dev(\Br_{n+1}) ~.
\Eeq
where $\Delta\gamma := \dot{\gamma}~\Delta t$.  Define the trial stress
\Beq
  \Bs^{\Trial} := \Bs_n + 2~\mu~\Delta~t~\Beta_{n+1} ~.
\Eeq
Then
\Beq
  \Bs_{n+1} = \Bs^{\Trial} - 2~\mu~\Delta\gamma_{n+1}~\Dev(\Br_{n+1}) ~.
\Eeq
Also recall that the back stress is given by
\Beq
  \dot{\Bbeta} = \dot{\gamma}~\Dev{\Bh}^{\beta}
\Eeq
The evolution equation for the back stress can be integrated to get
\Beq
  \Bbeta_{n+1} - \Bbeta_n = \Delta\gamma_{n+1}~\Dev(\Bh)^{\beta}_{n+1} ~.
\Eeq
Now,
\Beq
  \Bxi_{n+1} = \Bs_{n+1} - \Bbeta_{n+1} ~.
\Eeq
Plugging in the expressions for $\Bs_{n+1}$ and $\Bbeta_{n+1}$, we get
\Beq
  \Bxi_{n+1} = \Bs^{\Trial} - 2~\mu~\Delta\gamma_{n+1}~\Dev(\Br_{n+1}) 
     - \Bbeta_n - \Delta\gamma_{n+1}~\Dev(\Bh)^{\beta}_{n+1} ~.
\Eeq
Define
\Beq
  \Bxi^{\Trial} := \Bs^{\Trial} - \Bbeta_n ~.
\Eeq
Then
\Beq
  \Bxi_{n+1} = \Bxi^{\Trial} - \Delta\gamma_{n+1}(2~\mu~\Dev(\Br_{n+1}) + \Dev(\Bh)^{\beta}_{n+1}) ~.
\Eeq
Similarly, the evolution of the plastic strain is given by
\Beq
  \Ve^p_{n+1} = \Ve^p_{n} + \Delta\gamma_{n+1}~h^{\alpha}_{n+1}
\Eeq
and the porosity evolves as
\Beq
  \phi_{n+1} = \phi_n + \Delta\gamma_{n+1}~h^{\phi}_{n+1} ~.
\Eeq
The yield condition is discretized as
\Beq
  f(\Bs_{n+1}, \Bbeta_{n+1}, \Ve^p_{n+1}, \phi_{n+1}, \dot{\Ve}_{n+1}, T_{n+1}, \dots) = 
  f(\Bxi_{n+1}, \Ve^p_{n+1}, \phi_{n+1}, \dot{\Ve}_{n+1}, T_{n+1}, \dots) = 0 ~.
\Eeq
{\bf Important:} We assume that the derivatives with respect to $\dot{\Ve}$ and $T$ are
small enough to be neglected.

\subsection{Newton iterations}
We now have the following equations that have to be solved for $\Delta\gamma_{n+1}$:
\Beq
  \Bal
  \Bxi_{n+1} & = \Bxi^{\Trial} - \Delta\gamma_{n+1}(2~\mu~\Dev(\Br_{n+1}) + \Dev(\Bh)^{\beta}_{n+1})\\
  \Ve^p_{n+1} & = \Ve^p_{n} + \Delta\gamma_{n+1}~h^{\alpha}_{n+1} \\
  \phi_{n+1} & = \phi_n + \Delta\gamma_{n+1}~h^{\phi}_{n+1}  \\
  f(\Bxi_{n+1}, \Ve^p_{n+1}, \phi_{n+1}, \dot{\Ve}_{n+1}, T_{n+1}, \dots) & = 0 ~.
  \Eal
\Eeq

Recall that if $g(\Delta\gamma) = 0$ is a nonlinear equation that we have to solve
for $\Delta\gamma$, an iterative Newton method can be expressed as
\Beq
  \Delta\gamma^{(k+1)} = \Delta\gamma^{(k)} - \left[\Deriv{g}{\Delta\gamma}\right]^{-1}_{(k)}~g^{(k)} ~.
\Eeq
Define 
\Beq
  \delta\gamma := \Delta\gamma^{(k+1)} - \Delta\gamma^{(k)} ~.
\Eeq
Then, the iterative scheme can be written as
\Beq
  g^{(k)} + \left[\Deriv{g}{\Delta\gamma}\right]^{(k)}~\delta\gamma  = 0~.
\Eeq
In our case we have
\Beq
  \Bal
  \Ba(\Delta\gamma) = 0 &= -\Bxi + \Bxi^{\Trial} - \Delta\gamma(2~\mu~\Dev(\Br) + \Dev(\Bh)^{\beta})\\
  b(\Delta\gamma) = 0 &= -\Ve^p + \Ve^p_{n} + \Delta\gamma~h^{\alpha} \\
  c(\Delta\gamma) = 0 & = -\phi + \phi_n + \Delta\gamma~h^{\phi}  \\
  f(\Delta\gamma) = 0 &= f(\Bxi, \Ve^p, \phi, \dot{\Ve}, T, \dots) 
  \Eal
\Eeq
Therefore,
\Beq
  \Bal
  \Deriv{\Ba}{\Delta\gamma} & = 
   -\Partial{\Bxi}{\Delta\gamma}  - (2~\mu~\Dev(\Br) + \Dev(\Bh)^{\beta})
   - \Delta\gamma~\left(2~\mu~\Partial{\Dev(\Br)}{\Delta\gamma} + 
        \Partial{\Dev(\Bh)^{\beta}}{\Delta\gamma}\right) \\
   & =
   -\Partial{\Bxi}{\Delta\gamma}  - (2~\mu~\Dev(\Br) + \Dev(\Bh)^{\beta})
   - \Delta\gamma~\left(
      2~\mu~\Partial{\Dev(\Br)}{\Bxi}:\Partial{\Bxi}{\Delta\gamma} + 
      2~\mu~\Partial{\Dev(\Br)}{\Ve^p}~\Partial{\Ve^p}{\Delta\gamma} + 
      2~\mu~\Partial{\Dev(\Br)}{\phi}~\Partial{\phi}{\Delta\gamma} + 
      \right. \\
   & \qquad \qquad
      \left.
      \Partial{\Dev(\Bh)^{\beta}}{\Bxi}:\Partial{\Bxi}{\Delta\gamma} + 
      \Partial{\Dev(\Bh)^{\beta}}{\Ve^p}~\Partial{\Ve^p}{\Delta\gamma} +
      \Partial{\Dev(\Bh)^{\beta}}{\phi}~\Partial{\phi}{\Delta\gamma} 
      \right) \\
  \Deriv{b}{\Delta\gamma} & = -\Partial{\Ve^p}{\Delta\gamma} +  h^{\alpha} 
    + \Delta\gamma~\left(\Partial{h^{\alpha}}{\Bxi}:\Partial{\Bxi}{\Delta\gamma} + 
                        \Partial{h^{\alpha}}{\Ve^p}~\Partial{\Ve^p}{\Delta\gamma} + 
                        \Partial{h^{\alpha}}{\phi}~\Partial{\phi}{\Delta\gamma}\right) \\
  \Deriv{c}{\Delta\gamma} & = -\Partial{\phi}{\Delta\gamma} +  h^{\phi} 
    + \Delta\gamma~\left(\Partial{h^{\phi}}{\Bxi}:\Partial{\Bxi}{\Delta\gamma} + 
                        \Partial{h^{\phi}}{\Ve^p}~\Partial{\Ve^p}{\Delta\gamma} + 
                        \Partial{h^{\phi}}{\phi}~\Partial{\phi}{\Delta\gamma}\right) \\
  \Deriv{f}{\Delta\gamma} & 
     = \Partial{f}{\Bxi}:\Partial{\Bxi}{\Delta\gamma} + 
          \Partial{f}{\Ve^p}~\Partial{\Ve^p}{\Delta\gamma} +
          \Partial{f}{\phi}~\Partial{\phi}{\Delta\gamma} ~.
  \Eal
\Eeq
Now, define
\Beq
   \Delta\Bxi := \Partial{\Bxi}{\Delta\gamma}~\delta\gamma ~;~~
   \Delta\Ve^p := \Partial{\Ve^p}{\Delta\gamma}~\delta\gamma ~;~~
   \Delta\phi := \Partial{\phi}{\Delta\gamma}~\delta\gamma ~.
\Eeq
Then
\Beq
  \Bal
  \Ba^{(k)} & - \Delta\Bxi - [2~\mu~\Dev(\Br^{(k)}) + \Dev(\Bh)^{\beta (k)}]~\delta\gamma \\
   & \qquad \qquad
    - 2~\mu~\Delta\gamma~\left(
      \Partial{\Dev(\Br^{(k)})}{\Bxi}:\Delta\Bxi + 
      \Partial{\Dev(\Br^{(k)})}{\Ve^p}~\Delta\Ve^p + 
      \Partial{\Dev(\Br^{(k)})}{\phi}~\Delta\phi 
      \right) \\
   & \qquad \qquad
    - \Delta\gamma~\left(
      \Partial{\Dev(\Bh)^{\beta (k)}}{\Bxi}:\Delta\Bxi + 
      \Partial{\Dev(\Bh)^{\beta (k)}}{\Ve^p}~\Delta\Ve^p +
      \Partial{\Dev(\Bh)^{\beta (k)}}{\phi}~\Delta\phi
      \right)  = 0\\
  b^{(k)} & - \Delta\Ve^p + h^{\alpha}~\delta\gamma 
    + \Delta\gamma~\left(\Partial{h^{\alpha (k)}}{\Bxi}:\Delta\Bxi + 
                        \Partial{h^{\alpha (k)}}{\Ve^p}~\Delta\Ve^p +
                        \Partial{h^{\alpha (k)}}{\phi}~\Delta\phi\right)
     = 0 \\
  c^{(k)} & - \Delta\phi + h^{\phi}~\delta\gamma 
    + \Delta\gamma~\left(\Partial{h^{\phi (k)}}{\Bxi}:\Delta\Bxi + 
                        \Partial{h^{\phi (k)}}{\Ve^p}~\Delta\Ve^p +
                        \Partial{h^{\phi (k)}}{\phi}~\Delta\phi\right)
     = 0 \\
  f^{(k)} & + \Partial{f^{(k)}}{\Bxi}:\Delta\Bxi + 
          \Partial{f^{(k)}}{\Ve^p}~\Delta\Ve^p +
          \Partial{f^{(k)}}{\phi}~\Delta\phi 
      = 0
  \Eal
\Eeq
Because the derivatives of $\Br^{(k)}, \Bh^{\alpha (k)}, \Bh^{\beta (k)}, \Bh^{\phi (k)}$ with respect 
to $\Bxi, \Ve^p, \phi$ may be difficult to calculate, we instead use a semi-implicit scheme in our 
implementation where the quantities $\Br$, $h^{\alpha}$, $\Bh^{\beta}$, and $h^{\phi}$ are evaluated 
at $t_n$.  Then the problematic derivatives disappear and we are left with
\Beq
  \Bal
  \Ba^{(k)} - \Delta\Bxi - [2~\mu~\Dev(\Br_n) + \Dev(\Bh)^{\beta}_n]~\delta\gamma & = 0\\
  b^{(k)} - \Delta\Ve^p + h^{\alpha}_n~\delta\gamma & = 0 \\
  c^{(k)} - \Delta\phi + h^{\phi}_n~\delta\gamma & = 0 \\
  f^{(k)} + \Partial{f^{(k)}}{\Bxi}:\Delta\Bxi + 
          \Partial{f^{(k)}}{\Ve^p}~\Delta\Ve^p +
          \Partial{f^{(k)}}{\phi}~\Delta\phi & = 0
  \Eal
\Eeq
We now force $\Ba^{(k)}$, $b^{(k)}$, and $c^{(k)}$ to be zero at all times, leading
to the expressions
\Beq
  \Bal
  \Delta\Bxi & = - [2~\mu~\Dev(\Br_n) + \Dev(\Bh)^{\beta}_n]~\delta\gamma \\
  \Delta\Ve^p & =  h^{\alpha}_n~\delta\gamma \\
  \Delta\phi & =  h^{\phi}_n~\delta\gamma \\
  f^{(k)} + \Partial{f^{(k)}}{\Bxi}:\Delta\Bxi + 
          \Partial{f^{(k)}}{\Ve^p}~\Delta\Ve^p +
          \Partial{f^{(k)}}{\phi}~\Delta\phi & = 0
  \Eal
\Eeq
Plugging the expressions for $\Delta\Bxi$, $\Delta\Ve^p$, $\Delta\phi$ from the 
first three equations into the fourth gives us
\Beq
  f^{(k)} -\Partial{f^{(k)}}{\Bxi}:[2~\mu~\Dev(\Br_n) + \Dev(\Bh)^{\beta}_n]~\delta\gamma +
          h^{\alpha}_n~\Partial{f^{(k)}}{\Ve^p}~\delta\gamma  + 
          h^{\phi}_n~\Partial{f^{(k)}}{\phi}~\delta\gamma  = 0 
\Eeq
or
\Beq
  \Delta\gamma^{(k+1)} - \Delta\gamma^{(k)} = \delta\gamma = 
   \cfrac{f^{(k)}}
   {\Partial{f^{(k)}}{\Bxi}:[2~\mu~\Dev(\Br_n) + \Dev(\Bh^{\beta}_n)] - 
   h^{\alpha}_n~\Partial{f^{(k)}}{\Ve^p} -
   h^{\phi}_n~\Partial{f^{(k)}}{\phi} } ~.
\Eeq

\subsection{Algorithm}
The following stress update algorithm is used for each (plastic) time step:
\begin{enumerate}
  \item Initialize:
  \Beq
    k = 0 ~;~~ (\Ve^p)^{(k)} = \Ve^p_n ~;~~ \phi^{(k)} = \phi_n ~;~~
    \Bbeta^{(k)} = \Bbeta_n ~;~~ \Delta\gamma^{(k)} = 0 ~;~~
    \Bxi^{(k)} = \Bxi^{\Trial}~.
  \Eeq
  \item Check yield condition:
  \Beq
    f^{(k)} := f(\Bxi^{(k)}, (\Ve^p)^{(k)}, \phi^{(k)}, \dot{\Ve}_n, T_n, \dots)
  \Eeq
  If $f^{(k)} < \text{tolerance}$ then 
  go to step 5 else go to step 3.
  \item Compute updated $\delta\gamma^{(k)}$ using
  \Beq
    \delta\gamma^{(k)} = 
     \cfrac{f^{(k)}}
     {\Partial{f^{(k)}}{\Bxi}:[2~\mu~\Dev(\Br_n) + \Dev(\Bh^{\beta}_n)] - 
     h^{\alpha}_n~\Partial{f^{(k)}}{\Ve^p} - 
     h^{\phi}_n~\Partial{f^{(k)}}{\phi}} ~.
  \Eeq
  Compute
  \Beq
  \Bal
    \Delta\Bxi^{(k)} & = -[2~\mu~\Dev(\Br_n) + \Dev(\Bh^{\beta}_n)]~\delta\gamma^{(k)} \\
    (\Delta\Ve^p)^{(k)} & =  h^{\alpha}_n~\delta\gamma^{(k)}   \\
    \Delta\phi^{(k)} & =  h^{\phi}_n~\delta\gamma^{(k)}   
  \Eal
  \Eeq
  \item Update variables:
  \Beq
    \Bal
    (\Ve^p)^{(k+1)} & = (\Ve^p)^{(k)} + (\Delta\Ve^p)^{(k)} \\
    \phi^{(k+1)} & = \phi^{(k)} + \Delta\phi^{(k)} \\
    \Bxi^{(k+1)} & = \Bxi^{(k)} + \Delta\Bxi^{(k)} \\
    \Delta\gamma^{(k+1)} & = \Delta\gamma^{(k)} + \delta\gamma^{(k)}
    \Eal
  \Eeq
  Set $k \leftarrow k+1$ and go to step 2.
  \item Update and calculate back stress and the deviatoric part of Cauchy stress:
  \Beq
    \Ve^p_{n+1} = (\Ve^p)^{(k)} ~;~~
    \phi_{n+1} = \phi^{(k)} ~;~~
    \Bxi_{n+1} = \Bxi^{(k)} ~;~~
    \Delta\gamma_{n+1} = \Delta\gamma^{(k)}
  \Eeq
  and
  \Beq
    \Bal
    \widehat{\Bbeta}_{n+1} & = \widehat{\Bbeta}_n + \Delta\gamma_{n+1}~\Bh^{\beta}(\Bxi_{n+1}, \Ve^p_{n+1}, \phi_{n+1}) \\
    \Bbeta_{n+1} & = \widehat{\Bbeta}_{n+1} - \Third~\Tr(\widehat{\Bbeta}_{n+1})~\BI \\
    \Bs_{n+1} & = \Bxi_{n+1} + \Bbeta_{n+1}
    \Eal
  \Eeq
  \item Update the temperature and the Cauchy stress
  \Beq
    \Bal
    T_{n+1} & = T_n + 
     \cfrac{\chi_{n+1}~\Delta t}{\rho_{n+1}~C_p}~\sigma^{n+1}_y~\dot{\Ve^p}_{n+1} 
     = T_n + 
     \cfrac{\chi_{n+1}~\Delta\gamma_{n+1}}{\rho_{n+1}~C_p}~\sigma^{n+1}_y~h^{\alpha}_{n+1} \\
    p_{n+1} & = p(J_{n+1}) \\ 
    \kappa_{n+1} & = J_{n+1}~\left[\Deriv{p(J)}{J}\right]_{n+1} \\
    \Bsig_{n+1} & = \left[p_{n+1} - 3~\kappa_{n+1}~\alpha~(T_{n+1}-T_0)\right]~\BI + \Bs_{n+1}
    \Eal
  \Eeq
\end{enumerate}

\section{Examples}
Let us now look at a few examples.
\subsection{Example 1}
Consider the case of $J_2$ plasticity with the yield condition
\Beq
  f := \sqrt{\frac{3}{2}} \Norm{\Bs-\Bbeta}{} - \sigma_y(\Ve^p, \dot{\Ve}, T, \dots) = 
       \sqrt{\frac{3}{2}} \Norm{\Bxi}{} - \sigma_y(\Ve^p, \dot{\Ve}, T, \dots) \le 0 
\Eeq
where $\Norm{\Bxi} = \sqrt{\Bxi:\Bxi}$. Assume the associated flow rule
\Beq
  \Bd^p = \dot{\gamma}~\Br = \dot{\gamma}~\Partial{f}{\Bsig} = \dot{\gamma}~\Partial{f}{\Bxi} ~.
\Eeq
Then
\Beq
  \Br = \Partial{f}{\Bxi} = \sqrt{\frac{3}{2}}~\cfrac{\Bxi}{\Norm{\Bxi}{}} 
\Eeq
and
\Beq
  \Bd^p = \sqrt{\frac{3}{2}}~\dot{\gamma}~\cfrac{\Bxi}{\Norm{\Bxi}{}} ~;~~
  \Norm{\Bd^p}{} = \sqrt{\frac{3}{2}}~\dot\gamma ~.
\Eeq
The evolution of the equivalent plastic strain is given by
\Beq
  \dot{\Ve^p} = \dot{\gamma}~h^{\alpha} = \sqrt{\cfrac{2}{3}}~\Norm{\Bd^p}{} = \dot{\gamma}~.
\Eeq
This definition is consistent with the definition of equivalent plastic strain
\Beq
  \Ve^p = \int_0^t \dot{\Ve}^p~d\tau = 
   \int_0^t \sqrt{\cfrac{2}{3}}~\Norm{\Bd^p}{}~d\tau ~.
\Eeq
The evolution of porosity is given by (there is no evolution of porosity)
\Beq
  \dot{\phi} = \dot{\gamma}~h^{\phi} = 0
\Eeq
The evolution of the back stress is given by the Prager kinematic hardening rule
\Beq
  \dot{\widehat{\Bbeta}} = \dot{\gamma}~\Bh^{\beta} = \frac{2}{3}~H'~\Bd^p 
\Eeq
where $\widehat{\Bbeta}$ is the back stress and
$H'$ is a constant hardening modulus.  Also, the trace of $\Bd^p$ is 
\Beq
  \Tr(\Bd^p) = \sqrt{\frac{3}{2}}~\dot{\gamma}~\cfrac{\Tr(\Bxi)}{\Norm{\Bxi}{}}~.
\Eeq
Since $\Bxi$ is deviatoric, $\Tr(\Bxi) = 0$ and hence $\Bd^p = \Beta^p$.
Hence, $\widehat{\Bbeta} = \Bbeta$ (where $\Bbeta$ is the deviatoric part of $\widehat{\Bbeta}$), and
\Beq
  \dot{\Bbeta} = \sqrt{\frac{2}{3}}~H'~\dot{\gamma}~\cfrac{\Bxi}{\Norm{\Bxi}{}} ~.
\Eeq

These relation imply that
\Beq
  \boxed{
  \Bal
    \Br & = \sqrt{\frac{3}{2}}~\cfrac{\Bxi}{\Norm{\Bxi}{}} \\
     h^{\alpha} & = 1 \\
     h^{\phi} & = 0 \\
    \Bh^{\beta} & = \sqrt{\frac{2}{3}}~H'~\cfrac{\Bxi}{\Norm{\Bxi}{}} ~.
  \Eal
  }
\Eeq
We also need some derivatives of the yield function.  These are
\Beq
  \Bal
  \Partial{f}{\Bxi} & = \Br \\
  \Partial{f}{\Ve^p} & = -\Partial{\sigma_y}{\Ve^p} \\
  \Partial{f}{\phi} & = 0 ~.
  \Eal
\Eeq

Let us change the kinematic hardening model and use the Armstrong-Frederick
model instead, i.e.,
\Beq
  \dot{\Bbeta} = \dot{\gamma}~\Bh^{\beta} = \frac{2}{3}~H_1~\Bd^p - H_2~\Bbeta~\Norm{\Bd^p}{} ~.
\Eeq
Since
\Beq
  \Bd^p = \sqrt{\frac{3}{2}}~\dot{\gamma}~\cfrac{\Bxi}{\Norm{\Bxi}{}}
\Eeq
we have
\Beq
  \Norm{\Bd^p}{} = 
   \sqrt{\frac{3}{2}}~\dot{\gamma}~\cfrac{\Norm{\Bxi}{}}{\Norm{\Bxi}{}} = 
   \sqrt{\frac{3}{2}}~\dot{\gamma} ~.
\Eeq
Therefore,
\Beq
  \dot{\Bbeta} = \sqrt{\frac{2}{3}}~H_1~\dot{\gamma}~\cfrac{\Bxi}{\Norm{\Bxi}{}} 
    - \sqrt{\frac{3}{2}}~H_2~\dot{\gamma}~\Bbeta ~.
\Eeq
Hence we have
\Beq
  \boxed{
  \Bh^{\beta} = \sqrt{\frac{2}{3}}~H_1~\cfrac{\Bxi}{\Norm{\Bxi}{}} 
    - \sqrt{\frac{3}{2}}~H_2~\Bbeta ~.
   }
\Eeq

\subsection{Example 2}
Let us now consider a Gurson type yield condition with kinematic hardening.  In this
case the yield condition can be written as
\Beq
  f := \cfrac{3~\Bxi:\Bxi}{2~\sigma_y^2} + 
     2~q_1~\phi^{*}~\cosh\left(\cfrac{q_2~\Tr(\Bsig)}{2~\sigma_y}\right)
     - [1 + q_3~(\phi^*)^2]
\Eeq
where $\phi$ is the porosity and
\Beq
  \phi^* = \begin{cases}
             \phi & \text{for}~ \phi \le \phi_c \\
             \phi_c - \cfrac{\phi_u^* - \phi_c}{\phi_f - \phi_c}~(\phi - \phi_c) & 
              \text{for}~ \phi > \phi_c
           \end{cases}
\Eeq
Final fracture occurs for $\phi = \phi_f$ or when $\phi_u^* = 1/q_1$.  

Let us use an associated flow rule
\Beq
  \Bd^p = \dot{\gamma}~\Br = \dot{\gamma}~\Partial{f}{\Bsig} ~.
\Eeq
Then
\Beq
  \Br = \Partial{f}{\Bsig} = \cfrac{3~\Bxi}{\sigma_y^2} + \cfrac{q_1~q_2~\phi^{*}}{\sigma_y}~
   \sinh\left(\cfrac{q_2~\Tr(\Bsig)}{2~\sigma_y}\right)~\BI ~.
\Eeq
In this case
\Beq
  \Tr(\Br) = \cfrac{3~q_1~q_2~\phi^{*}}{\sigma_y}~\sinh\left(\cfrac{q_2~\Tr(\Bsig)}{2~\sigma_y}\right)
  \ne 0 
\Eeq
Therefore,
\Beq
  \Bd^p \ne \Beta^p ~.
\Eeq

For the evolution equation for the plastic strain we use
\Beq
  (\Bsig-\widehat{\Bbeta}):\Bd^p = (1 - \phi)~\sigma_y~\dot{\Ve}^p
\Eeq
where $\dot{\Ve}^p$ is the effective plastic strain rate in the matrix material.  Hence,
\Beq
  \dot{\Ve}^p = \dot{\gamma}~h^{\alpha}
    = \dot{\gamma}~\cfrac{(\Bsig - \widehat{\Bbeta}):\Br}{(1 - \phi)~\sigma_y} ~.
\Eeq

The evolution equation for the porosity is given by
\Beq
  \dot{\phi} = (1 - \phi)~\Tr(\Bd^p) + A~\dot{\Ve^p}
\Eeq
where
\Beq
A = \cfrac{f_n}{s_n \sqrt{2\pi}} \exp [-1/2 (\Ve^p - \Ve_n)^2/s_n^2]
\Eeq
and $ f_n $ is the volume fraction of void nucleating particles, 
$ \Ve_n $ is the mean of the normal distribution of nucleation strains, and 
$ s_n $ is the standard deviation of the distribution.

Therefore,
\Beq
  \dot{\phi} = \dot{\gamma}~h^{\phi} =
    \dot{\gamma}~\left[(1 - \phi)~\Tr(\Br) + A~
    \cfrac{(\Bsig - \widehat{\Bbeta}):\Br}{(1 - \phi)~\sigma_y}\right] ~.
\Eeq

If the evolution of the back stress is given by the Prager kinematic hardening rule
\Beq
  \dot{\widehat{\Bbeta}} = \dot{\gamma}~\Bh^{\beta} = \frac{2}{3}~H'~\Bd^p 
\Eeq
where $\widehat{\Bbeta}$ is the back stress, then
\Beq
  \dot{\widehat{\Bbeta}} = \frac{2}{3}~H'~\dot{\gamma}~\Br ~.
\Eeq
Alternatively, if we use the Armstrong-Frederick model, then
\Beq
  \dot{\widehat{\Bbeta}} = \dot{\gamma}~\Bh^{\beta} = 
   \frac{2}{3}~H_1~\Bd^p - H_2~\widehat{\Bbeta}~\Norm{\Bd^p}{} ~.
\Eeq
Plugging in the expression for $\Bd^p$, we have
\Beq
  \dot{\widehat{\Bbeta}} = \dot{\gamma}~
  \left[\frac{2}{3}~H_1~\Br - H_2~\widehat{\Bbeta}~\Norm{\Br}{}\right] ~.
\Eeq
Therefore, for this model,
\Beq
  \boxed{
  \Bal
  \Br & = \cfrac{3~\Bxi}{\sigma_y^2} + \cfrac{q_1~q_2~\phi^{*}}{\sigma_y}~
   \sinh\left(\cfrac{q_2~\Tr(\Bsig)}{2~\sigma_y}\right)~\BI  \\
  h^{\alpha} &  
    = \cfrac{(\Bsig - \Bbeta):\Br}{(1 - \phi)~\sigma_y} \\
  h^{\phi} & = 
    (1 - \phi)~\Tr(\Br) + A~
    \cfrac{(\Bsig - \widehat{\Bbeta}):\Br}{(1 - \phi)~\sigma_y}  \\
  \Bh^{\beta} & = 
   \frac{2}{3}~H_1~\Br - H_2~\widehat{\Bbeta}~\Norm{\Br}{}
  \Eal
  }
\Eeq
The other derivatives of the yield function that we need are
\Beq
  \Bal
  \Partial{f}{\Bxi} & = \cfrac{3~\Bxi}{\sigma_y^2} \\
  \Partial{f}{\Ve^p} & = \Partial{f}{\sigma_y}~\Partial{\sigma_y}{\Ve^p} 
   = -\left[\cfrac{3~\Bxi:\Bxi}{\sigma_y^3} +
     \cfrac{q_1~q_2~\phi^*~\Tr(\Bsig)}{\sigma_y^2}~
     \sinh\left(\cfrac{q_2~\Tr(\Bsig)}{2~\sigma_y}\right)\right]~
     \Partial{\sigma_y}{\Ve^p}\\
  \Partial{f}{\phi} & = 2~q_1~\Deriv{\phi^*}{\phi}~
    \cosh\left(\cfrac{q_2~\Tr(\sigma)}{2~\sigma_y}\right) 
    - 2~q_3~\phi^*~\Deriv{\phi^*}{\phi} ~.
  \Eal
\Eeq


