\chapter{Cam-Clay model based on Borja et al. 1997}

\section{Introduction}
The Cam-clay plasticity model and its modified ellipsoidal version \cite{roscoe1963,roscoe1968,schofield1968,wroth1986} is widely considered to be an accurate model for the prediction of the compressive and shear behavior of clays.  The Borja model \cite{borja1990,borja1991,borja1997,borja1998} extends the original Cam-clay model to large deformations and uses a hyperelastic model and large strain elastic-plastic kinematics.

The Borja Cam-clay model and its implementation in Vaango are discussed in this chapter.

\section{Quantities that are needed in a Vaango implementation}
The implementation of a hyperelastic-plastic model in Vaango typically (but not always) involves the following:
\begin{enumerate}
  \item an elasticity model factory that creates an elasticity model that provides the simulation with a pressure and a deviatoric stress for a given (elastic) deformation gradient.
  \item a plasticity model factory that creates:
        \begin{enumerate}
          \item a yield condition factory that compute the yield function for a given stress and internal variable state,
          \item a flow rule factory that gives the value of the plastic potential for a given state of stress/internal variables.  The flow rule factory and yield condition factory are typically assumed to be identical (i.e., plastic flow is associated),
          \item an internal variable factory that is used to update internal variables and compute hardening moduli.
        \end{enumerate}
\end{enumerate}
The models returned by the various factories for Borja cam-clay are discussed below.

\subsection{Elasticity}
The elastic strain energy density in Borja's model has the form
\[
  W(\Ve^e_v,\Ve^e_s) = W_\Tvol(\Ve^e_v) + W_\Tdev(\Ve^e_v, Ve^e_s)
\]
where
\[
   \Bal
    W_\Tvol(\Ve^e_v) & = -p_0\kappatilde\,\exp\left(-\frac{\Ve^e_v - \Ve^e_{v0}}{\kappatilde}\right) \\
    W_\Tdev(\Ve^e_v,\Ve^e_s) & =  \tfrac{3}{2}\,\mu\,(\Ve^e_s)^2
   \Eal
\]
where $\Ve^e_{v0}$ is the volumetric strain corresponding to a mean normal compressive stress $p_0$ 
(tension positive), $\kappatilde$ is the elastic compressibility index, and the shear modulus is given by
\[
  \mu = \mu_0 + \frac{\alpha}{\kappatilde}\,W_\Tvol(\Ve^e_v) 
      = \mu_0 - \alpha p_0\,\exp\left(-\frac{\Ve^e_v - \Ve^e_{v0}}{\kappatilde}\right) 
      = \mu_0 - \mu_\Tvol\,.
\]
The parameter $\alpha$ determines the extent of coupling between the volumetric and deviatoric 
responses.  For consistency with isotropic elasticity, Rebecca Brannon suggests that $\alpha = 0$ (citation?). 

The stress invariants $p$ and $q$ are defined as
\[
  \Bal
    p &= \Partial{W}{\Ve^e_v} = p_0\left[1 + \tfrac{3}{2}\,\frac{\alpha}{\kappatilde}\,(\Ve^e_s)^2\right]
         \exp\left(-\frac{\Ve^e_v - \Ve^e_{v0}}{\kappatilde}\right) 
       = p_0\,\beta\,\exp\left(-\frac{\Ve^e_v - \Ve^e_{v0}}{\kappatilde}\right) \\
    q &= \Partial{W}{\Ve^e_s} = 3\left[\mu_0 - \alpha p_0\exp\left(-\frac{\Ve^e_v - \Ve^e_{v0}}{\kappatilde}\right) 
         \right]\Ve^e_s  = 3\mu\,\Ve^e_s\,.
  \Eal
\]
The derivatives of the stress invariants are
\[
  \Bal
    \Partial{p}{\Ve^e_v} & = -\frac{p_0}{\kappatilde}
         \left[1 + \tfrac{3}{2}\,\frac{\alpha}{\kappatilde}\,(\Ve^e_s)^2\right]
         \,\exp\left(-\frac{\Ve^e_v - \Ve^e_{v0}}{\kappatilde}\right) 
        = -\frac{p}{\kappatilde} \\
    \Partial{p}{\Ve^e_s} & = \Partial{q}{\Ve^e_v} = \frac{3\alpha p_0 \Ve^e_s}{\kappatilde}
         \,\exp\left(-\frac{\Ve^e_v - \Ve^e_{v0}}{\kappatilde}\right) = \frac{3\alpha p}{\beta \kappatilde}\,\Ve^e_s 
         = \frac{3\mu_\Tvol}{\kappatilde}\,\Ve^e_s\\
    \Partial{q}{\Ve^e_s} & = 3 \left[\mu_0 - \alpha p_0
         \,\exp\left(-\frac{\Ve^e_v - \Ve^e_{v0}}{\kappatilde}\right) \right] = 3 \mu\,.
  \Eal
\]

\subsection{Plasticity}
For plasticity we use a Cam-Clay yield function of the form
\[
   f = \left(\frac{q}{M}\right)^2 + p(p-p_c) 
\]
where $M$ is the slope of the critical state line and the consolidation pressure $p_c$ is an internal variable that 
evolves according to 
\[
   \frac{1}{p_c}\,\Deriv{p_c}{t} = \frac{1}{\lambdatilde - \kappatilde}\,\Deriv{\Ve^p_v}{t} \,.
\]
The derivatives of $f$ that are of interest are
\[
   \Bal
     \Partial{f}{p} & = 2p - p_c \\
     \Partial{f}{q} & = \frac{2q}{M^2} \,.
   \Eal
\]
If we integrate the equation for $p_c$ from $t_{n}$ to $t_{n+1}$, we can show that
\[
   (p_c)_{n+1} = (p_c)_n \exp\left[\frac{(\Ve_v^e)_\Trial - (\Ve_v^e)_{n+1}}{\lambdatilde - \kappatilde}\right] \,.
\]
The derivative of $p_c$ that is of interest is
\[
   \Partial{p_c}{(\Ve_v^e)_{n+1}} = -\frac{(p_c)_n}{\lambdatilde-\kappatilde}\,\exp\left[\frac{(\Ve^e_v)_\Trial - (\Ve_v^e)_{n+1}}{\lambdatilde-\kappatilde}\right] \,.
\]

\section{Stress update based Rich Reguiero's notes}
The volumetric and deviatoric components of the elastic strain $\Beps^e$ are defined
as follows:
\[
   \BeT^e = \Beps^e - \tfrac{1}{3}\Ve^e_v\,\Bone = \Beps^e - \tfrac{1}{3}\Tr(\Beps^e)\,\Bone
   \quad \Tand \quad
   \Ve^e_s = \sqrt{\tfrac{2}{3}}\Norm{\BeT^e}{}  = \sqrt{\tfrac{2}{3}}\sqrt{\BeT^e:\BeT^e} \,.
\]
The stress tensor is decomposed into a volumetric and a deviatoric component
\[
   \Bsig = p\,\Bone + \sqrt{\tfrac{2}{3}}\, q\, \BnT \quad \text{with} \quad
   \BnT = \cfrac{\BeT^e}{\Norm{\BeT^e}{}} = \sqrt{\tfrac{2}{3}}\, \cfrac{\BeT^e}{\Ve^e_s} \,.
\]
The models used to determine $p$ and $q$ are
\[
  \Bal
    p &= p_0\beta\exp\left[-\frac{\Ve^e_v - \Ve^e_{v0}}{\kappatilde}\right] \quad \text{with} \quad
     \beta = 1 + \tfrac{3}{2}\,\frac{\alpha}{\kappatilde}\,(\Ve^e_s)^2 \\
    q &= 3\mu\Ve^e_s \,.
  \Eal
\]
The strains are updated using
\[
  \Beps^e = \Beps^e_{\rm trial} - \Delta\gamma\,\Partial{f}{\Bsig}
  \quad \text{where} \quad \Beps^e_{\rm trial} = \Beps^e_n + \Delta\Beps
     = \Beps^e_n + (\Beps - \Beps_n) \,.
\]

{\footnotesize
{\bf Remark 1:}  The interface with MPMICE, among other things in Vaango, requires the 
computation of the quantity $dp/dJ$.  Since $J$ does not appear in the above equation we
proceed as explained below.
\[
  \Bal
   J & = \det(\BF) = \det(\Bone + \GradX{\Bu}) = \det(\Bone + \Beps) \\
     & = 1 + \Tr\Beps + \Half\left[(\Tr\Beps)^2 - \Tr(\Beps^2)\right] + \det(\Beps) \,.
     & = 1 + \Ve_v + \Half\left[\Ve_v^2 - \Tr(\Beps^2)\right] + \det(\Beps) \,.
  \Eal
\]
Also,
\[
   J = \frac{\rho_0}{\rho} = \frac{V}{V_0} \quad \Tand \quad
   \Ve_v = \frac{V-V_0}{V_0} = \frac{V}{V_0} - 1 = J - 1 \,.
\]
We use the relation $J = 1 + \Ve_v$ while keeping in mind that this is {\em true only for
infinitesimal strains and plastic incompressibility} for which 
$\Ve_v^2$, $\Tr(\Beps^2)$, and $\det(\Beps)$ are zero.  Under these conditions
\[
   \Partial{p}{J} = \Partial{p}{\Ve_v}\,\Partial{\Ve_v}{J} = \Partial{p}{\Ve_v} \quad \Tand 
   \quad
   \Partial{p}{\rho} = \Partial{p}{\Ve_v}\,\Partial{\Ve_v}{J}\,\Partial{J}{\rho} 
      = -\frac{J}{\rho}\,\Partial{p}{\Ve_v} \,.
\]

{\bf Remark 2:} MPMICE also needs the density at a given pressure.  For the Borja model, with
$\Ve_v = J-1 = \rho_0/\rho -1$, we have
\[
  \rho = \rho_0\left[1 + \Ve_{v0} + \kappatilde\ln\left(\frac{p}{p_0\beta}\right)\right]^{-1}\,.
\]

{\bf Remark 3:}  The quantity $q$ is related to the deviatoric part of the Cauchy stress, $\BsT$
as follows:
\[
   q = \sqrt{3J_2} \quad \text{where} \quad J_2 = \Half\,\BsT:\BsT \,.
\]
The shear modulus relates the deviatoric stress $\BsT$ to the deviatoric strain $\BeT^e$.  We
assume a relation of the form
\[
   \BsT = 2\mu\BeT^e \,.
\]
Note that the above relation assumes a linear elastic type behavior.  Then we get the Borja 
shear model:
\[
  q = \sqrt{\tfrac{3}{2}\,\BsT:\BsT} = \sqrt{\tfrac{3}{2}}\,(2\mu)\,\sqrt{\BeT^e:\BeT^e}
     = \sqrt{\tfrac{3}{2}}\,(2\mu)\,\sqrt{\tfrac{3}{2}}\,\Ve^e_s = 3\mu\Ve^e_s \,.
\]
}



\subsection{Elastic-plastic stress update}
For elasto-plasticity we start with a yield function of the form
\[
   f = \left(\frac{q}{M}\right)^2 + p(p-p_c) \le 0 
   \quad \text{where} \quad
  \frac{1}{p_c}\,\Deriv{p_c}{t} = \frac{1}{\lambdatilde- \kappatilde}\,\Deriv{\Ve_v^p}{t} \,.
\]
Integrating the ODE for $p_c$ with the initial condition $p_c(t_n) = (p_c)_n$, at $t = t_{n+1}$, 
\[
   (p_c)_{n+1} = (p_c)_n \exp\left[\frac{(\Ve_v^p)_{n+1} - (\Ve_v^p)_n}{\lambdatilde - \kappatilde}\right] \,.
\]
From the additive decomposition of the strain into elastic and plastic parts, and if the elastic trial 
strain is defined as 
\[
   (\Ve_v^e)_\Trial := (\Ve_v^e)_n + \Delta\Ve_v
\]
we have
\[
   \Ve_v^p = \Ve_v - \Ve_v^e \quad \implies \quad
   (\Ve_v^p)_{n+1} - (\Ve_v^p)_n = (\Ve_v)_{n+1} - (\Ve_v^e)_{n+1} - (\Ve_v)_{n} + (\Ve_v^e)_{n} 
                               = \Delta\Ve_v + (\Ve_v^e)_{n} - (\Ve_v^e)_{n+1} 
                               = (\Ve_v^e)_\Trial - (\Ve_v^e)_{n+1} \,.
\]
Therefore we can write
\[
   (p_c)_{n+1} = (p_c)_n \exp\left[\frac{(\Ve_v^e)_\Trial - (\Ve_v^e)_{n+1}}{\lambdatilde - \kappatilde}\right] \,.
\]
The flow rule is assumed to be given by
\[
   \Partial{\Beps^p}{t} = \gamma\,\Partial{f}{\Bsig} \,.
\]
Integration of the PDE with backward Euler gives
\[
  \Beps^p_{n+1} = \Beps^p_n + \Delta t\,\gamma_{n+1}\,\left[\Partial{f}{\Bsig}\right]_{n+1} 
             = \Beps^p_n + \Delta\gamma\,\left[\Partial{f}{\Bsig}\right]_{n+1} \,.
\]
This equation can be expressed in terms of the trial elastic strain as follows.
\[
  \Beps_{n+1}-\Beps^e_{n+1} = \Beps_n - \Beps^e_n + \Delta\gamma\,\left[\Partial{f}{\Bsig}\right]_{n+1} 
\]
or
\[
  \Beps^e_{n+1} = \Delta\Beps + \Beps^e_n - \Delta\gamma\,\left[\Partial{f}{\Bsig}\right]_{n+1} 
             = \Beps^e_\Trial - \Delta\gamma\,\left[\Partial{f}{\Bsig}\right]_{n+1} \,.
\]
In terms of the volumetric and deviatoric components
\[
  (\Ve_v^e)_{n+1} = \Tr(\Beps^e_{n+1}) = \Tr(\Beps^e_\Trial) - \Delta\gamma\,\Tr\left[\Partial{f}{\Bsig}\right]_{n+1} = (\Ve_v^e)_\Trial - \Delta\gamma\,\Tr\left[\Partial{f}{\Bsig}\right]_{n+1} 
\]
and
\[
  \BeT^e_{n+1} = \BeT^e_\Trial - \Delta\gamma\,\left[\left(\Partial{f}{\Bsig}\right)_{n+1}
      - \tfrac{1}{3}\,\Tr\left(\Partial{f}{\Bsig}\right)_{n+1}\Bone\right] \,.
\]
With $\BsT = \Bsig - p\Bone$, we have
\[
  \Partial{f}{\Bsig} = \Partial{f}{\BsT}:\Partial{\BsT}{\Bsig} + \Partial{f}{p}\,\Partial{p}{\Bsig}
    = \Partial{f}{\BsT}:[\SfI^{(s)} - \tfrac{1}{3}\,\Bone\otimes\Bone] 
      + \Partial{f}{p}\,\Bone
    = \Partial{f}{\BsT} - \tfrac{1}{3}\,\Tr\left[\Partial{f}{\BsT}\right]\Bone
      + \Partial{f}{p}\,\Bone
\]
and
\[
  \tfrac{1}{3}\,\Tr\left[\Partial{f}{\Bsig}\right]\Bone = 
    \tfrac{1}{3}\left(\Tr\left[\Partial{f}{\BsT}\right] - \Tr\left[\Partial{f}{\BsT}\right]
      + 3\Partial{f}{p}\right)\Bone  =  \Partial{f}{p}\,\Bone \,.
\]

{\footnotesize
{\bf Remark 4:}  Note that, because $\Bsig = \Bsig(p, q, p_c)$ the 
chain rule should contain a contribution from $p_c$:
\[
  \Partial{f}{\Bsig} = \Partial{f}{q}\,\Partial{q}{\Bsig} + \Partial{f}{p}\,\Partial{p}{\Bsig}
                       + \Partial{f}{p_c}\,\Partial{p_c}{\Bsig} \,.
\]
However, the Borja implementation does not consider that extra term.  Also note that for the present model
\[
  \Bsig = \Bsig(p(\Ve^e_v,\Ve^e_s,\Ve^p_v, \Ve^p_s), \BsT(\Ve^e_v, \Ve^e_s, \Ve^p_v,\Ve^p_s), p_c(\Ve^p_v))
\]
}

Therefore, for situations where $\Tr(\partial f/\partial \BsT) = \Bzero$, we have
\[
  \Partial{f}{\Bsig} - \tfrac{1}{3}\,\Tr\left[\Partial{f}{\Bsig}\right]\Bone = 
     \Partial{f}{\BsT} - \tfrac{1}{3}\,\Tr\left[\Partial{f}{\BsT}\right]\Bone =
     \Partial{f}{\BsT} \,.
\]
The deviatoric strain update can be written as
\[
  \BeT^e_{n+1} = \BeT^e_\Trial - \Delta\gamma\,\left(\Partial{f}{\BsT}\right)_{n+1} 
\]
and the shear invariant update is
\[
  (\Ve_s^e)_{n+1} = \sqrt{\tfrac{2}{3}}\,
  \sqrt{\BeT^e_{n+1}:\BeT^e_{n+1}}
  = \sqrt{\tfrac{2}{3}}\,\sqrt{\BeT^e_\Trial:\BeT^e_\Trial 
     - 2\Delta\gamma\,\left[\Partial{f}{\BsT}\right]_{n+1}:\BeT^e_\Trial
     + (\Delta\gamma)^2\left[\Partial{f}{\BsT}\right]_{n+1}:\left[\Partial{f}{\BsT}\right]_{n+1}}
\]
The derivative of $f$ can be found using the chain rule (for smooth $f$):
\[
   \Partial{f}{\Bsig} = \Partial{f}{p}\,\Partial{p}{\Bsig} + \Partial{f}{q}\,\Partial{q}{\Bsig}
     = (2p - p_c)\,\Partial{p}{\Bsig} + \frac{2q}{M^2}\,\Partial{q}{\Bsig} \,.
\]
Now, with $p = 1/3\,\Tr(\Bsig)$ and $q = \sqrt{3/2\,\Bs:\Bs}$, we have
\[
  \Bal
   \Partial{p}{\Bsig} & = \Partial{}{\Bsig}\left[\tfrac{1}{3}\,\Tr(\Bsig)\right] = \tfrac{1}{3}\,\Bone \\
   \Partial{q}{\Bsig} & = \Partial{}{\Bsig}\left[\sqrt{\tfrac{3}{2}\,\BsT:\BsT}\right]
     = \sqrt{\tfrac{3}{2}}\,\frac{1}{\sqrt{\BsT:\BsT}}\,\Partial{\BsT}{\Bsig}:\BsT
     = \sqrt{\tfrac{3}{2}}\,\frac{1}{\Norm{\BsT}{}}\,\left[\SfI^{(s)}-\tfrac{1}{3}\Bone\otimes\Bone\right]:\BsT
     = \sqrt{\tfrac{3}{2}}\,\frac{\BsT}{\Norm{\BsT}{}}\,.
  \Eal
\]
Therefore,
\[
   \Partial{f}{\Bsig} = \frac{2p - p_c}{3}\,\Bone + \sqrt{\tfrac{3}{2}}\,\frac{2q}{M^2}\,\frac{\BsT}{\Norm{\BsT}{}} \,.
\]
Recall that
\[
   \Bsig = p\,\Bone + \sqrt{\tfrac{2}{3}}\, q\, \BnT = p\,\Bone + \BsT \,.
\]
Therefore,
\[
   \BsT = \sqrt{\tfrac{2}{3}}\,q\,\BnT 
   \quad \Tand \quad
   \Norm{\BsT}{} = \sqrt{\BsT:\BsT} = \sqrt{\tfrac{2}{3}\,q^2\,\BnT:\BnT} = 
     \sqrt{\tfrac{2}{3}\,q^2\,\frac{\BeT^e:\BeT^e}{\Norm{\BeT^e}{}^2}}
     = \sqrt{\tfrac{2}{3}\,q^2} = \sqrt{\tfrac{2}{3}}\,q \,.
\]
So we can write
\Beq
   \Partial{f}{\Bsig} = \frac{2p - p_c}{3}\,\Bone + \sqrt{\tfrac{3}{2}}\,\frac{2q}{M^2}\,\BnT \,.
\Eeq
Using the above relation we have
\[
   \Partial{f}{p} = \tfrac{1}{3}\,\Tr\left[\Partial{f}{\Bsig}\right] = 2p-p_c 
   \quad \Tand \quad
   \Partial{f}{\BsT}  = \Partial{f}{\Bsig} - \Partial{f}{p}\Bone
                      = \sqrt{\tfrac{3}{2}}\,\frac{2q}{M^2}\,\BnT \,.
\]
The strain updates can now be written as
\[
  \Bal
  (\Ve_v^e)_{n+1} & =  (\Ve_v^e)_\Trial - \Delta\gamma\,[2p_{n+1} - (p_c)_{n+1}] \\
  \BeT^e_{n+1} & = \BeT^e_\Trial - \sqrt{\tfrac{3}{2}}\,\Delta\gamma\,\left(\frac{2q_{n+1}}{M^2_{n+1}}\right)
           \BnT_{n+1}  \\
  (\Ve_s^e)_{n+1} & = 
    \sqrt{\tfrac{2}{3}}\,\sqrt{\BeT^e_\Trial:\BeT^e_\Trial 
     - \sqrt{6}\,(\Delta\gamma)^2\left(\frac{2q_{n+1}}{M^2_{n+1}}\right)\BnT_{n+1}:\BeT^e_\Trial
     + \tfrac{3}{2}\,(\Delta\gamma)^4\,\left(\frac{2q_{n+1}}{M^2_{n+1}}\right)^2 } \,.
  \Eal
\]
From the second equation above,
\[
  \BnT_{n+1}:\BeT^e_\Trial = 
   \BnT_{n+1}:\BeT^e_{n+1} + \sqrt{\tfrac{3}{2}}\,\Delta\gamma\,\left(\frac{2q_{n+1}}{M^2_{n+1}}\right)
           \BnT_{n+1}:\BnT_{n+1} = 
   \frac{\BeT^e_{n+1}:\BeT^e_{n+1}}{\Norm{\BeT^e_{n+1}}{}} + 
      \sqrt{\tfrac{3}{2}}\,\Delta\gamma\,\left(\frac{2q_{n+1}}{M^2_{n+1}}\right) =
   \Norm{\BeT^e_{n+1}}{} + 
      \sqrt{\tfrac{3}{2}}\,\Delta\gamma\,\left(\frac{2q_{n+1}}{M^2_{n+1}}\right) \,.
\]
Also notice that
\[
  \BeT^e_\Trial:\BeT^e_\Trial = \BeT^e_{n+1}:\BeT^e_{n+1} + 2\,\sqrt{\tfrac{3}{2}}\,\Delta\gamma\,
     \left(\frac{2q_{n+1}}{M^2_{n+1}}\right)\BeT^e_{n+1}:\BnT_{n+1} +
     \left[\sqrt{\tfrac{3}{2}}\,\Delta\gamma\,\left(\frac{2q_{n+1}}{M^2_{n+1}}\right)\right]^2
\]
or,
\[
  \Norm{\BeT^e_\Trial}{}^2 = \left[\Norm{\BeT^e_{n+1}}{} + \sqrt{\tfrac{3}{2}}\,\Delta\gamma\,\left(\frac{2q_{n+1}}{M^2_{n+1}}\right)\right]^2 \,.
\]
Therefore,
\[
  \BnT_{n+1}:\BeT^e_\Trial = \Norm{\BeT^e_\Trial}{} 
\]
and we have
\[
  (\Ve_s^e)_{n+1} = 
    \sqrt{\tfrac{2}{3}}\,\sqrt{\Norm{\BeT^e_\Trial}{}^2
     - \sqrt{6}\,(\Delta\gamma)^2\left(\frac{2q_{n+1}}{M^2_{n+1}}\right)\Norm{\BeT^e_\Trial}{}
     + \tfrac{3}{2}\,(\Delta\gamma)^4\,\left(\frac{2q_{n+1}}{M^2_{n+1}}\right)^2 } 
    = \sqrt{\tfrac{2}{3}}\,\Norm{\BeT^e_\Trial}{} - \Delta\gamma\,\left(\frac{2q_{n+1}}{M^2_{n+1}}\right) \,.
\]
The elastic strain can therefore be updated using
\[
  \Bal 
    (\Ve_v^e)_{n+1} & =  (\Ve_v^e)_\Trial - \Delta\gamma\,[2p_{n+1} - (p_c)_{n+1}] \\
    (\Ve_s^e)_{n+1} & =  (\Ve_s^e)_\Trial - \Delta\gamma\,\left(\frac{2q_{n+1}}{M^2_{n+1}}\right) \,.
  \Eal 
\]
The consistency condition is needed to close the above equations
\[
   f = \left(\frac{q_{n+1}}{M}\right)^2 + p_{n+1}[p_{n+1}-(p_c)_{n+1}] = 0  \,.
\]
The unknowns are $(\Ve_v^e)_{n+1}$, $(\Ve_s^e)_{n+1}$ and $\Delta\gamma$.  Note that we can express
the three equations as
\Beq
  \Bal 
    (\Ve_v^e)_{n+1} & =  (\Ve_v^e)_\Trial - \Delta\gamma\,\left[\Partial{f}{p}\right]_{n+1} \\
    (\Ve_s^e)_{n+1} & =  (\Ve_s^e)_\Trial - \Delta\gamma\,\left[\Partial{f}{q}\right]_{n+1} \\
    f_{n+1} & = 0 \,.
  \Eal 
\Eeq

\subsection{Newton iterations}
The three nonlinear equations in the three unknowns can be solved using Newton iterations
for smooth yield functions.  Let us define the residual as
\[
   \Mr(\Mx) = \begin{bmatrix} 
    (\Ve_v^e)_{n+1} -  (\Ve_v^e)_\Trial + \Delta\gamma\,\left[\Partial{f}{p}\right]_{n+1} \\
    (\Ve_s^e)_{n+1} -  (\Ve_s^e)_\Trial + \Delta\gamma\,\left[\Partial{f}{q}\right]_{n+1} \\
    f_{n+1} \end{bmatrix} =: \begin{bmatrix} r_1 \\ r_2 \\ r_3 \end{bmatrix}
   \quad \text{where} \quad
   \Mx = \begin{bmatrix} (\Ve_v^e)_{n+1} \\ (\Ve_s^e)_{n+1} \\ f_{n+1} \end{bmatrix} 
        =: \begin{bmatrix} x_1 \\ x_2 \\ x_3 \end{bmatrix} \,.
\]
The Newton root finding algorithm is :
\begin{algorithm}
  \begin{algorithmic}
    \Require $\Mx^0$
    \State $k \leftarrow 0$
    \While {$\Mr(\Mx^k) \ne 0$}
      \State $\Mx^{k+1} \Leftarrow \Mx^k - \left[\left(\Partial{\Mr}{\Mx}\right)^{-1}\right]_{\Mx^k}\cdot
              \Mr(\Mx^k)$
      \State $k \leftarrow k+1$
    \EndWhile
  \end{algorithmic}
\end{algorithm}

To code the algorithm we have to find the derivatives of the residual with respect to the primary variables.
Let's do the terms one by one.  For the first row,
\[
  \Bal
  \Partial{r_1}{x_1} & = \Partial{}{\Ve_v^e}\left[\Ve_v^e -  (\Ve_v^e)_\Trial + \Delta\gamma\,(2p-p_c)\right] 
     = 1 + \Delta\gamma\left(2\Partial{p}{\Ve_v^e} - \Partial{p_c}{\Ve_v^e}\right) \\
  \Partial{r_1}{x_2} & = \Partial{}{\Ve_s^e}\left[\Ve_v^e -  (\Ve_v^e)_\Trial + \Delta\gamma\,(2p-p_c)\right] 
     = 2\Delta\gamma\,\Partial{p}{\Ve_s^e}\\
  \Partial{r_1}{x_3} & = \Partial{}{\Delta\gamma}\left[\Ve_v^e -  (\Ve_v^e)_\Trial + \Delta\gamma\,(2p-p_c)\right]
     = 2p - p_c = \Partial{f}{p}
  \Eal
\]
where
\[
  \Bal
   \Partial{p}{\Ve_v^e} & = -\frac{p_0\,\beta}{\kappatilde}\,\exp\left[-\frac{\Ve_v^e - \Ve_{v0}^e}{\kappatilde}\right] = \frac{p}{\kappatilde} \quad, \quad
   \Partial{p_c}{\Ve_v^e} = \frac{(p_c)_n}{\kappatilde-\lambdatilde}\,\exp\left[\frac{\Ve_v^e - (\Ve_{v}^e)_\Trial}{\kappatilde-\lambdatilde}\right] \quad \Tand \\
   \Partial{p}{\Ve_s^e} & = \frac{3\,p_0\,\alpha\,\Ve_s^e}{\kappatilde}\,\exp\left[-\frac{\Ve_v^e - \Ve_{v0}^e}{\kappatilde}\right] \,.
  \Eal
\]
For the second row,
\[
  \Bal
  \Partial{r_2}{x_1} & = \Partial{}{\Ve_v^e}\left[\Ve_s^e -  (\Ve_s^e)_\Trial + \Delta\gamma\,\frac{2q}{M^2}\right] 
     = \frac{2\Delta\gamma}{M^2}\,\Partial{q}{\Ve_v^e} \\
  \Partial{r_2}{x_2} & = \Partial{}{\Ve_s^e}\left[\Ve_s^e -  (\Ve_s^e)_\Trial + \Delta\gamma\,\frac{2q}{M^2}\right] 
     = 1 + \frac{2\Delta\gamma}{M^2}\,\Partial{q}{\Ve_s^e}\\
  \Partial{r_2}{x_3} & = \Partial{}{\Delta\gamma}\left[\Ve_s^e -  (\Ve_s^e)_\Trial + \Delta\gamma\,\frac{2q}{M^2}\right]
     = \frac{2q}{M^2} = \Partial{f}{q}
  \Eal
\]
where
\[
  \Partial{q}{\Ve_v^e} = -\frac{3p_0\,\alpha\,\Ve_s^e}{\kappatilde}\,\exp\left[-\frac{\Ve_v^e - \Ve_{v0}^e}{\kappatilde}\right] = \Partial{p}{\Ve_s^e}
  \quad \Tand \quad 
  \Partial{q}{\Ve_s^e} = 3\mu_0 + 3p_0\,\alpha\,\exp\left[-\frac{\Ve_v^e - \Ve_{v0}^e}{\kappatilde}\right] = 3\mu \,.
\]
For the third row, 
\[
  \Bal
  \Partial{r_3}{x_1} & = \Partial{}{\Ve_v^e}\left[\frac{q^2}{M^2} + p\,(p - p_c)\right]
     = \frac{2q}{M^2}\,\Partial{q}{\Ve_v^e} + (2p - p_c)\,\Partial{p}{\Ve_v^e} - p\,\Partial{p_c}{\Ve_v^e} 
     = \Partial{f}{q}\,\Partial{q}{\Ve_v^e} + \Partial{f}{p}\,\Partial{p}{\Ve_v^e} - p\,\Partial{p_c}{\Ve_v^e} \\
  \Partial{r_3}{x_2} & = \Partial{}{\Ve_s^e}\left[\frac{q^2}{M^2} + p\,(p - p_c)\right]
     = \frac{2q}{M^2}\,\Partial{q}{\Ve_s^e} + (2p - p_c)\,\Partial{p}{\Ve_s^e} 
     = \Partial{f}{q}\,\Partial{q}{\Ve_s^e} + \Partial{f}{p}\,\Partial{p}{\Ve_s^e} \\
  \Partial{r_3}{x_3} & = \Partial{}{\Delta\gamma}\left[\frac{q^2}{M^2} + p\,(p - p_c)\right]
     =  0 \,.
  \Eal
\]
We have to invert a matrix in the Newton iteration process.  Let us see whether we can make this 
quicker to do.  The Jacobian matrix has the form
\[
  \Partial{\Mr}{\Mx} = \begin{bmatrix}\Partial{r_1}{x_1} & \Partial{r_1}{x_2} & \Partial{r_1}{x_3} \\
     \Partial{r_2}{x_1} & \Partial{r_2}{x_2} & \Partial{r_2}{x_3} \\
     \Partial{r_3}{x_1} & \Partial{r_3}{x_2} & \Partial{r_3}{x_3} \end{bmatrix} 
     = \begin{bmatrix} \MAmat & \MBmat \\ \MCmat & 0 \end{bmatrix}
\]
where
\[
  \MAmat = \begin{bmatrix}\Partial{r_1}{x_1} & \Partial{r_1}{x_2} \\
                       \Partial{r_2}{x_1} & \Partial{r_2}{x_2} \end{bmatrix} \,,\quad
  \MBmat = \begin{bmatrix}\Partial{r_1}{x_3} \\ \Partial{r_2}{x_3} \end{bmatrix} \,,\quad \Tand \quad
  \MCmat = \begin{bmatrix}\Partial{r_3}{x_1} & \Partial{r_3}{x_2} \end{bmatrix} \,.
\]
We can also break up the $\Mx$ and $\Mr$ matrices:
\[
  \Delta\Mx = \Mx^{k+1}-\Mx^k = \begin{bmatrix} \Delta\Mx^{vs} \\ \Delta x_3 \end{bmatrix} \,, \quad
  \Mr = \begin{bmatrix} \Mr^{vs} \\ r_3 \end{bmatrix}
  \quad \text{where} \quad \Mr^{vs} = \begin{bmatrix} r_1 \\ r_2 \end{bmatrix}  
   \quad \Tand \quad \Delta\Mx^{vs} = \begin{bmatrix}\Delta x_1 \\\Delta x_2 \end{bmatrix}  \,.
\]
Then
\[
  \begin{bmatrix} \Delta\Mx^{vs} \\ \Delta x_3 \end{bmatrix} 
   = - \begin{bmatrix} \MAmat & \MBmat \\ \MCmat & 0 \end{bmatrix}^{-1} 
                    \begin{bmatrix} \Mr^{vs} \\ r_3 \end{bmatrix} 
  \quad \implies \quad 
   \begin{bmatrix} \MAmat & \MBmat \\ \MCmat & 0 \end{bmatrix} 
  \begin{bmatrix} \Delta\Mx^{vs} \\ \Delta x_3 \end{bmatrix}  = 
                    -\begin{bmatrix} \Mr^{vs} \\ r_3 \end{bmatrix} 
\]
or
\[
   \MAmat\,\Delta\Mx^{vs} + \MBmat\,\Delta x_3 = -\Mr^{vs} \quad \Tand \quad
   \MCmat\,\Delta\Mx^{vs} = -r_3 \,.
\]
From the first equation above,
\[
  \Delta\Mx^{vs} = -\MAmat^{-1}\,\Mr^{vs} - \MAmat^{-1}\,\MBmat\,\Delta x_3 \,.
\]
Plugging in the second equation gives
\[
   r_3 = \MCmat\,\MAmat^{-1}\,\Mr^{vs} + \MCmat\,\MAmat^{-1}\,\MBmat\,\Delta x_3 \,.
\]
Rearranging,
\[
  \Delta x_3 = x_3^{k+1} - x_3^k = \frac{-\MCmat\,\MAmat^{-1}\,\Mr^{vs} + r_3}{\MCmat\,\MAmat^{-1}\,\MBmat} \,.
\]
Using the above result,
\[
  \Delta\Mx^{vs} = -\MAmat^{-1}\,\Mr^{vs} - \MAmat^{-1}\,\MBmat\,\left(\frac{-\MCmat\,\MAmat^{-1}\,\Mr^{vs} + r_3}{\MCmat\,\MAmat^{-1}\,\MBmat}\right) \,.
\]
We therefore have to invert only a $2 \times 2 $ matrix.

\subsection{Tangent calculation: elastic}
We want to find the derivative of the stress with respect to the strain:
\Beq
   \Partial{\Bsig}{\Beps} = \Bone\otimes\Partial{p}{\Beps} + 
      \sqrt{\tfrac{2}{3}}\,\BnT\otimes\Partial{q}{\Beps} + 
      \sqrt{\tfrac{2}{3}}\,q\,\Partial{\BnT}{\Beps}  \,.
\Eeq
For the first term above,
\[
   \Partial{p}{\Beps} = p_0\,\exp\left[-\frac{\Ve^e_v - \Ve^e_{v0}}{\kappatilde}\right]\Partial{\beta}{\Beps}
      - p_0\,\frac{\beta}{\kappatilde}\,
        \exp\left[-\frac{\Ve^e_v - \Ve^e_{v0}}{\kappatilde}\right]\Partial{\Ve^e_v}{\Beps} 
     = p_0\,\exp\left[-\frac{\Ve^e_v - \Ve^e_{v0}}{\kappatilde}\right]\left(\Partial{\beta}{\Beps} -
            \frac{\beta}{\kappatilde}\,\Partial{\Ve^e_v}{\Beps} \right) \,.
\]
Now,
\[
   \Partial{\beta}{\Beps} = \frac{3\alpha}{\kappatilde}\,\Ve^e_s\,\Partial{\Ve^e_s}{\Beps} \,.
\]
Therefore, 
\[
  \Partial{p}{\Beps} = \frac{p_0}{\kappatilde}\,
      \exp\left[-\frac{\Ve^e_v - \Ve^e_{v0}}{\kappatilde}\right]\left(3\alpha\,\Ve^e_s\Partial{\Ve^e_s}{\Beps} -
            \beta\,\Partial{\Ve^e_v}{\Beps} \right) \,.
\]
We now have to figure out the other derivatives in the above expression.  First,
\[
  \Partial{\Ve^e_s}{\Beps} = \sqrt{\tfrac{2}{3}}\,\frac{1}{\sqrt{\BeT^e:\BeT^e}}\,\Partial{\BeT^e}{\Beps}:\BeT^e =
     \sqrt{\tfrac{2}{3}}\,\frac{1}{\Norm{\BeT^e}{}}\,
     \left(\Partial{\Beps^e}{\Beps} - \tfrac{1}{3}\Bone\otimes\Partial{\Ve^e_v}{\Beps} \right):\BeT^e\,.
\]
For the special situation where all the strain is elastic, $\Beps = \Beps^e$, and (see Wikipedia 
article on tensor derivatives)
\[
  \Partial{\Beps^e}{\Beps} = \Partial{\Beps}{\Beps} = \SfI^{(s)} \quad \Tand \quad
  \Partial{\Ve^e_v}{\Beps} = \Partial{\Ve_v}{\Beps} = \Bone \,.
\]
That gives us
\[
  \Partial{\Ve^e_s}{\Beps} = \sqrt{\tfrac{2}{3}}\,\frac{1}{\Norm{\BeT^e}{}}\,
     \left(\SfI^{(s)} - \tfrac{1}{3}\Bone\otimes\Bone \right):\BeT^e
   = \sqrt{\tfrac{2}{3}}\,\frac{1}{\Norm{\BeT^e}{}}\,
     \left[\BeT^e - \tfrac{1}{3}\Tr(\BeT^e)\Bone\right] \,.
\]
But $\Tr(\BeT^e)=0$ because this is the deviatoric part of the strain and we have
\[
  \boxed{
  \Partial{\Ve^e_s}{\Beps} = \sqrt{\tfrac{2}{3}}\,\frac{\BeT^e}{\Norm{\BeT^e}{}} = \sqrt{\tfrac{2}{3}}\,\BnT 
  }
  \quad \Tand \quad
  \boxed{
  \Partial{\Ve^e_v}{\Beps} = \Bone \,.
  }
\]
Using these, we get
\Beq
  \Partial{p}{\Beps} = \frac{p_0}{\kappatilde}\,
      \exp\left[-\frac{\Ve^e_v - \Ve^e_{v0}}{\kappatilde}\right]\left(\sqrt{6}\,\alpha\,\Ve^e_s\,\BnT -
            \beta\,\Bone \right) \,.
\Eeq
The derivative of $q$ with respect to $\Beps$ can be calculated in a similar way, i.e.,
\[
  \Partial{q}{\Beps} = 3\mu\,\Partial{\Ve^e_s}{\Beps} + 3\Ve^e_s\,\Partial{\mu}{\Beps}
   = 3\mu\,\Partial{\Ve^e_s}{\Beps} - 3\frac{p_0}{\kappatilde}\,\alpha\,\Ve^e_s\,
      \exp\left[-\frac{\Ve^e_v - \Ve^e_{v0}}{\kappatilde}\right]\,\Partial{\Ve^e_v}{\Beps} \,.
\]
Using the expressions in the boxes above, 
\Beq
  \Partial{q}{\Beps} = \sqrt{6}\,\mu\,\BnT - 3\frac{p_0}{\kappatilde}\,
      \exp\left[-\frac{\Ve^e_v - \Ve^e_{v0}}{\kappatilde}\right]\,\alpha\,\Ve^e_s\,\Bone \,.
\Eeq
Also,
\[
   \Partial{\BnT}{\Beps} = \sqrt{\tfrac{2}{3}}\,\left[\frac{1}{\Ve^e_s}\,\Partial{\BeT^e}{\Beps}
     - \frac{1}{(\Ve^e_s)^2}\,\BeT^e\otimes\Partial{\Ve^e_s}{\Beps}\right] \,.
\]
Using the previously derived expression, we have
\[
   \Partial{\BnT}{\Beps} = \sqrt{\tfrac{2}{3}}\,\frac{1}{\Ve^e_s}\,\left[
        \SfI^{(s)} - \tfrac{1}{3}\,\Bone\otimes\Bone
     - \sqrt{\tfrac{2}{3}}\,\frac{1}{\Ve^e_s}\,\frac{\BeT^e\otimes\BeT^e}{\Norm{\BeT^e}{}}\right] 
\]
or
\Beq
   \Partial{\BnT}{\Beps} = \sqrt{\tfrac{2}{3}}\,\frac{1}{\Ve^e_s}\,\left[
        \SfI^{(s)} - \tfrac{1}{3}\,\Bone\otimes\Bone - \BnT\otimes\BnT\right] \,.
\Eeq
Plugging the expressions for these derivatives in the original equation, we get
\[
  \Bal
   \Partial{\Bsig}{\Beps} & = \frac{p_0}{\kappatilde}\,
      \exp\left[-\frac{\Ve^e_v - \Ve^e_{v0}}{\kappatilde}\right]
      \left(\sqrt{6}\,\alpha\,\Ve^e_s\,\Bone\otimes\BnT - \beta\,\Bone\otimes\Bone \right) + 
      2\mu\,\BnT\otimes\BnT - \sqrt{6}\frac{p_0}{\kappatilde}\,
      \exp\left[-\frac{\Ve^e_v - \Ve^e_{v0}}{\kappatilde}\right]\,\alpha\,\Ve^e_s\,\BnT\otimes\Bone +\\
      & \qquad \qquad \tfrac{2}{3}\,\frac{q}{\Ve^e_s}\,\left[
        \SfI^{(s)} - \tfrac{1}{3}\,\Bone\otimes\Bone - \BnT\otimes\BnT\right] \,.
  \Eal
\]
Reorganizing,
\Beq
  \boxed{
  \Bal
   \Partial{\Bsig}{\Beps} & = \frac{\sqrt{6}\,p_0\,\alpha\,\Ve^e_s}{\kappatilde}\,
      \exp\left[-\frac{\Ve^e_v - \Ve^e_{v0}}{\kappatilde}\right](\Bone\otimes\Bn + \Bn\otimes\Bone) - 
      \left(\frac{p_0\beta}{\kappatilde}\, \exp\left[-\frac{\Ve^e_v - \Ve^e_{v0}}{\kappatilde}\right]
       +\tfrac{2}{9}\,\frac{q}{\Ve_s^e}\right) \Bone\otimes\Bone + \\
     & \qquad \qquad 2\left(\mu - \tfrac{1}{3}\,\frac{q}{\Ve^e_s}\right)\,\BnT\otimes\BnT 
           + \tfrac{2}{3}\,\frac{q}{\Ve^e_s}\,\SfI^{(s)}\,.
  \Eal
  }
\Eeq

\subsection{Tangent calculation: elastic-plastic}
From the previous section recall that 
\[
   \Partial{\Bsig}{\Beps} = \Bone\otimes\Partial{p}{\Beps} + 
      \sqrt{\tfrac{2}{3}}\,\BnT\otimes\Partial{q}{\Beps} + 
      \sqrt{\tfrac{2}{3}}\,q\,\Partial{\BnT}{\Beps}  
\]
where
\[
  \Bal
  \Partial{p}{\Beps} & = \frac{p_0}{\kappatilde}\,
      \exp\left[-\frac{\Ve^e_v - \Ve^e_{v0}}{\kappatilde}\right]\left(3\alpha\,\Ve^e_s\Partial{\Ve^e_s}{\Beps} -
            \beta\,\Partial{\Ve^e_v}{\Beps} \right) \,,\qquad
  \Partial{q}{\Beps}  = 3\mu\,\Partial{\Ve^e_s}{\Beps} - 3\frac{p_0}{\kappatilde}\,\alpha\,\Ve^e_s \,
      \exp\left[-\frac{\Ve^e_v - \Ve^e_{v0}}{\kappatilde}\right]\,\Partial{\Ve^e_v}{\Beps} \quad \Tand \\
   \Partial{\BnT}{\Beps} & = \sqrt{\tfrac{2}{3}}\,\left[\frac{1}{\Ve^e_s}\,\Partial{\BeT^e}{\Beps}
     - \frac{1}{(\Ve^e_s)^2}\,\BeT^e\otimes\Partial{\Ve^e_s}{\Beps}\right] \,.
  \Eal
\]
The total strain is equal to the elastic strain for the purely elastic case and the tangent is relatively
straightforward to calculate.  For the elastic-plastic case we have
\[
   \Beps^e_{n+1} = \Beps^e_\Trial - \Delta\gamma\left[\Partial{f}{\Bsig}\right]_{n+1} \,.
\]
Dropping the subscript $n+1$ for convenience, we have
\[
   \Partial{\Beps^e}{\Beps} = \Partial{\Beps^e_\Trial}{\Beps} 
     - \Partial{f}{\Bsig}\otimes\Partial{\Delta\gamma}{\Beps}
     - \Delta\gamma\,\Partial{}{\Beps}\left[\Partial{f}{\Bsig}\right]
     = \SfI^{(s)}
     - \left[\frac{2p-p_c}{3}\,\Bone + \sqrt{\tfrac{3}{2}}\,\frac{2q}{M^2}\,\BnT\right]
       \otimes\Partial{\Delta\gamma}{\Beps}
     - \Delta\gamma\,\Partial{}{\Beps}
     \left[\frac{2p-p_c}{3}\,\Bone + \sqrt{\tfrac{3}{2}}\,\frac{2q}{M^2}\,\BnT\right] \,.
\]

\section{Caveats}
The Cam-Clay implementation in Vaango behaves reasonably for moderate strains but is
known to fail to converge for high-rate applications that involve very large plastic
strains.

