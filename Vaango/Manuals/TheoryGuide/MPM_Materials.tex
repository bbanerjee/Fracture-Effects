\chapter{MPM Material Models}
In this chapter we discuss some general features of the MPM material
models.  Individual models are complex and are discussed in separate
chapters.  Notation and definitions that are used frequently are
also elaborated upon here.

\section{Notation and definitions}
A primary assumption made in many of the material models in Vaango is that
stresses and (moderate) strains can be additively decomposed into volumetric and deviatoric parts.

\subsection{Volumetric-deviatoric decomposition}
The volumetric-deviatoric decomposition of stress ($\Bsig$) is expressed as
\Beq
  \Bsig = p\BI + \Bs
\Eeq
where the mean stress ($p$) and the deviatoric stress ($\Bs$) are given by
\Beq
  p = \Third\Tr(\Bsig) = \Third\Bsig:\BI = \tfrac{1}{3}(\sigma_{11} + \sigma_{22} + \sigma_{33})
  \quad \Tand \quad
  \Bs = \Bsig - p\BI = 
    \begin{bmatrix} 
       \sigma_{11} - p & \sigma_{12} & \sigma_{13} \\
       \sigma_{12} & \sigma_{22} - p & \sigma_{23} \\
       \sigma_{13} & \sigma_{23} & \sigma_{33} - p
    \end{bmatrix} \,.
\Eeq

Similarly, the volumeric-deviatoric split of the strain ($\BVeps$) is expressed as
\Beq
  \BVeps = \Third \Veps_v \BI + \Beta
\Eeq
where the volumetric strain ($\Veps_v$) and the deviatoric strain ($\Beta$) are defined as
\Beq
  \Veps_v = \Tr(\BVeps) = \BVeps:\BI = \Veps_{11} + \Veps_{22} + \Veps_{33}
  \quad \Tand \quad
  \Beta = \BVeps - \Third \Veps_v\BI = 
    \begin{bmatrix} 
       \Veps_{11} - \Third \Veps_v & \Veps_{12} & \Veps_{13} \\
       \Veps_{12} & \Veps_{22} - \Third \Veps_v & \Veps_{23} \\
       \Veps_{13} & \Veps_{23} & \Veps_{33} - \Third \Veps_v
    \end{bmatrix} \,.
\Eeq

\subsection{Stress invariants}
The principal invariants and principal deviatoric invariants of the stress are used in 
several models.  Frequently used invariants are:
\Beq
  \Bal
    I_1 & = \Tr(\Bsig) = \sigma_{11} + \sigma_{22} + \sigma_{33} \\
    I_2 & = \Half\left[\Tr(\Bsig)^2 - \Tr(\Bsig^2)\right]
          = \sigma_{11} \sigma_{22} + \sigma_{22} \sigma_{33} + \sigma_{33} \sigma_{11}
            -  (\sigma_{12}^2 + \sigma_{23}^2 + \sigma_{13}^2) \\
    I_3 & = \det(\Bsig)
          = \sigma_{11}\sigma_{22}\sigma_{33} + 2\sigma_{12}\sigma_{23}\sigma_{13}
            - \sigma_{12}^2 \sigma_{33} - \sigma_{23}^2 \sigma_{11}
            - \sigma_{13}^2 \sigma_{22} \\
    J_2 & = \Half \Bs:\Bs 
          = \tfrac{1}{6}\left[(\sigma_{11} - \sigma_{22})^2 + 
             (\sigma_{22} - \sigma_{33})^2 + (\sigma_{33} - \sigma_{11})^2\right] +
             (\sigma_{12}^2 + \sigma_{23}^2 + \sigma_{31}^2) \\
    J_3 & = \det(\Bs) = \tfrac{2}{27} I_1^3 - \Third I_1 I_2 + I_3
  \Eal
\Eeq
Alternatives to $I_1$, $J_2$ and $j_3$ are $p$, $q$, and $\theta$, defined as
\Beq
  p := \frac{1}{3} I_1~,~~ q := \sqrt{3 J_2}~,~~
  \cos3\theta := \left(\cfrac{r}{q}\right)^3 = \frac{3\sqrt{3}}{2} \frac{J_3}{J_2^{3/2}} ~,~~
  r^3 = \frac{27}{2} J_3\,.
\Eeq
A geometric accurate view of the stress state and yield surfaces is obtained if 
the isomorphic cylindrical coordinates $z$, $\rho$, and $\theta$ are used instead, where
\Beq
  z := \frac{I_1}{\sqrt{3}} = \sqrt{3} p ~,~~
  \rho := \sqrt{2 J_2} = \sqrt{\frac{2}{3}} q ~,~~
  \cos3\theta := \frac{3\sqrt{3}}{2} \frac{J_3}{J_2^{3/2}} \,.
\Eeq

\subsection{Effective stress and strain}
The effective stress and strain (sometimes also referred to as the shear stress and shear strain
in the code) are defined such that the product is equal to the plastic work done.  These
measures are strictly applicable only to $J_2$ plasticity models but have also been used elsewhere.

The effective stress is defined as
\Beq \label{eq:eff_stress}
  \sigma_\Teff = q = \sqrt{3J_2} = \sqrt{\tfrac{3}{2} \Bs:\Bs}
    = \sqrt{\tfrac{1}{2}\left[(\sigma_{11} - \sigma_{22})^2 + 
             (\sigma_{22} - \sigma_{33})^2 + (\sigma_{33} - \sigma_{11})^2\right] +
             3 (\sigma_{12}^2 + \sigma_{23}^2 + \sigma_{31}^2) } \,.
\Eeq

The effective strain is defined as
\Beq \label{eq:eff_strain}
  \Veps_\Teff = \sqrt{\tfrac{2}{3} \Beta:\Beta}
\Eeq
so that
\Beq
  \sigma_\Teff \Veps_\Teff = \sqrt{(\Bs:\Bs) (\Beta:\Beta)} \,.
\Eeq
From the definition of $\Beta$ we see that
\Beq
  \Bal
  \Beta:\Beta & = \BVeps:\BVeps - \tfrac{2}{3} \Tr(\BVeps)\BI:\BVeps + \tfrac{1}{9}\left[\Tr(\BVeps)\right]^2 \BI:\BI
     = \BVeps:\BVeps - \tfrac{2}{3} \left[\Tr(\BVeps)\right]^2 + \tfrac{1}{3}\left[\Tr(\BVeps)\right]^2
     = \BVeps:\BVeps - \Third \left[\Tr(\BVeps)\right]^2 \\
     &= \Veps_{11}^2 + \Veps_{22}^2 + \Veps_{33}^2 + 2\Veps_{12}^2 + 2\Veps_{23}^2 + 2\Veps_{13}^2
       - \Third\left[ \Veps_{11}^2 + \Veps_{22}^2 + \Veps_{23}^2 + 2\Veps_{11}\Veps_{22} + 
       2\Veps_{22}\Veps_{33} + 2\Veps_{11}\Veps_{33} \right] \\
     &= \Third\left[(\Veps_{11} - \Veps_{22})^2 + (\Veps_{22} - \Veps_{33})^2 + (\Veps_{33} - \Veps_{11})^2 \right] + 2(\Veps_{12}^2 + \Veps_{23}^2 + \Veps_{13}^2)
  \Eal
\Eeq
Therefore,
\Beq
  \Veps_\Teff = \sqrt{
      \tfrac{2}{3}\left[\Third\left[(\Veps_{11} - \Veps_{22})^2 + (\Veps_{22} - \Veps_{33})^2 + (\Veps_{33} - \Veps_{11})^2\right] + 2(\Veps_{12}^2 + \Veps_{23}^2 + \Veps_{13}^2)\right]}
\Eeq
For volume preserving plastic deformations, $\Tr(\BVeps) = 0$, and we have
\Beq
  \Veps_\Teff = \sqrt{
     \tfrac{2}{3}\left(\Veps_{11}^2 + \Veps_{22}^2 + \Veps_{33}^2\right) + 
     \tfrac{4}{3}\left(\Veps_{12}^2 + \Veps_{23}^2 + \Veps_{13}^2\right)} 
     = \sqrt{
     \tfrac{2}{3}\left(\Veps_{11}^2 + \Veps_{22}^2 + \Veps_{33}^2\right) + 
     \tfrac{1}{3}\left(\gamma_{12}^2 + \gamma_{23}^2 + \gamma_{13}^2\right)} \,.
\Eeq

\subsection{Equivalent plastic strain}
For models where an effective plastic strain is computed, the definition is
\Beq
  \Veps_p^\Teff(t) = \int_0^t \sqrt{\tfrac{2}{3} \dot{\BVeps}_p(\tau):\dot{\BVeps}_p(\tau)} d\tau
\Eeq
where $\dot{\BVeps}_p(t)$ is the plastic strain rate.

\subsection{Velocity gradient, rate-of-deformation, deformation gradient}
The velocity gradient is represented by $\BlT$ and the defromation gradient by $\BF$.  The
rate-of-deformation is
\Beq
  \BdT = \Half(\BlT + \BlT^T) \,.
\Eeq


\section{Material models available in Vaango}
The MPM material models implemented in \Vaango were originally
chosen for the following purposes:
\begin{itemize}
  \item To verify the \Textsfc{accuracy} of the material point method (MPM)
        and to validate the \Textsfc{coupling} between the computational fluid 
        dynamics code (ICE) and MPM.
  \item To model the elastic-plastic deformation of \Textsfc{metals} 
        and the consequent damage in the regimes of 
        both high and low strain rates and high and low temperatures.
  \item To model \Textsfc{polymer bonded explosives} and \Textsfc{polymers}
        under various strain rates and temperatures.
  \item To model the deformation of \Textsfc{biological tissues}.
  \item To model the explosive deformation of \Textsfc{rocks and soils}.
\end{itemize}

As of \Vaango \version, the material models that have been implemented are:
\begin{enumerate}
  \item Rigid material
  \item Ideal gas material
  \item Water material
  \item Membrane material
  \item Programmed burn material
  \item Tabular equation of state
  \item Murnaghan equation of state
  \item JWL++ equation of state
  \item Hypoelastic material
  \item Hypoelastic material with manufactured solutions
  \item Hypoelastic material implementation in FORTRAN
  \item Polar-orthotropic hypoelastic material
  \item Compressible neo-Hookean hyperelastic material
  \item Compressible neo-Hookean hyperelastic material with manufactured solutions
  \item Compressible neo-Hookean hyperelastic material with damage
  \item Unified explicit/implicit compressible Neo-Hookean hyperelastic material with damage
  \item Compressible Neo-Hookean hyperelastic-J$_2$ plastic material with damage
  \item Compressible Mooney-Rivlin hyperelastic material
  \item Compressible neo-Hookean material for shells
  \item Transversely isotropic hyperelastic material
  \item The $p$-$\alpha$ model for porous materials
  \item Viscoelastic material written in FORTRAN for damping
  \item Simplified Maxwell viscoelastic material
  \item Visco-SCRAM model for viscoelastic materials with cracks
  \item Visco-SCRAM hotspot model
  \item Tabular plasticity model
  \item Tabular plasticity model with cap
  \item Hypoelastic J$_2$ plasticity model with damage for high-rates
  \item Viscoplastic J$_2$ plasticity model
  \item Mohr-Coulomb material
  \item Drucker-Prager material with deformation induced elastic anisotropy
  \item CAM-Clay model for soils
  \item Nonlocal Drucker-Prager material
  \item Arenisca material for rocks and soils
  \item Arenisca3 material for rocks and soils
  \item Arena material for partially saturated soils
  \item Arena-mixture material for mixes of partially saturated sand and clay
  \item Brannon's soil model
  \item Soil foam model
\end{enumerate}
A small subset of these models also have implementations that can be used with
Implicit \MPM.

Some of these models can work with multiple sub-models such as elasticity model
or yield condition.  As of \Vaango \version, the implemented sub-models are:
\begin{enumerate}
  \item Equations of state:
  \begin{enumerate}
    \item Pressure model for Air
    \item Pressure model for Borja's CAMClay
    \item Pressure model for Granite
    \item Pressure model for hyperelastic materials
    \item Pressure model for Hypoelastic materials
    \item Pressure model for Mie-Gruneisen equation of state
    \item Mie-Gruneisen energy-based equation of state for pressure
    \item Pressure model for Water
  \end{enumerate}
  \item Shear modulus models:
  \begin{enumerate}
    \item Constant shear modulus model
    \item Shear modulus model for Borja's CAMClay
    \item Shear modulus model by Nadal and LePoac
    \item Mechanical Threshold Stress shear modulus model
    \item Preston-Tonks-Wallace shear modulus model
    \item Steinberg-Guinan shear modulus model
  \end{enumerate}
  \item Combined elastic modulus models:
  \begin{enumerate}
    \item Constant elastic modulus model
    \item Tabular elastic modulus model
    \item Neural net elastic modulus model
    \item Arena elastic modulus model
    \item Arena mixture elastic modulus model
    \item Arenisca elastic modulus model
  \end{enumerate}
  \item Yield condition models:
  \begin{enumerate}
    \item Yield condition for Arena model
    \item Yield condition for Arena mixture model
    \item Yield condition for Arenisca3 model
    \item Yield condition for CamClay model
    \item Yield condition for Gurson model
    \item Yield condition for Tabular plasticity with Cap
    \item Yield condition for Tabular plasticity with
    \item Yield condition for vonMises J$_2$ plasticity
    \item Classic Mohr-Coulomb model
    \item Sheng's Mohr-Coulomb model
  \end{enumerate}
  \item Plastic flow stress models 
  \begin{enumerate}
    \item Isotropic hardening plastic flow model
    \item Johnson-Cook plastic flow model
    \item Mechanical Threshold Stress plastic flow model
    \item Preston-Tonks-Wallace plastic flow model
    \item Steinberg-Guinan plastic flow model
    \item SuvicI viscoplastic flow model
    \item Zerilli-Armstrong metal plastic flow model
    \item Zerilli-Armstrong polymer plastic flow model
  \end{enumerate}
  \item Plastic internal variable models:
  \begin{enumerate}
    \item Arena internal variable model
    \item Borja internal variable model
    \item Brannan's soil model internal variable model
    \item Tabular plasticity with cap internal variable model
  \end{enumerate}
  \item Kinematic hardening models:
  \begin{enumerate}
    \item Prager kinematic hardening model
    \item Armstrong-Frederick kinematic hardening model
    \item Arena kinematic hardening model
  \end{enumerate}
  \item Damage models
  \begin{enumerate}
    \item Becker's damage model
    \item Drucker and Becker combined damage model
    \item Drucker loss of stability model
    \item Johnson-Cook damage model
    \item Hancock-MacKenzie damage model
  \end{enumerate}
  \item Melting model
  \begin{enumerate}
    \item Constant melting temperature model
    \item Linear melting temperature model
    \item BPS melting model
    \item Steinberg-Guinan melting temperature model
  \end{enumerate}
  \item Specific heat model 
  \begin{enumerate}
    \item Constant specific heat
    \item Cubic specific heat model
    \item Copper specific heat model
    \item Steel specific heat model
  \end{enumerate}
\end{enumerate}

\subsection{Material models for the validation of MPM}
The models that have been implemented for the verification of MPM
are:
\begin{itemize}
   \item Isotropic hypoelastic model using the Jaumann rate of 
         stress.
     \begin{enumerate}
        \item  MPM predictions have been compared with exact results 
               for thick cylinders under internal pressure for small
               strains, three-point beam bending, etc.
        \item  MPM predictions for the strain/stress contours for
               a set of disks in contact have been found to match
               experimental results.
     \end{enumerate}
   \item Isotropic hyperelastic material models for Mooney-Rivlin
         rubber and a modified compressible Neo-Hookean material. 
         Isotropic strain hardening plasticity for the Neo-Hookean
         material.
     \begin{enumerate}
        \item  A billet compression problem has been simulated using
               MPM and the results have been found to closely 
               match finite element simulations.
        \item  MPM simulations for a thick cylinder under internal
               pressure with plastic deformation (perfect plasticity)
               compare well with the exact solution.
     \end{enumerate}
\end{itemize}

\subsection{Material models for metals}
The material models for metals are used to determine
the state of stress  for an applied deformation
rate and deformation gradient at each material point.  The strain 
rates can vary from $10^{-3}$/s to $10^6$/s and temperatures in 
the container can vary from 250 K to 1000 K.  For example, in
an explosion inside a container, plasticity dominates
the deformation of the container during the expansion of the 
explosive gases inside.  At high strain rates the volumetric
response of the container is best obtained using an equation of 
state.  After the plastic strain in the container
has reached a threshold value a damage/erosion model is required to
rupture the container.

Two plasticity models with strain rate and temperature dependency 
are the Johnson-Cook and the Mechanical Threshold Stress (MTS) 
models.  The volumetric response is calculated using a modified
Mie-Gruneisen equation of state.  A damage model that ties in well 
with the Johnson-Cook plasticity model is the Johnson-Cook damage 
model.  The erosion algorithm either removes the contribution
of the mass of the material point or forces the material point
to undergo no tension or shear under further loading.

The stress update at each material point is performed using either
of the two methods discussed below.
\begin{itemize}
   \item Isotropic Hypoelastic-plastic material model using an 
         additive decomposition of the rate of deformation tensor.
     \begin{enumerate}
        \item The rate of deformation tensor a material point
              is calculated using the grid velocities.
        \item An incremental update of the left stretch and the
              rate of rotation tensors is calculated.
        \item The stress and the rate of deformation are rotated
              into the material coordinates.
        \item A trial elastic deviatoric stress state is calculated.
        \item The flow stress is calculated using the plasticity
              model and compared with the vonMises yield condition.
        \item If the stress state is elastic, an update of the 
              stress is computed using the Mie-Gruneisen equation
              of state or the isotropic hypoelastic constitutive
              equation.
        \item If the stress state is plastic, all the strain rate 
              is considered to the plastic and an elastic correction
              along with a radial return step move the stress state
              to the yield surface.  The hydrostatic part of the 
              stress is calculated using the equation of state or
              the hypoelastic constitutive equation.
        \item A scalar damage parameter is calculated and used
              to determine whether material points are to be eroded
              or not.
        \item Stresses and deformation rates are rotated back to the
              laboratory coordinates.
     \end{enumerate}
   \item Isotropic Hyperelastic-plastic material model using a 
         multiplicative decomposition of the deformation gradient.
     \begin{enumerate}
        \item The velocity gradient at a material point
              is calculated using the grid velocities.
        \item An incremental update of the deformation gradient and
              the left Cauchy-Green tensor is calculated.
        \item A trial elastic deviatoric stress state is calculated
              assuming a compressible Neo-Hookean elastic model.
        \item The flow stress is calculated using the plasticity
              model and compared with the vonMises yield condition.
        \item If the stress state is elastic, an update of the 
              stress is computed using the Mie-Gruneisen equation
              of state or the compressible Neo-Hookean constitutive
              equation.
        \item If the stress state is plastic, all the strain rate 
              is considered to the plastic and an elastic correction
              along with a radial return step move the stress state
              to the yield surface.  The hydrostatic part of the 
              stress state is calculated using the Mie-Gruneisen
              equation of state or the Neo-Hookean model.
        \item A scalar damage parameter is calculated and used
              to determine whether material points are to be eroded
              or not.
     \end{enumerate}
\end{itemize}

\subsection{Material models for the explosive}
The explosive is modeled using the \Textsfc{ViscoSCRAM} constitutive 
model.  Since large deformations or strains are not expected in 
the explosive, a small strain formulation has been implemented into 
\Vaango.  The model consists of five generalized Maxwell elements
arranged in parallel, crack growth, friction at the crack 
interfaces and heating due to friction and reactions at the 
crack surfaces.  The implementation has been verified with 
experimental data and found to be accurate.


