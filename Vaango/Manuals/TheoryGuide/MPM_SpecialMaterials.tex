\chapter{Special material models}
\Vaango contains a few material models that are designed for special problems. These are
discussed in this chapter.

\section{Rigid material}
\Textmag{Applicable to:} \Textsfc{explicit} and \Textsfc{implicit} \MPM

This material model assumes that 
\Beq
  \Bsig(\BF, t) = \Bzero \quad \Tand \quad \BF(t) = \BI \,.
\Eeq
The model is a rough approximation of a rigid body as long as there is no
contact between objects.  Upon contact, the model should ideally transition into the
form
\Beq
  \Bsig(\BF, t) = \infty \quad \Tand \quad \BF(t) = \BI \,.
\Eeq
This situation is approximated using the \Textsfc{specified body contact}
algorithm which is applicable only in certain directions.  Rigid materials were 
designed to act as rigid surfaces against which deformable objects
could be compressed.  The specified body contat algorithm can simulate the 
interaction of a single ``master'' rigid body with deformable objects.

\section{Ideal gas material}
\Textmag{Applicable to:} \Textsfc{explicit} \MPM only

The ideal gas material assumes that the stress at a particle is
\Beq
  \Bsig(\BF, t) = \begin{cases}
                    \pbar(\BF)\,\BI & \quad\text{for}\quad p \ge 0 \\
                    0       & \quad\text{for}\quad p < 0
                  \end{cases}
\Eeq
where $\pbar = -p$, and the pressure $p$ is computed with an isentropic ideal 
gas equation of state:
\Beq
  p = p_{\Tref}\left[\exp(\gamma\,\bar{\Veps_v}) - 1 \right]
  ~;~~ \bar{\Veps_v}  
       = \begin{cases}
           -\ln\left(J\right) & \quad \text{for} \quad J < 1 \\
           0 & \quad \text{for} \quad J \ge 1
         \end{cases}
\Eeq
where $J = \det(\BF)$.

A rate of change of temperature ($T$) can also be computed by the model:
\Beq
  \Deriv{T}{t} = \frac{1}{\Delta t}\left(1 - \frac{J_{n+1}}{J_n}\right) \left(\frac{p}{\rho C_v}\right) 
\Eeq
where $J_{n+1} = J(t_{n+1}$, $J_n = J(t_n)$, $\rho$ is the mass density,
and $C_v$ is the constant volume specific heat.

\section{Water material}
\Textmag{Applicable to:} \Textsfc{explicit} \MPM only

This material models water~\cite{water_model_ref}, and assumes that the stress is given by
\Beq
  \Bsig(\BF, t) = \pbar(\BF) \BI + 2\mu\,\Beta.
\Eeq
where $\mu$ is a shear viscosity, $\BdT$ is the symmetric part of the velocity gradient,
\Beq
  \pbar = -p \quad \Tand \quad \Beta = \BdT - \Third \Tr(\BdT) \BI \,.
\Eeq
The pressure is given by:
\begin{equation}
  p = \kappa\left[J^{-\gamma} - 1\right] ~,~~ J = \det(\BF)
\end{equation}
where $\kappa$ the bulk modulus and $\gamma$ is a model parameter.  
It has not been validated, but gives qualitatively reasonable behavior.

\section{Murnaghan material}
\Textmag{Applicable to:} \Textsfc{explicit} \MPM only

This material is based on the equation of state proposed in ~\cite{Murnaghan1944}.
The stress is given by
\Beq
  \Bsig(\BF, t) = \pbar(\BF) \BI + 2\mu\,\Beta.
\Eeq
where $\mu$ is a shear viscosity, $\BdT$ is the symmetric part of the velocity gradient,
\Beq
  \pbar = -p \quad \Tand \quad \Beta = \BdT - \Third \Tr(\BdT) \BI \,.
\Eeq
The pressure is given by:
\begin{equation}
  p = \frac{\kappa}{\kappa'}\left[J^{-\kappa'} - 1\right] ~,~~ J = \det(\BF)
\end{equation}
where $\kappa$ the initial bulk modulus and $\kappa' = d\kappa/dp$ is a constant.

\section{JWL++ material}
\Textmag{Applicable to:} \Textsfc{explicit} \MPM only

The \Textsfc{JWL++} material is a combination of the Murnaghan and JWL models
along with a burn algorithm to convert from one to the other~\cite{JWL2000}. 
A small viscous component is added to the JWL model to stabilize behavior.

The stress is given by
\Beq
  \Bsig(\BF, t) = \pbar(\BF) \BI + 2\mu\,\Beta.
\Eeq
where $\mu$ is a shear viscosity, $\BdT$ is the symmetric part of the velocity gradient,
\Beq
  \pbar = -p \quad \Tand \quad \Beta = \BdT - \Third \Tr(\BdT) \BI \,.
\Eeq

The burn rate is computed as
\Beq
  \dot{f} = (1 - f) G \,p^b
\Eeq
where $f$ is the volume fraction of the reactant, $G, b$ are fit parameters, and
$p$ is the pressure, computed using
\Beq
  p = (1 - f) p_{\text{m}} + f p_{\text{jwl}} \,.
\Eeq

The Murnaghan pressure ($p_{\text{m}}$) is given by:
\begin{equation}
  p_{\text{m}} = \frac{1}{nK}\left[J^{-n} - 1\right] ~,~~ J = \det(\BF)
\end{equation}
where $K = 1/\kappa$, $\kappa$ is the initial bulk modulus, and $n = \kappa' = d\kappa/dp$ 
is a constant.

The JWL pressure ($p_{\text{jwl}}$) is given by
\Beq
  p_{\text{jwl}} = A \exp(-R_1 J) + B \exp(-R_2 J) + C J^{-(1+\omega)}
\Eeq
where $A$, $B$, $C$, $\omega$ are fit parameters, $R_1$, $R_2$ are fit rate parameters, 
and $J = \det\BF$.


