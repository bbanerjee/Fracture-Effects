Recall from equation~\eqref{eq:discrete_MPM_equations} that the \MPM discretized momentum
equations can be written as a semi-algebraic system
\Beq \label{eq:discrete_MPM_equations_1}
  \sum_h M_{gh} \dot{\Bv}_h = \Bf_g^{\Text} -\Bf_g^{\Tint} + \Bf_g^{\Tbody} 
      ~;~~ g = 1 \dots N_g
\Eeq
where the mass matrix ($\BM$), the internal force vector ($\Bf_g^{\Tint}$), 
the body force vector ($\Bf_g^{\Tbody}$), and the  external force vector ($\Bf_g^{\Text}$)
at grid node $g$ as
\Beq \label{eq:fext_fint_def}
  \Bal
   &M_{gh} := \sum_p \rho_p\IntOmegap Y_p(\Bx)~S_g(\Bx)~S_h(\Bx)~d\Omega \\
   &\Bf_g^{\Tint} := \sum_p V_p \Bsig_p \cdot \Av{\Grad{S_{gp}}} \\
   &\Bf_g^{\Tbody} := \sum_p m_p\Bb_p~\Av{S_{gp}} \\
   &\Bf_g^{\Text} := \IntGammat \Bart(\Bx)~S_g(\Bx)~d\Gamma \,.
  \Eal
\Eeq

While the \MPM background grid is reset after each time increment, \MPM does not
require it to be reset during each iteration of an implicit integration process.
Therefore, during a time step, we can carry a grid displacement variable $\Bu$ that 
can be used to compute grid accelerations $\Ba_g = \dot{\Bv}_g$ and discarded at
the end of a time step.

Let us express \eqref{eq:discrete_MPM_equations_1} in matrix form as
\Beq
  \BM_g \BaT_g = \BfT_g^{\Text} - \BfT_g^{\Tint} + \BfT_g^{\Tbody} 
\Eeq
Define the residual as
\Beq \label{eq:mpm_residual}
  \Br(\BuT_g^{n+1}, t_{n+1}) = \BM_g \BaT_g^{n+1} - \BfT^{\Text}(\BuT_g^{n+1}, t_{n+1}) + 
     \BfT^{\Tint}(\BuT_g^{n+1}, t_{n+1}) - \BfT^{\Tbody}(\BuT_g^{n+1}, t_{n+1}) = \Bzero .
\Eeq
where the superscipt $n+1$ indicates quantities at time $t_{n+1}$ and $\BuT_g$ is the 
$N_g \times 3$ matrix of grid
node displacements.  We use a Newmark-$\beta$ method to integrate the acceleration:
\Beq \label{eq:newmark_beta}
  \Bal
    \BuT_g^{n+1} & = \BuT^\star + \beta  \BaT_g^{n+1} (\Delta t)^2 \\
    \BvT_g^{n+1} & = \BvT^\star + \gamma \BaT_g^{n+1} \Delta t 
  \Eal
\Eeq
where
\Beq
  \Bal
    \BuT^\star & = \BuT_g^n + \BvT_g^n \Delta t + \Half (1 - 2\beta) \BaT_g^n (\Delta t)^2 \\
    \BvT^\star & = \BvT_g^n + (1 - \gamma) \BaT_g^n \Delta t \,.
  \Eal
\Eeq

\subsection{Newton's method}
In the \Vaango implementation, the residual is expressed in terms of the displacement.  
However, we can implement the implicit method in terms of the velocity with no loss of
generality.  Thus using \eqref{eq:mpm_residual} and \eqref{eq:newmark_beta}, we have
\Beq \label{eq:mpm_residual_1}
  \Br(\BvT_g^{n+1}, t_{n+1}) = \frac{1}{\gamma \Delta t} \BM_g (\BvT_g^{n+1} - \BvT^\star) - 
      \BfT^{\Text}(\BvT_g^{n+1}, t_{n+1}) + 
     \BfT^{\Tint}(\BvT_g^{n+1}, t_{n+1}) - \BfT^{\Tbody}(\BvT_g^{n+1}, t_{n+1}) = \Bzero .
\Eeq
The problem then reduces to finding the solution $\BvT_g^{n+1}$ of the nonlinear system of equations
\eqref{eq:mpm_residual_1}. Newton's method is used in \Vaango with the starting 
value of $\BvT_g^{n+1} = \BvT_g^\star$.  Dropping the subscript $g$ temporarily for convenience,
and denoting the current Newton iteration by the subscript $k$, we can linearize the residual at 
$\BvT_k^{n+1}$ using a Taylor expansion:
\Beq
  \Bzero = \Br(\BvT_{k+1}^{n+1}, t_{n+1}) = 
     \Br(\BvT_k^{n+1}, t_{n+1}) + \Partial{\Br(\BvT_k^{n+1}, t_{n+1})}{\BvT}(\BvT^{n+1}_{k+1} - \BvT^{n+1}_k) \,.
\Eeq
Rearranging the above equation,
\Beq \label{eq:Newton_step}
  \Delta\BvT = \BvT^{n+1}_{k+1} - \BvT^{n+1}_k = 
     -\left[\Partial{\Br(\BvT_k^{n+1}, t_{n+1})}{\BvT}\right]^{-1}
     \Br(\BvT_k^{n+1}, t_{n+1})  = -\BK^{-1} \Br(\BvT_k^{n+1}, t_{n+1})\,.
\Eeq
This iterative process is continued until $\Delta \BvT$ is smaller than a given tolerance.
The tangent matrix $\BK$, of size $N_g \times N_g$, is 
\Beq
  \BK = \Partial{\Br(\BvT_k^{n+1}, t_{n+1})}{\BvT} \,.
\Eeq
This matrix is decomposed and evaluated separately for the internal and external forces, i.e., 
\Beq
  \BK = \Partial{\Br(\BvT_k^{n+1}, t_{n+1})}{\BvT} 
      = \frac{1}{\gamma \Delta t} \BM_g  - \Partial{}{\BvT}\left[\BfT^{\Text}(\BvT_g^{n+1}, t_{n+1})\right] + 
        \Partial{}{\BvT}\left[\BfT^{\Tint}(\BvT_g^{n+1}, t_{n+1})\right] - 
        \Partial{}{\BvT}\left[\BfT^{\Tbody}(\BvT_g^{n+1}, t_{n+1})\right] \,.
\Eeq
Alternatively,
\Beq
  \BK = \frac{1}{\gamma \Delta t} \BM_g  - \BK^{\Text}(\BvT_g^{n+1}, t_{n+1}) +  
         \BK^{\Tint}(\BvT_g^{n+1}, t_{n+1}) -  \BK^{\Tbody}(\BvT_g^{n+1}, t_{n+1}) \,.
\Eeq

\subsection{Tangent stiffness matrix}
The contribution to the tangent matrix ($\BK$) from the internal forces is called the
\Textsfc{tangent stiffness matrix} ($\BK^\Tint$).  Since an updated Lagrangian formulation is used
in \MPM, we can compute the tangent stiffness using the configuration at time $t_n$ as
the reference configuration.

Recall from \eqref{eq:fext_fint_def} that for explicit \MPM we used
\Beq \label{eq:f_g_int_0}
  \Bf_g^{\Tint} = \sum_p V_p \Bsig_p \cdot \Av{\Grad{S_{gp}}}  
  \quad \text{where} \quad
  \Av{S_{gp}} := \frac{1}{V_p} \IntOmegap Y_p(\Bx)~S_g(\Bx)~d\Omega \,.
\Eeq
For the computation of the tangent matrix, it is preferable to start from the
weak form of the momentum equation \eqref{eq:weak_form}:
\Beq \label{eq:f_int_weak}
  I = \IntOmega \Bsig:\Grad{\Bw}~d\Omega  
\Eeq
which leads to integral form of equation \eqref{eq:f_g_int_0} (see \eqref{eq:integral_weak_form}):
\Beq  \label{eq:f_g_int}
  \Bf_g^{\Tint} = \sum_p\IntOmegap Y_p(\Bx)\Bsig_p \cdot \Grad{S_g}~d\Omega \,.
\Eeq
Also, since we are typically working with rates of stress in the constitutive models, it
is preferable to express all quantities in terms of stress rates that are objective.
It is easier to work with the Lagrangian PK-1 stress ($\BP$) at the beginning of the timestep
rather than the spatial Cauchy stress ($\Bsig$).

To convert from the spatial description \eqref{eq:f_int_weak} to a Lagrangian material 
description, observe that
\Beq
  \underset{n+1}{\nabla}{\Bw} = \Partial{\Bw}{\Bx^{n+1}} = \Partial{\Bw}{\Bx^n} \cdot \Partial{\Bx^n}{\Bx^{n+1}} 
    = \Partial{\Bw}{\Bx^n} \cdot (\Delta\BF_n^{n+1})^{-1}
    = \Partial{\Bw}{\Bx^n} \cdot \frac{\Delta\BF_c^T}{J_n^{n+1}}
    = \underset{n}{\nabla} \Bw \cdot \frac{\Delta\BF_c^T}{J_n^{n+1}}
\Eeq
where, with $\BF$ as the deformation gradient, 
\Beq
  \BF_{n+1} = \Delta\BF_n^{n+1} \BF_n~,~~ J_n^{n+1} = \det(\Delta\BF_n^{n+1}) ~,~~
  \Delta\BF_c = \text{cofactor}(\Delta\BF_n^{n+1}) \,.
\Eeq
Therefore,
\Beq
  \Bal 
   I & = \int_{\Omega^{n+1}} \Bsig^{n+1} : \underset{n+1}{\nabla}{\Bw}~d\Omega^{n+1}
       = \int_{\Omega^n} \Bsig^{n+1} : \underset{n+1}{\nabla}{\Bw}~J_n^{n+1}\,d\Omega^n \\
     & = \int_{\Omega^n} \Bsig^{n+1} : \left(\underset{n}{\nabla}{\Bw}\cdot \Delta\BF_c^T\right)\,d\Omega^n 
       = \int_{\Omega^n} \left(\Bsig^{n+1} \cdot \Delta\BF_c\right) : \underset{n}{\nabla}{\Bw}\,d\Omega^n \\
     & = \int_{\Omega^n} \BP^n : \underset{n}{\nabla}{\Bw}\,d\Omega^n 
  \Eal
\Eeq
where $\BP^n$ is the first Piola-Kirchhof stress.  Following the same process as used to derive
\eqref{eq:integral_weak_form}, we get
\Beq  \label{eq:f_g_int_lagrangian}
  \Bf_g^{\Tint} = \sum_p\int_{\Omega_p^n} Y_p(\Bx^n)\BP_p^n \cdot \underset{n}{\nabla}{S_g}~d\Omega^n \,.
\Eeq
Taking the material time derivative of \eqref{eq:f_g_int_lagrangian}, we have
\Beq  \label{eq:rate_f_g_int_lagrangian}
  \dot{\Bf}_g^{\Tint} = \sum_p\int_{\Omega_p^n} Y_p(\Bx^n)\dot{\BP}_p^n \cdot \underset{n}{\nabla}{S_g}~d\Omega^n \,.
\Eeq
Since the rate of the first Piola-Kircchoff stress is not objective, it is easier to work with
the rate of the second Piola-Kirchhoff stress ($\BS$):
\Beq \label{eq:rate_P}
  \BP = \BS \cdot \BF^T \quad \implies \quad
  \dot{\BP} = \dot{\BS} \cdot \BF^T + \BS \cdot \dot{\BF}^T \,.
\Eeq
Substitution of \eqref{eq:rate_P} into \eqref{eq:rate_f_g_int_lagrangian} gives
\Beq
  \dot{\Bf}_g^{\Tint} = \sum_p\int_{\Omega_p^n} Y_p(\Bx^n)\left[\dot{\BS}_p^n \cdot (\BF_p^n)^T + \BS_p^n \cdot (\dot{\BF}_p^n)^T\right] \cdot \underset{n}{\nabla}{S_g}~d\Omega^n \,.
\Eeq
Separating out the two components, we have
\Beq
  \dot{\Bf}_g^{\Tint} = 
    \sum_p\int_{\Omega_p^n} Y_p(\Bx^n) \dot{\BS}_p^n \cdot (\BF_p^n)^T \cdot \underset{n}{\nabla}{S_g}~d\Omega^n + 
    \sum_p\int_{\Omega_p^n} Y_p(\Bx^n) \BS_p^n \cdot (\dot{\BF}_p^n)^T \cdot \underset{n}{\nabla}{S_g}~d\Omega^n \,.
\Eeq
The rate of the internal force can then be expressed as
\Beq
  \dot{\Bf}_g^{\Tint} = \dot{\Bf}_g^{\Tmat} + \dot{\Bf}_g^{\Tgeo}
\Eeq
where the material and geometric rates of the internal forces are defined as
\Beq \label{eq:fdot_mat_geo}
  \Bal
   \dot{\Bf}_g^{\Tmat} & :=  \sum_p\int_{\Omega_p^n} Y_p(\Bx^n) \dot{\BS}_p^n \cdot (\BF_p^n)^T \cdot \underset{n}{\nabla}{S_g}~d\Omega^n  \\
   \dot{\Bf}_g^{\Tgeo} & :=  \sum_p\int_{\Omega_p^n} Y_p(\Bx^n) \BS_p^n \cdot (\dot{\BF}_p^n)^T \cdot \underset{n}{\nabla}{S_g}~d\Omega^n \,.
  \Eal
\Eeq
We can now use the constitutive relation between the second Piola-Kirchhoff stress and the Green strain ($\BE$)
\Beq
  \dot{\BS} = \CalC : \dot{\BE}
\Eeq
and the relationship between the velocity gradient and the rate of change of the deformation gradient
\Beq
  \dot{\BF} = \BlT \cdot \BF
\Eeq
to write \eqref{eq:fdot_mat_geo} as
\Beq \label{eq:fdot_mat_geo_1}
  \Bal
   \dot{\Bf}_g^{\Tmat} & :=  \sum_p\int_{\Omega_p^n} Y_p(\Bx^n) \left[\CalC_p^n : \dot{\BE}_p^n\right] \cdot (\BF_p^n)^T \cdot \underset{n}{\nabla}{S_g}~d\Omega^n  \\
   \dot{\Bf}_g^{\Tgeo} & :=  \sum_p\int_{\Omega_p^n} Y_p(\Bx^n) \BS_p^n \cdot \left[(\BF_p^n)^T \cdot (\BlT_p^n)^T\right] \cdot \underset{n}{\nabla}{S_g}~d\Omega^n \,.
  \Eal
\Eeq
Noting that
\Beq
  \Partial{\Bf}{t} = \Partial{\Bf}{\Bv} \cdot \Partial{\Bv}{t} = \Partial{\Bf}{\Bv} \cdot \Ba
\Eeq
we have
\Beq \label{eq:fdot_mat_geo_2}
  \Bal
   \Partial{\Bf_g^{\Tmat}}{\Bv} \cdot \Ba_g & :=  \left[\sum_p\int_{\Omega_p^n} Y_p(\Bx^n) 
     \left[\CalC_p^n : \Partial{\BE_p^n}{\Bv}\right] \cdot (\BF_p^n)^T \cdot \underset{n}{\nabla}{S_g}~d\Omega^n\right] \cdot \Ba_g  \\
   \Partial{\Bf_g^{\Tgeo}}{\Bv} \cdot \Ba_g & :=  \left[\sum_p\int_{\Omega_p^n} Y_p(\Bx^n) 
     \BS_p^n \cdot \left[(\BF_p^n)^T \cdot (\BlT_p^n)^T\right] \cdot \underset{n}{\nabla}{S_g}~d\Omega^n\right] \,.
  \Eal
\Eeq

