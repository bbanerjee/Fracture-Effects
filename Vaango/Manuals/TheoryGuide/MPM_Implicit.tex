Recall from equation~\eqref{eq:discrete_MPM_equations} that the \MPM discretized momentum
equations can be written as a semi-algebraic system
\Beq \label{eq:discrete_MPM_equations_1}
  \sum_h M_{gh} \dot{\Bv}_h = \Bf_g^{\Text} -\Bf_g^{\Tint} + \Bf_g^{\Tbody} 
      ~;~~ g = 1 \dots N_g
\Eeq
where the mass matrix ($\BM$), the internal force vector ($\Bf_g^{\Tint}$), 
the body force vector ($\Bf_g^{\Tbody}$), and the  external force vector ($\Bf_g^{\Text}$)
at grid node $g$ as
\Beq \label{eq:fext_fint_def}
  \Bal
   &M_{gh} := \sum_p \rho_p\IntOmegap Y_p(\Bx)~S_g(\Bx)~S_h(\Bx)~d\Omega \\
   &\Bf_g^{\Tint} := \sum_p V_p \Bsig_p \cdot \Av{\Grad{S_{gp}}} \\
   &\Bf_g^{\Tbody} := \sum_p m_p\Bb_p~\Av{S_{gp}} \\
   &\Bf_g^{\Text} := \IntGammat \Bart(\Bx)~S_g(\Bx)~d\Gamma \,.
  \Eal
\Eeq

While the \MPM background grid is reset after each time increment, \MPM does not
require it to be reset during each iteration of an implicit integration process.
Therefore, during a time step, we can carry a grid displacement variable $\Bu$ that 
can be used to compute grid accelerations $\Ba_g = \dot{\Bv}_g$ and discarded at
the end of a time step.

Let us express \eqref{eq:discrete_MPM_equations_1} in matrix form as
\Beq
  \BM_g \BaT_g = \BfT_g^{\Text} - \BfT_g^{\Tint} + \BfT_g^{\Tbody} 
\Eeq
Define the residual as
\Beq \label{eq:mpm_residual}
  \Br(\BuT_g^{n+1}, t_{n+1}) = \BM_g \BaT_g^{n+1} - \BfT^{\Text}(\BuT_g^{n+1}, t_{n+1}) + 
     \BfT^{\Tint}(\BuT_g^{n+1}, t_{n+1}) - \BfT^{\Tbody}(\BuT_g^{n+1}, t_{n+1}) = \Bzero .
\Eeq
where the superscipt $n+1$ indicates quantities at time $t_{n+1}$ and $\BuT_g$ is the 
$N_g \times 3$ matrix of grid
node displacements.  We use a Newmark-$\beta$ method to integrate the acceleration:
\Beq \label{eq:newmark_beta}
  \Bal
    \BuT_g^{n+1} & = \BuT^\star + \beta  \BaT_g^{n+1} (\Delta t)^2 \\
    \BvT_g^{n+1} & = \BvT^\star + \gamma \BaT_g^{n+1} \Delta t 
  \Eal
\Eeq
where
\Beq
  \Bal
    \BuT^\star & = \BuT_g^n + \BvT_g^n \Delta t + \Half (1 - 2\beta) \BaT_g^n (\Delta t)^2 \\
    \BvT^\star & = \BvT_g^n + (1 - \gamma) \BaT_g^n \Delta t \,.
  \Eal
\Eeq

\subsection{Newton's method}
In the \Vaango implementation, the residual is expressed in terms of the displacement.  
We are required to do this because tangents needed in Newton's method are easier to compute 
when forces can be expressed in the form
\Beq
  \BfT = \BK \cdot \BuT
\Eeq
where $\BK$ is the stiffness matrix.  If we were to use the velocity as the primary variable,
as in explicit MPM, we would need rates of the forces instead:
\Beq
  \dot{\BfT} = \BK \cdot \BvT + \dot{\BK} \cdot \BuT \,.
\Eeq
The extra term involving the rate of change of the stiffness matrix complicates the process
and we avoid it in \Vaango.

Then, using \eqref{eq:mpm_residual} and \eqref{eq:newmark_beta}, we have
\Beq \label{eq:mpm_residual_1}
  \Br(\BuT_g^{n+1}, t_{n+1}) = \frac{1}{\beta \Delta t^2} \BM_g (\BuT_g^{n+1} - \BuT^\star) - 
      \BfT^{\Text}(\BuT_g^{n+1}, t_{n+1}) + 
     \BfT^{\Tint}(\BuT_g^{n+1}, t_{n+1}) - \BfT^{\Tbody}(\BuT_g^{n+1}, t_{n+1}) = \Bzero .
\Eeq
The problem then reduces to finding the solution $\BuT_g^{n+1}$ of the nonlinear system of equations
\eqref{eq:mpm_residual_1}. Newton's method is used in \Vaango with the starting 
value of $\BuT_g^{n+1} = \BuT_g^\star$.  Dropping the subscript $g$ temporarily for convenience,
and denoting the current Newton iteration by the subscript $k$, we can linearize the residual at 
$\BuT_k^{n+1}$ using a Taylor expansion:
\Beq
  \Bzero = \Br(\BuT_{k+1}^{n+1}, t_{n+1}) = 
     \Br(\BuT_k^{n+1}, t_{n+1}) + \Partial{\Br(\BuT_k^{n+1}, t_{n+1})}{\BuT}(\BuT^{n+1}_{k+1} - \BuT^{n+1}_k) \,.
\Eeq
Rearranging the above equation,
\Beq \label{eq:Newton_step}
  \Delta\BuT = \BuT^{n+1}_{k+1} - \BuT^{n+1}_k = 
     -\left[\Partial{\Br(\BuT_k^{n+1}, t_{n+1})}{\BuT}\right]^{-1}
     \Br(\BuT_k^{n+1}, t_{n+1})  = -\BK^{-1} \Br(\BuT_k^{n+1}, t_{n+1})\,.
\Eeq
This iterative process is continued until $\Delta \BuT$ is smaller than a given tolerance.
The tangent matrix $\BK$, of size $N_g \times N_g$, is 
\Beq
  \BK = \Partial{\Br(\BuT_k^{n+1}, t_{n+1})}{\BuT} \,.
\Eeq
This matrix is decomposed and evaluated separately for the internal and external forces, i.e., 
\Beq
  \BK = \Partial{\Br(\BuT_k^{n+1}, t_{n+1})}{\BuT} 
      = \frac{1}{\beta \Delta t^2} \BM_g  - \Partial{}{\BuT}\left[\BfT^{\Text}(\BuT_g^{n+1}, t_{n+1})\right] + 
        \Partial{}{\BuT}\left[\BfT^{\Tint}(\BuT_g^{n+1}, t_{n+1})\right] - 
        \Partial{}{\BuT}\left[\BfT^{\Tbody}(\BuT_g^{n+1}, t_{n+1})\right] \,.
\Eeq
Alternatively,
\Beq
  \BK = \frac{1}{\beta \Delta t^2} \BM_g  - \BK^{\Text}(\BuT_g^{n+1}, t_{n+1}) +  
         \BK^{\Tint}(\BuT_g^{n+1}, t_{n+1}) -  \BK^{\Tbody}(\BuT_g^{n+1}, t_{n+1}) \,.
\Eeq

\subsection{Tangent stiffness matrix}
The contribution to the tangent matrix ($\BK$) from the internal forces is called the
\Textsfc{tangent stiffness matrix} ($\BK^\Tint$).  Since an updated Lagrangian formulation is used
in \MPM, we can compute the tangent stiffness using the configuration at time $t_n$ as
the reference configuration.

Recall from \eqref{eq:fext_fint_def} that for explicit \MPM we used
\Beq \label{eq:f_g_int_0}
  \Bf_g^{\Tint} = \sum_p V_p \Bsig_p \cdot \Av{\Grad{S_{gp}}}  
  \quad \text{where} \quad
  \Av{S_{gp}} := \frac{1}{V_p} \IntOmegap Y_p(\Bx)~S_g(\Bx)~d\Omega \,.
\Eeq
For the computation of the tangent matrix, it is preferable to start from the
weak form of the momentum equation \eqref{eq:weak_form}:
\Beq \label{eq:f_int_weak}
  I = \IntOmega \Bsig:\Grad{\Bw}~d\Omega  
\Eeq
which leads to integral form of equation \eqref{eq:f_g_int_0} (see \eqref{eq:integral_weak_form}):
\Beq  \label{eq:f_g_int}
  \Bf_g^{\Tint} = \sum_p\IntOmegap Y_p(\Bx)\Bsig_p \cdot \Grad{S_g}~d\Omega \,.
\Eeq
Also, since we are typically working with rates of stress in the constitutive models, it
is preferable to express all quantities in terms of stress rates that are objective.
It is easier to work with the Lagrangian PK-1 stress ($\BP$) at the beginning of the timestep
rather than the spatial Cauchy stress ($\Bsig$).

To convert from the spatial description \eqref{eq:f_int_weak} to a Lagrangian material 
description, observe that
\Beq
  \underset{n+1}{\nabla}{\Bw} = \Partial{\Bw}{\Bx^{n+1}} = \Partial{\Bw}{\Bx^n} \cdot \Partial{\Bx^n}{\Bx^{n+1}} 
    = \Partial{\Bw}{\Bx^n} \cdot (\Delta\BF_n^{n+1})^{-1}
    = \Partial{\Bw}{\Bx^n} \cdot \frac{\Delta\BF_c^T}{J_n^{n+1}}
    = \underset{n}{\nabla} \Bw \cdot \frac{\Delta\BF_c^T}{J_n^{n+1}}
\Eeq
where, with $\BF$ as the deformation gradient, 
\Beq
  \BF^{n+1} = \Delta\BF_n^{n+1} \BF^n~,~~ J_n^{n+1} = \det(\Delta\BF_n^{n+1}) ~,~~
  \Delta\BF_c = \text{cofactor}(\Delta\BF_n^{n+1}) \,.
\Eeq
Therefore,
\Beq
  \Bal 
   I & = \int_{\Omega^{n+1}} \Bsig^{n+1} : \underset{n+1}{\nabla}{\Bw}~d\Omega^{n+1}
       = \int_{\Omega^n} \Bsig^{n+1} : \underset{n+1}{\nabla}{\Bw}~J_n^{n+1}\,d\Omega^n \\
     & = \int_{\Omega^n} \Bsig^{n+1} : \left(\underset{n}{\nabla}{\Bw}\cdot \Delta\BF_c^T\right)\,d\Omega^n 
       = \int_{\Omega^n} \left(\Bsig^{n+1} \cdot \Delta\BF_c\right) : \underset{n}{\nabla}{\Bw}\,d\Omega^n \\
     & = \int_{\Omega^n} \BP^n : \underset{n}{\nabla}{\Bw}\,d\Omega^n 
  \Eal
\Eeq
where $\BP^n$ is the first Piola-Kirchhoff stress.  Following the same process as used to derive
\eqref{eq:integral_weak_form}, we get
\Beq  \label{eq:f_g_int_lagrangian}
  \Bf_g^{\Tint} = \sum_p\int_{\Omega_p^n} Y_p(\Bx^n)\BP_p^n \cdot \underset{n}{\nabla}{S_g}~d\Omega^n \,.
\Eeq
Taking the material time derivative of \eqref{eq:f_g_int_lagrangian}, we have
\Beq  \label{eq:rate_f_g_int_lagrangian}
  \dot{\Bf}_g^{\Tint} = \sum_p\int_{\Omega_p^n} Y_p(\Bx^n)\dot{\BP}_p^n \cdot \underset{n}{\nabla}{S_g}~d\Omega^n \,.
\Eeq
Since the rate of the first Piola-Kirchhoff stress is not objective, it is easier to work with
the rate of the second Piola-Kirchhoff stress ($\BS$):
\Beq \label{eq:rate_P}
  \BP = \BF \cdot \BS \quad \implies \quad
  \dot{\BP} = \dot{\BF} \cdot \BS + \BF \cdot \dot{\BS} \,.
\Eeq
Substitution of \eqref{eq:rate_P} into \eqref{eq:rate_f_g_int_lagrangian} gives
\Beq
  \dot{\Bf}_g^{\Tint} = \sum_p\int_{\Omega_p^n} Y_p(\Bx^n)\left[\dot{\BF}_p^{n+1} \cdot \BS_p^n + \BF_p^{n+1} \cdot \dot{\BS}_p^n\right] \cdot \underset{n}{\nabla}{S_g}~d\Omega^n \,.
\Eeq
Separating out the two components, we have
\Beq
  \dot{\Bf}_g^{\Tint} = 
    \sum_p\int_{\Omega_p^n} Y_p(\Bx^n) \dot{\BF}_p^{n+1} \cdot \BS_p^n \cdot \underset{n}{\nabla}{S_g}~d\Omega^n + 
    \sum_p\int_{\Omega_p^n} Y_p(\Bx^n) \BF_p^{n+1} \cdot \dot{\BS}_p^n \cdot \underset{n}{\nabla}{S_g}~d\Omega^n \,.
\Eeq
The rate of the internal force can then be expressed as
\Beq
  \dot{\Bf}_g^{\Tint} = \dot{\Bf}_g^{\Tgeo} + \dot{\Bf}_g^{\Tmat}
\Eeq
where the geometric and material rates of the internal forces are defined as
\Beq \label{eq:fdot_mat_geo}
  \Bal
   \dot{\Bf}_g^{\Tgeo} & :=  \sum_p\int_{\Omega_p^n} Y_p(\Bx^n) \dot{\BF}_p^{n+1} \cdot \BS_p^n \cdot \underset{n}{\nabla}{S_g}~d\Omega^n  \\
   \dot{\Bf}_g^{\Tmat} & :=  \sum_p\int_{\Omega_p^n} Y_p(\Bx^n) \BF_p^{n+1} \cdot \dot{\BS}_p^n \cdot \underset{n}{\nabla}{S_g}~d\Omega^n \,.
  \Eal
\Eeq
We can now use the constitutive relation between the second Piola-Kirchhoff stress and the
Green strain ($\BE$), the expression for the Green strain in terms of the deformation gradient,
the relationship between the velocity gradient ($\BlT$) and the rate of
change of the deformation gradient, and the definition of the rate-of-deformation ($\BdT$)
\Beq
  \dot{\BS} = \CalC : \dot{\BE}~,~~ \BE = \Half\left(\BF^T \cdot \BF - \BI\right) ~,~~
  \dot{\BF} = \BlT \cdot \BF, \quad \Tand \quad \BdT = \Half(\BlT + \BlT^T)
\Eeq
to write the material and geometric rates of the internal force in \eqref{eq:fdot_mat_geo} as
\Beq \label{eq:fdot_mat_geo_1}
  \Bal
   \dot{\Bf}_g^{\Tmat} 
   & =  \sum_p\int_{\Omega_p^n} Y_p(\Bx^n) \BF_p^{n+1} \cdot \left[\CalC_p^n : \dot{\BE}_p^n\right] 
        \cdot \underset{n}{\nabla}{S_g}~d\Omega^n  \\
   & = \Half \sum_p\int_{\Omega_p^n} Y_p(\Bx^n) \BF_p^{n+1} \cdot \left[
        \CalC_p^n :
        \left((\dot{\BF}_p^{n+1})^T \cdot \BF_p^{n+1} + (\BF_p^{n+1})^T \cdot \dot{\BF}_p^{n+1}\right)
        \right] \cdot \underset{n}{\nabla}{S_g}~d\Omega^n  \\
   & = \Half \sum_p\int_{\Omega_p^n} Y_p(\Bx^n) \BF_p^{n+1} \cdot \left[
        \CalC_p^n :
        \left((\BF_p^{n+1})^T \cdot (\BlT_p^{n+1})^T \cdot \BF_p^{n+1} +
              (\BF_p^{n+1})^T \cdot \BlT_p^{n+1} \cdot \BF_p^{n+1}\right)
        \right] \cdot \underset{n}{\nabla}{S_g}~d\Omega^n  \\
   & = \sum_p\int_{\Omega_p^n} Y_p(\Bx^n) \BF_p^{n+1} \cdot \left[
        \CalC_p^n :
        \left((\BF_p^{n+1})^T \cdot \BdT_p^{n+1} \cdot \BF_p^{n+1}\right)
        \right] \cdot \underset{n}{\nabla}{S_g}~d\Omega^n  \\
   \dot{\Bf}_g^{\Tgeo} & :=
     \sum_p\int_{\Omega_p^n} Y_p(\Bx^n) \left[(\BF_p^{n+1})^T \cdot (\BlT_p^{n+1})^T\right] \cdot \BS_p^n \cdot \underset{n}{\nabla}{S_g}~d\Omega^n \,.
  \Eal
\Eeq
Recall the interpolation of the velocity from the grid nodes ($h$) to particles ($p$) can be
computed using
\Beq
  \Bv_p(\Bx^{n+1}) = \sum_h \Bv_h^{n+1} S_h(\Bx^{n+1})
\Eeq
Therefore,
\Beq \label{eq:vel_grad_implicit}
  \Bal
  \BlT_p^{n+1} & = \underset{n+1}{\nabla}\Bv_p^{n+1}(\Bx^{n+1}) = 
    \sum_h \Bv_h^{n+1} \otimes \underset{n+1}{\nabla}S_h \\
  \BdT_p^{n+1} & = \Half\sum_h 
    \left[\Bv_h^{n+1} \otimes \underset{n+1}{\nabla}S_h +
          \underset{n+1}{\nabla}S_h \otimes \Bv_h^{n+1}\right] \,.
  \Eal
\Eeq
Using \eqref{eq:vel_grad_implicit} in \eqref{eq:fdot_mat_geo_1}, we have
\Beq \label{eq:fint_rates}
  \Bal
  \dot{\Bf}_g^{\Tmat} & = 
    \Half \sum_h \sum_p\int_{\Omega_p^n} Y_p(\Bx^n) \BF_p^{n+1} \cdot \left[
        \CalC_p^n :
        \left((\BF_p^{n+1})^T \cdot 
        \left[\underset{n+1}{\nabla}S_h \otimes \Bv_h^{n+1} +
              \Bv_h^{n+1} \otimes \underset{n+1}{\nabla}S_h\right] 
        \cdot \BF_p^{n+1}\right)
        \right] \cdot \underset{n}{\nabla}{S_g}~d\Omega^n  \\
   \dot{\Bf}_g^{\Tgeo} & =
     \sum_h \sum_p\int_{\Omega_p^n} Y_p(\Bx^n) 
         \left[(\BF_p^{n+1})^T \cdot \left(\underset{n+1}{\nabla}S_h \otimes \Bv_h^{n+1}\right)\right] 
          \cdot \BS_p^n \cdot \underset{n}{\nabla}{S_g}~d\Omega^n \,.
  \Eal
\Eeq
Since both the second Piola-Kirchhoff stress and the Green strain are symmetric, the
tensor $\CalC$ has the symmetries $C_{ijkl} = C_{jikl} = C_{jilk}$.  For hyperelastic materials
we have the additional symmetry $C_{ijkl} = C_{klij}$. We can take advantage of these symmetries
to simplify the above expressions.  The first term in the expression for the rate of the 
material internal force contains an expression of the form
\Beq
  \BAv := \BF \cdot \left[ 
           \CalC : \left(\BF^T \cdot \left[\tilde{\BGv} \otimes \Bv \right] \cdot \BF\right) 
           \right] \cdot \BGv =: \Balpha \cdot \Bv
\Eeq
while the second term contains
\Beq
  \BBv := \BF \cdot \left[ 
           \CalC : \left(\BF^T \cdot \left[\Bv \otimes \tilde{\BGv}\right] \cdot \BF\right) 
           \right] \cdot \BGv =: \Bbeta \cdot \Bv
\Eeq
where
\Beq
  \BGv = \BGv_g := \underset{n}{\nabla}S_g \quad \Tand \quad
  \tilde{\BGv} = \BGv_h := \underset{n+1}{\nabla}S_h \,.
\Eeq
In index notation,
\Beq
  \Bal
    A_r & = F_{ri} C_{ijk\ell} F^T_{km} \tilde{G}_m v_n F_{n\ell} G_j 
          = F_{ri} C_{ijk\ell} F_{mk} \tilde{G}_m F_{n\ell} G_j v_n
          = G_j (\BF \cdot \CalC \cdot \BF^T)_{rjkn} (\tilde{\BGv} \cdot \BF)_k v_n\\
        & = G_j (\tilde{\BGv} \cdot \BF)_k (\BF \cdot \CalC \cdot \BF^T)_{rjkn} v_n
          =: \alpha_{rn} v_n \\
    B_r & = F_{ri} C_{ijk\ell} F^T_{kn} v_n \tilde{G}_m F_{m\ell} G_j 
          = F_{ri} C_{ijk\ell} F_{nk} \tilde{G}_m F_{m\ell} G_j v_n
          = G_j (\BF \cdot \CalC \cdot \BF^T)_{rj\ell n} (\tilde{\BGv} \cdot \BF)_{\ell} v_n \\
        & = G_j (\tilde{\BGv} \cdot \BF)_{\ell} (\BF \cdot \CalC \cdot \BF^T)_{rj\ell n} v_n 
          =: \beta_{rn} v_n = \alpha_{rn} v_n = A_r 
  \Eal
\Eeq
Similarly for the geometrically nonlinear component, we have
\Beq
  \BCv := \left[\BF^T \cdot \left(\tilde{\BGv} \otimes \Bv\right)\right] 
          \cdot \BS \cdot \BGv =: \Bgamma \cdot \Bv\,.
\Eeq
In index notation
\Beq
  C_r = F_{ri} \tilde{G}_{i} v_n S_{nk} G_{k} = (\tilde{\BGv} \cdot \BF^T)_r (\BGv \cdot \BS)_{n} v_n
      =: \gamma_{rn} v_n \,.
\Eeq
We can now express \eqref{eq:fint_rates} as
\Beq \label{eq:fint_rates_1}
  \Bal
  \dot{\Bf}_g^{\Tmat} & = 
     \sum_h \left[\sum_p\int_{\Omega_p^n} Y_p(\Bx^n) \,\Balpha ~d\Omega^n\right]\cdot\Bv_h^{n+1}  \\
   \dot{\Bf}_g^{\Tgeo} & =
     \sum_h \left[\sum_p\int_{\Omega_p^n} Y_p(\Bx^n) \,\Bgamma ~d\Omega^n\right]\cdot\Bv_h^{n+1} 
  \Eal
\Eeq
where
\Beq
  \Bal
    \left[\Balpha\right]_{i\ell} & = G_j \left[\tilde{\BGv} \cdot \BF_p^{n+1}\right]_k \left[\BF_p^{n+1} \cdot 
       \CalC_p^n \cdot \left(\BF_p^{n+1}\right)^T\right]_{ijk\ell}  \\
    [\Bgamma]_{i\ell} & = \left[\tilde{\BGv} \cdot \left(\BF_p^{n+1}\right)^T\right]_i \left[\BGv \cdot \BS_p^n\right]_{\ell} 
  \Eal
\Eeq
Using
\Beq
  \dot{\Bf} = \Partial{\Bf}{\Bu}\cdot\Bv
\Eeq
where $\Bu$ is the displacement, we notice from \eqref{eq:fint_rates_1} that
\Beq \label{eq:df_du_mat_geo}
  \Bal
  (\BK^\Tmat)_{gh} = \Partial{\Bf_g^{\Tmat}}{\Bu_h^{n+1}} & = 
     \sum_p\int_{\Omega_p^n} Y_p(\Bx^n) \,\Balpha ~d\Omega^n \\
  (\BK^\Tgeo)_{gh} = \Partial{\Bf_g^{\Tgeo}}{\Bu_h^{n+1}} & = 
     \sum_p\int_{\Omega_p^n} Y_p(\Bx^n) \,\Bgamma ~d\Omega^n \,.
  \Eal
\Eeq
If we now set the current configuration as the reference configuration (see~\cite{Ogden1997}, section 6.1.3),
we have
\Beq
  \BF_p^{n+1} = \BI~,~~ \BS_p^n = \Bsig_p^{n+1}~,~~ \Bx^n = \Bx^{n+1} ~,~~
  d\Omega_n = d\Omega_{n+1}~,~~ \BGv = \tilde{\BGv}~,~~ 
  \CalC_p^n = (\CalC^\sigma)_p^{n+1} \,.
\Eeq
Therefore,
\Beq
  (\BK^\Tint)_{gh}(\BuT_g^{n+1}, t_{n+1}) = (\BK^\Tmat)_{gh} + (\BK^\Tgeo)_{gh} 
\Eeq
where
\Beq \label{eq:K_mat_geo}
  \Bal
  (\BK^\Tmat)_{gh} & = \sum_p\int_{\Omega_p^{n+1}} Y_p(\Bx^{n+1}) \,
     \tilde{\BGv}_g \cdot (\CalC^\sigma)_p^{n+1} \cdot \tilde{\BGv}_h ~d\Omega^{n+1} \\
  (\BK^\Tgeo)_{gh} & = \sum_p\int_{\Omega_p^{n+1}} Y_p(\Bx^{n+1}) \,
     (\tilde{\BGv}_h \otimes \tilde{\BGv}_g) \cdot \Bsig_p^{n+1} 
     ~d\Omega^{n+1} \,.
  \Eal
\Eeq
An efficient way of converting these relations to Voigt form is possible only in
the case where the grid basis functions are trilinear.  For \GIMP and \CPDI basis functions,
the problem becomes more complex and have not been implemented in \Vaango.

\subsection{External force stiffness matrix}
Recall from \eqref{eq:discretized_def} that the external force is given by
\Beq
  \Bf_g^{\Text} := \IntGammat \Bart(\Bx)~S_g(\Bx)~d\Gamma \,.
\Eeq
To find the contribution to the stiffness matrix from the external force, note that
\Beq
  \dot{\Bf}_g^{\Text} := 
    \IntGammat \left[\dot{\Bart}(\Bx)~S_g(\Bx) + \Bart(\Bx)~(\Grad{S_g}\cdot\Bv_g)\right]~d\Gamma =
    \IntGammat \left[\dot{\Bart}(\Bx)~S_g(\Bx) + (\Bart(\Bx) \otimes \tilde{\BGv}_g) \cdot\Bv_g\right]~d\Gamma 
\Eeq
We make the simplifying assumption that
\Beq
  \dot{\Bart}(\Bx) = \widetilde{t}(\Bx) \Bv_g
\Eeq
to get 
\Beq
  \dot{\Bf}_g^{\Text} := 
    \IntGammat \left[\widetilde{t}(\Bx)~S_g(\Bx) \BI + \Bart(\Bx) \otimes \tilde{\BGv}_g\right] \cdot\Bv_g~d\Gamma 
\Eeq
Therefore,
\Beq
  (\BK^\Text)_{gh}  = 
    \IntGammat \left[\widetilde{t}(\Bx)~S_g(\Bx) \BI + \Bart(\Bx) \otimes \tilde{\BGv}_g\right]\delta_{gh}~d\Gamma 
\Eeq

\subsection{Body force stiffness matrix}
The body force is given by
\Beq
  \Bf_g^{\Text} := \sum_p\IntOmegap \rho_p Y_p(\Bx) S_g(\Bx)\Bb_p~d\Omega \\
\Eeq
In \Vaango we assume that the body force does not vary with deformation.
Therefore,
\Beq
  (\BK^\Tbody)_{gh}  = 0 \,.
\Eeq


