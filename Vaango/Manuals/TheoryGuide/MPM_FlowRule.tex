\chapter{Flow rule}
Plastic flow rules in \Vaango have the form
\Beq \label{eq:flow_rule}
  \BdT^p = \dot{\BVeps}^p = \dot{\lambda}\BM
\Eeq
where $\BdT^p = \dot{\BVeps}^p$ is the plastic strain rate tensor, $\lambda$ is 
the consistency parameter, and $\BM$ is a unit tensor in the direction of the
plastic strain rate.

\section{Associated plasticity}
For associated plasticity, \Vaango uses the classical approach in plasticity theory
and assumes that $\BM = \hat{\BN}$ is the unit normal to the yield surface:
\Beq
  \dot{\BVeps}^p = \dot{\lambda}\hat{\BN} 
  ~,~~ \hat{\BN} = \frac{\Partial{f}{\Bsig}}{\Norm{\Partial{f}{\Bsig}}{}}
\Eeq
where $f$ is the yield function.

\section{Non-associated plasticity}
\Vaango use two approaches for non-associated plasticity ($\BM \ne \hat{\BN}$).  
The first approach, implemented in the Mohr-Coulomb model, is to use a separate
plastic potential ($g$) to compute $\BM$:
\Beq
  \dot{\BVeps}^p = \dot{\lambda} \hat{\BM} 
  ~,~~ \hat{\BM} = \frac{\Partial{g}{\Bsig}}{\Norm{\Partial{g}{\Bsig}}{}}
\Eeq
The plastic potential is assumed to have the same form as the yield function,
but different parameters to match experimental data on dilatation.

An alternative approach, used in Arenisca3 and Arena, is to compute the
direction of the plastic strain rate tensor using
\Beq
  \dot{\BVeps}^p = \dot{\lambda} \hat{\BM} 
  ~,~~ \hat{\BM} = \frac{\Dev(\BN) + \beta\, \Tr(\BN)}{\Norm{\Dev(\BN) + \beta\, \Tr(\BN)}{}}
\Eeq
where $\beta$ is an adjustable parameter.

