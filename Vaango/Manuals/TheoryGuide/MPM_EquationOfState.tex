\chapter{Equation of state models} \label{ch:EOS}
In the isotropic metal plasticity models implemented in \Vaango, the volumetric 
part of the Cauchy stress can be calculated using an equation of state.  The 
equations of state that are implemented in \Vaango are described below.

\section{Hypoelastic equation of state}
In this case we assume that the stress rate is given by
\Beq
    \dot{\Bsig} = \lambda~\Tr(\BdT^e)~\BI + 2~\mu~\BdT^e
\Eeq
where $\Bsig$ is the Cauchy stress, $\BdT^e$ is the elastic part of
the rate of deformation, and $\lambda, \mu$ are constants.

If $\Dev(\BdT^e)$ is the deviatoric part of $\BdT^e$ then we can write
\Beq \label{eq:sigdot_hypo}
    \dot{\Bsig} = \left(\lambda + \frac{2}{3}~\mu\right)~\Tr(\BdT^e)~\BI + 
        2~\mu~\Dev(\BdT^e) = \kappa~\Tr(\BdT^e)~\BI + 2~\mu~\Dev(\BdT^e) ~.
\Eeq
If we split $\Bsig$ into a volumetric and a deviatoric part, i.e.,
$\Bsig = p~\BI + \Bs$, take the time derivative to get
$\dot{\Bsig} = \dot{p}~\BI + \dot{\Bs}$, and compare the result with
\eqref{eq:sigdot_hypo}, we see that
\Beq
    \dot{p} = \kappa~\Tr(\BdT^e) ~.
\Eeq
In addition we assume that $\BdT = \BdT^e + \BdT^p$.  If we also assume that 
the plastic volume change is negligible ($\Tr(\BdT^p) \approx 0$), 
which is reasonable for a void-free metal matrix, we have
\Beq
    \dot{p} = \kappa~\Tr(\BdT) ~.
\Eeq
This is the equation that is used to calculate the pressure $p$ in the 
default hypoelastic equation of state.  For a forward Euler integration step,
\Beq
    \boxed{
    p_{n+1} = p_n + \kappa~\Tr(\BdT_{n+1})~\Delta t ~.
    }
\Eeq
To get the derivative of $p$ with respect to $J$, where $J = \det(\BF)$,
we note that
\Beq
    \dot{p} = \Partial{p}{J}~\dot{J} = \Partial{p}{J}~J~\Tr(\BdT) ~.
\Eeq
Therefore,
\Beq
    \boxed{
    \Partial{p}{J} = \cfrac{\kappa}{J} ~.
    }
\Eeq

This model is invoked in \Vaango using
\lstset{language=XML}
\begin{lstlisting}
  <equation_of_state type="default_hypo">
  </equation_of_state>
\end{lstlisting}

\section{Default hyperelastic equation of state}
In this model the pressure is computed using the relation
\begin{equation}
  p = \Half~\kappa~\left(J^e - \cfrac{1}{J^e}\right)
\end{equation}
where $\kappa$ is the bulk modulus and $J^e$ is determinant of the elastic 
part of the deformation gradient.

We can also compute
\begin{equation}
  \Deriv{p}{J} = \Half~\kappa~\left(1 + \cfrac{1}{(J^e)^2}\right) ~.
\end{equation}

The metal plasticity implementations in \Vaango assume that the volume
change of the matrix during plastic deformation can be neglected, i.e., $J^e = J$.

This model is invoked using
\lstset{language=XML}
\begin{lstlisting}
  <equation_of_state type="default_hyper">
  </equation_of_state>
\end{lstlisting}

\section{Mie-Gruneisen equation of state}
The pressure ($p$) is calculated using a Mie-Gr{\"u}neisen equation of state 
of the form (\cite{Wilkins1999,Zocher2000})
\begin{equation} \label{eq:EOSMG_upd}
  p_{n+1} =  - \frac{\rho_0~C_0^2~(1 - J^e_{n+1})
           [1 - \Gamma_0 (1 - J^e_{n+1})/2]}
           {[1 - S_{\alpha}(1 - J^e_{n+1})]^2} - \Gamma_0~e_{n+1} 
  ~;~~~ J^e := \det{\BF^e} 
\end{equation}
where $C_0$ is the bulk speed of sound, $\rho_0$ is the initial mass density,
$\Gamma_0$ is the Gr{\"u}neisen's gamma at the reference state,
$S_{\alpha} = dU_s/dU_p$ is a linear Hugoniot slope coefficient,
$U_s$ is the shock wave velocity, $U_p$ is the particle velocity, and
$e$ is the internal energy density (per unit reference volume), $\BF^e$ is
the elastic part of the deformation gradient.  For isochoric plasticity,
\begin{equation*}
  J^e = J = \det(\BF) = \cfrac{\rho_0}{\rho} ~.
\end{equation*}
  The internal energy is computed using
  \begin{equation}
    E = \frac{1}{V_0} \int C_v dT \approx \frac{C_v (T-T_0)}{V_0}
  \end{equation}
  where $V_0 = 1/\rho_0$ is the reference specific volume at temperature 
  $T = T_0$, and $C_v$ is the specific heat at constant volume.

Also,
\Beq
  \boxed{
  \Partial{p}{J^e} = 
  \cfrac{\rho_0~C_0^2~[1 + (S_{\alpha}-\Gamma_0)~(1-J^e)]}
        {[1-S_{\alpha}~(1-J^e)]^3} - \Gamma_0~\Partial{e}{J^e}.
  }
\Eeq
We neglect the $\Partial{e}{J^e}$ term in our calculations.

This model is invoked in Vaango using
\lstset{language=XML}
\begin{lstlisting}
  <equation_of_state type="mie_gruneisen">
    <C_0>5386</C_0>
    <Gamma_0>1.99</Gamma_0>
    <S_alpha>1.339</S_alpha>
    <rho_0> 7200 </rho_0>
  </equation_of_state>
\end{lstlisting}

An alternative formulation is also available that can be used for models where a 
linear Hugoniot is not accurate enough.  A cubic model can be used in that
formulation.
\Beq
  p_{n+1} =  - \frac{\rho_0~C_0^2~(1 - J^e_{n+1})
           [1 - \Gamma_0 (1 - J^e_{n+1})/2]}
           {[1 - S_{\alpha}(1 - J^e_{n+1}) - S_2 (1 - J^e_{n+1})^2 - S_3 (1 - J^e_{n+1})^3]^2} - \Gamma_0~e_{n+1} 
  ~;~~~ J^e := \det{\BF^e} 
\Eeq
This model is invoked using the label \Textbfc{mie\_gruneisen\_energy}.

