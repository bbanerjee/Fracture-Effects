\chapter{Damage models}
The damage models implemented in \Vaango are described in this chapter.  The
most common model evolves a scalar damage parameter that can either be
used to flag failure when a critical value is reached or to modify the
stress as in continuum damage mechanics.

\section{Hancock-MacKenzie model}
The Hancock-MacKenzie model~\cite{Hancock1976} evolves a scalar damage parameter ($D$)
using the rule:
\Beq
  \dot{D} = (1/1.65)\dot{\Veps}_p \exp\left(\frac{\Tr(\Bsig)}{2\sigma_{\Teq}}\right)
\Eeq
where $ D = 0 $ for virgin material, $ \dot{\Veps}_p $ is the 
equivalent plastic strain rate, $\Bsig$ is the Cauchy stress, and
$\sigma_\Teq = \sqrt{3J_2}$ is the von Mises equivalent stress.

The input tags for the damage model are :
\lstset{language=XML}
\begin{lstlisting}
  <damage_model type="hancock_mackenzie">
    <D0>0.05</D1>
    <Dc>3.44</D2>
  </damage_model>
\end{lstlisting}

\section{Johnson-Cook model}
The Johnson-Cook damage model ~\cite{Johnson1985} depends on temperature, plastic
strain, and strain rate.
The damage evolution rule for the damage parameter ($D$) can be written as
\Beq
  \dot{D} = \frac{\dot{\Veps}_p}{\Veps_p^f}
\Eeq
where
\Beq
  \Veps_p^f = 
    \left[D_1 + D_2 \exp \left(\frac{D_3}{3} \sigma^\star\right)\right]
    \left[1+ D_4 \ln(\dot{\Veps}_p^\star)\right]
    \left[1+D_5 T^\star\right]\,.
\Eeq
The damage parameter $D$ has a value of 0 for virgin material
and a value of 1 at fracture, $\Veps_p^f$ is the fracture strain, 
$D_1, D_2, D_3, D_4, D_5$ are constants.
In the above equation,
\Beq
  \sigma^\star= \frac{\Tr(\Bsig)}{\sigma_{\Teq}}
\Eeq
where $\Bsig$ is the Cauchy stress and $\sigma_\Teq$ is the von Mises equivalent stress.
The scaled plastic strain rate and temperature are defined as
\Beq
 \dot{\Veps}_p^\star = \frac{\dot{\Veps}_p}{\dot{\Veps}_{p0}} ~,~~
 T^\star = \frac{T-T_0}{T_m-T_0}
\Eeq
where $\dot{\Veps}_{p0}$ is a reference strain rate and $T_0$ is a reference temperature,
$T_m$ is the melting temperature, and $\dot{\Veps}_p$ is the equivalent plastic strain rate.

The input tags for the damage model are :
\lstset{language=XML}
\begin{lstlisting}
  <damage_model type="johnson_cook">
    <D1>0.05</D1>
    <D2>3.44</D2>
    <D3>-2.12</D3>
    <D4>0.002</D4>
    <D5>0.61</D5>
  </damage_model>
\end{lstlisting}

An initial damage distribution can be created using the following tags
\lstset{language=XML}
\begin{lstlisting}
  <evolve_damage>                 true  </evolve_damage>
  <initial_mean_scalar_damage>    0.005  </initial_mean_scalar_damage>
  <initial_std_scalar_damage>     0.001 </initial_std_scalar_damage>
  <critical_scalar_damage>        1.0   </critical_scalar_damage>
  <initial_scalar_damage_distrib> gauss </initial_scalar_damage_distrib>
\end{lstlisting}

