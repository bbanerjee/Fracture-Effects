\chapter{Internal variable evolution}
Internal variables are used to model isotropic hardening/softening behavior in 
\Vaango. The evolution of these internal variables is assumed to be 
given by first-order differential equations of the form
\Beq \label{eq:iv_evolution}
  \dot{\Beta} = \dot{\lambda} \Bh_\eta
\Eeq
where $\Beta$ is the internal variable, $\lambda$ is the consistency parameter,
and $\Bh_\eta$ is a hardening/softening modulus.  

\begin{NoteBox}
  The kinematic hardening backstress is also an internal variable.  Equations
  for the evolution of backstress are given in Chapter~\ref{ch:kin_hard}.
\end{NoteBox}

\section{Equivalent plastic strain}
Recall from the flow rule \eqref{eq:flow_rule} that
\Beq
  \dot{\BVeps}^p = \dot{\lambda}\BM
\Eeq
where $\BM$ is a unit tensor ($\BM:\BM = 1$).  Therefore,
using the definition of the equivalent plastic strain 
rate from \eqref{eq:eq_plastic_strain_rate},
\Beq \label{eq:consistency_rate}
  \dot{\BVeps}^p:\dot{\BVeps}^p = \left(\dot{\lambda}\right)^2
  \quad \implies \quad
  \dot{\lambda} = \sqrt{\dot{\BVeps}^p:\dot{\BVeps}^p} = \dot{\Veps}_p^\Teq \,.
\Eeq
Therefore, from the definition of the equivalent plastic strain in
\eqref{eq:eq_plastic_strain}, we see that the evolution rule for the 
equivalent plastic strain can be expressed in the the form \eqref{eq:iv_evolution}
as
\Beq
  \dot{\Veps}_p^\Teq = \dot{\lambda} h_{\Veps_p} ~,~~ h_{\Veps_p} = 1 \,.
\Eeq

\section{Porosity}
The evolution of porosity is assumed to be given by the sum of the rate of 
void growth and the rate of void nucleation~\cite{Ramaswamy1998a}.  In
\Vaango these rates are computed as ~\cite{Chu1980}:
\begin{align}
  \dot{\phi} &= \dot{\phi}_{\text{nucl}} + \dot{\phi}_{\text{grow}} \\
  \dot{\phi}_{\text{grow}} & = (1-\phi) \Tr(\dot{\BVeps}^p) \\
  \dot{\phi}_{\text{nucl}} & = \frac{f_n}{(s_n \sqrt{2\pi})}
          \exp\left[-\Half \frac{(\Veps^\Teq_p - \Veps_n)^2}{s_n^2}\right]
          \dot{\Veps}_p^\Teq
\end{align}
where $\dot{\BVeps}_p$ is the plastic strain rate, $f_n$ is the volume 
fraction of void nucleating particles , $\Veps_n$ is the mean of the 
distribution of nucleation strains, and $s_n$ is the standard 
deviation of the distribution.

From the flow rule \eqref{eq:flow_rule},
\Beq
  \Tr(\dot{\BVeps}^p) = \dot{\lambda}\Tr(\BM)
\Eeq
Therefore, using \eqref{eq:consistency_rate},
\Beq
  \dot{\phi} = \dot{\lambda} h_\phi ~,~~
    h_\phi = (1-\phi)\Tr(\BM) + \frac{f_n}{(s_n \sqrt{2\pi})}
             \exp\left(-\Half \frac{(\Veps^\Teq_p - \Veps_n)^2}{s_n^2}\right)\,.
\Eeq

\Vaango allows for the possibly of porosity to be different in each MPM particle.
The inputs tags for defining the porosity and its distribution are:
\lstset{language=XML}
\begin{lstlisting}
  <evolve_porosity> true </evolve_porosity>
  <initial_mean_porosity>         0.005 </initial_mean_porosity>
  <initial_std_porosity>          0.001 </initial_std_porosity>
  <critical_porosity>             0.3   </critical_porosity>
  <frac_nucleation>               0.1   </frac_nucleation>
  <meanstrain_nucleation>         0.3   </meanstrain_nucleation>
  <stddevstrain_nucleation>       0.1   </stddevstrain_nucleation>
  <initial_porosity_distrib>      gauss </initial_porosity_distrib>
\end{lstlisting}

\section{Backstress}
The backstress evolution rule can also be expressed in terms of the consistency
parameter in the form
\Beq
   \dot{\Bbeta} = \dot{\lambda} \Bh_\beta
\Eeq
For the Ziegler-Prager model in \eqref{eq:kin_ZP},
\Beq
  \Bh_\beta = \tfrac{2}{3} \beta H \BM
\Eeq
where $\BM$ is the unit tensor in the direction of th eplastic flow rate.

For the Armstrong-Frederick model in \eqref{eq:kin_AF},
\Beq
  \Bh_\beta = \tfrac{2}{3} \beta H_1 \BM - \beta H_2 \Bbeta \,.
\Eeq

\section{Damage}
The evolution of damage models in \Vaango is detailed in Chapter~\ref{ch:damage}.
These models have the general form
\Beq
  \dot{D} = g(\Bsig, T)\,\dot{\Veps}_p^\Teq  = \dot{\lambda} h_D ~,~~ h_D = g(\Bsig,T)
\Eeq
where $D$ is the damage parameter and $g(\Bsig, T)$ is a damage function.

\section{Temperature}
The rise in temperature due to plastic dissipation can also be treated as an
internal variable that causes softening.  This may be considered to
be equivalent to treating the plastic work as an internal variable.

The evolution of temperature ($T$) due to plastic work is given by the equation
\begin{equation}
  \dot{T} = \cfrac{\chi}{\rho C_p} \Bsig:\dot{\BVeps^p} 
\end{equation}
where $\chi$ is the Taylor-Quinney coefficient, $\rho$ is the density, and
$C_p$ is the specific heat.

Expressed in terms of the consistency parameter,
\Beq
  \dot{T} = \dot{\lambda} h_T ~,~~ h_T = \cfrac{\chi}{\rho C_p} \Bsig:\BM 
\Eeq
where $\BM$ is the plastic flow rate direction defined in the flow rule.

