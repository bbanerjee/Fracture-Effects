\chapter{Yield condition}

The yield condition models in \Vaango are of two types: yield conditions that are
tightly tied to material models such as CamClay, Arenisca3, Arena, Mohr-Coulomb 
etc. and those that can be switched in the input file.  This chapter discusses
those yield conditions that can be easily substituted while simulating isotropic
metal plasticity.  The other yield conditions are described in the chapters that
deal with specific models.

\section{von Mises yield}
  The von Mises yield function implemented in \Vaango has the form
  \Beq
    f = \sigma^\xi_{\Teq} - \sigma_y(\Veps^p, \dot{\Veps^p}, \phi, T, \dots)
  \Eeq 
  where $\sigma_y$ is the flow stress, $\Veps^p$ is the equivalent plastic strain,
  $\dot{\Veps^p}$ is the equivalent plastic strain rate, $\phi$ is the
  porosity, and $T$ is the temperature.  The equivalent stress is defined as
  \Beq \label{eq:equiv_stress}
    \sigma^\xi_{\Teq} = \sqrt{3J^\xi_2} = \sqrt{\tfrac{3}{2} \Bxi:\Bxi} ~,\quad
    \Bxi =  \BsT - \Dev(\Bbeta) ~,~~ \BsT = \Bsig - \Third \Tr(\Bsig)\BI
  \Eeq
  where $\Bsig$ is the Cauchy stress and $\Bbeta$ is the kinematic hardening backstress.

  The normal to the yield surface is 
  \Beq
    \BN = \Partial{f}{\Bsig}
        = \Partial{f}{\Bxi}:\Partial{\Bxi}{\Bsig}
        = \Partial{f}{\Bxi}:\Partial{\Bxi}{\BsT}:\Partial{\BsT}{\Bsig}
  \Eeq
  Noting that
  \Beq
    \Partial{\BsT}{\Bsig} = \Tsym(\SfI) - \Third \BI\otimes\BI
    \quad \Tand \quad
    \Partial{\Bxi}{\BsT} = \Tsym(\SfI)
  \Eeq
  where $\SfI$ is the fourth-order identity tensor and $\BI$ is the second-order
  identity tensor, we have
  \Beq \label{eq:df_dxi}
    \Bal
      \BN & = \Partial{f}{\Bxi}:\Tsym(\SfI):(\Tsym(\SfI) - \Third \BI\otimes\BI)
            = \Partial{f}{\Bxi}:(\Tsym(\SfI) - \Third \BI\otimes\BI) \\
          & = \Partial{f}{\Bxi} - \Third \Tr\left(\Partial{f}{\Bxi}\right)\BI \,.
    \Eal
  \Eeq
  Next we compute the derivative of $f$:
  \Beq
    \Partial{f}{\Bxi} = \Partial{f}{\sigma^\xi_\Teq} \Partial{\sigma^\xi_\Teq}{\Bxi}
     = \sqrt{\tfrac{3}{2}} \frac{\Bxi}{\sqrt{\Bxi:\Bxi}} 
    \quad \implies \quad
    \Tr\left(\Partial{f}{\xi}\right) = 0 \,.
  \Eeq
  Therefore,
  \Beq
    \BN = \sqrt{\tfrac{3}{2}} \frac{\Bxi}{\Norm{\Bxi}{}}
  \Eeq
  The unit normal to the yield surface is
  \Beq
    \hat{\BN} = \frac{\Bxi}{\Norm{\Bxi}{}} \,.
  \Eeq
  
  The von Mises yield condition is the default for metal plasticity and
  can be invoked using the tag
  \lstset{language=XML}
  \begin{lstlisting}
    <yield_condition type="von_mises"/>
  \end{lstlisting}

\section{The Gurson-Tvergaard-Needleman (GTN) yield condition}
  The Gurson-Tvergaard-Needleman (GTN) yield 
  condition~\cite{Gurson1977,Tver1984} depends on porosity.  
  The GTN yield function can be written as
  \begin{equation}
    f = \left(\frac{\sigma^\xi_{\Teq}}{\sigma_y}\right)^2 +
    2 q_1 \phi_\star \cosh \left(q_2 \frac{\Tr(\Bsig^\xi)}{2\sigma_y}\right) -
    (1+q_3 \phi_\star^2) 
  \end{equation}
  where $\Bsig^\xi = \Bsig - \Bbeta$, $\Bsig$ is the Cauchy stress,
  $\Bbeta$ is the backstress, 
  $\sigma^\xi_\Teq$ is the equivalent stress defined in \eqref{eq:equiv_stress},
  $\sigma_y$ is the flow stress of the void-free material, 
  $q_1,q_2,q_3$ are material constants, and $\phi_\star$ is the porosity 
  function defined as
  \begin{equation}
    \phi_\star = 
    \begin{cases}
      \phi & \text{for}~~ \phi \le \phi_c,\\ 
      \phi_c + k (\phi - \phi_c) & \text{for}~~ \phi > \phi_c 
    \end{cases}
  \end{equation}
  where $k$ is a constant and $\phi$ is the porosity (void volume fraction).  

  The normal to the yield surface is
  \Beq
    \BN = \Partial{f}{\Bsig}
        = \Partial{f}{\Bxi}:\Partial{\Bxi}{\Bsig} + \Partial{f}{I_1^\xi}\Partial{I_1^\xi}{\Bsig}
        = \Partial{f}{\Bxi}:\Partial{\Bxi}{\BsT}:\Partial{\BsT}{\Bsig} + 
          \Partial{f}{I_1^\xi}\Partial{I_1^\xi}{I_1}\Partial{I_1}{\Bsig}
  \Eeq
  where $I_1 = \Tr(\Bsig)$ and $I_1^\xi = \Tr(\Bsig^\xi) = \Tr(\Bsig - \Bbeta)$.  
  Using \eqref{eq:df_dxi},
  \Beq
    \BN = \Partial{f}{\Bxi} - \Third \Tr\left(\Partial{f}{\Bxi}\right)\BI +
          \Partial{f}{I_1^\xi}\Partial{I_1^\xi}{I_1}\Partial{I_1}{\Bsig} \,.
  \Eeq
  Noting that
  \Beq
    \Partial{I_1^\xi}{I_1} = 1 \quad \Tand \quad
    \Partial{I_1}{\Bsig} = \BI
  \Eeq
  we have
  \Beq
    \BN = \Partial{f}{\Bxi} - \Third \Tr\left(\Partial{f}{\Bxi}\right)\BI +
          \Partial{f}{I_1^\xi} \BI \,.
  \Eeq
  Computation of the derivatives of $f$ gives
  \Beq
    \Partial{f}{\Bxi} = \Partial{f}{\sigma^\xi_\Teq} \Partial{\sigma^\xi_\Teq}{\Bxi}
      = \sqrt{\tfrac{3}{2}} \left(\frac{2\sigma^\xi_{\Teq}}{\sigma_y^2}\right) \frac{\Bxi}{\Norm{\Bxi}{}}
      = \frac{3\Bxi}{\sigma_y^2} 
    \quad \implies \quad
    \Tr\left(\Partial{f}{\xi}\right) = 0 \,.
  \Eeq
  and
  \Beq
    \Partial{f}{I_1^\xi} = 
    \frac{q_1 q_2 \phi_\star}{\sigma_y} \sinh \left(q_2 \frac{\Tr(\Bsig^\xi)}{2\sigma_y}\right)\,.
  \Eeq
  Therefore,
  \Beq
    \BN = \frac{3\Bxi}{\sigma_y^2}
      + \frac{q_1 q_2 \phi_\star}{\sigma_y} \sinh \left(q_2 \frac{\Tr(\Bsig^\xi)}{2\sigma_y}\right) \BI\,.
  \Eeq
  The unit normal to the GTN yield surface is given by
  \Beq
    \hat{\BN} = \frac{\BN}{\Norm{\BN}{}} \,.
  \Eeq
  The GTN yield condition is invoked using
  \lstset{language=XML}
  \begin{lstlisting}
    <yield_condition type="gurson">
      <q1> 1.5 </q1>
      <q2> 1.0 </q2>
      <q3> 2.25 </q3>
      <k> 4.0 </k>
      <f_c> 0.05 </f_c>
    </yield_condition>
  \end{lstlisting}

\section{The Rousselier yield condition}
  The Rousselier yield condition~\cite{Bernauer2002} is another porosity-based yield
  condition that has been used for ductile tearing simulations.

  The yield function is
  \Beq
    f  = \frac{\sigma^\xi_{\Teq}}{1-\phi} +
         D \sigma_1 \phi \exp \left(\frac{\Tr(\Bsig^\xi)}{3(1-\phi)\sigma_1}\right) -
         \sigma_y 
  \Eeq
  where $D,\sigma_1$ are material constants, and the remaining quantities have been
  defined in the previous section.

  The normal to the yield surface is
  \Beq
    \BN = \Partial{f}{\Bxi} - \Third \Tr\left(\Partial{f}{\Bxi}\right)\BI +
          \Partial{f}{I_1^\xi} \BI 
  \Eeq
  where
  \Beq
    \Partial{f}{\Bxi} = \Partial{f}{\sigma^\xi_\Teq} \Partial{\sigma^\xi_\Teq}{\Bxi}
      = \sqrt{\tfrac{3}{2}} \frac{1}{1-\phi} \frac{\Bxi}{\Norm{\Bxi}{}}
    \quad \implies \quad
    \Tr\left(\Partial{f}{\xi}\right) = 0 
  \Eeq
  and
  \Beq
    \Partial{f}{I_1^\xi} = 
      \frac{D\phi}{3(1-\phi)}\exp \left(\frac{\Tr(\Bsig^\xi)}{3(1-\phi)\sigma_1}\right) \,.
  \Eeq
  Therefore,
  \Beq
    \BN = \frac{1}{1-\phi}\left[
       \sqrt{\tfrac{3}{2}} \frac{\Bxi}{\Norm{\Bxi}{}} + 
       \frac{D\phi}{3}\exp \left(\frac{\Tr(\Bsig^\xi)}{3(1-\phi)\sigma_1}\right)\right] \,.
  \Eeq


