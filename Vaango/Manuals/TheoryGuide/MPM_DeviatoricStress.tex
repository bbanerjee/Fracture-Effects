\chapter{Deviatoric stress models} \label{ch:DevStress}

Isotropic plasticity models in \Vaango typically assume hypoelasticity,
for which the stress rate is given by
\Beq
  \dot{\Bsig} = \dot{p}~\BI + \dot{\Bs}
              = \kappa~\Tr(\BdT^e)~\BI + 2~\mu~\Dev(\BdT^e) 
\Eeq
where $\Bsig = p\BI + \Bs$ is the Cauchy stress, $p = \Tr(\Bsig)$, 
$\Bs$ is the deviatoric stress, $\BdT^e$ is the elastic part of
the rate of deformation, and $\kappa, \mu$ are the bulk and shear
moduli.  

The pressure is computed using an equation of state as
described in the chapter~\ref{ch:EOS}.  The deviatoric stress
is computed using the relation
\Beq
  \dot{\Bs} = 2~\mu~\Dev(\BdT^e) \,.
\Eeq

If a forward Euler stress update is used, we have
\Beq
  \Bs_{n+1} = \Bs_{n} + 2\mu\Dev(\BdT^e_{n+1}) \Delta t \,.
\Eeq

For linear elastic materials, the shear modulus can vary with temperature
and pressure.  Several shear modulus models area available in \Vaango
for computing the value for a given state.  

For linear viscoelastic materials to be used with plasticity, a Maxwell 
model is available in \Vaango where the deviatoric stress is computed 
as a sum of Maxwell elements:
\Beq
  \Bs_{n+1} = \Bs_{n} + 2\sum_j\mu_j\Dev(\BdT^e_{n+1}) \Delta t \,.
\Eeq

\section{Shear modulus models} \label{sec:ModelShear}

Shear modulus models that are available in \Vaango are described below.

  \subsection{Constant shear modulus}
  The default model gives a constant shear modulus.  The model is 
  invoked using
  \lstset{language=XML}
  \begin{lstlisting}
    <shear_modulus_model type="constant_shear">
      <shear_modulus> 1.0e8 </shear_modulus>
    </shear_modulus_model>
  \end{lstlisting}

  \subsection{Mechanical Threshold Stress shear modulus}
  The simplest model is of the form suggested by \cite{Varshni1970} 
  (\cite{Chen1996})
  \begin{equation} \label{eq:MTSShear}
    \mu(T) = \mu_0 - \frac{D}{exp(T_0/T) - 1}
  \end{equation}
  where $\mu_0$ is the shear modulus at 0K, and $D, T_0$ are material
  constants. 

  The model is invoked using
  \lstset{language=XML}
  \begin{lstlisting}
    <shear_modulus_model type="mts_shear">
      <mu_0>28.0e9</mu_0>
      <D>4.50e9</D>
      <T_0>294</T_0>
    </shear_modulus_model>
  \end{lstlisting}

  \subsection{SCG shear modulus}
  The Steinberg-Cochran-Guinan (SCG) shear modulus 
  model (\cite{Steinberg1980,Zocher2000}) is pressure dependent and
  has the form
  \begin{equation} \label{eq:SCGShear}
    \mu(p,T) = \mu_0 + \Partial{\mu}{p} \frac{p}{\eta^{1/3}} +
         \Partial{\mu}{T}(T - 300) ; \quad
    \eta = \rho/\rho_0
  \end{equation}
  where, $\mu_0$ is the shear modulus at the reference state($T$ = 300 K, 
  $p$ = 0, $\eta$ = 1), $p$ is the pressure, and $T$ is the temperature.
  When the temperature is above $T_m$, the shear modulus is instantaneously
  set to zero in this model.

  The model is invoked using
  \lstset{language=XML}
  \begin{lstlisting}
  <shear_modulus_model type="scg_shear">
    <mu_0> 81.8e9 </mu_0>
    <A> 20.6e-12 </A>
    <B> 0.16e-3 </B>
  </shear_modulus_model>
  \end{lstlisting}

  \subsection{Nadal-LePoac (NP) shear modulus}
  A modified version of the SCG model has been developed by 
  \cite{Nadal2003} that attempts to capture the sudden drop in the
  shear modulus close to the melting temperature in a smooth manner.
  The Nadal-LePoac (NP) shear modulus model has the form
  \begin{equation} \label{eq:NPShear}
    \mu(p,T) = \frac{1}{\mathcal{J}(\That)}
      \left[
        \left(\mu_0 + \Partial{\mu}{p} \cfrac{p}{\eta^{1/3}} \right)
        (1 - \That) + \frac{\rho}{Cm}~k_b~T\right]; \quad
    C := \cfrac{(6\pi^2)^{2/3}}{3} f^2
  \end{equation}
  where
  \begin{equation}
    \mathcal{J}(\That) := 1 + \exp\left[-\cfrac{1+1/\zeta}
        {1+\zeta/(1-\That)}\right] \quad
       \text{for} \quad \That:=\frac{T}{T_m}\in[0,1+\zeta],
  \end{equation}
  $\mu_0$ is the shear modulus at 0 K and ambient pressure, $\zeta$ is
  a material parameter, $k_b$ is the Boltzmann constant, $m$ is the atomic
  mass, and $f$ is the Lindemann constant.

  The model is invoked using
  \lstset{language=XML}
  \begin{lstlisting}
    <shear_modulus_model type="np_shear">
      <mu_0>26.5e9</mu_0>
      <zeta>0.04</zeta>
      <slope_mu_p_over_mu0>65.0e-12</slope_mu_p_over_mu0>
      <C> 0.047 </C>
      <m> 26.98 </m>
    </shear_modulus_model>
  \end{lstlisting}
  
  \subsection{Preston-Tonks-Wallace (PTW) shear modulus}
  The PTW shear model~\cite{Preston2003} is a simplified version of the SCG shear model.
  This model suggests computing the shear modulus using
  \Beq
    \mu(p, T) = \mu_0 \left(1 + \beta \frac{\pbar}{\eta^{1/3}}\right) \left(1 - \alpha_p \frac{T}{T_m}\right)
  \Eeq
  where $\mu_0$ is the shear modulus at room temperature and pressure, $\pbar = -p$,
  $\alpha_p$ is a material parameter, $T_m$ is the melting temperature, and
  \Beq
    \eta = \frac{\rho}{\rho_0} ~,~~ \beta = \frac{\Deriv{\mu}{p}}{\mu_0} \,.
  \Eeq

  \subsection{Borja's shear modulus model}
  Borja's deviatoric stress model~\cite{borja1998} assumes that the deviatoric part of
  the elastic strain energy density has the form
  \Beq
    W_\Tdev(\Ve^e_v,\Ve^e_s) =  \tfrac{3}{2}\,\mu\,(\Ve^e_s)^2
  \Eeq
  where $\Ve^e_v = \Tr(\Beps^e_v)$ is the volumetric part of the elastic strain, 
  $\Ve^e_s = \sqrt{2/3 \Dev(\Beps^e_v):\Dev(\Beps^e_v)}$ is the 
  deviatoric part of the elastic strain, and $\mu$ is the shear modulus.

  The shear modulus in the Borja model is computed as
  \Beq
    \mu(p) = \mu_0 - \alpha p_0\,\exp\left(-\frac{\Ve^e_v - \Ve^e_{v0}}{\kappatilde}\right) 
  \Eeq
  where $\mu_0$ is a reference shear modulus, $\Ve^e_{v0}$ is the volumetric strain corresponding 
  to a mean normal compressive stress $p_0$, and $\kappatilde$ is the elastic compressibility index.
  

