\chapter{Isotropic metal plasticity}

This section describes the current implementation of the hypoelastic-
plastic model.  The stress update algorithm is a slightly modified 
version of the approach taken by 
Nemat-Nasser et al. (1991,1992)~\cite{Nemat1991,Nemat1992}, Wang (1994)~\cite{Wang1994}, 
Maudlin (1996)~\cite{Maudlin1996}, and Zocher et al. (2000)~\cite{Zocher2000}.  

\subsection{Simplified theory for hypoelastic-plasticity}
A simplified version of the theory behind the stress update algorithm
(in the contex of von Mises plasticity) is given below.

  Following ~\cite{Maudlin1996}, the rotated spatial rate of deformation 
  tensor ($\Bd$) is decomposed into an elastic part ($\Bd^e$) and a 
  plastic part ($\Bd^p$)
  \begin{equation}
     \Bd = \Bd^e + \Bd^p
  \end{equation}
  If we assume plastic incompressibility ($\Tr{(\Bd^p)} = 0$), we get
  \begin{equation}
     \Beta = \Beta^e + \Beta^p
  \end{equation}
  where $\Beta$, $\Beta^e$, and $\Beta^p$ are the deviatoric parts of $\Bd$,
  $\Bd^e$, and $\Bd^p$, respectively.  For isotropic materials, the hypoelastic
  constitutive equation for deviatoric stress is
  \begin{equation}
    \dot{\Bs} = 2\mu(\Beta - \Beta^p)
  \end{equation}
  where $\Bs$ is the deviatoric part of the stress tensor and $\mu$ is the
  shear modulus.  We assume that the flow stress obeys the Huber-von Mises
  yield condition
  \begin{equation}
    f := \sqrt{\frac{3}{2}}\norm{\Bs} - \sigma_y \le 0  ~~\text{or,}~~
    F := \frac{3}{2} \Bs:\Bs - \sigma_y^2 \le 0 
  \end{equation}
  where $\sigma_y$ is the flow stress.  Assuming an associated flow rule,
  and noting that $\Bd^p = \Beta^p$, we have
  \begin{equation}
    \Beta^p = \Bd^p = \lambda\Partial{f}{\Bsig} 
                    = \Lambda\Partial{F}{\Bsig} = 3\Lambda\Bs
  \end{equation}
  where $\Bsig$ is the stress.  Let $\Bu$ be a tensor proportional to the 
  plastic straining direction, and define $\gamma$ as
  \begin{equation}
    \Bu = \sqrt{3} \frac{\Bs}{\norm{\Bs}}; \quad
    \gamma := \sqrt{3}\Lambda\norm{\Bs}  \quad \Longrightarrow
    \gamma\Bu = 3\Lambda\Bs
  \end{equation}
  Therefore, we have
  \begin{equation} \label{eq:stresseqn}
    \Beta^p = \gamma\Bu; \quad  
    \dot{\Bs} = 2\mu(\Beta - \gamma\Bu)
  \end{equation}
  From the consistency condition, if we assume that the deviatoric stress
  remains constant over a timestep, we get 
  \begin{equation}
    \gamma = \frac{\Bs:\Beta}{\Bs:\Bu}
  \end{equation}
  which provides an initial estimate of the plastic strain-rate.  To obtain
  a semi-implicit update of the stress using equation (\ref{eq:stresseqn}), we
  define
  \begin{equation}\label{eq:taueqn}
    \tau^2 := \frac{3}{2} \Bs:\Bs = \sigma_y^2
  \end{equation}
  Taking a time derivative of equation (\ref{eq:taueqn}) gives us
  \begin{equation}\label{eq:taudot}
    \sqrt{2} \dot{\tau} = \sqrt{3} \frac{\Bs:\dot{\Bs}}{\norm{\Bs}}
  \end{equation}
  Plugging equation (\ref{eq:taudot}) into equation (\ref{eq:stresseqn})$_2$
  we get
  \begin{equation}\label{eq:tau}
    \dot{\tau} = \sqrt{2}\mu(\Bu:\Beta - \gamma\Bu:\Bu)
               = \sqrt{2} \mu (d - 3\gamma)
  \end{equation} 
  where $d = \Bu:\Beta$.  If the initial estimate of the plastic strain-rate
  is that all of the deviatoric strain-rate is plastic, then we get an 
  approximation to $\gamma$, and the corresponding error 
  ($\gamma_{\text{er}}$) given by
  \begin{equation}\label{eq:gammaer}
    \gamma_{\text{approx}} = \frac{d}{3}; \quad
    \gamma_{\text{er}} = \gamma_{\text{approx}} - \gamma = \frac{d}{3} - \gamma
  \end{equation}
  The incremental form of the above equation is
  \begin{equation}\label{eq:delgamma}
    \Delta\gamma = \frac{d^*\Delta t}{3} - \Delta\gamma_{\text{er}}
  \end{equation}
  Integrating equation (\ref{eq:tau}) from time $t_n$ to time $t_{n+1} = 
  t_n + \Delta t$, and using equation (\ref{eq:delgamma}) we get
  \begin{equation}\label{eq:taun}
    \tau_{n+1} = \tau_n + \sqrt{2}\mu(d^*\Delta t - 3\Delta\gamma)
               = \tau_n + 3\sqrt{2}\mu\Delta\gamma_{\text{er}}
  \end{equation}
  where $d^*$ is the average value of $d$ over the timestep.
  Solving for $\Delta\gamma_{\text{er}}$ gives
  \begin{equation}\label{eq:delgammaer}
    \Delta\gamma_{\text{er}} = \cfrac{\tau_{n+1} - \tau_n}{3\sqrt{2}\mu}
      = \cfrac{\sqrt{2}\sigma_y - \sqrt{3}\norm{\Bs_n}}{6\mu}
  \end{equation}
  The direction of the total strain-rate ($\Bu^{\eta}$) and the
  direction of the plastic strain-rate ($\Bu^s$) are given by 
  \begin{equation}
    \Bu^{\eta} = \frac{\Beta}{\norm{\Beta}} ; \quad
    \Bu^{s} = \frac{\Bs}{\norm{\Bs}} 
  \end{equation}
  Let $\theta$ be the fraction of the time increment that sees elastic
  straining.  Then
  \begin{equation}\label{eq:theta}
    \theta = \frac{d^* - 3\gamma_n}{d^*}
  \end{equation}
  where $\gamma_n = d_n/3$ is the value of $\gamma$ at the beginning of the 
  timestep.  We also assume that 
  \begin{equation}\label{eq:dstar}
    d^* = \sqrt{3}\Beta:[(1-\theta)\Bu^{\eta} + \frac{\theta}{2}
              (\Bu^{\eta}+\Bu^{s})]
  \end{equation}
  Plugging equation (\ref{eq:theta}) into equation (\ref{eq:dstar}) we get
  a quadratic equation that can be solved for $d^*$ as follows
  \begin{equation}
    \frac{2}{\sqrt{3}} (d^*)^2 - (\Beta:\Bu^s + \norm{\Beta}) d^*
       + 3\gamma_n (\Beta:\Bu^s - \norm{\Beta}) = 0
  \end{equation}
  The real positive root of the above quadratic equation is taken as the
  estimate for $d$.  The value of $\Delta\gamma$ can now be calculated
  using equations (\ref{eq:delgamma}) and (\ref{eq:delgammaer}).  A
  semi-implicit estimate of the deviatoric stress can be obtained at this
  stage by integrating equation (\ref{eq:stresseqn})$_2$
  \begin{align}
    \tilde{\Bs}_{n+1} & = \Bs_n + 2\mu\left(\eta\Delta t  - \sqrt{3}\Delta\gamma
         \cfrac{\tilde{\Bs}_{n+1}}{\norm{\Bs_{n+1}}}\right) \\
     & = \Bs_n + 2\mu\left(\eta\Delta t  - \frac{3}{\sqrt{2}}\Delta\gamma
         \cfrac{\tilde{\Bs}_{n+1}}{\sigma_y}\right)
  \end{align}
  Solving for $\tilde{\Bs}_{n+1}$, we get
  \begin{equation}
    \tilde{\Bs}_{n+1} = \cfrac{\Bs_{n+1}^{\text{trial}}}
           {1 + 3\sqrt{2}\mu\cfrac{\Delta\gamma}{\sigma_y}}
  \end{equation}
  where $\Bs_{n+1}^{\text{trial}} = \Bs_n + 2\mu\Delta t\Beta$.
  A final radial return adjustment is used to move the stress to the yield
  surface 
  \begin{equation}
    \Bs_{n+1} = \sqrt{\frac{2}{3}}\sigma_y \cfrac{\tilde{\Bs}_{n+1}}
                {\norm{\tilde{\Bs}_{n+1}}}
  \end{equation}
  A pathological situation arises if $\gamma_n = \Bu_n:\Beta_n$ is less
  than or equal to zero or 
  $\Delta\gamma_{\text{er}} \ge \frac{d^*}{3} \Delta t $.  
  This can occur is the rate of plastic deformation
  is small compared to the rate of elastic deformation or if the timestep
  size is too small (see~\cite{Nemat1992}).  In such situations, we use a
  locally implicit stress update that uses Newton iterations (as
  discussed in \cite{Simo1998}, page 124) to compute $\tilde{\Bs}$.

\section{Models}
  Below are some of the strain-rate, strain, and temperature dependent models 
  for metals that are implemented in Vaango.


\subsection{Melting Temperature}
  \paragraph{Default model}
  The default model is to use a constant melting temperature.  This model
  is invoked using
  \lstset{language=XML}
  \begin{lstlisting}  
      <melting_temp_model type="constant_Tm">
      </melting_temp_model>
  \end{lstlisting}  
  
  \paragraph{SCG melt model}
  We use a pressure dependent relation to determine the melting 
  temperature ($T_m$).  The Steinberg-Cochran-Guinan (SCG) melt model 
  (\cite{Steinberg1980}) has been used for our simulations of copper.  
  This model is based on a modified Lindemann law and has the form
  \begin{equation} \label{eq:TmSCG}
    T_m(\rho) = T_{m0} \exp\left[2a\left(1-\frac{1}{\eta}\right)\right]
              \eta^{2(\Gamma_0-a-1/3)}; \quad
    \eta = \frac{\rho}{\rho_0}
  \end{equation}
  where $T_{m0}$ is the melt temperature at $\eta = 1$, 
  $a$ is the coefficient of the first order volume correction to 
  Gr{\"u}neisen's gamma ($\Gamma_0$).  

  This model is invoked with
  \lstset{language=XML}
  \begin{lstlisting}
    <melting_temp_model type="scg_Tm">
      <T_m0> 2310.0 </T_m0>
      <Gamma_0> 3.0 </Gamma_0>
      <a> 1.67 </a>
    </melting_temp_model>
  \end{lstlisting}

  \paragraph{BPS melt model}
  An alternative melting relation that is based on dislocation-mediated
  phase transitions - the Burakovsky-Preston-Silbar (BPS) model
  (\cite{Burakovsky2000}) can also be used.  This model has been used to 
  determine the melt temperature for 4340 steel.  The BPS model has the form
  \begin{align}
    T_m(p) & = T_m(0)
      \left[\cfrac{1}{\eta} + 
            \cfrac{1}{\eta^{4/3}}~\cfrac{\mu_0^{'}}{\mu_0}~p\right]~; 
    \quad
    \eta = \left(1 + \cfrac{K_0^{'}}{K_0}~p\right)^{1/K_0^{'}} 
    \label{eq:TmBPS}\\
    T_m(0) & = \cfrac{\kappa\lambda\mu_0~v_{WS}}{8\pi\ln(z-1)~k_b}
               \ln\left(\cfrac{\alpha^2}{4~b^2\rho_c(T_m)}\right)
  \end{align}
  where $p$ is the pressure, $\eta = \rho/\rho_0$ is the compression, 
  $\mu_0$ is the shear modulus at room temperature and zero pressure, 
  $\mu_0^{'} = \partial\mu/\partial p$ is the derivative of the shear modulus 
  at zero pressure, $K_0$ is the bulk modulus at room temperature and
  zero pressure, $K_0^{'} = \partial K/\partial p$ is the derivative of the 
  bulk modulus at zero pressure, $\kappa$ is a constant, $\lambda = b^3/v_{WS}$
  where $b$ is the magnitude of the Burgers' vector, $v_{WS}$ is the 
  Wigner-Seitz volume, $z$ is the coordination number, $\alpha$ is a 
  constant, $\rho_c(T_m)$ is the critical density of dislocations, and
  $k_b$ is the Boltzmann constant.

  This model is invoked with
  \lstset{language=XML}
  \begin{lstlisting}
    <melting_temp_model type="bps_Tm">
      <B0> 137e9 </B0>
      <dB_dp0> 5.48 <dB_dp0>
      <G0> 47.7e9 <G0>
      <dG_dp0> 1.4 <dG_dp0>
      <kappa> 1.25 <kappa>
      <z> 12 <z>
      <b2rhoTm> 0.64 <b2rhoTm>
      <alpha> 2.9 <alpha>
      <lambda> 1.41 <lambda>
      <a> 3.6147e-9<a>
      <v_ws_a3_factor> 1/4 <v_ws_a3_factor>
      <Boltzmann_Constant> <Boltzmann_Constant>
    </melting_temp_model>
  \end{lstlisting}

  \subsection{Adiabatic Heating and Specific Heat}
  A part of the plastic work done is converted into heat and used to update the 
  temperature of a particle.  The increase in temperature ($\Delta T$) due to 
  an increment in plastic strain ($\Delta\epsilon_p$) is given by the equation
  \begin{equation}
    \Delta T = \cfrac{\chi\sigma_y}{\rho C_p} \Delta \epsilon_p
  \end{equation}
  where $\chi$ is the Taylor-Quinney coefficient, and $C_p$ is the specific
  heat.  The value of the Taylor-Quinney coefficient is taken to be 0.9
  in all our simulations (see \cite{Ravi2001} for more details on the 
  variation of $\chi$ with strain and strain rate).

  The Taylor-Quinney coefficient is taken as input in the ElasticPlastic model
  using the tags
  \lstset{language=XML}
  \begin{lstlisting}
    <taylor_quinney_coeff> 0.9 </taylor_quinney_coeff>
  \end{lstlisting}

  \paragraph{Default specific heat model}
  The default model returns a constant specific heat and is invoked using
  \lstset{language=XML}
  \begin{lstlisting}
    <specific_heat_model type="constant_Cp">
    </specific_heat_model>
  \end{lstlisting}

  \paragraph{Specific heat model for copper}
  The specific heat model for copper is of the form
  \Beq
    Cp = 
    \begin{cases}
      A_0~T^3 - B_0~T^2 + C_0~T - D_0 & \text{if} ~~T < T_0 \\
      A_1~T + B_1 & \text{if} ~~T \ge T_0 ~.
    \end{cases}
  \Eeq
  The model is invoked using
  \lstset{language=XML}
  \begin{lstlisting}
  <specific_heat_model type = "copper_Cp"> </specific_heat_model>
  \end{lstlisting}

  \paragraph{Specific heat model for steel}
  A relation for the dependence of $C_p$ upon temperature is
  used for the steel (\cite{Lederman1974}).
  \begin{align}
    C_p & = \begin{cases}
            A_1 + B_1~t + C_1~|t|^{-\alpha} & \text{if}~~ T < T_c \\
            A_2 + B_2~t + C_2~t^{-\alpha^{'}} & \text{if}~~ T > T_c 
          \end{cases} \label{eq:CpSteel}\\
    t & = \cfrac{T}{T_c} - 1 
  \end{align}
  where $T_c$ is the critical temperature at which the phase transformation
  from the $\alpha$ to the $\gamma$ phase takes place, and $A_1, A_2, B_1, B_2,
  \alpha, \alpha^{'}$ are constants.

  The model is invoked using
  \lstset{language=XML}
  \begin{lstlisting}
  <specific_heat_model type = "steel_Cp"> </specific_heat_model>
  \end{lstlisting}

  The heat generated at a material point is conducted away at the end of a 
  time step using the transient heat equation.  The effect of conduction on
  material point temperature is negligible (but non-zero) for the high 
  strain-rate problems simulated using Vaango.

\subsection{Adding new models}
  In the parallel implementation of the stress update algorithm, sockets have 
  been added to allow for the incorporation of a variety of plasticity, damage, 
  yield, and bifurcation models without requiring any change in the stress 
  update code.  The algorithm is shown in Algorithm~\ref{algo1}.  The
  equation of state, plasticity model, yield condition, damage model, and
  the stability criterion are all polymorphic objects created using a 
  factory idiom in C++~(\cite{Coplien1992}).
  \begin{table}[p]
    \caption{Stress Update Algorithm} \label{algo1}
    \vspace{12pt}
    \begin{tabbing}
    \quad \=\quad \=\quad \=\quad \=\quad \kill
    {\bf Persistent}:Initial moduli, temperature, porosity, \\
      \>\>        scalar damage, equation of state, plasticity model, \\
      \>\>        yield condition, stability criterion, damage model\\
    {\bf Temporary}:Particle state at time $t$ \\
    {\bf Output:} Particle state at time $t+\Delta t$\\ \\

    {\bf For} {\it all the patches in the domain}\\
      \> Read the particle data and initialize updated data storage\\
      \> {\bf For} {\it all the particles in the patch}\\ 
      \>\>   Compute the velocity gradient and the rate of deformation tensor\\ 
      \>\>   Compute the deformation gradient and the rotation tensor\\
      \>\>   Rotate the Cauchy stress and the rate of deformation tensor \\
      \>\>\> to the material configuration\\
      \>\>   Compute the current shear modulus and melting temperature\\
      \>\>   Compute the pressure using the equation of state,  \\
      \>\>\>  update the hydrostatic stress, and  \\
      \>\>\>  compute the trial deviatoric stress\\
      \>\>   Compute the flow stress using the plasticity model\\
      \>\>   Evaluate the yield function\\
      \>\>   {\bf If} {\it particle is elastic} \\
      \>\>\>     Update the elastic deviatoric stress from the trial stress\\
      \>\>\>     Rotate the stress back to laboratory coordinates\\
      \>\>\>     Update the particle state\\
      \>\>   {\bf Else} \\
      \>\>\>     Compute the elastic-plastic deviatoric stress\\
      \>\>\>     Compute updated porosity, scalar damage, and \\
      \>\>\>\>       temperature increase due to plastic work\\
      \>\>\>     Compute elastic-plastic tangent modulus and 
                     evaluate stability condition\\
      \>\>\>     Rotate the stress back to laboratory coordinates\\
      \>\>\>     Update the particle state\\
      \>\>   {\bf End If} \\
      \>\>  {\bf If} 
             {\it Temperature $>$ Melt Temperature} or 
             {\it Porosity $>$ Critical Porosity} or
             {\it Unstable}\\
      \>\>\>       Tag particle as failed\\
      \>\>  {\bf End If} \\
      \>\> Convert failed particles into a material with a different 
           velocity field \\
      \> {\bf End For} \\
    {\bf End For}
    \end{tabbing}
  \end{table}
      
Addition of a new model requires the following steps (the example below is only
for the flow stress model but the same idea applies to other models) :
\begin{enumerate}
    \item Creation of a new class that encapsulates the plasticity 
    model.  The template for this class can be copied from the
    existing plasticity models.  The data that is unique to 
    the new model are specified in the form of 
    \begin{itemize}
      \item A structure containing the constants for the plasticity
            model.
      \item Particle variables that specify the variables that 
            evolve in the plasticity model.
    \end{itemize}
    \item The implementation of the plasticity model involves the
    following steps.
    \begin{itemize}
      \item Reading the input file for the model constants in the
            constructor.
      \item Adding the variables that evolve in the plasticity model
            appropriately to the task graph.
      \item Adding the appropriate flow stress calculation method.
    \end{itemize}
    \item The \verb+PlasticityModelFactory+ is then modified so that
          it recognizes the added plasticity model.
\end{enumerate}

\subsection{Damage Models and Failure}
Only the Johnson-Cook damage evolution rule has been added to the 
DamageModelFactory so far.  The damage model framework is designed 
to be similar to the plasticity model framework.  New models can
be added using the approach described in the previous section.

  A particle is tagged as ``failed'' when its temperature is greater than the
  melting point of the material at the applied pressure.  An additional
  condition for failure is when the porosity of a particle increases beyond a
  critical limit and the strain exceeds the fracture strain of the material.  

  Another condition for failure is when a material bifurcation 
  condition such as the Drucker stability postulate is satisfied.  Upon failure,
  a particle is either removed from the computation by setting the stress to
  zero or is converted into a material with a different velocity field 
  which interacts with the remaining particles via contact.  Either approach
  leads to the simulation of a newly created surface.  More details of the
  approach can be found in ~\cite{Banerjee2004a,Banerjee2004c,Banerjee2005}.

  \subsection{Porosity model}
  The evolution of porosity is calculated as the sum of the rate of growth 
  and the rate of nucleation~\cite{Ramaswamy1998a}.  The rate of growth of
  porosity and the void nucleation rate are given by the following equations
  ~\cite{Chu1980}
  \begin{align}
    \dot{f} &= \dot{f}_{\text{nucl}} + \dot{f}_{\text{grow}} \\
    \dot{f}_{\text{grow}} & = (1-f) \text{Tr}(\BD_p) \\
    \dot{f}_{\text{nucl}} & = \cfrac{f_n}{(s_n \sqrt{2\pi})}
            \exp\left[-\Half \cfrac{(\epsilon_p - \epsilon_n)^2}{s_n^2}\right]
            \dot{\epsilon}_p
  \end{align}
  where $\BD_p$ is the rate of plastic deformation tensor, $f_n$ is the volume 
  fraction of void nucleating particles , $\epsilon_n$ is the mean of the 
  distribution of nucleation strains, and $s_n$ is the standard 
  deviation of the distribution.

  The inputs tags for porosity are of the form
  \lstset{language=XML}
  \begin{lstlisting}
    <evolve_porosity> true </evolve_porosity>
    <initial_mean_porosity>         0.005 </initial_mean_porosity>
    <initial_std_porosity>          0.001 </initial_std_porosity>
    <critical_porosity>             0.3   </critical_porosity>
    <frac_nucleation>               0.1   </frac_nucleation>
    <meanstrain_nucleation>         0.3   </meanstrain_nucleation>
    <stddevstrain_nucleation>       0.1   </stddevstrain_nucleation>
    <initial_porosity_distrib>      gauss </initial_porosity_distrib>
  \end{lstlisting}

  \subsection{Damage model}
  After the stress state has been determined on the basis of the yield condition
  and the associated flow rule, a scalar damage state in each material point can
  be calculated using the Johnson-Cook model ~\cite{Johnson1985}.  
  The Johnson-Cook model has an explicit dependence on temperature, plastic
  strain, ans strain rate.

  The damage evolution rule for the Johnson-Cook damage model can be written as
  \begin{equation}
    \dot{D} = \cfrac{\dot{\epsilon_p}}{\epsilon_p^f} ~;~~
    \epsilon_p^f = 
      \left[D_1 + D_2 \exp \left(\cfrac{D_3}{3} \sigma^*\right)\right]
      \left[1+ D_4 \ln(\dot{\epsilon_p}^*)\right]
      \left[1+D_5 T^*\right]~;~~
    \sigma^*= \cfrac{\text{Tr}(\Bsig)}{\sigma_{eq}}~;~~
  \end{equation}
  where $D$ is the damage variable which has a value of 0 for virgin material
  and a value of 1 at fracture, $\epsilon_p^f$ is the fracture strain, 
  $D_1, D_2, D_3, D_4, D_5$ are constants, $\Bsig$ is the Cauchy stress, and
  $T^*$ is the scaled temperature as in the Johnson-Cook plasticity model.

  The input tags for the damage model are :
  \lstset{language=XML}
  \begin{lstlisting}
    <damage_model type="johnson_cook">
      <D1>0.05</D1>
      <D2>3.44</D2>
      <D3>-2.12</D3>
      <D4>0.002</D4>
      <D5>0.61</D5>
    </damage_model>
  \end{lstlisting}

  An initial damage distribution can be created using the following tags
  \lstset{language=XML}
  \begin{lstlisting}
    <evolve_damage>                 true  </evolve_damage>
    <initial_mean_scalar_damage>    0.005  </initial_mean_scalar_damage>
    <initial_std_scalar_damage>     0.001 </initial_std_scalar_damage>
    <critical_scalar_damage>        1.0   </critical_scalar_damage>
    <initial_scalar_damage_distrib> gauss </initial_scalar_damage_distrib>
  \end{lstlisting}

  \subsection{Erosion algorithm}
  Under normal conditions, the heat generated at a material point is conducted 
  away at the end of a time step using the heat equation.  If special adiabatic 
  conditions apply (such as in impact problems), the heat is accumulated at a 
  material point and is not conducted to the surrounding particles.  This 
  localized heating can be used to determine whether a material point has
  melted.

  The determination of whether a particle has failed can be made on the 
  basis of either or all of the following conditions:
  \begin{itemize}
    \item The particle temperature exceeds the melting temperature.
    \item The TEPLA-F fracture condition~\cite{Johnson1988} is satisfied.
       This condition can be written as
       \begin{equation}
         (f/f_c)^2 + (\epsilon_p/\epsilon_p^f)^2 = 1
       \end{equation}
       where $f$ is the current porosity, $f_c$ is the maximum 
       allowable porosity, $\epsilon_p$ is the current plastic strain, and
       $\epsilon_p^f$ is the plastic strain at fracture.
    \item An alternative to ad-hoc damage criteria is to use the concept of 
       bifurcation to determine whether a particle has failed or not.  Two
       stability criteria have been explored in this paper - the Drucker
       stability postulate~\cite{Drucker1959} and the loss of hyperbolicity
       criterion (using the determinant of the acoustic tensor)
       \cite{Rudnicki1975,Perzyna1998}.  
  \end{itemize}

  The simplest criterion that can be used is the Drucker stability postulate 
  \cite{Drucker1959} which states that time rate of change of the rate of 
  work done by a material cannot be negative.  Therefore, the material is 
  assumed to become unstable (and a particle fails) when
  \begin{equation}
    \dot\Bsig:\BD^p \le 0
  \end{equation}

  Another stability criterion that is less restrictive is the acoustic
  tensor criterion which states that the material loses stability if the 
  determinant of the acoustic tensor changes sign~\cite{Rudnicki1975,Perzyna1998}.  
  Determination of the acoustic tensor requires a search for a normal vector 
  around the material point and is therefore computationally expensive.  A 
  simplification of this criterion is a check which assumes that the direction 
  of instability lies in the plane of the maximum and minimum principal 
  stress~\cite{Becker2002}.  In this approach, we assume that the strain is 
  localized in a band with normal $\Bn$, and the magnitude of the velocity 
  difference across the band is $\Bg$.  Then the bifurcation condition 
  leads to the relation 
  \begin{equation} 
    R_{ij} g_{j} = 0 ~;~~~
    R_{ij} = M_{ikjl} n_k n_l + M_{ilkj} n_k n_l - \sigma_{ik} n_j n_k
  \end{equation} 
  where $M_{ijkl}$ are the components of the co-rotational tangent
  modulus tensor and $\sigma_{ij}$ are the components of the co-rotational 
  stress tensor.  If $\det(R_{ij}) \le 0 $, then $g_j$ can be arbitrary and 
  there is a possibility of strain localization.  If this condition for 
  loss of hyperbolicity is met,  then a particle deforms in an unstable 
  manner and failure can be assumed to have occurred at that particle.  
  We use a combination of these criteria to simulate failure.

  Since the material in the container may unload locally after fracture, the 
  hypoelastic-plastic stress update may not work accurately under certain 
  circumstances.  An improvement would be to use a hyperelastic-plastic stress 
  update algorithm.  Also, the plasticity models are temperature dependent.
  Hence there is the issue of severe mesh dependence due to change of the
  governing equations from hyperbolic to elliptic in the softening regime
  ~\cite{Hill1975,Bazant1985,Tver1990}.  Viscoplastic stress update models or 
  nonlocal/gradient plasticity models~\cite{Ramaswamy1998,Hao2000} can be used 
  to eliminate some of these effects and are currently under investigation. 

  The tags used to control the erosion algorithm are in two places.  
  In the \verb|<MPM> </MPM>| section the following flags can be set
  \lstset{language=XML}
  \begin{lstlisting}
    <erosion algorithm = "ZeroStress"/>
    <create_new_particles>           false      </create_new_particles>
    <manual_new_material>            false      </manual_new_material>
  \end{lstlisting}
  If the erosion algorithm is \verb|"none"| then no particle failure is done.

  In the \verb|<constitutive_model type="elastic_plastic">| section, the 
  following flags can be set
  \lstset{language=XML}
  \begin{lstlisting}
    <evolve_porosity>               true  </evolve_porosity>
    <evolve_damage>                 true  </evolve_damage>
    <do_melting>                    true  </do_melting>
    <useModifiedEOS>                true  </useModifiedEOS>
    <check_TEPLA_failure_criterion> true  </check_TEPLA_failure_criterion>
    <check_max_stress_failure>      false </check_max_stress_failure>
    <critical_stress>              12.0e9 </critical_stress>
  \end{lstlisting}

\section{Implementation}
The elastic response is assumed to be isotropic.  The material
constants that are taken as input for the elastic response are the
bulk and shear modulus.  The flow rule is determined from the input
and the appropriate plasticity model is created using the 
\verb+PlasticityModelFactory+ class.  The damage evolution rule
is determined from the input and a damage model is created using
the \verb+DamageModelFactory+ class.  The equation of state 
that is used to determine the pressure is also determined from the
input.  The equation of state model is created using the 
\verb+MPMEquationOfStateFactory+ class.

In addition, a damage evolution variable ($D$) is stored at each time 
step (this need not be the case and will be transfered to the 
damage models in the future).  The left stretch and rotation are 
updated incrementally at each
time step (instead of performing a polar decomposition) and the 
rotation tensor is used to rotate the Cauchy stress and rate of deformation
to the material coordinates at each time step (instead of using a 
objective stress rate formulation).  

Any evolution variables for the plasticity model, damage model or the
equation of state are specified in the class that encapsulates the 
particular model.  

The flow stress is calculated from the plasticity model using a 
function call of the form
\lstset{language=C++}
\begin{lstlisting}
    double flowStress = d_plasticity->computeFlowStress(tensorEta, tensorS, 
                                                        pTemperature[idx],
                                                        delT, d_tol, matl, idx);
\end{lstlisting}
A number of plasticity models can be evaluated using the inputs in the
\verb+computeFlowStress+ call.  The variable \verb+d_plasticity+ is
polymorphic and can represent any of the plasticity models that can be
created by the plasticity model factory.  The plastic evolution variables
are updated using a polymorphic function along the lines of
\verb+computeFlowStress+.

The equation of state is used to calculate the hydrostatic stress using
a function call of the form
\lstset{language=C++}
\begin{lstlisting}
    Matrix3 tensorHy = d_eos->computePressure(matl, bulk, shear, 
                                              tensorF_new, tensorD, 
                                              tensorP, pTemperature[idx], 
                                              rho_cur, delT);
\end{lstlisting}

Similarly, the damage model is called using a function of the type
\lstset{language=C++}
\begin{lstlisting}
    double damage = d_damage->computeScalarDamage(tensorEta, tensorS, 
                                                  pTemperature[idx],
                                                  delT, matl, d_tol, 
                                                  pDamage[idx]);
\end{lstlisting}

Therefore, the plasticity, damage and equation of state models are 
easily be inserted into any other type of stress update algorithm 
without any change being needed in them as can be seen in the 
hyperelastic-plastic stress update algorithm discussed below.

