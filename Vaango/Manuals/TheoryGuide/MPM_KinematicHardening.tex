\chapter{Kinematic hardening models}
Kinematic hardening in \Vaango is modeled with a backstress ($\Bbeta$) that is
subtracted from the stress while evaluating the yield condition.  In Arena, the
pore water pressure ($\pbar_w = -p_w$) acts as a backstress, i.e.,
\Beq
  \Bbeta = \pbar_w \BI \,.
\Eeq
For metals, the backstress can either be ignored or modeled using the approaches described
in this chapter.

\section{Prager model}
If the evolution of the backstress is given by the Prager kinematic hardening rule, we have
\Beq
  \dot{\Bbeta} = \frac{2}{3}~\beta H~\dot{\BVeps}^p 
\Eeq
where $\Bbeta$ is the backstress, $\beta H$ is a constant hardening modulus, and $\dot{\Bveps}^p$ is
the plastic strain rate.

The Prager model is invoked using
\lstset{language=XML}
\begin{lstlisting}
<kinematic_hardening_model type="prager_hardening">
  <beta> 1.0 </beta>
  <hardening_modulus>1.5e6</hardening_modulus>
</kinematic_hardening_model>
\end{lstlisting}

\section{Armstrong-Frederick model}
The Armstrong-Frederick model evolves the backstress using
\Beq
  \dot{\Bbeta} = \frac{2}{3}~\beta H_1~\dot{\BVeps}^p - \beta H_2~\Bbeta~\Norm{\dot{\BVeps}^p}{} 
\Eeq
where $\beta$, $H_1$ and $H_2$ are material parameters.

The Armstrong-Frederick model is invoked using
\lstset{language=XML}
\begin{lstlisting}
<kinematic_hardening_model type="armstrong_frederick_hardening">
  <beta> 1.0 </beta>
  <hardening_modulus_1>1.5e6</hardening_modulus_1>
  <hardening_modulus_2>1.5e4</hardening_modulus_2>
</kinematic_hardening_model>
\end{lstlisting}

