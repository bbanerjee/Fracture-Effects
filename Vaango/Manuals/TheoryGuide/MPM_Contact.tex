The default behavior of \MPM is to handle interactions between objects using velocities on
the background grid.  However, beyond some simple situations, contact requires the application
of contact laws.  In the \Vaango implementation of friction contact, Coulomb friction is
assumed.  Alternative types of contact, such as adhesive contact. could also be implemented by
changing the contact law.

The purpose of the various contact algorithms in \Vaango is to correct the grid velocities
such that a particular set of contact assumptions are satisfied.  The two main algorithms
are \Textsfc{friction\_bard}, which is based on~\cite{Bard2001}, and \Textsfc{friction\_LR},
which is described in~\cite{Nairn2020}.

Let $m_p$, $\Bv_p$, $\Bp_p$ be the mass, velocity, and momentum of particle $p$. Also, let $m_g$,
$\Bv_g$, $\Bp_g$ be the mass, velocity, and momentum at a grid point $g$ due to nearby particles
in the region of influence.

Let the interpolation from grid nodes to particles be given by
\Beq
  \Ba_p = \sum_g S_g(\Bx_p) \Ba_g 
\Eeq
where $\Ba_p$ is a particle quantity, $\Ba_g$ is a grid quantity, and $S_g(\Bx_p)$ is a scalar
interpolation function evaluated at the particle position $\Bx_p$.  The quantities $\Ba_p$ and $\Ba_g$
are expressed as column vectors.  For scalar quantities, the column vectors are of size $1 \times 1$.
For vector quantities the size is $3 \times 1$, and for second-order symmetric tensors the vector
has dimensions $6 \times 1$. For example, for a typical 3D vector,
\Beq
  \begin{bmatrix} a_1 \\ a_2 \\ a_3 \end{bmatrix}_p = \sum_g S_g(\Bx_p)
  \begin{bmatrix} a_1 \\ a_2 \\ a_3 \end{bmatrix}_g \,.
\Eeq
The interpolation from grid to particles can then be expressed in matrix form as
\Beq
  \Ba_p = \BA_g \cdot \Bs_{gp}
\Eeq
where, if $S_g$ is the number of contributing grid nodes, 
\Beq
  \BA_g = \begin{bmatrix}
             \begin{bmatrix} a_1 \\ a_2 \\ a_3 \end{bmatrix}_1 &
             \begin{bmatrix} a_1 \\ a_2 \\ a_3 \end{bmatrix}_2 &
             \dots & 
             \begin{bmatrix} a_1 \\ a_2 \\ a_3 \end{bmatrix}_{N_g} 
           \end{bmatrix}
  ~,~~
  \Bs_{gp} = \begin{bmatrix} S_1(\Bx_p) \\ S_2(\Bx_p) \\ \vdots \\ S_{N_g}(\Bx_p) \end{bmatrix} \,.
\Eeq

Because the number of particles can differ from the number of grid nodes, the projection of
quantities from particles to the grid requires a different operator.
\begin{NoteBox}
An important underlying assumption in MPM is that
continuum field quantities have two equivalent representations -- a grid representation and
a particle representation.  For instance, the representation of a vector field $\Ba$ can be both of
\Beq
  \Ba(\Bx) = \sum_g \Ba(\Bx_g) S_g(\Bx_g) \quad \Tand \quad
  \Ba(\Bx) = \sum_p \Ba(\Bx_p) \chi_p(\Bx_p) \,.
\Eeq

\end{NoteBox}
Let that projection
operation from particles to grid nodes be
\Beq
  \Ba_g = \sum_p T_p(\Bx_g) \Ba_p 
\Eeq

