The default behavior of \MPM is to handle interactions between objects using velocities on
the background grid.  However, beyond some simple situations, contact requires the application
of contact laws.  In the \Vaango implementation of friction contact, Coulomb friction is
assumed.  Alternative types of contact, such as adhesive contact. could also be implemented by
changing the contact law.

The purpose of the various contact algorithms in \Vaango is to correct the grid velocities
such that a particular set of contact assumptions are satisfied.  Many of these algorithms
require the computaion of surface normals.

\subsection{Definitions}
Let $m_p$, $\Bv_p$, $\Bp_p$ be the mass, velocity, and momentum of particle $p$. Also, let $m_g$,
$\Bv_g$, $\Bp_g$ be the mass, velocity, and momentum at a grid point $g$ due to nearby particles
in the region of influence.  Consider $N_\alpha$ objects that can potentially be in contact and index
then by the superscript $\alpha$.
Then, from \eqref{eq:vol_weighted_mom}, we have
\Beq
  m_g^\alpha = V_g^\alpha \rho_g^\alpha = \sum_p V_p^\alpha \rho_p^\alpha \bar{S}_{gp}
             = \sum_p m_p^\alpha \bar{S}_{gp} \quad \Tand \quad
  \Bp_g^\alpha = \sum_p \Bp_p^\alpha \bar{S}_{gp} \,.
\Eeq
In matrix notation,
\Beq
  \BmT_g^\alpha = \SfS^T \BmT_p^\alpha \quad \Tand \quad
  \BpT_g^\alpha = \SfS^T \BpT_p^\alpha \,.
\Eeq
Similarly, from \eqref{eq:mass_weighted_vel}, we have
\Beq 
  m_g^\alpha \Bv_g^\alpha = \sum_p m_p^\alpha \Bv_p^\alpha \bar{S}_{gp}
  \quad \implies \quad
  \Bv_g^\alpha = \frac{1}{m_g^\alpha} \sum_p m_p^\alpha \Bv_p^\alpha \bar{S}_{gp} \,.
\Eeq
In matrix form,
\Beq
  \BvT_g^\alpha = \SfS^\plus_\alpha \BvT_p^\alpha 
  \quad \text{where} \quad
  \SfS^\plus_\alpha = \left(\BmT_g^\alpha\right)^{-1} \SfS^T \BmT_p^\alpha \,.
\Eeq
Based on a local conservation of momentum, we define a \Textsfc{center-of-mass velocity}, $\Bv_g^{\Tcm}$,
at grid node $g$ for all the contacting objects:
\Beq \label{eq:center_of_mass_vel}
  \Bv_g^{\Tcm} = \frac{\sum_\alpha m_g^\alpha \Bv_g^\alpha}{\sum_\alpha m_g^\alpha} \,.
\Eeq
We also define an \Textsfc{effective grid mass}, $m_g^\Teff$, as
\Beq \label{eq:eff_mass}
  \frac{1}{m_g^\Teff} = \sum_\alpha \frac{1}{m_g^\alpha} \,.
\Eeq

\subsection{Computing surface normals and tractions}
Surface normals are typically estimated from the gradient of mass at grid nodes.  If $m_p$ is the mass 
of particle $p$, then the normal at grid node $g$ due to the particles in its region of influence
is
\Beq \label{eq:normal_g}
  \Bn_g = \sum_p m_p \Grad{\bar{S}}_{gp} \,.
\Eeq
Normals are converted to unit vectors before they ar used in \Vaango computations.

Surface tractions at the nodes are computed by projecting particle stresses ($\Bsig_p$) to grid nodes:
\Beq
  \bar{\Bt}_g = \Bn_g \cdot \left(\sum_p \Bsig_p \bar{S}_{gp}\right) \,.
\Eeq

\subsection{Basic contact algorithm}
The most basic contact algroithm in \Vaango is called ``single-velocity contact''  The center-of-mass
velocity is computed using \eqref{eq:center_of_mass_vel}.  Upon contact detection, grid nodes
that participate are assigned this velocity:
\Beq
   \Bv_g = \Bv_g^{\Tcm} \,.
\Eeq

\subsection{Contact with a specified master}
A slightly more complex algorithm is the ``master''-based contact which is called
``specified-velocity contact'' in \Vaango.  In this model, a selected master material is
assigned velocities, $\Bv_g^m = \Bv^m(t)$, where $m$ is the index of the master material.
The grid node velocities of the materials are then adjusted according to
\Beq
  \Bv_g \leftarrow \Bv_g -  \left[\Bn^m_g \cdot (\Bv_g - \Bv_g^m)\right]\Bn^m_g 
\Eeq
where $\Bn^m_g$ is the normal for the master material computed using \eqref{eq:normal_g}.
This type of contact is useful for imposing boundary conditions on objects.

\subsection{Frictional contact algorithms}
The two main frictional contact algorithms
are \Textsfc{friction\_bard}, which is based on~\cite{Bard2001}, and \Textsfc{friction\_LR},
which is described in~\cite{Nairn2020}.


\subsubsection{Bardenhagen et al. algorithm}
In the algorithm developed in~\cite{Bard2001}, a contact interface is defined as the set of
nodes for which individual grid velocities associated with each object differ from the
center of mass velocity:
\Beq
  \Bv_g^\alpha - \Bv_g^\Tcm \ne 0 \,.
\Eeq
Once this condition is identified, the surface normal $\Bn_g^\alpha$ is computed
from the mass distribution around node $g$, and the surface normal traction
$\Bt_g^\alpha$ is computed from the stresses in surrounding material points.

The contact condition is
\Beq \label{eq:contact_constraint_bard}
(\Bv_g^\alpha - \Bv_g^\Tcm)\cdot\Bn_g^\alpha > 0 \quad \Tand \quad
\Bt_g^\alpha \cdot \Bn_g^\alpha < 0 \,.
\Eeq
This condition indicates compressive stress at node $g$.  If this condition is not satisfied,
the objects are assumed to have separated.

To enforce \eqref{eq:contact_constraint_bard}, the grid node velocities are adjusted
such that momentum is conserved, i.e.,
\Beq
  \Delta (v_n)_g^\alpha = \Delta \Bv_g^\alpha \cdot \Bn_g^\alpha \quad \Tand \quad
  \Delta (v_t)_g^\alpha = \Delta \Bv_g^\alpha \cdot \left[
     \Bn_g^\alpha \times \frac{\Delta \Bv_g^\alpha \times \Bn_g^\alpha}%
                              {\Norm{\Delta \Bv_g^\alpha \times \Bn_g^\alpha}{}} \right]
\Eeq
where
\Beq
  \Delta \Bv_g^\alpha := \Bv_g^\alpha - \Bv_g^\Tcm \,.
\Eeq
Normal contact is enforced by adjusting material velocities by $\Delta (v_n)_g^\alpha$.
The tangential contact is enforced using Coulomb friction with the tangential velocity
determined using $\mu \Delta (v_n)_g^\alpha$ where $\mu$ is the friction coefficient. If
$\Delta (v_t)_g^\alpha < \mu \Delta (v_n)_g^\alpha$, the no-slip condition is enforced.
Otherwise, the tangential components of the nodal velocities are updated with a reduced
friction coefficient
\Beq
  \mu_\Tred = \Tmin\left(\mu, \frac{|\Delta (v_t)_g^\alpha|}{|\Delta (v_n)_g^\alpha|}\right) \,.
\Eeq

Returning to the problem of computing object outward normals at a grid point, the
traditional approach is to compute volume gradients using the set of particles influencing
a node:
\Beq \label{eq:vol_grad}
  \Bg_g^\alpha = \sum_{p^\alpha} \overbar{\BGv}_{gp} V_p 
\Eeq
where the gradients $\overbar{\BGv}_{gp}$ are as defined in \eqref{eq:G_mpm}, \eqref{eq:G_gimp}, and
\eqref{eq:G_cpdi}.  The normal to an object is calculated using
\Beq
  \Bn^\alpha = \frac{\Bg_g^\alpha}{\Norm{\Bg_g^\alpha}{}} \,.
\Eeq
For multiple objects, an average gradient can be computed for better accuracy.

\subsubsection{Nairn et al. algorithm}
The more recent algorithm by~\cite{Nairn2020} uses a logistic regression step to determine
contact.  The underlying approach is similar to that used in ~\cite{Bard2001}.  Since the approach
is at its simplest when only two objects are involved at a grid point, we will describe only that
case below.  Most situations with contact between multiple objects see~\cite{Nairn2020}.

Let the two objects be indexed by $\alpha$ and $\beta$.  Let $\Bp_g^{\alpha 0}$ and $\Bp_g^{\beta 0}$
be the particle momenta projected to the grid.  We would like to compute the momentum correction
$\Delta \Bp$ so that momentum is conserved after contact.  Let the corrected momenta be
\Beq \label{eq:p_alpha_beta}
  \Bp_g^\alpha = \Bp_g^{\alpha 0} + \Delta \Bp \quad  \Tand \quad
  \Bp_g^\beta = \Bp_g^{\beta 0} - \Delta \Bp  \,.
\Eeq
If we restrict relative motion between objects at a grid point to the tangent plane, and let
$\Bthat$ be the direction of relative motion, then
\Beq \label{eq:v_alpha_beta}
  \Bv_g^\beta - \Bv_g^\alpha = k \Bthat \quad \implies \quad
  \frac{\Bp_g^\beta}{m_g^\beta} - \frac{\Bp_g^\alpha}{m_g^\alpha} = k \Bthat
  \quad \implies \quad
  m_g^\alpha \Bp_g^\beta - m_g^\beta \Bp_g^\alpha = m_g^\alpha m_g^\beta k \Bthat \,.
\Eeq
From the definition of the center-of-mass velocity in \eqref{eq:center_of_mass_vel} and
the effective grid mass~\eqref{eq:eff_mass}, we have
\Beq 
  \Bv_g^{\Tcm} = \frac{m_g^\alpha \Bv_g^\alpha + m_g^\beta \Bv_g^\beta}{m_g^\alpha + m_g^\beta} 
     = \frac{\Bp_g^\alpha + \Bp_g^\beta}{m_g^\alpha + m_g^\beta}  \quad \Tand \quad
  m_g^\Teff = \frac{m_g^\alpha m_g^\beta}{m_g^\alpha + m_g^\beta} \,.
\Eeq
Therefore,
\Beq \label{eq:p_alpha_beta_1}
  \Bp_g^\alpha + \Bp_g^\beta =  \frac{\Bv_g^{\Tcm} m_g^\alpha m_g^\beta}{m_g^\Teff} \,.
\Eeq
Solving for $\Bp_g^\alpha, \Bp_g^\beta$ from equations \eqref{eq:v_alpha_beta} and
\eqref{eq:p_alpha_beta_1}, we have
\Beq
  \Bp_g^\alpha = m_g^\alpha \Bv_g^\Tcm - m_g^\Teff k \Bthat \quad \Tand \quad
  \Bp_g^\beta = m_g^\beta \Bv_g^\Tcm + m_g^\Teff k \Bthat \,.
\Eeq
Therefore, from \eqref{eq:p_alpha_beta},
\Beq
\Delta \Bp = \Bp_g^\alpha - \Bp_g^{\alpha 0}
           = m_g^\alpha \Bv_g^\Tcm - m_g^\Teff k \Bthat - m_g^\alpha \Bv_g^{\alpha 0} 
           = m_g^\alpha (\Bv_g^\Tcm - \Bv_g^{\alpha 0}) - m_g^\Teff k \Bthat \,.
\Eeq
The quantity
\Beq
  \Delta \Bp^0 := m_g^\alpha (\Bv_g^\Tcm - \Bv_g^{\alpha 0})
\Eeq
is the initial change of momentum before tangential correction.
Since the contact force ($\Bf^c$) is given by the rate of change of momentum due to contact, we
have
\Beq
  \Bf^{c0} = \frac{\Delta \Bp^0}{\Delta t} \quad \Tand \quad
  \Bf^c = \frac{\Delta \Bp}{\Delta t} = \Bf^{c0} - \frac{m_g^\Teff k \Bthat}{\Delta t} \,.
\Eeq
where $\Delta t$ is the timestep size.
The contact compressive traction is found from the normal component of the contact force needed to
prevent interpenetration:
\Beq
  T^c_n = -\frac{1}{A^c}\,\Bf^{c0}\cdot\Bn
\Eeq
where $A^c$ is the contact area.  The tangential contact traction can be found using a
contact law:
\Beq
  T^c_t = \frac{1}{A^c}\,\Bf^c\cdot\Bthat
        = \frac{1}{A^c}\left[\Bf^{c0}\cdot\Bthat - \frac{m_g^\Teff k}{\Delta t}\right] \,.
\Eeq
Therefore, if $T^c_t = T^c_t(T^c_n)$ is a contact law, 
\Beq
  k = \frac{\Delta t}{m_g^\Teff} \left[\Bf^{c0} \cdot \Bthat - A^c T^c_t(T^c_n)\right] \,.
\Eeq
We can compute $k$ using the contact law $T^c_t(T^c_n)$ and then adjust $\Bv_g^\alpha$
and $\Bv_g^\beta$ using \eqref{eq:p_alpha_beta}.  Note that this process is identical to that
used in the Bardenhagen at al. algorithm.  The main difficulty is in finding where contact has occurred
and the quantities $\Bn$, $\Bthat$, and $A^c$.

The basic contact identification condition used in this approach, and in the Bardenhagen et al.
approach, is
\Beq \label{eq:nairn_condition}
  (\Bv_g^\beta - \Bv_g^\alpha)\cdot \frac{\Bg_g^\alpha - \Bg_g^\beta}{\Norm{\Bg_g^\alpha - \Bg_g^\beta}{}}
    < 0 \quad \Tand \quad T^c_n > 0
\Eeq
where $\Bg_g$ is the volume gradient defined in \eqref{eq:vol_grad}.  This condition is a variation
of \eqref{eq:contact_constraint_bard} and is a necessary, but not sufficient, condition to detect
whether the two objects are approaching each other and in contact.  However, $T^c_n > 0$ even when
the objects are not touching and a separation condition is needed to correctly identify contact.

The logistic regression approach developed in ~\cite{Nairn2020} attempts to identify contact
without having to rely purely on grid information.  This approach requires a set of
particles (point-cloud) in the neighborhood of a grid point that satisfies \eqref{eq:nairn_condition}.
The aim of this technique is to identify a plane within the point-cloud that best separates
the two objects.  The normal to this plane is the contact normal $\Bn_g$.  The logistic function
penalizes points as a function of their distance from a preferred separation plane.

The logistic regression method for separation detection is described next.  Let $\BxT_p$ be
the homogeneous coordinate representation of a particle position, i.e., $\BXv_p = (\Bx_p, 1)
 =: (X_1, X_2, X_3, X_4)$,
where $\Bx_p = (x_p^1, x_p^2, x_p^3)$ is the particle position.
Let $\BNv$ be the corresponding normal vector of the separation plane, i.e., $\BNv = (\Bn, N_4)$
where $\Bn = (n_1, n_2, n_3) =: (N_1, N_2, N_3)$ is the normal to the separation plane and
$n_4$ is an offset.  The equation of the desired separation plane is
\Beq
  \BXv \cdot \BNv = 0
\Eeq
where $\BXv$ is the vector of particle positions.  Let there be $P$ particles in the point-cloud
consisting of points from objects $\alpha$ and $\beta$.  Define a particle label $c_p$ as:
\Beq
  c_p = \begin{cases}
       -1 & \quad \text{for particles in object}~ \alpha \\
        1 & \quad \text{for particles in object}~ \beta.
       \end{cases}
\Eeq
The objective function that has to be minimized is the error
\Beq
  E = \sum_{p=1}^P w_p \left[\CalL(\BXv_p, \BNv) - c_p\right]^2 + \sum_{j=1}^4 \lambda_j^2 N_j^2 
\Eeq
where $w_p$ are weights, $\lambda_j^2$ are penalty factors that help regularize the error function and
$\CalL$ is the logistic function given by
\Beq
  \CalL(\BXv,\BNv) = \frac{2}{1 + \exp(-\BXv\cdot\BNv)} - 1 \,.
\Eeq
The minimum of $E$ is achieved when $\Partial{E}{\BNv} = \Bzero$ and $\Partial{E}{\lambda} = 0$.
From the first requirement
\Beq \label{eq:dE_dN}
  \Partial{E}{N_i} = 
    2 \sum_{p=1}^P w_p \left[\CalL(\BXv_p, \BNv) - c_p\right] \Partial{\CalL}{N_i} +
    2 \sum_{j=1}^4 \lambda_j^2 N_j \Partial{N_j}{N_i} = 0 
\Eeq
where
\Beq
  \Bal
  \Partial{\CalL}{N_i} & = 2\Partial{}{N_i}\left[1 + \exp(-X_m N_m)\right]^{-1}
    = -2\left[1 + \exp(-X_m N_m)\right]^{-2}\Partial{}{N_i}\exp(-X_m N_m) \\
  & = 2\left[1 + \exp(-X_m N_m)\right]^{-2}\exp(-X_m N_m) X_m\Partial{N_m}{N_i} \\
  & = 2\left[1 + \exp(-X_m N_m)\right]^{-2}\exp(-X_m N_m) X_i \,.
  \Eal
\Eeq
Define
\Beq
  \theta_p := -\BXv_p \cdot \BNv \quad \Tand \quad \phi_p := 1 + \exp(\theta_p) \,.
\Eeq
Then, in vector form, 
\Beq \label{eq:dL_dN}
  \Partial{\CalL}{\BNv}  = \frac{2\exp(\theta_p)}{\phi_p^2} \,\BXv
  \quad \Tand \quad \CalL(\BXv_p,\BNv) = \frac{2}{\phi_p} - 1\,.
\Eeq
Returning to \eqref{eq:dE_dN}, we can write
\Beq
  \Partial{E}{N_i} = 
    2 \sum_{p=1}^P w_p \left[\CalL(\BXv_p, \BNv) - c_p\right] \Partial{\CalL}{N_i} +
    2 \lambda_i^2 N_i = 0 \,.
\Eeq
Similarly,
\Beq
  \Partial{E}{\lambda_i} = 2 \sum_{j=1}^4 \lambda_j \Partial{\lambda_j}{\lambda_i} N_j^2
  = \lambda_i N_i^2 = 0
\Eeq
Then, in vector form, the system of equations needed to solve for $\BNv$ and $\Blambda$ is
\Beq \label{eq:nonlinear_N}
  \boxed{
  \Bal
    & \sum_{p=1}^P w_p \left[\CalL(\BXv_p,\BNv) - c_p\right]\Partial{\CalL}{\BNv} +
     (\Blambda \odot \Blambda) \odot \BNv = 0 \quad  \Tand\\
    & \Blambda \odot (\BNv \odot \BNv) = 0
  \Eal
  }
\Eeq
where
\Beq
  \BNv = (N_1, N_2, N_3, N_4) ~,~
  \Blambda = (\lambda_1, \lambda_2, \lambda_3, \lambda_4) ~,~
  \Ba\odot\Bb = (a_1 b_1, a_2 b_2, a_3 b_3, a_4 b_4) \,.
\Eeq
The second set of equations in \eqref{eq:nonlinear_N} suggest that the solution will
improve as $\Blambda \rightarrow 0$.
Given a vector $\Blambda$, the first equation in \eqref{eq:nonlinear_N} can be solved for $\BNv$ using
Newton's method.
Define
\Beq \label{eq:Y_N}
  \BYv(\BNv) := \sum_{p=1}^P w_p \left[\CalL(\BXv_p,\BNv) - c_p\right]\Partial{\CalL}{\BNv} +
     (\Blambda \odot \Blambda) \odot \BNv \,.
\Eeq
Then, with $\BIv$ denoting the $4 \times 4$ identity matrix, 
\Beq \label{eq:dY_dN}
\Partial{\BYv}{\BNv} = \sum_{p=1}^P w_p 
   \left[\Partial{\CalL}{\BNv}\otimes\Partial{\CalL}{\BNv} +
         \left[\CalL(\BXv_p,\BNv) - c_p\right]\PPartial{\CalL}{\BNv}\right] +
    (\Blambda \odot \Blambda) \BIv \,.
\Eeq
Then Newton's method gives the iterative rule
\Beq
  \BNv^{k+1} = \BNv^k - \left[\Partial{\BYv}{\BNv}\right]^{-1}_{\BNv^k} \cdot \BYv(\BNv^k)
\Eeq
Given appropriate starting values, this method with converge to the solution except in
situations where $\BXv \cdot \BNv < 0$.  It is preferable to normalize $\BNv$ in those situations
where the exponential becomes too large.

Once the $\BNv$ vector has been found, the unit normal to the separation plane is
determine by normalizing $\Bn = (N_1, N_2, N_3)$.  Contact occurs at particle $p$ if 
\Beq
  \underset{p\in \beta}{\Tmin}\, (\Bx_p \cdot \frac{\Bn}{\Norm{\Bn}{}} - R_p) - 
  \underset{p\in \alpha}{\Tmax}\, (\Bx_p \cdot \frac{\Bn}{\Norm{\Bn}{}} + R_p) < 0
\Eeq
where $R_p$ is the distance from the centroid of particle $p$ to its deformed edge along $\Bn$.


