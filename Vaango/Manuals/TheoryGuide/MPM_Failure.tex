\chapter{Material failure}

\section{Introduction}
The primary technique used in \Vaango to simulate failure is damage evolution.
A particle is tagged as ``failed'' when its temperature is greater than the
melting point of the material at the applied pressure.  Failure is also 
flagged when the porosity of a particle is greater critical limit (typically 0.9)
and the strain exceeds the fracture strain of the material.  

An alternative approach that can be used in the metal plasticity models
implemented in \Vaango is to test material stability conditions 
to determine and propagate failure.  Upon failure detection,
a particle is either removed from the computation by setting the stress to
zero or is converted into a material with a different velocity field 
which interacts with the remaining particles via contact.  Either approach
leads to the simulation of a newly created surface.  More details of the
approach can be found in ~\cite{Banerjee2004a,Banerjee2004c,Banerjee2005}.

\section{Erosion algorithm}
In metal plasticity simulations, the heat generated at a material point is conducted 
away at the end of a time step using the heat equation.  If special adiabatic 
conditions apply (such as in impact problems), the heat is accumulated at a 
material point and is not conducted to the surrounding particles.  This 
localized heating can be used to determine whether a material point has
melted.

The determination of whether a particle has failed can be made on the 
basis of either or all of the following conditions:
\begin{itemize}
  \item The particle temperature exceeds the melting temperature.
  \item The TEPLA-F fracture condition~\cite{Johnson1988} is satisfied.
     This condition can be written as
     \begin{equation}
       \left(\frac{\phi}{\phi_c}\right)^2 + 
       \left(\frac{\Veps_p^\Teq}{\Veps_p^f}\right)^2 = 1
     \end{equation}
     where $\phi$ is the current porosity, $\phi_c$ is the maximum 
     allowable porosity, $\Veps_p^\Teq$ is the current equivalent plastic strain, and
     $\Veps_p^f$ is the equivalent plastic strain at fracture.
  \item An alternative to ad-hoc damage criteria is to use the concept of 
     material stability bifurcation to determine whether a particle has 
     failed or not.
\end{itemize}

Since the material unloads locally after fracture, the 
hypoelastic-plastic stress update may not work accurately under certain 
circumstances.  An improvement would be to use a hyperelastic-plastic stress 
update algorithm.  Also, the plasticity models are temperature dependent.
Hence there is the issue of severe mesh dependence due to change of the
governing equations from hyperbolic to elliptic in the softening regime
~\cite{Hill1975,Bazant1985,Tver1990}.  Viscoplastic stress update models or 
nonlocal/gradient plasticity models~\cite{Ramaswamy1998,Hao2000} can be used 
to eliminate some of these effects.  Such models that have been
implemented in \Vaango are discussed later.

The tags used to control the erosion algorithm are in two places.  
In the \verb|<MPM> </MPM>| section the following flags can be set
\lstset{language=XML}
\begin{lstlisting}
  <erosion algorithm = "ZeroStress"/>
  <create_new_particles>           false      </create_new_particles>
  <manual_new_material>            false      </manual_new_material>
\end{lstlisting}
If the erosion algorithm is \verb|"none"| then no particle failure is done.

In the \verb|<constitutive_model type="elastic_plastic">| section, the 
following flags can be set
\lstset{language=XML}
\begin{lstlisting}
  <evolve_porosity>               true  </evolve_porosity>
  <evolve_damage>                 true  </evolve_damage>
  <do_melting>                    true  </do_melting>
  <useModifiedEOS>                true  </useModifiedEOS>
  <check_TEPLA_failure_criterion> true  </check_TEPLA_failure_criterion>
  <check_max_stress_failure>      false </check_max_stress_failure>
  <critical_stress>              12.0e9 </critical_stress>
\end{lstlisting}

\section{Material stability conditions}
\subsection{Drucker's condition}
The simplest criterion that can be used is the Drucker stability postulate 
\cite{Drucker1959} which states that time rate of change of the rate of 
work done by a material cannot be negative.  Therefore, the material is 
assumed to become unstable (and a particle fails) when
\begin{equation}
  \dot\Bsig:\BdT^p \le 0
\end{equation}

\subsection{Acoustic tensor criterion}
Another stability criterion that is less restrictive is the acoustic
tensor criterion which states that the material loses stability if the 
determinant of the acoustic tensor changes sign~\cite{Rudnicki1975,Yamamoto1978,Perzyna1998}.  

We assume that the strain is localized in a thin band with normal $\Bn$.  The band is 
assumed be homogeneous
but has slightly different material properties than the surrounding material.

To develop the bifurcation relations~\cite{Yamamoto1978}, assume that $\Bv^b$ is the 
velocity of a material point in the band ($\Omega_b$) and $\Bv^0$ is the velocity of the 
material outside the band ($\Omega_0$).  The deformation of the material outside the band
is assumed to be uniform.  The deformation within the band is also assumed to
be homogeneous.

We assume that stresses and rates of deformation have been rotated to the undeformed
configuration using the polar decomposition of the deformation gradient.

Consider the case where the local coordinates of points in the band are expressed in
terms of an orthonormal basis $\Be_1, \Be_2, \Be_3$ where $\Be_2 = \Bn$.  Then
a point $\Bx$ inside (or outside) the band can be expressed as $\Bx = x_i \Be_i$.

Continuity and the homogeneity of deformation in the two regions requires that 
only the velocity in the $\Bn$ direction can be different in the two regions.
This implies
\Beq
  \Bv^b(\Bx) - \Bv^0(\Bx) = \Bf(\Bn \cdot \Bx) = \Bf(x_2) 
\Eeq
where $\Bf$ is an unknown function.
The velocity gradient can be computed from the above relation as
\Beq \label{eq:gradv_b}
  \Grad{\Bv^b} = \Grad{\Bv^0} + \Deriv{\Bf}{x_2} \otimes \Bn = \Grad{\Bv^0} + \Bq \otimes \Bn
\Eeq
where
\Beq
  \Bq := \Deriv{\Bf}{x_2} = \begin{cases}
                              \Bzero   & \quad\text{for}\quad \Bx \in \Omega_0 \\
                              \Bq & \quad\text{for}\quad \Bx \in \Omega_b 
                            \end{cases}
\Eeq
The rate of deformation in the band is
\Beq \label{eq:d_b}
  \BdT^b = \Half\left[\Grad{\Bv^b} + (\Grad{\Bv^b})^T\right]
         = \Half\left[\Grad{\Bv^0} + (\Grad{\Bv^0})^T\right] + \Half(\Bq \otimes \Bn + \Bn \otimes \Bq)
         = \BdT^0 + \Half(\Bq \otimes \Bn + \Bn \otimes \Bq) \,.
\Eeq
The stress rate is related to the rate of deformation by
\Beq \label{eq:sigdot_1}
  \dot{\Bsig}^0 = \SfC^0 : \BdT^0 ~,~~
  \dot{\Bsig}^b = \SfC^b : \BdT^b = \SfC^b : \BdT^0 + \Half \SfC^b : (\Bq \otimes \Bn + \Bn \otimes \Bq) \,.
\Eeq
The minor symmetry of $\SfC$ implies that
\Beq \label{eq:sigdot_2}
  \dot{\Bsig}^b = \SfC^b : \BdT^0 +  \SfC^b : (\Bq \otimes \Bn) \,.
\Eeq
Homogeneity of the deformation also implies that
\Beq
  \Bn \cdot \Bsig^b = \Bn \cdot \Bsig^0 \,.
\Eeq
Taking the material time derivative of the above gives
\Beq \label{eq:ndot_rel}
  \dot{\Bn} \cdot \Bsig^b + \Bn \cdot \dot{\Bsig}^b = 
    \dot{\Bn} \cdot \Bsig^0 + \Bn \cdot \dot{\Bsig}^0 \,.
\Eeq
We need an expression for $\dot{\Bn}$.  To find that, note that if $\Bn_0$ and $\Bn$ are the unit normals 
to the band in the reference and current configurations, using Nanson's formula, we have
\Beq \label{eq:nanson}
  \Bn da = J (\BF^{-T} \cdot \Bn_0) dA ~,~~ J = \det\BF  \,.
\Eeq
Taking the time derivative of \eqref{eq:nanson},
\Beq 
  \dot{\Bn} da + \Bn \dot{da} = \Deriv{J}{t} (\BF^{-T} \cdot \Bn_0) dA + 
                                J \left(\Deriv{\BF^{-1}}{t}\right)^T \cdot \Bn_0 dA 
\Eeq
For the derivative of $J$, we have
\Beq
  \Deriv{J}{t} = \Partial{J}{\BF} : \dot{\BF} = J\BF^{-T}:\dot{\BF} = J \Tr(\dot{\BF}\cdot\BF^{-1})
    = J \Tr(\Grad{\Bv}) \,. 
\Eeq
We can also show that
\Beq
  \Deriv{\BF^{-1}}{t} = -\BF^{-1} \cdot \dot{\BF} \cdot \BF^{-1} 
                      = -\BF^{-1} \cdot \Grad{\Bv}\,.
\Eeq
Therefore, using \eqref{eq:nanson},
\Beq \label{eq:nanson_deriv}
  \Bal
  \dot{\Bn} da + \Bn \dot{da} &= J \Tr(\Grad{\Bv}) (\BF^{-T} \cdot \Bn_0) dA - 
                                 J (\Grad{\Bv})^T \cdot \BF^{-T} \cdot \Bn_0 dA  \\
    & = \Bn \cdot \left[\Tr(\Grad{\Bv}) \BI - \Grad{\Bv} \right] da \,.
  \Eal
\Eeq
To find $\dot{da}$, we compute a dot product of both sides of \eqref{eq:nanson} to get
\Beq 
  da^2 = J^2 (\BF^{-T} \cdot \Bn_0)\cdot(\BF^{-T} \cdot \Bn_0) dA^2 
       =  J^2 \Bn_0 \cdot (\BF^{-1}\cdot\BF^{-T}) \cdot \Bn_0 dA^2 \,.
\Eeq
The material time derivative of the above expression is
\Beq\label{eq:da_sq}
  2 da \dot{da} = 2 J \Deriv{J}{t} \Bn_0 \cdot (\BF^{-1}\cdot\BF^{-T}) \cdot \Bn_0 dA^2 + 
                  J^2 \Bn_0 \cdot \Deriv{}{t}(\BF^{-1}\cdot\BF^{-T}) \cdot \Bn_0 dA^2
\Eeq
For the derivative of the $\BF^{-1}$ product, we have
\Beq
  \Bal
  \Deriv{}{t}(\BF^{-1}\cdot\BF^{-T}) &= 
    \Deriv{\BF^{-1}}{t}\cdot\BF^{-T} + \BF^{-1}\cdot\left(\Deriv{\BF^{-1}}{t}\right)^T = 
    - \BF^{-1} \cdot \dot{\BF} \cdot \BF^{-1} \cdot \BF^{-T} 
    - \BF^{-1} \cdot \BF^{-T} \cdot \dot{\BF}^{T} \cdot \BF^{-T} \\
   & = - \BF^{-1} \cdot \Grad{\Bv} \cdot \BF^{-T} 
       - \BF^{-1} \cdot (\Grad{\Bv})^T \cdot \BF^{-T} 
     = - 2 \BF^{-1} \cdot \BdT \cdot \BF^{-T} \,.
  \Eal
\Eeq
Substitution into \eqref{eq:da_sq} gives
\Beq
  \Bal
  da \dot{da} &= J^2 \Tr(\Grad{\Bv}) \Bn_0 \cdot (\BF^{-1}\cdot\BF^{-T}) \cdot \Bn_0 dA^2 - 
                J^2 \Bn_0 \cdot (\BF^{-1}\cdot\BdT\cdot\BF^{-T}) \cdot \Bn_0 dA^2 \\
    &= \Tr(\Grad{\Bv}) (J \BF^{-T} \cdot \Bn_0 dA) \cdot (J \BF^{-T}\cdot \Bn_0 dA)  - 
                (J \BF^{-T} \cdot \Bn_0 dA) \cdot \BdT \cdot (J \BF^{-T} \cdot \Bn_0 dA) \\
    &= \Tr(\Grad{\Bv}) \Bn \cdot \Bn da^2  - \Bn \cdot \BdT \cdot \Bn da^2 
     = \Tr(\Grad{\Bv}) da^2  - \Bn \cdot \BdT \cdot \Bn da^2 \,.
  \Eal
\Eeq
Therefore,
\Beq \label{eq:da_dot}
  \dot{da} = \left[\Tr(\Grad{\Bv}) - \Bn \cdot \BdT \cdot \Bn \right] da 
           = \left[\Tr(\Grad{\Bv}) - \Bn \cdot \Grad{\Bv} \cdot \Bn \right] da \,.
\Eeq
Plugging \eqref{eq:da_dot} into \eqref{eq:nanson_deriv}, we have
\Beq 
  \Bal
  \dot{\Bn} da + 
  \Bn \left[\Tr(\Grad{\Bv}) - \Bn \cdot \BdT \cdot \Bn \right] da 
  = \Bn \cdot \left[\Tr(\Grad{\Bv}) \BI - \Grad{\Bv} \right] da \,.
  \Eal
\Eeq
That gives us the expression for $\dot{\Bn}$ that we seek,
\Beq \label{eq:ndot}
  \dot{\Bn} = \Bn \cdot \left[\BdT \cdot (\Bn \otimes \Bn) - \Grad{\Bv}\right] 
            = \Bn \cdot \Grad{\Bv} \cdot \left(\Bn \otimes \Bn - \BI\right)  \,.
\Eeq
Using \eqref{eq:ndot} in \eqref{eq:ndot_rel}, we have
\Beq 
  \Bn \cdot \left[(\dot{\Bsig}^b - \dot{\Bsig}^0) + 
     \Grad{\Bv^b} \cdot (\Bn \otimes \Bn - \BI) \cdot (\Bsig^b - \Bsig^0)\right] = 0\,.
\Eeq
Substituting equations \eqref{eq:sigdot_1}, \eqref{eq:sigdot_2}, and \eqref{eq:gradv_b},
\Beq 
  \Bn \cdot \left[(\SfC^b : \BdT^0 +  \SfC^b : (\Bq \otimes \Bn) - \SfC^0 : \BdT^0 ) + 
     \left(\Grad{\Bv^0} + \Bq \otimes\Bn\right) \cdot 
     (\Bn \otimes \Bn - \BI) \cdot (\Bsig^b - \Bsig^0)\right] = 0\,.
\Eeq
Using the symmetry of stress and the projection $\Bn\otimes\Bn - \BI$, we can reorganize
the above expression into 
\Beq 
  \Bn \cdot \left[
    \left(\SfC^b + (\Bsig^b - \Bsig^0) \cdot (\Bn \otimes \Bn - \BI) \cdot \SfI\right) : (\Bq \otimes\Bn) + 
    (\SfC^b - \SfC^0) : \BdT^0 +  
      (\Bsig^b - \Bsig^0) \cdot (\Bn \otimes \Bn - \BI) \cdot \Grad{\Bv^0} 
    \right] = 0\,.
\Eeq
Further rearrangement leads to
\Beq
  \left[\Bn \cdot \left(\SfC^b + (\Bsig^b - \Bsig^0) \cdot (\Bn \otimes \Bn - \BI) \cdot \SfI\right) \cdot \Bn\right] \cdot \Bq  =  -\Bn \cdot \left[ (\SfC^b - \SfC^0) : \BdT^0 +  
      (\Bsig^b - \Bsig^0) \cdot (\Bn \otimes \Bn - \BI) \cdot \Grad{\Bv^0}\right] 
\Eeq
This equation has a solution ($\Bq$) only if 
\Beq
  \det\left[\Bn \cdot \left(\SfC^b + (\Bsig^b - \Bsig^0) \cdot (\Bn \otimes \Bn - \BI) \cdot \SfI\right) \cdot \Bn\right] \ne 0 \,.
\Eeq
The canonical bifurcation condition is obtained if $\Bsig^b = \Bsig^0$:
\Beq \label{eq:acoustic_tensor}
  \det(\BA) := \det(\Bn \cdot \SfC \cdot \Bn) = 0 
\Eeq
where $\BA$ is the \Textmag{acoustic tensor}.

Evaluation of the acoustic tensor requires a search for a normal vector 
around the material point and is therefore computationally expensive.  

\subsection{Becker's simplification}
A simplification of this criterion is a check which assumes that the direction 
of instability lies in the plane of the maximum and minimum principal 
stress~\cite{Becker2002}.  

Let the principal stresses be $\sigma_1 > \sigma_2 > \sigma_3$, and the 
corresponding principal directions (eigenvectors) are $\BE_1, \BE_2, \BE_3$.
We can express the unit normal to the band in this basis as $\Bn = n_i \BE_i$.

The components of the tangent modulus in this coordinate system are given by
\Beq
  C'_{ijk\ell} = Q_{im} Q_{jn} Q_{kp} Q_{\ell q} C_{mnpq} \,.
\Eeq
where the $3 \times 3$ matrix used for this coordinate transformation, $\BQ$, is given by
\Beq
  \BQ^T := \begin{bmatrix} \BE_1 & \BE_2 & \BE_3 \end{bmatrix}
\Eeq
Then the acoustic tensor in \eqref{eq:acoustic_tensor} has the components
\begin{equation} 
  A_{jk} = C'_{ijk\ell} n_i n_\ell 
\end{equation} 
If $\det(A_{jk}) = 0 $, then $q_j = df_j/dx_2$ can be arbitrary and 
there is a possibility of strain localization.  Also, this condition indicates
when a material transitions from stable behavior where $\det(A_{jk} > 0$.
If this condition for loss of hyperbolicity is met,  then a particle deforms in an unstable 
manner and failure can be assumed to have occurred at that particle.  

Becker's simplification is to consider only selected components of the acoustic tensor
by assuming that the stress state in the band can be approximated as a planar
tension problem.  Then the acoustic tensor takes the form
\Beq
  \BA = \begin{bmatrix}
          C'_{1111} n_1^2 + C'_{3113} n_3^2 & 0 & (C'_{1133} + C'_{3131}) n_1 n_3 \\
          0 & C'_{1221} n_1^2 + C'_{3223} n_3^2 & 0 \\
          (C'_{3311} + C'_{1313}) n_1 n_3  & 0 & C'_{1331} n_1^2 + C'_{3333} n_3^2 
        \end{bmatrix}
\Eeq
Without loss of generality, we can divide this matrix by $n_1^2$ to get
\Beq
  \tilde{\BA} = \begin{bmatrix}
          C'_{1111} + C'_{3113} \frac{n_3^2}{n_1^2}  & 0 & (C'_{1133} + C'_{3131}) \frac{n_3}{n_1} \\
          0 & C'_{1221} + C'_{3223} \frac{n_3^2}{n_1^2} & 0 \\
          (C'_{3311} + C'_{1313}) \frac{n_3}{n_1}  & 0 & C'_{1331} + C'_{3333} \frac{n_3^2}{n_1^2} 
        \end{bmatrix}
\Eeq
Let $\alpha := n_3/n_1$.  Then we can write
\Beq
  \tilde{\BA} = \begin{bmatrix}
          C'_{1111} + C'_{3113} \alpha^2  & 0 & (C'_{1133} + C'_{3131}) \alpha \\
          0 & C'_{1221} + C'_{3223} \alpha^2 & 0 \\
          (C'_{3311} + C'_{1313}) \alpha  & 0 & C'_{1331} + C'_{3333} \alpha^2 
        \end{bmatrix}
\Eeq
and we have
\Beq
  \det(\tilde{\BA}) = (C'_{1111} + C'_{3113} \alpha^2) \alpha^2 (C'_{1331} + C'_{3333} \alpha^2) - 
                      (C'_{1133} + C'_{3131}) \alpha (C'_{1221} + C'_{3223} \alpha^2) (C'_{3311} + C'_{1313}) \alpha
\Eeq
Setting the determinant to zero allows us to get the following quadratic equation in $\beta := \alpha^2$:
\Beq
  (C'_{1111} + C'_{3113} \beta) (C'_{1331} + C'_{3333} \beta) - 
  (C'_{1133} + C'_{3131}) (C'_{1221} + C'_{3223} \beta) (C'_{3311} + C'_{1313}) = 0 \,. 
\Eeq
We can express the above in Voigt notation (convention $11$, $22$, $33$, $23$, $31$, $12$) as
\Beq
  (\Chat_{11} + \Chat_{55} \beta) (\Chat_{55} + \Chat_{33} \beta) - 
  (\Chat_{13} + \Chat_{55}) (\Chat_{66} + \Chat_{44} \beta) (\Chat_{31} + \Chat_{55}) = 0 \,. 
\Eeq
or
\Beq
  a_1 \beta^2 + a_2 \beta + a_3 = 0
\Eeq
where,
\Beq
  \Bal
  a_1 &= \Chat_{33} \Chat_{55} \\
  a_2 &= \Chat_{11} \Chat_{33} - \Chat_{13} \Chat_{31} \Chat_{44}
        - \Chat_{13} \Chat_{44} \Chat_{55} - \Chat_{31} \Chat_{44} \Chat_{55}
        - \Chat_{44} \Chat_{55}^2 + \Chat_{55}^2 \\
  a_3 &= \Chat_{11} \Chat_{55} - \Chat_{13} \Chat_{31} \Chat_{66} 
        - \Chat_{13} \Chat_{55} \Chat_{66} - \Chat_{31} \Chat_{55} \Chat_{66} \,.
  \Eal
\Eeq
The four roots are
\Beq
  \frac{n_3}{n_1} = \pm\sqrt{\frac{-a_2 \pm \sqrt{a_2^2 - 4 a_1 a_3}}{2a_1}} \,.
\Eeq
If there are no real roots, a band cannot form and there is no bifurcation.  If there are four real roots then
bifurcation is possible.  Two real roots indicate an intermediate condition that may not be realized in
practice but is considered stable in \Vaango.

More explicitly, for unstable deformation,
\Beq
  a_2^2 - 4 a_1 a_3 \ge 0 \quad \Tand \quad
  \frac{-a_2 \pm \sqrt{a_2^2 - 4 a_1 a_3}}{2a_1} \ge 0 \,.
\Eeq
If these conditions are satisfied, the \MPM particle is assumed to have failed.
