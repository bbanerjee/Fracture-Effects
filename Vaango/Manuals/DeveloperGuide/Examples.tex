\chapter{Examples}

This chapter will describe a set of example problems showing various
stages of algorithm complexity and how the \Vaango framework is used to
solve the discretized form of the solutions.  Emphasis will not be on
the most efficient or fast algorithms, but intead will demonstrate
straightforward implementations of well known algorithms within the
\Vaango Framework.  Several examples will be given that show an
increasing level of complexity.

All examples described are found in the directory
\Textsfc{src/CCA/Components/Examples}.

\section{Poisson1}

Poisson1 solves Poisson's equation on a grid using Jacobi
iteration.  Since this is not a time dependent problem and \Vaango is
fundamentally designed for time dependent problems, each Jacobi
iteration is considered to be a timestep.  The timestep specified and
computed is a fixed value obtained from the input file and has no
bearing on the actual computation.

The following equation is discretized and solved using an iterative
method. Each timestep is one iteration. At the end of the timestep, we
the residual is computed showing the convergence of the solution and
the next iteration is computed until.

The following shows a simplified form of the Poisson1 of the .h and
.cc files found in the Examples directory.  The argument list for some
of the methods are eliminated for readibility purposes.  Please refer
to the actual source for a complete description of the arguments
required for each method.

\begin{lstlisting}[language=Cpp]
  class Poisson1 : public UintahParallelComponent, public SimulationInterface {
  public:
    Poisson1(const ProcessorGroup* myworld);
    virtual ~Poisson1();
    virtual void problemSetup(const ProblemSpecP& params, const ProblemSpecP& restart_prob_spec, GridP& grid, SimulationStateP&);
    virtual void scheduleInitialize(const LevelP& level,SchedulerP& sched);
    virtual void scheduleComputeStableTimestep(const LevelP& level,SchedulerP&);
    virtual void scheduleTimeAdvance(const LevelP& level,SchedulerP&);

  private:
    void initialize(const ProcessorGroup*, const PatchSubset* patches, const MaterialSubset* matls, DataWarehouse* old_dw, DataWarehouse* new_dw);
    void computeStableTimestep(const ProcessorGroup*,const PatchSubset* patches, const MaterialSubset* matls,DataWarehouse* old_dw, DataWarehouse* new_dw);
    void timeAdvance(const ProcessorGroup,const PatchSubset* patches, const MaterialSubset* matls,DataWarehouse* old_dw, DataWarehouse* new_dw);

    SimulationStateP sharedState_;
    double delt_;
    const VarLabel* phi_label;
    const VarLabel* residual_label;
    SimpleMaterial* mymat_;

    Poisson1(const Poisson1&);
    Poisson1& operator=(const Poisson1&);
  };
\end{lstlisting}

The private methods and data shown are the functions that are function
pointers referred to in the task descriptions.  The \Textbfc{VarLabel}
data type stores the names of the various data that can be referenced
uniquely by the data warehouse.  The \Textbfc{SimulationStateP} data
type is essentially a global variable that stores information about
the materials that are needed by other internal \Vaango framework
components.  \Textbfc{SimpleMaterial} is a data type that refers to the
material properties.

Within each schedule function, i.e. \Textbfc{sheduleInitialize},
\Textbfc{scheduleComputeStableTimestep}, and
\Textbfc{scheduleTimeAdvance}, a task is specified that has a function
pointer associated with it.  The function pointers point to the actual
implementation of the specific task and have a different argument list
than the associated schedule method.

The typical task implementation, i.e. \Textbfc{timeAdvance()} contains
the following arguments: \Textbfc{ProcessorGroup},
\Textbfc{PatchSubset}, \Textbfc{MaterialSubset}, and two
\Textbfc{DataWarehouse} objects.  The purpose of the
\Textbfc{ProcessorGroup} is to hold various MPI information such as the
\Textbfc{MPI\_Communicator}, the rank of the process and the number of
processes that are actually being used.

\subsection{Description of Scheduling Functions}

The actual implementation with descriptions are presented following
the code snippets.

\begin{lstlisting}[language=Cpp]
  Poisson1::Poisson1(const ProcessorGroup* myworld)
  : UintahParallelComponent(myworld)
  {
    phi_label = VarLabel::create("phi", 
    NCVariable<double>::getTypeDescription());
    residual_label = VarLabel::create("residual", 
    sum_vartype::getTypeDescription());

  }

  Poisson1::~Poisson1()
  {
    VarLabel::destroy(phi_label);
    VarLabel::destroy(residual_label);
  }
\end{lstlisting}

Typical constructor and destructor for simple examples where the data
label names (\Textbfc{phi} and \Textbfc{residual}) are created for data
wharehouse storage and retrieval.

\begin{lstlisting}[language=Cpp]
  void Poisson1::problemSetup(const ProblemSpecP& params, const ProblemSpecP& restart_prob_spec, GridP& /*grid*/, SimulationStateP& sharedState)
  {
    sharedState_ = sharedState;
    ProblemSpecP poisson = params->findBlock("Poisson");

    poisson->require("delt", delt_);

    mymat_ = scinew SimpleMaterial();

    sharedState->registerSimpleMaterial(mymat_);
  }
\end{lstlisting}

The \Textbfc{problemSetup} is based in a xml description of the input
file.  The input file is parsed and the delt tag is set.  The
\Textbfc{sharedState} is assigned and is used to register a material
and store it for later use by the \Vaango internals.

\begin{lstlisting}[language=Cpp]
  void Poisson1::scheduleInitialize(const LevelP& level, SchedulerP& sched)
  {
    Task* task = scinew Task("Poisson1::initialize",
    this, &Poisson1::initialize);

    task->computes(phi_label);
    task->computes(residual_label);
    sched->addTask(task, level->eachPatch(), sharedState_->allMaterials());
  }
\end{lstlisting}

A task is defined which contains a name and a function pointer,
i.e. \Textbfc{initialize} which is described later in Poisson1.cc The
task defines two variables that are computed in the
\Textbfc{initialize} function, \Textbfc{phi} and \Textbfc{residual}.  The
task is then added to the scheduler.  This task is only computed once
at the beginning of the simulation.

\begin{lstlisting}[language=Cpp]
  void Poisson1::scheduleComputeStableTimestep(const LevelP& level, SchedulerP& sched)
  {
    Task* task = scinew Task("Poisson1::computeStableTimestep",
    this, &Poisson1::computeStableTimestep);

    task->requires(Task::NewDW, residual_label);
    task->computes(sharedState_->get_delt_label());
    sched->addTask(task, level->eachPatch(), sharedState_->allMaterials());
  }

\end{lstlisting}

A task is defined for the computing the stable timestep and uses the
function pointer, \Textbfc{computeStableTimestep} defined later in
Poisson1.cc.  This requires data from the New DataWarehouse, and the
next timestep size is computed and stored.

\begin{lstlisting}[language=Cpp]
  void
  Poisson1::scheduleTimeAdvance( const LevelP& level, SchedulerP& sched)
  {
    Task* task = scinew Task("Poisson1::timeAdvance", this, &Poisson1::timeAdvance);

    task->requires(Task::OldDW, phi_label, Ghost::AroundNodes, 1);
    task->computes(phi_label);
    task->computes(residual_label);
    sched->addTask(task, level->eachPatch(), sharedState_->allMaterials());
  }
\end{lstlisting}

The \Textbfc{timeAdvance} function is the main function that describes
the computational algorithm.  For simple examples, the entire
algorithm is usually defined by one task with a small set of data
dependencies.  However, for more complicated algorithms, it is best to
break the algorithm down into a set of tasks with each task describing
its own set of data dependencies.

For this example, a single task is described and the
\Textbfc{timeAdvance} function pointer is specified.  Data from the
previous timestep (\Textbfc{OldDW}) is required for the current
timestep.  For a simple seven (7) point stencil, only one level of
ghost cells is required.  The algorithm is set up for nodal values,
the ghost cells are specified by the the \Textbfc{Ghost::AroundNodes}
syntax.  The task computes both the new data values for phi and a
residual.

\subsection{Description of Computational Functions}

\begin{lstlisting}[language=Cpp]
  void Poisson1::computeStableTimestep(const ProcessorGroup* pg, const PatchSubset* /*patches*/, const MaterialSubset* /*matls*/, DataWarehouse*, DataWarehouse* new_dw)
  {
    if(pg->myrank() == 0){
      sum_vartype residual;
      new_dw->get(residual, residual_label);
      cerr << "Residual=" << residual << '\n';
    }
    new_dw->put(delt_vartype(delt_), sharedState_->get_delt_label());
  }
\end{lstlisting}

In this particular example, no timestep is actually computed, instead
the original timestep specified in the input file is used and stored
in the data warehouse (\Textbfc{new\_dw$\to$put(delt\_vartype(delt\_)},\\
  \Textbfc{sharedState\_$\to$get\_delt\_label())}).  The residual computed in the
main \Textbfc{timeAdvance} function is retrieved from the data
warehouse and printed out to standard error for only the processor
with a rank of 0.

\begin{lstlisting}[language=Cpp]
  void Poisson1::initialize(const ProcessorGroup*, const PatchSubset* patches, const MaterialSubset* matls, DataWarehouse* /*old_dw*/, DataWarehouse* new_dw)
  {
    int matl = 0;
    for(int p=0;p<patches->size();p++){
      const Patch* patch = patches->get(p);
\end{lstlisting}

    The node centered variable (\Textbfc{NCVariable\textless double\textgreater phi}) has space
    reserved in the DataWarehouse for the given patch and the given
    material (\Textbfc{int matl = 0;}).  The \Textbfc{phi} variable is
    initialized to 0 for every grid node on the patch.

\begin{lstlisting}[language=Cpp]
      NCVariable<double> phi;
      new_dw->allocateAndPut(phi, phi_label, matl, patch);
      phi.initialize(0.);
\end{lstlisting}

    The boundary faces on the xminus face of the computational domain are
    specified and set to a value of 1.  All other boundary values are set
    to a value of 0 as well as the internal nodes via the
    \Textbfc{phi.initialize(0.)} construct.  \Vaango provides helper
    functions for determining which nodes are on the boundaries.  In
    addition, there are convenient looping constructs such as
    \Textbfc{NodeIterator} that alleviate the need to specify triply nested
    loops for visiting each node in the domain.

\begin{lstlisting}[language=Cpp]
      if(patch->getBCType(Patch::xminus) != Patch::Neighbor){
        IntVector l,h;
        patch->getFaceNodes(Patch::xminus, 0, l, h);

        for(NodeIterator iter(l,h); !iter.done(); iter++){
          phi[*iter]=1;
        }
      }
      new_dw->put(sum_vartype(-1), residual_label);
    }
  }
\end{lstlisting}

The initial residual value of -1 is stored at the beginning of the
simulation.

The main computational algorithm is defined in the
\Textbfc{timeAdvance} function.  The overall algorithm is based on a
simple Jacobi iteration step.

\begin{lstlisting}[language=Cpp]
  void Poisson1::timeAdvance(const ProcessorGroup*, const PatchSubset* patches, const MaterialSubset* matls, DataWarehouse* old_dw, DataWarehouse* new_dw)
  {
    int matl = 0;
    for(int p=0;p<patches->size();p++){
      const Patch* patch = patches->get(p);
      constNCVariable<double> phi;
\end{lstlisting}

    Data from the previous timestep is retrieved from the data warehouse
    and copied to the current timestep's phi variable (\Textbfc{newphi}).

\begin{lstlisting}[language=Cpp]
      old_dw->get(phi, phi_label, matl, patch, Ghost::AroundNodes, 1);
      NCVariable<double> newphi;

      new_dw->allocateAndPut(newphi, phi_label, matl, patch);
      newphi.copyPatch(phi, newphi.getLowIndex(), newphi.getHighIndex());
\end{lstlisting}

    The indices for the patch are obtained and altered depending on
    whether or not the patch's internal boundaries are on the coincident
    with the grid domain.  If the patch boundaries are the same as the
    grid domain, the boundary values are not overwritten since the lower
    and upper indices are modified to only specify internal nodal grid
    points.


    %%% The implementation of more sensible boundary condition
    %%% specifications needs to be added here.

\begin{lstlisting}[language=Cpp]
      double residual=0;
      IntVector l = patch->getNodeLowIndex__New();
      IntVector h = patch->getNodeHighIndex__New();

      l += IntVector(patch->getBCType(Patch::xminus) == Patch::Neighbor?0:1,
      patch->getBCType(Patch::yminus) == Patch::Neighbor?0:1,
      patch->getBCType(Patch::zminus) == Patch::Neighbor?0:1);
      h -= IntVector(patch->getBCType(Patch::xplus)  == Patch::Neighbor?0:1,
      patch->getBCType(Patch::yplus)  == Patch::Neighbor?0:1,
      patch->getBCType(Patch::zplus)  == Patch::Neighbor?0:1);
\end{lstlisting}

    The Jacobi iteration step is applied at each internal grid node.  The
    residual is computed based on the old and new values and stored as a
    reduction variable (\Textbfc{sum\_vartype}) in the data warehouse.

\begin{lstlisting}[language=Cpp]
      //__________________________________
      //  Stencil
      for(NodeIterator iter(l, h);!iter.done(); iter++){
        IntVector n = *iter;

        newphi[n]=(1./6)*(
        phi[n+IntVector(1,0,0)] + phi[n+IntVector(-1,0,0)] +
        phi[n+IntVector(0,1,0)] + phi[n+IntVector(0,-1,0)] +
        phi[n+IntVector(0,0,1)] + phi[n+IntVector(0,0,-1)]);

        double diff = newphi[n] - phi[n];
        residual += diff * diff;
      }
      new_dw->put(sum_vartype(residual), residual_label);
    }
  }
\end{lstlisting}

\subsection{Input file}
The input file that is used to run this example is given below and is
given in
\Textsfc{SCIRun/src/Packages/Uintah/StandAlone/inputs/Examples/poisson1.ups}.
Relevant sections of the input file that can be modified are found in
the \Textbfc{\textless Time\textgreater } section, and the \Textbfc{\textless Grid\textgreater} section,
specifically, the number of patches and the grid resolution.

\begin{lstlisting}[language=XML]
<Uintah_specification>
  <Meta>
    <title>Poisson1 test</title>
  </Meta>
  <SimulationComponent>
    <type> poisson1 </type>
  </SimulationComponent>
  <Time>
    <maxTime>       1.0       </maxTime>
    <initTime>      0.0       </initTime>
    <delt_min>      0.00001   </delt_min>
    <delt_max>      1         </delt_max>
    <max_Timesteps> 100        </max_Timesteps>
    <timestep_multiplier>  1  </timestep_multiplier>
  </Time>
  <DataArchiver>
    <filebase>poisson.uda</filebase>
    <outputTimestepInterval>1</outputTimestepInterval>
    <save label = "phi"/>
    <save label = "residual"/>
    <checkpoint cycle = "2" timestepInterval = "1"/>
  </DataArchiver>
  <Poisson>
    <delt>.01</delt>
    <maxresidual>.01</maxresidual>
  </Poisson>
  <Grid>
    <Level>
      <Box label = "1">
        <lower>     [0,0,0]        </lower>
        <upper>     [1.0,1.0,1.0]  </upper>
        <resolution>[50,50,50]     </resolution>
        <patches>   [2,1,1]        </patches>
      </Box>
    </Level>
  </Grid>
</Uintah_specification>
\end{lstlisting}

\subsubsection{Running the Poisson1 Example}
To run the poisson1.ups example, 
\begin{lstlisting}[language=sh,backgroundcolor=\color{background}]
  cd ~/SCIRun/dbg/Packages/Uintah/StandAlone/
\end{lstlisting}
create a symbolic link to the inputs directory:
\begin{lstlisting}[language=sh,backgroundcolor=\color{background}]
  ln -s ~/SCIRun/src/Packages/Uintah/StandAlone/inputs
\end{lstlisting}
For a single processor run type the following:
\begin{lstlisting}[language=sh,backgroundcolor=\color{background}]
  vaango inputs/Examples/poisson1.ups
\end{lstlisting}
For a two processor run, type the following:
\begin{lstlisting}[language=sh,backgroundcolor=\color{background}]
  mpirun -np 2 vaango -mpi inputs/Examples/poisson1.ups
\end{lstlisting}
Changing the number of patches in the poisson1.ups and the resolution,
enables you to run a more refined problem on more processors.

\Textbfc{ADVICE:} For non-AMR problems, it is advised to have at least
the same number of patches as processors.  You can always have more
patches than processors, but you cannot have fewer patches than
processors.


\section{Poisson2}

The next example also solves the Poisson's equation but instead of
iterating in time, the subscheduler feature iterates within a given
timestep, thus solving the problem in one timestep.  The use of the
subcheduler is important for implementing algorithms which solve
non-linear problems which require iterating on a solution for each
timestep.

The majority of the schedule and computational functions are similar
to the \Textbfc{Poisson1} example and are not repeated.  Only the revised
code is presented with explanations about the new features of Uintah.

\begin{lstlisting}[language=Cpp]
  void Poisson2::scheduleTimeAdvance( const LevelP& level, SchedulerP& sched)
  {
    Task* task = scinew Task("timeAdvance", this, &Poisson2::timeAdvance, level, sched.get_rep());
    task->hasSubScheduler();
    task->requires(Task::OldDW, phi_label, Ghost::AroundNodes, 1);
    task->computes(phi_label);
    LoadBalancer* lb = sched->getLoadBalancer();
    const PatchSet* perproc_patches = lb->getPerProcessorPatchSet(level);
    sched->addTask(task, perproc_patches, sharedState_->allMaterials());
  }

\end{lstlisting}

Within this function, the task is specified with two additional
arguments, the level and the scheduler \Textbfc{sched.get\_rep()}.  The
task must also set the flag that a subscheduler will be used within
the scheduling of the various tasks.  Similar code to the
\Textbfc{Poisson1} example is used to specify what data is required and
computed during the actual task execution.  In addition, a
loadbalancer component is required to query the patch distribution for
each level of the grid.  The task is then added to the top level
scheduler with the requisite information, i.e. patches and materials.

The actual implementation of the \Textbfc{timeAdvance} function is also
different from the \Textbfc{Poisson1} example.  The code is specified
below with text explaining the use of the subscheduler.  The new
feature of the subscheduler shows the creation of a the
\Textbfc{iterate} task within the subscheduler.  This task will perform
the actual Jacobi iteration for a given timestep.

\begin{lstlisting}[language=Cpp]
  void Poisson2::timeAdvance(const ProcessorGroup* pg, const PatchSubset* patches, const MaterialSubset* matls, DataWarehouse* old_dw, DataWarehouse* new_dw, LevelP level, Scheduler* sched)
  {
  \end{lstlisting}

  The subscheduler is instantiated and initialized. 

  \begin{lstlisting}[language=Cpp]
    SchedulerP subsched = sched->createSubScheduler();
    subsched->initialize();
    GridP grid = level->getGrid();
  \end{lstlisting}

  An \Textbfc{iterate} task is created and added to the subscheduler.
  The typical computes and requires are specified for a 7 point stencil
  used in Jacobi iteration scheme with one layer of ghost cells.  The
  new task is added to the subscheduler.  A residual variable is only
  computed within the subscheduler and not passed back to the main
  scheduler.  This is in contrast to the \Textbfc{phi} variable which was
  specified in \Textbfc{scheduleTimeAdvance} in the computes, as well as
  being specified in the computes for the subscheduler.  Any variables
  that are only computed and used in an iterative step of an algorithm
  do not need to be added to the dependency specification for the top
  level task.

  \begin{lstlisting}[language=Cpp]
    // Create the tasks
    Task* task = scinew Task("iterate", this, &Poisson2::iterate);
    task->requires(Task::OldDW, phi_label, Ghost::AroundNodes, 1);
    task->computes(phi_label);
    task->computes(residual_label);
    subsched->addTask(task, level->eachPatch(), sharedState_->allMaterials());
  \end{lstlisting}

  The subscheduler has its own data wharehouse that is separate from the
  top level's scheduler's data warehouse.  This data warehouse must be
  initialized and any data from the top level's data warehouse must be
  passed to the subscheduler's version.  This data resides in the data
  warehouse position \Textbfc{NewDW}.

  \begin{lstlisting}[language=Cpp]
    // Compile the scheduler
    subsched->advanceDataWarehouse(grid);
    subsched->compile();

    int count = 0;
    double residual;
    subsched->get_dw(1)->transferFrom(old_dw, phi_label, patches, matls);
  \end{lstlisting}

  Within each iteration, the following must occur for the subscheduler:
  the data warehouse's new data must be moved to the \Textbfc{OldDW}
  position, since any new values will be stored in \Textbfc{NewDW} and
  the old values cannot be overwritten.  The \Textbfc{OldDW} is referred
  to in the subscheduler via the \Textbfc{subsched-\textgreater get\_dw(0)} and the
  \Textbfc{NewDW} is referred to in the subscheduler via
  \Textbfc{subsched-\textgreater get\_dw(1)}.  Once the iteration is deemed to have
  met the tolerance, the data from the subscheduler is transferred to
  the scheduler's data warehouse.

  \begin{lstlisting}[language=Cpp]
    // Iterate
    do {
      subsched->advanceDataWarehouse(grid);
      subsched->get_dw(0)->setScrubbing(DataWarehouse::ScrubComplete);
      subsched->get_dw(1)->setScrubbing(DataWarehouse::ScrubNonPermanent);
      subsched->execute();    

      sum_vartype residual_var;
      subsched->get_dw(1)->get(residual_var, residual_label);
      residual = residual_var;

      if(pg->myrank() == 0)
        cerr << "Iteration " << count++ << ", residual=" << residual << '\n';
    } while(residual > maxresidual_);

    new_dw->transferFrom(subsched->get_dw(1), phi_label, patches, matls);
  }

\end{lstlisting}

The iteration cycle is identical to \Textbfc{Poisson1}'s
\Textbfc{timeAdvance} algorithm using Jacobi iteration with a 7 point
stencil.  Refer to the discussion about the algorithm implementation
in the \Textbfc{Poisson1} description.

\begin{lstlisting}[language=Cpp]
  void Poisson2::iterate(const ProcessorGroup*, const PatchSubset* patches, const MaterialSubset* matls, DataWarehouse* old_dw, DataWarehouse* new_dw)
  {
    for(int p=0;p<patches->size();p++){
      const Patch* patch = patches->get(p);
      for(int m = 0;m<matls->size();m++){
        int matl = matls->get(m);
        constNCVariable<double> phi;
        old_dw->get(phi, phi_label, matl, patch, Ghost::AroundNodes, 1);
        NCVariable<double> newphi;
        new_dw->allocateAndPut(newphi, phi_label, matl, patch);
        newphi.copyPatch(phi, newphi.getLow(), newphi.getHigh());
        double residual=0;
        IntVector l = patch->getNodeLowIndex__New();
        IntVector h = patch->getNodeHighIndex__New(); 
        l += IntVector(patch->getBCType(Patch::xminus) == Patch::Neighbor?0:1,
        patch->getBCType(Patch::yminus) == Patch::Neighbor?0:1,
        patch->getBCType(Patch::zminus) == Patch::Neighbor?0:1);
        h -= IntVector(patch->getBCType(Patch::xplus) == Patch::Neighbor?0:1,
        patch->getBCType(Patch::yplus) == Patch::Neighbor?0:1,
        patch->getBCType(Patch::zplus) == Patch::Neighbor?0:1);
        for(NodeIterator iter(l, h);!iter.done(); iter++){
          newphi[*iter]=(1./6)*(
          phi[*iter+IntVector(1,0,0)]+phi[*iter+IntVector(-1,0,0)]+
          phi[*iter+IntVector(0,1,0)]+phi[*iter+IntVector(0,-1,0)]+
          phi[*iter+IntVector(0,0,1)]+phi[*iter+IntVector(0,0,-1)]);
          double diff = newphi[*iter]-phi[*iter];
          residual += diff*diff;
        }
        new_dw->put(sum_vartype(residual), residual_label);
      }
    }
  }
\end{lstlisting}

The input file
\Textsfc{src/StandAlone/inputs/Examples/poisson2.ups}
is very similar to the \Textbfc{poisson1.ups} file shown above.  The only
additional tag that is used is the \Textbfc{\textless maxresidual \textgreater} specifying
the tolerance within the iteration performed in the subscheduler.

To run the poisson2 input file execute the following in the \Textsfc{dbg} build
\Textsfc{StandAlone} directory:

\begin{lstlisting}[language=sh,backgroundcolor=\color{background}]
  vaango inputs/Examples/poisson2.ups
\end{lstlisting}


\section{Burger}
In this example, the inviscid Burger's equation is solved in three dimensions:
\Beq
  \frac{du}{dt} = -u \frac{du}{dx}  
\Eeq
with the initial conditions:
\Beq
  u = \sin(\pi x) + \sin(2\pi y) + \sin(3\pi z)
\Eeq
using Euler's method to advance in time.  The majority of the code is
very similar to the Poisson1 example code with the differences shown
below.

The initialization of the grid values for the unknown variable, u, is
done at every grid node using the NodeIterator construct.  The x,y,z
values for a given grid node is determined using the function,
\Textbfc{patch-\textgreater getNodePosition(n)}, where n is the nodal index in
i,j,k space.

\begin{lstlisting}[language=Cpp]
  void Burger::initialize(const ProcessorGroup*, const PatchSubset* patches, const MaterialSubset* matls, DataWarehouse*, DataWarehouse* new_dw)
  {
    int matl = 0;
    for(int p=0;p<patches->size();p++){
      const Patch* patch = patches->get(p);

      NCVariable<double> u;
      new_dw->allocateAndPut(u, u_label, matl, patch);

      //Initialize
      // u = sin( pi*x ) + sin( pi*2*y ) + sin(pi*3z )
      IntVector l = patch->getNodeLowIndex__New();
      IntVector h = patch->getNodeHighIndex__New();
      
      for( NodeIterator iter=patch->getNodeIterator__New(); !iter.done(); iter++ ){
        IntVector n = *iter;
        Point p = patch->nodePosition(n);
        u[n] = sin( p.x() * 3.14159265358 ) + sin( p.y() * 2*3.14159265358)  +  sin( p.z() * 3*3.14159265358);
      }
    }
  }
\end{lstlisting}

The \Textbfc{timeAdvance} function is quite similar to the Poisson1's
\Textbfc{timeAdvance} routine.  The relavant differences are only
shown.
\begin{lstlisting}[language=Cpp]
  void Burger::timeAdvance(const ProcessorGroup*, const PatchSubset* patches, const MaterialSubset* matls, DataWarehouse* old_dw, DataWarehouse* new_dw)
  {
    int matl = 0;
    //Loop for all patches on this processor
    for(int p=0;p<patches->size();p++){
      const Patch* patch = patches->get(p);
      . . . . .
\end{lstlisting}

    The grid spacing and timestep values are stored.
\begin{lstlisting}[language=Cpp]
      // dt, dx
      Vector dx = patch->getLevel()->dCell();
      delt_vartype dt;
      old_dw->get(dt, sharedState_->get_delt_label());
      . . . . .  
\end{lstlisting}

    Refer to the description in \Textbfc{Poisson1} about the specification
    of the \Textbfc{NodeIterator} limits.  The Euler algorithm is applied to solve
    the ordinary differential equation in time.
\begin{lstlisting}[language=Cpp]
      //Iterate through all the nodes
      for(NodeIterator iter(l, h);!iter.done(); iter++){    
        IntVector n = *iter;
        double dudx = (u[n+IntVector(1,0,0)] - u[n-IntVector(1,0,0)]) /(2.0 * dx.x());
        double dudy = (u[n+IntVector(0,1,0)] - u[n-IntVector(0,1,0)]) /(2.0 * dx.y());
        double dudz = (u[n+IntVector(0,0,1)] - u[n-IntVector(0,0,1)]) /(2.0 * dx.z());
        double du = - u[n] * dt * (dudx + dudy + dudz);
        new_u[n]= u[n] + du;
      }
\end{lstlisting}

    Zero flux Neumann boundary conditions are applied to the node points
    on each of the grid faces.
\begin{lstlisting}[language=Cpp]
      //__________________________________
      // Boundary conditions: Neumann
      // Iterate over the faces encompassing the domain
      vector<Patch::FaceType>::const_iterator iter;
      vector<Patch::FaceType> bf;
      patch->getBoundaryFaces(bf);
      for (iter  = bf.begin(); iter != bf.end(); ++iter){
        Patch::FaceType face = *iter;

        IntVector axes = patch->faceAxes(face);
        int P_dir = axes[0]; // find the principal dir of that face

        IntVector offset(0,0,0);
        if (face == Patch::xminus || face == Patch::yminus || face == Patch::zminus){
          offset[P_dir] += 1; 
        }
        if (face == Patch::xplus || face == Patch::yplus || face == Patch::zplus){
          offset[P_dir] -= 1;
        }

        Patch::FaceIteratorType FN = Patch::FaceNodes;
        for (CellIterator iter = patch->getFaceIterator__New(face,FN);!iter.done(); iter++){
          IntVector n = *iter;
          new_u[n] = new_u[n + offset];
        }
      }
    }
  }
\end{lstlisting}

The input file for the Burger (\Textttc{burger.ups}) problem is very
similar to the \Textttc{poisson1.ups} with the addition, that the
timestep increment used in the \Textbfc{timeAdvance} is quite small,
1.e-4 for stability reasons.

To run the Burger input file execute the following in the \Textsfc{dbg} build
\Textsfc{StandAlone} directory:
\begin{lstlisting}[language=sh,backgroundcolor=\color{background}]
  vaango inputs/Examples/burger.ups
\end{lstlisting}

