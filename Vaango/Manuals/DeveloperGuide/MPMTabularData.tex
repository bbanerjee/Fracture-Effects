\section{Models that use tabular data}
The tabular data infrastructure in the \Textsfc{Models} directory has the 
potential to be used for many constitutive models.  A brief description
of the implementation and design choices is given in this section.

The existing tabular data handling classes in \Vaango were designed to
deal with multidimensional structured data.  However, some \MPM models
need unstructured tabular data as input.  The \Textbfc{TabularData}
class was designed to handle unstructured tables containing (at most)
four independent variables.  However, it can be extended to deal
with more independent variables at the cost of added complexity.

The input tabular data are expected in the following JSON format:

\begin{lstlisting}[language=JSON]
{"Vaango_tabular_data": {
  "Meta" : {
    "title" : "Test data"
  },
  "Data" : {
    "Salinity" : [0.1, 0.2],
    "Data" : [{
      "Temperature" : [100, 200, 300],
      "Data" : [{
        "Volume" : [0.1, 0.2, 0.3, 0.4, 0.5, 0.6, 0.7, 0.8],
        "Pressure" : [10, 20, 30, 40, 50, 60, 70, 80]
        }, {
        "Volume" : [0.15, 0.25, 0.35],
        "Pressure" : [100, 200, 300]
        }, {
        "Volume" : [0.05, 0.45, 0.75],
        "Pressure" : [1000, 2000, 3000]
        }] 
      }, {
      "Temperature" : [0, 400],
      "Data" : [{
        "Volume" : [0.1, 0.2, 0.3, 0.4],
        "Pressure" : [15, 25, 35, 45]
        }, {
        "Volume" : [0.1, 0.45, 0.65],
        "Pressure" : [150, 250, 350]
        }]
      }]
    }
  }
}
\end{lstlisting}

\subsection{The TableContainers class}
To keep things easier to understand, the \Textbfc{TabularData} class has been
split into two parts: a \Textbfc{TableContainers} class and the main \Textbfc{TabularData}
class.

The \Textbfc{TableContainers} class contains the following:
\begin{enumerate}
  \item The independent and dependent variable name and data are stored in the \Textbfc{TableVar}
        structure which has the form:
\begin{lstlisting}[language=Cpp]
  struct TableVar
  {
    std::string name;
    std::unordered_map<IndexKey, std::vector<double>, IndexHash, IndexEqual>
      data;
    TableVar() {}
    TableVar(const std::string name) { this->name = name; }
  };
\end{lstlisting}
        The data for each variable is stored as a vector of doubles and associated with a 
        key that is determined by the indices of the associated independent variables.
        For easier reading, the structure TableVar is given the aliases
        \Textbfc{IndependentVar} and \Textbfc{DependentVar}:
\begin{lstlisting}[language=Cpp]
  using IndependentVar = TableVar;
  using DependentVar = TableVar;
\end{lstlisting}

  \item The \Textbfc{IndexKey} object that is used to locate a particular table variable data vector
        has an associated a \Textbfc{IndexHash} functor and an \Textbfc{IndexEqual} equality 
        operator functor as shown below:
\begin{lstlisting}[language=Cpp]
  struct IndexKey
  {
    std::uint8_t _ii;
    std::uint8_t _jj;
    std::uint8_t _kk;
    std::uint8_t _ll;
    IndexKey(std::uint8_t ii, std::uint8_t jj, std::uint8_t kk, std::uint8_t ll)
      : _ii(ii) , _jj(jj) , _kk(kk) , _ll(ll)
    {
    }
  };
\end{lstlisting}
\begin{lstlisting}[language=Cpp]
  struct IndexEqual
  {
    bool operator()(const IndexKey& lhs, const IndexKey& rhs) const
    {
      return (lhs._ii == rhs._ii && lhs._jj == rhs._jj && lhs._kk == rhs._kk &&
              lhs._ll == rhs._ll);
    }
  };
\end{lstlisting}
\begin{lstlisting}[language=Cpp]
  struct IndexHash
  {
    std::size_t operator()(const IndexKey& key) const
    {
      std::size_t hashval = key._ii;
      hashval *= 37;
      hashval += key._jj;
      hashval *= 37;
      hashval += key._kk;
      hashval *= 37;
      hashval += key._ll;
      return hashval;
    }
  };
\end{lstlisting}

\end{enumerate}

\subsection{The TabularData class}
The \Textbfc{TabularData} class header file uses, among other things, the
\Textsfc{json.hpp} header-only JSON library that is included as a git submodule
in \Vaango.  The current version only allows for linear interpolation and leaves
higher order interpolation to the calling programs.

\begin{lstlisting}[language=Cpp]
#ifndef VAANGO_MPM_CONSTITUTIVE_MODEL_TABULAR_DATA_H
#define VAANGO_MPM_CONSTITUTIVE_MODEL_TABULAR_DATA_H

#include <CCA/Components/MPM/ConstitutiveModel/Models/TableContainers.h>
#include <submodules/json/src/json.hpp>

namespace Vaango {

class TabularData
{
public:
  // Constructors, assignment, destructor
  TabularData() {};
  TabularData(Uintah::ProblemSpecP& ps);
  TabularData(const TabularData& table);
  ~TabularData() = default;
  TabularData& operator=(const TabularData& table);

  // Initialization and output
  void initialize();
  void outputProblemSpec(Uintah::ProblemSpecP& ps);

  // Add variables
  void addIndependentVariable(const std::string& varName);
  std::size_t addDependentVariable(const std::string& varName);

  // A setup method
  void setup();

  // A method to modify the table data if needed
  template <int dim> void translateAlongNormals(const std::vector<Uintah::Vector>& vec, const double& shift);
  template <int dim> void translateIndepVar1ByIndepVar0();

  // File read methods
  template <int dim> void readJSONTableFromFile(const std::string& tableFile);
  template <int dim> void readJSONTable(const nlohmann::json& doc, const std::string& tableFile);

  // A method for interpolation
  template <int dim> DoubleVec1D interpolate(const std::array<double, dim>& indepValues) const;
  template <int dim> DoubleVec1D interpolateLinearSpline( const std::array<double, dim>& indepValues, const IndepVarPArray& indepVars, const DepVarPArray& depVars) const;

  // Some get and set methods
  std::size_t getNumIndependents() const { return d_indepVars.size(); }
  std::size_t getNumDependents() const { return d_depVars.size(); }
  const IndepVarPArray& getIndependentVars() const { return d_indepVars; }
  const DepVarPArray& getDependentVars() const { return d_depVars; }

  DoubleVec1D getIndependentVarData(const std::string& name, const IndexKey& index) const;
  DoubleVec1D getDependentVarData(const std::string& name, const IndexKey& index) const;
  void setIndependentVarData(const std::string& name, const IndexKey& index, const DoubleVec1D& data);
  void setDependentVarData(const std::string& name, const IndexKey& index, const DoubleVec1D& data);

private:
  std::string d_filename;
  std::string d_indepVarNames;
  std::string d_depVarNames;
  std::string d_interpType;

  IndepVarPArray d_indepVars;
  DepVarPArray d_depVars;

  // A parser for variable names
  std::vector<std::string> parseVariableNames(const std::string& vars);

  // JSON related methods
  nlohmann::json loadJSON(std::stringstream& inputStream, const std::string& fileName);
  nlohmann::json getContentsJSON(const nlohmann::json& doc, const std::string& fileName);
  std::string getTitleJSON(const nlohmann::json& contents, const std::string& fileName);
  nlohmann::json getDataJSON(const nlohmann::json& contents, const std::string& fileName);
  DoubleVec1D getVectorJSON(const nlohmann::json& object, const std::string key, const std::string& tableFile);
  DoubleVec1D getDoubleArrayJSON(const nlohmann::json& object, const std::string key, const std::string& tableFile);

  // Search and interpolation methods
  std::size_t findLocation(const double& value, const DoubleVec1D& varData) const;
  double computeParameter(const double& input, const std::size_t& startIndex, const DoubleVec1D& data) const;
  double computeInterpolated(const double& tval, const std::size_t& startIndex, const DoubleVec1D& data) const;
};
}
#endif // VAANGO_MPM_CONSTITUTIVE_MODEL_TABULAR_DATA_H
\end{lstlisting}

The \Textbfc{TabularData} class uses the following aliases:
\begin{lstlisting}[language=Cpp]
using IndexKey = TableContainers::IndexKey;

using DoubleVec1D = std::vector<double>;
using DoubleVec2D = std::vector<DoubleVec1D>;

using IndependentVarP = std::unique_ptr<TableContainers::IndependentVar>;
using DependentVarP = std::unique_ptr<TableContainers::DependentVar>;

using IndepVarPArray = std::vector<IndependentVarP>;
using DepVarPArray = std::vector<DependentVarP>;
\end{lstlisting}

\subsection{A TabularData implementation}
The current implementation of \Textbfc{TabularData} was designed with the
\Textbfc{TabularPlasticity} model in mind, specifically, the need to be able to
represent and interpolate between several unloading curves at different values
of plastic strain.

\subsubsection{Construction}
A \Textbfc{TabularData} object is created during input file processing using
the following constructor.  A copy constructor and an assignment operator are
also provided.
\begin{lstlisting}[language=Cpp]
// Read input file and construct
TabularData::TabularData(ProblemSpecP& ps)
{
  ps->require("independent_variables", d_indepVarNames);
  ps->require("dependent_variables", d_depVarNames);
  ps->require("filename", d_filename);
  ProblemSpecP interp = ps->findBlock("interpolation");
  if (!interp) {
    d_interpType = "linear";
  } else {
    if (!interp->getAttribute("type", d_interpType)) {
      throw ProblemSetupException("**ERROR** Interpolation \
        tag needs type=linear/cubic", __FILE__, __LINE__);
    }
  }
  initialize();
}
\end{lstlisting}
The \Textsfc{independent\_variables} and \Textsfc{dependent\_variables} have to
be specified in the format \Textsfc{var1, var2, var3}.  

The constructor calls the \Textsfc{initialize} method that parses
the variable names from the input lists and allocates heap memory for
the tabular data.
\begin{lstlisting}[language=Cpp]
void
TabularData::initialize()
{
  std::vector<std::string> indepVarNames = parseVariableNames(d_indepVarNames);
  std::vector<std::string> depVarNames = parseVariableNames(d_depVarNames);
  for (const auto& name : indepVarNames) {
    addIndependentVariable(name);
  }
  for (const auto& name : depVarNames) {
    addDependentVariable(name);
  }
}
\end{lstlisting}

The variable name parser splits the strings in the input file using "," as
a separator.  Variable names may contain spaces in the middle but have to
be separated by commas.
\begin{lstlisting}[language=Cpp]
std::vector<std::string>
TabularData::parseVariableNames(const std::string& vars)
{
  std::vector<std::string> varNames = Vaango::Util::split(vars, ',');
  for (auto& name : varNames) {
    name = Vaango::Util::trim(name);
  }
  return varNames;
}
\end{lstlisting}

Variables are created on the stack and added to the list
of independent variables.
\begin{lstlisting}[language=Cpp]
void
TabularData::addIndependentVariable(const std::string& varName)
{
  d_indepVars.push_back(std::make_unique<IndependentVar>(varName));
}
\end{lstlisting}

Each set of independent variables can have more that one dependent variable.
These are created next.
\begin{lstlisting}[language=Cpp]
std::size_t
TabularData::addDependentVariable(const std::string& varName)
{
  // Check if dependent variable already exists
  // and just return index if true
  for (auto ii = 0u; ii < d_depVars.size(); ii++) {
    if (d_depVars[ii]->name == varName) {
      return ii;
    }
  }
  // Name does not exist, add the new variable and return its index
  d_depVars.push_back(std::make_unique<DependentVar>(varName));
  return d_depVars.size() - 1;
}
\end{lstlisting}
The copy constructor calls the assignment operator and carries out a deep
copy of the data.
\begin{lstlisting}[language=Cpp]
TabularData::TabularData(const TabularData& table) {
  *this = table;
}

TabularData& 
TabularData::operator=(const TabularData& table) {
  if (this != &table) {
    d_indepVarNames = table.d_indepVarNames;
    d_depVarNames = table.d_depVarNames;
    d_filename = table.d_filename;
    for (auto ii = 0u; ii < table.d_indepVars.size(); ii++) {
      addIndependentVariable(table.d_indepVars[ii]->name);
      d_indepVars[ii]->data = table.d_indepVars[ii]->data;
    }
    for (auto ii = 0u; ii < table.d_depVars.size(); ii++) {
      addDependentVariable(table.d_depVars[ii]->name);
      d_depVars[ii]->data = table.d_depVars[ii]->data;
    }
  }
  return *this;
}
\end{lstlisting}

\subsubsection{Setup}
The \Textsfc{TabularData} data can be populated either by reading a file or
directly using a JSON object.  {\Red \textbf{Warning}: Do not expect the 
data to be populated automatically just because you have specified a file name in the input.}
You have to explicitly call the \Textsfc{setup} method in the calling program
after you have created the \Textbfc{TabularData} object.

The \Textsfc{setup} method is listed below.  At present we allow for only three
independent variables, but more can be added with template specializations.
\begin{lstlisting}[language=Cpp]
void
TabularData::setup()
{
  if (d_indepVars.size() == 1) {
    readJSONTableFromFile<1>(d_filename);
  } else if (d_indepVars.size() == 2) {
    readJSONTableFromFile<2>(d_filename);
  } else if (d_indepVars.size() == 3) {
    readJSONTableFromFile<3>(d_filename);
  } else {
    std::ostringstream out;
    out << "**ERROR**" << " More than three independent variables not allowed in " << d_filename;
    throw ProblemSetupException(out.str(), __FILE__, __LINE__);
  }
}
\end{lstlisting}

\subsubsection{Reading the JSON data from a file}
The process of reading is the JSON data consists of loading the file into an 
input stream and then converting the stream int a JSON object:
\begin{lstlisting}[language=Cpp]
template <int dim>
void
TabularData::readJSONTableFromFile(const std::string& tableFile)
{
  std::ifstream inputFile(tableFile);
  if (!inputFile) {
    std::ostringstream out;
    out << "**ERROR**" << " Cannot read tabular input data file " << tableFile;
    throw ProblemSetupException(out.str(), __FILE__, __LINE__);
  }

  std::stringstream inputStream;
  inputStream << inputFile.rdbuf();
  inputFile.close();

  json doc = loadJSON(inputStream, tableFile);
  readJSONTable<dim>(doc, tableFile);
}

json
TabularData::loadJSON(std::stringstream& inputStream, const std::string& tableFile)
{
  json doc;
  try {
    doc << inputStream;
  } catch (std::invalid_argument err) {
    std::ostringstream out;
    out << "**ERROR**" << " Cannot parse tabular input data file " << tableFile << "\n" << " Please check that the file is valid JSON using a linter";
    throw ProblemSetupException(out.str(), __FILE__, __LINE__);
  }
  return doc;
}
\end{lstlisting}

\subsubsection{Reading the data from a JSON object}
After the JSON object has been created, we can interpret the contents according
to the number of independent variables it contains.  For example, if there
are three independent variables, we specialize the \Textsfc{readJSONTable} method
as follows:
\begin{lstlisting}[language=Cpp]
template <>
void
TabularData::readJSONTable<3>(const json& doc, const std::string& tableFile)
{
  json contents = getContentsJSON(doc, tableFile);
  std::string title = getTitleJSON(contents, tableFile);
  json data = getDataJSON(contents, tableFile);

  DoubleVec1D indepVar0Data = getDoubleArrayJSON(data, d_indepVars[0]->name, tableFile);
  d_indepVars[0]->data.insert({ IndexKey(0, 0, 0, 0), indepVar0Data });

  json data0 = getDataJSON(data, tableFile);
  for (auto ii = 0u; ii < indepVar0Data.size(); ii++) {

    DoubleVec1D indepVar1Data = getDoubleArrayJSON(data0[ii], d_indepVars[1]->name, tableFile);
    d_indepVars[1]->data.insert({ IndexKey(ii, 0, 0, 0), indepVar1Data });

    json data1 = getDataJSON(data0[ii], tableFile);
    for (auto jj = 0u; jj < indepVar1Data.size(); jj++) {

      int index = 0;
      for (const auto& indepVar : d_indepVars) {
        if (index > 1) {
          DoubleVec1D indepVar2Data = getDoubleArrayJSON(data1[jj], indepVar->name, tableFile);
          indepVar->data.insert({ IndexKey(ii, jj, 0, 0), indepVar2Data });
        }
        index++;
      }

      for (const auto& depVar : d_depVars) {
        DoubleVec1D depVar2Data = getDoubleArrayJSON(data1[jj], depVar->name, tableFile);
        depVar->data.insert({ IndexKey(ii, jj, 0, 0), depVar2Data });
      }
    }
  }
}
\end{lstlisting}
The helper functions \Textsfc{getContentsJSON}, \Textsfc{getDataJSON}, and \Textsfc{getDoubleArrayJSON}
are encapsulate \Textsfc{get} methods provided by the JSON library.  Notice that the data are
expected to be in the order in which the independent variables have been specified.  Checks for
consistency are minimal.

\subsubsection{Translation of the data}
In some material submodels, the data have to be translated by a given amount.  These 
are accomplished by \Textsfc{translateIndepVar1ByIndepVar0<2>} for the tabular elastic moduli model,
and \Textsfc{translateAlongNormals<1>} for the yield condition model.

\subsubsection{Interpolation of the data}
The current implementation allows only for linear interpolation of the data. 
{\Red \textbf{Warning:} Extrapolation is strictly not allowed and will cause an exception to be thrown.}
The interpolation method has the form
\begin{lstlisting}[language=Cpp]
template <int dim>
DoubleVec1D
TabularData::interpolate(const std::array<double, dim>& indepValues) const
{
  return interpolateLinearSpline<dim>(indepValues, d_indepVars, d_depVars);
}
\end{lstlisting}
The linear interpolation method for three independent variables is listed below.
\begin{lstlisting}[language=Cpp]
template <>
DoubleVec1D
TabularData::interpolateLinearSpline<3>( const std::array<double, 3>& indepValues, const IndepVarPArray& indepVars, const DepVarPArray& depVars) const
{
  // Get the numbers of vars
  auto numDepVars = depVars.size();

  // First find the segment containing the first independent variable value
  // and the value of parameter s
  auto indepVarData0 = getIndependentVarData(indepVars[0]->name, IndexKey(0, 0, 0, 0));
  auto segLowIndex0 = findLocation(indepValues[0], indepVarData0);
  auto sval = computeParameter(indepValues[0], segLowIndex0, indepVarData0);

  // Choose the two vectors containing the relevant independent variable data
  // and find the segments containing the data
  DoubleVec1D depValsT;
  for (auto ii = segLowIndex0; ii <= segLowIndex0 + 1; ii++) {
    auto indepVarData1 = getIndependentVarData(indepVars[1]->name, IndexKey(ii, 0, 0, 0));
    auto segLowIndex1 = findLocation(indepValues[1], indepVarData1);
    auto tval = computeParameter(indepValues[1], segLowIndex1, indepVarData1);

    // Choose the last two vectors containing the relevant independent variable data
    // and find the segments containing the data
    DoubleVec1D depValsU;
    for (auto jj = segLowIndex1; jj <= segLowIndex1 + 1; jj++) {
      auto indepVarData2 = getIndependentVarData(indepVars[2]->name, IndexKey(ii, jj, 0, 0));
      auto segLowIndex2 = findLocation(indepValues[2], indepVarData2);
      auto uval = computeParameter(indepValues[2], segLowIndex2, indepVarData2);

      for (const auto& depVar : depVars) {
        auto depVarData = getDependentVarData(depVar->name, IndexKey(ii, jj, 0, 0));
        auto depvalU = computeInterpolated(uval, segLowIndex2, depVarData);
        depValsU.push_back(depvalU);
      }
    }

    // First interpolation step
    for (auto index = 0u; index < numDepVars; index++) {
      auto depvalT = (1 - tval) * depValsU[index] + tval * depValsU[index + numDepVars];
      depValsT.push_back(depvalT);
    }
  }

  // Second interpolation step
  DoubleVec1D depVals;
  for (auto index = 0u; index < numDepVars; index++) {
    auto depval = (1 - sval) * depValsT[index] + sval * depValsT[index + numDepVars];
    depVals.push_back(depval);
  }

  return depVals;
}
\end{lstlisting}
In the above, the \Textsfc{findLocation}, \Textsfc{computeParameter}, and \Textsfc{computeInterpolated}
methods are used to find the data points that are to be interpolated between.  Details can be
found in the \Vaango Theory Manual.
