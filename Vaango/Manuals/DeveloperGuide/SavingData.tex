\chapter{Saving Simulation Data}
The UCF automatically saves data as specified in the .ups file.  This
works for all data that is stored in the Data Warehouse.  However,
there are times when a component will need to save its own data.  (It
is preferable that the Data Warehouse be updated to manage this
Component data, but sometimes this is not expedient...)  In these
cases, here is an outline on how to save that data:

\begin{itemize}
\item Create a function that will save your data:
  \begin{lstlisting}[language=Cpp]
    void
    Component::saveMyData( const PatchSubset * patches )
    {
      ...
    }
  \end{lstlisting}
\item From the last \Textbfc{Task} in your algorithm, call the function:
  \begin{lstlisting}[language=Cpp]
    saveMyData( patches );
  \end{lstlisting}
\item In the function, you need to do several things:
  Make sure that it is an output timestep.  (Your component must have
  saved a pointer to the Data Archiver.  Most components already do
  this... if not, you can do it in \Textsfc{Component::problemSetup()}.)
  \begin{lstlisting}[language=Cpp]
    if( dataArchiver_->isOutputTimestep() ) {
      // Dump your data... (see below)
    }
  \end{lstlisting}

\item You need to create a directory to put the data in:
  \begin{lstlisting}[language=Cpp]
    const int     & timestep = d_sharedState->getCurrentTopLevelTimeStep();
    ostringstream   data_dir;

    // This gives you ".../simulation.uda.###/t#####/mydata/
    data_dir << dataArchiver()->getOutputLocation() << "/t" << setw(5) << setfill('0') << timestep
    << "/mydata";
    
    int result = MKDIR( data_dir.c_str(), 0777 );  // Make sure you #include <Core/OS/Dir.h>

    if( result != 0 ) {
      ostringstream error;
      error << "Component::saveMyData(): couldn't create directory '") + data_dir.str() + "' (" << result << ").";
      throw InternalError( error.str(), __FILE__, __LINE__);        
    }
  \end{lstlisting}

\item Loop over the patches and save your data... (Note, depending on
  how much data you have, you need to determine whether you want to
  write the files as a binary file, or as an ascii file.  Ascii files
  are easier for a human to look at for errors, but take a lot more
  time to read and space to write.)
  \begin{lstlisting}[language=Cpp]
    for( int pIndex = 0; pIndex < patches->size(); pIndex++ ){
      const Patch * patch = patches->get( pIndex );
      ostringstream file_name;
      // WARNING... not sure this is the correct naming convention... PLEASE VERIFY AND UPDATE THIS DOC.
      file_name << data_dir << "/p" << setw(5) << setfill('0') << pIndex;

      ofstream output( ceFileName, ios::out | ios::binary);
      if( !output ) {
        throw InternalError( string( "Component::saveMyData(): couldn't open file '") + file_name.str() + "'.",
        __FILE__, __LINE__);        
      }

      // Output data
      for( loop over data ) {
        output << data...
      }

      output.close();
    } // end for pIndex
  \end{lstlisting}

\end{itemize}
