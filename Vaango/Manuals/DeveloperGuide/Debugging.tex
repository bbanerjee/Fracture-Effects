\chapter{Debugging}

Debugging multi-processor software can be very difficult.  Below are
some hints on how to approach this.

\section{Debugger}

Use a debugger such as \Textsfc{GDB} to attach to the running program.
Note, if you are an Emacs user, running GDB through Emacs' gdb-mode
makes debugging even easier!

\subsection{Serial Debugging}

Whenever possible, debug \Textsfc{vaango} running in serial (ie. no MPI).
This is much easier for many reasons (though may also not be possible
in many situations).  To debug serially, use:

\begin{lstlisting}[language=sh,backgroundcolor=\color{background}]
gdb vaango <input_file>
\end{lstlisting}

\subsection{Parallel Debugging}

If it is necessary to use a parallel run of \Textsfc{vaango} to debug, a few things
must take place.

1) The helpful macro \Textttc{WAIT\_FOR\_DEBUGGER();} must be inserted into
the code (vaango.cc) just before the real execution of the components
begins \Textsfc{ctl-\textgreater run()}:

\begin{lstlisting}[language=Cpp]
WAIT_FOR_DEBUGGER();
ctl->run();
\end{lstlisting}

Recompile the code (will only take a few seconds), then run through
MPI as normal.  

\subsubsection{Caveats}

\begin{enumerate}
 \item  Unless you know that the error occurs on a specific processor, you will need to attach a debugger to \Textsfc{every}
process... so it behooves the developer to narrow the problem down to
as few processors as possible.

 \item  If you did know the processor, then the \Textttc{WAIT\_FOR\_DEBUGGER()} will
need to be placed inside an 'if' statement to check for the correct
MPI Rank.

 \item  Try to use only one node.  If you need more 'processors', you (most
likely) can double up on the same machine.  If \Textsfc{vaango} must span multiple
nodes, then you will need to log into each node separately to attach
debuggers (see below).
\end{enumerate}

\subsection{Running Vaango}

Run \Textsfc{vaango}:

\begin{lstlisting}[language=sh,backgroundcolor=\color{background}]
mpirun -np 4 -m mpihosts vaango inputs/MPM/thickCylinderMPM.ups
\end{lstlisting}

Create a new terminal window for each processor you are using.  (In
the above case, 4 windows.)  

You will notice that the output from \Textsfc{vaango} will include lines (one for
each processor used) like this:

\begin{lstlisting}[language=sh,backgroundcolor=\color{background}]
updraft2:22793 waiting for debugger
\end{lstlisting}

Attach to each process in the following manner (where PID is the
process id number (22793 in the above case)):

\begin{lstlisting}[language=sh,backgroundcolor=\color{background}]
cd Uintah/bin/StandAlone
gdb -p <PID>
\end{lstlisting}

Within GDB, you will then need to break each process out of the WAIT loop in the
following manner:

\begin{lstlisting}[language=sh,backgroundcolor=\color{background}]
up 2
set wait=false
cont
\end{lstlisting}

The above GDB commands must be run from each GDB session.  Once all
the WAITs are stopped, \Textsfc{vaango} will begin to run.



