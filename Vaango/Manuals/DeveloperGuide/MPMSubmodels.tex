\chapter{Submodels of MPM material models}
The previous chapter on the \Arena model showed that MPM constitutive models can
rapidly become unmanageable without sufficient object granularity.  To allow
for simpler model development and testing processes to be used in \Vaango, 
constitutive models are sometime broken up into submodels.

In \Vaango, we have the main constitutive models are located in the directory
\Textsfc{src/CCA/Components/MPM/ConstitutiveModel}.  Some of the models in
this directory depend of one of two sets of submodels.  The first set of 
submodels were designed circa 2004-2005 and are located in the directory
\Textsfc{...ConstitutiveModel/PlasticityModels}.  These models have been 
retained largely for backward compatibility with older plasticity models.
A more recent set of models (designed circa 2012) can be found in 
\Textsfc{...ConstitutiveModel/Models}.

\section{Models in the PlasticityModels directory}
\Textsfc{PlasticityModels} contains several submodels that
are primarily applicable to high stain-rate metal plasticity.  The
models in this folder are derived from the following base classes:
\begin{lstlisting}[language=sh, backgroundcolor=\color{background}]
PlasticityModels/
|-- DamageModel.h
|-- DevStressModel.h
|-- FlowModel.h
|-- KinematicHardeningModel.h
|-- MeltingTempModel.h
|-- MPMEquationOfState.h
|-- ShearModulusModel.h
|-- SpecificHeatModel.h
|-- StabilityCheck.h
|-- ViscoPlasticityModel.h
|-- YieldCondition.h
\end{lstlisting}

These models are created inside a \Textbfc{ConstitutiveModel} using the 
factory idiom.  The following factories are available:
\begin{lstlisting}[language=sh, backgroundcolor=\color{background}]
PlasticityModels/
|-- DamageModelFactory.cc
|-- DevStressModelFactory.cc
|-- FlowStressModelFactory.cc
|-- KinematicHardeningModelFactory.cc
|-- MeltingTempModelFactory.cc
|-- MPMEquationOfStateFactory.cc
|-- ShearModulusModelFactory.cc
|-- SpecificHeatModelFactory.cc
|-- StabilityCheckFactory.cc
|-- ViscoPlasticityModelFactory.cc
|-- YieldConditionFactory.cc
\end{lstlisting}

The material state is communicated to the submodels through the
two \textsf{struct}s:
\begin{lstlisting}[language=sh, backgroundcolor=\color{background}]
PlasticityModels/
|-- DeformationState.h
|-- PlasticityState.h
\end{lstlisting}

Each model factory can produce several types of submodel objects.  
The \Vaango implementation contains the following specialized
submodels.

\subsection{Damage models}
\begin{lstlisting}[language=sh, backgroundcolor=\color{background}]
PlasticityModels/
├── HancockMacKenzieDamage.cc
├── JohnsonCookDamage.cc
├── NullDamage.cc
\end{lstlisting}

\subsection{Deviatoric stress models}
\begin{lstlisting}[language=sh, backgroundcolor=\color{background}]
PlasticityModels/
├── HypoElasticDevStress.cc
├── HypoViscoElasticDevStress.cc
\end{lstlisting}

\subsection{Flow stress models}
\begin{lstlisting}[language=sh, backgroundcolor=\color{background}]
PlasticityModels/
├── IsoHardeningFlow.cc
├── JohnsonCookFlow.cc
├── MTSFlow.cc
├── PTWFlow.cc
├── SCGFlow.cc
├── ZAFlow.cc
├── ZAPolymerFlow.cc
\end{lstlisting}

\subsection{Kinematic hardening models}
\begin{lstlisting}[language=sh, backgroundcolor=\color{background}]
PlasticityModels/
├── ArmstrongFrederickKinematicHardening.cc
├── NoKinematicHardening.cc
├── PragerKinematicHardening.cc
\end{lstlisting}

\subsection{Melting temperature models}
\begin{lstlisting}[language=sh, backgroundcolor=\color{background}]
PlasticityModels/
├── BPSMeltTemp.cc
├── ConstantMeltTemp.cc
├── LinearMeltTemp.cc
├── SCGMeltTemp.cc
\end{lstlisting}

\subsection{Equation of state models}
\begin{lstlisting}[language=sh, backgroundcolor=\color{background}]
PlasticityModels/
├── DefaultHypoElasticEOS.cc
├── HyperElasticEOS.cc
├── MieGruneisenEOS.cc
├── MieGruneisenEOSEnergy.cc
\end{lstlisting}

\subsection{Shear modulus models}
\begin{lstlisting}[language=sh, backgroundcolor=\color{background}]
PlasticityModels/
├── ConstantShear.cc
├── MTSShear.cc
├── NPShear.cc
├── PTWShear.cc
├── SCGShear.cc
\end{lstlisting}

\subsection{Specific heat models}
\begin{lstlisting}[language=sh, backgroundcolor=\color{background}]
PlasticityModels/
├── ConstantCp.cc
├── CopperCp.cc
├── CubicCp.cc
├── SteelCp.cc
\end{lstlisting}

\subsection{Stability check models}
\begin{lstlisting}[language=sh, backgroundcolor=\color{background}]
PlasticityModels/
├── BeckerCheck.cc
├── DruckerBeckerCheck.cc
├── DruckerCheck.cc
├── NoneCheck.cc
\end{lstlisting}

\subsection{Viscoplasticity models}
\begin{lstlisting}[language=sh, backgroundcolor=\color{background}]
PlasticityModels/
├── SuvicI.cc
\end{lstlisting}

\subsection{Yield condition models}
\begin{lstlisting}[language=sh, backgroundcolor=\color{background}]
PlasticityModels/
├── GursonYield.cc
├── VonMisesYield.cc
\end{lstlisting}

\section{Using the models in PlasticityModels}
Suppose that you want to design a new constitutive model \Textbfc{MyModel}
but want to reuse some of the flow stress models in 
the \Textsfc{PlasticityModels} directory.
The following steps are needed to integrate the submodels into your new model.

\begin{enumerate}
  \item In your header file, \Textsfc{MyModel.h}, create a private pointer to the
        model:
\begin{lstlisting}[language=Cpp]
#ifndef __MPM_CM_MyModel_H__
#define __MPM_CM_MyModel_H__
#include <CCA/Components/MPM/ConstitutiveModel/PlasticityModels/FlowModel.h>
namespace Vaango {
  class MyModel {
    public:
        //......
    private:
        Uintah::FlowModel* d_flow;  
  };
}
#endif
\end{lstlisting}

  \item In your implementation file, \Textsfc{MyModel.cc}, create a copy of the
        model in the constructor and delete the copy in the destructor:
\begin{lstlisting}[language=Cpp]
#include <CCA/Components/MPM/ConstitutiveModel/MyModel.h>
#include <CCA/Components/MPM/ConstitutiveModel/PlasticityModels/FlowStressModelFactory.h>
// Constructor
MyModel::MyModel(Uintah::ProblemSpecP& ps, Uintah::MPMFlags* flags) : ConstitutiveModel(flags)
{
  d_flow = FlowStressModelFactory::create(ps);
  if (!d_flow) {
    std::ostringstream err;
    err << "An error occured in the FlowModelFactory that has \n"
        << " slipped through the existing bullet proofing. \n";
    throw Uintah::ProblemSetupExcepton(err.str(), __FILE__, __LINE__);
  }
}
// Copy constructor
MyModel::MyModel(const MyModel* model) : ConstitutiveModel(model)
{
  d_flow = FlowStressModelFactory::createCopy(model->d_flow);
}
// Destructor
MyModel::~MyModel()
{
  delete d_flow;
}
\end{lstlisting}

  \item To make sure that the details of the flow stress model are added to the output for
        restarting the simulation from a checkpoint, you will have to add the
        following to \Textsfc{MyModel.cc}:
\begin{lstlisting}[language=Cpp]
void
MyModel::outputProblemSpec(Uintah::ProblemSpecP& ps, bool output_cm_tag)
{
  Uintah::ProblemSpecP model_ps = ps;
  if (output_cm_tag) {
    model_ps = ps->appendChild("constitutive_model");
    model_ps->setAttribute("type", "my_model_tag");
  }
  d_flow->outputProblemSpec(model_ps);
}
\end{lstlisting}

  \item Some flow stress models have their own associated internal variables.  You will
        have to make sure that these are initialized, even if you don't plan to use a 
        model with submodel internal variables.
\begin{lstlisting}[language=Cpp]
// Set up particle state
void
MyModel::addParticleState(std::vector<const Uintah::VarLabel*>& from, std::vector<const Uintah::VarLabel*>& to)
{
  d_flow->addParticleState(from, to);
}
// Set up initialization task
void
MyModel::addInitialComputesAndRequires(Uintah::Task* task, const Uintah::MPMMaterial* matl, const Uintah::PatchSet* patch) const
{
  const Uintah::MaterialSubset* matlset = matl->thisMaterial();
  d_flow->addInitialComputesAndRequires(task, matl, patch);
}
// Do the actual initialization
void
MyModel::initializeCMData(const Uintah::Patch* patch, const Uintah::MPMMaterial* matl, Uintah::DataWarehouse* new_dw)
{
  Uintah::ParticleSubset* pset = new_dw->getParticleSubset(matl->getDWIndex(), patch);
  d_flow->initializeInternalVars(pset, new_dw);
}
\end{lstlisting}

  \item Now you are almost ready to use the flow stress model in the stress computation 
        logic.  To complete the process, you will have to add a task that makes sure 
        any submodel internal variables are updated correctly during the stress computation
        process:
\begin{lstlisting}[language=Cpp]
void
MyModel::addComputesAndRequires(Uintah::Task* task, const Uintah::MPMMaterial* matl, const Uintah::PatchSet* patches) const
{
  d_flow->addComputesAndRequires(task, matl, patches);
}
\end{lstlisting}

  \item Finally, you can use the flow stress model object for stress computation:
\begin{lstlisting}[language=Cpp]
void
MyModel::computeStressTensor(const Uintah::PatchSubset* patches, const Uintah::MPMMaterial* matl, Uintah::DataWarehouse* old_dw, Uintah::DataWarehouse* new_dw)
{
  // Set up initial state
  Uintah::PlasticityState state;
  state.initialTemperature = ...;
  state.initialDensity = ...;
  state.initialVolume = ...;
  state.initialBulkModulus = ...;
  state.initialShearModulus = ...;
  state.initialMeltTemp = ...;
  state.energy = ...;
  // Loop through patches
  for (int patchIndex = 0; patchIndex < patches->size(); patchIndex++) {
    const Patch* patch = patches->get(patchIndex);
    int dwi = matl->getDWIndex();
    // Get the particle set and do gets and allocations
    ParticleSubset* pset = old_dw->getParticleSubset(dwi, patch);
    d_flow->getInternalVars(pset, old_dw);
    d_flow->allocateAndPutInternalVars(pset, new_dw);
    // Loop through particles in a patch
    for (auto idx : *pset) {
      // Update the state
      state.strainRate = ...;
      state.plasticStrainRate = ...;
      state.plasticStrain = ...;
      state.pressure = ...;
      state.temperature = ...;
      state.density = ...;
      state.volume = ...;
      state.bulkModulus = ...;
      state.shearModulus = ...;
      state->meltingTemp = ...;
      state->specificHeat = ...;
      // Calculate the flow stress using the flow stress model
      state->yieldStress = d_flow->computeFlowStress(state, delT, d_tol, matl, idx);
      // .......
    } // End particle loop
  } // End patch loop
}
\end{lstlisting}
\end{enumerate}

\section{Creating a new model in PlasticityModels}
Let us now suppose that you want to add a new flow stress model called 
\Textbfc{MyFlow} to \Vaango.  You will have to use the following process to
create the model so that it can be used by the existing constitutive models
as well as the new model that you created in the previous section.
\begin{enumerate}
  \item First create a header file \Textsfc{MyFlow.h} in the \Textsfc{PlasticityModels}
        directory:
\begin{lstlisting}[language=Cpp]
#ifndef __MPM_CM_PM_MyFlow_H__
#define __MPM_CM_PM_MyFlow_H__
#include <CCA/Components/MPM/ConstitutiveModels/PlasticityModels/FlowModel.h>
....
namespace Vaango {
  class MyFlow : public FlowModel
  {
  public:
    // Model parameters
    struct MyFlowParameters {
      double parameter1;
      double parameter2;
      ....
    };

    // Internal variables
    constParticleVariable<double> pInternalVar1, pInternalVar2;
    ParticleVariable<double> pInternalVar1_new, pInternalVar2_new;

    // Internal variable labels
    const Uintah::VarLabel* pInternalVar1Label, pInternalVar2Label;
    const Uintah::VarLabel* pInternalVar1Label_preReloc, pInternalVar2Label_preReloc;

    // Constructors and destructor
    MyFlow(Uintah::ProblemSpecP& ps);
    MyFlow(const MyFlow* flow);
    ~MyFlow() override;

    // Delete assignment operator
    MyFlow& operator=(const MyFlow& flow) = delete;

    // Compute the flow stress 
    double computeFlowStress(const Uintah::PlasticityState* state, const double& delT, const double& tolerance, const Uintah::MPMMaterial* matl, const Uintah::particleIndex idx) override;

    // Calculate the plastic strain rate [epdot(tau,ep,T)]
    double computeEpdot(const Uintah::PlasticityState* state, const double& delT, const double& tolerance, const Uintah::MPMMaterial* matl, const Uintah::particleIndex idx) override;

    // Evaluate derivative of flow stress with respect to scalar and internal variables.
    void evalDerivativeWRTScalarVars(const Uintah::PlasticityState* state, const Uintah::particleIndex idx, Uintah::Vector& derivs) override;

    // Evaluate derivative of flow stress with respect to plastic strain
    double evalDerivativeWRTPlasticStrain(const Uintah::PlasticityState* state, const Uintah::particleIndex idx) override;

    // Evaluate derivative of flow stress with respect to strain rate.
    double evalDerivativeWRTStrainRate(const Uintah::PlasticityState* state, const Uintah::particleIndex idx) override;

    //Compute the elastic-plastic tangent modulus
    void computeTangentModulus(const Uintah::Matrix3& stress, const Uintah::PlasticityState* state, const double& delT, const Uintah::MPMMaterial* matl, const Uintah::particleIndex idx, Uintah::TangentModulusTensor& Ce, Uintah::TangentModulusTensor& Cep) override;

    // Flow stress update methods
    void updateElastic(const Uintah::particleIndex idx) override;
    void updatePlastic(const Uintah::particleIndex idx, const double& delGamma) override;

    // Shear modulus and melting temperature computations
    double computeShearModulus(const Uintah::PlasticityState* state) override;
    double computeMeltingTemp(const Uintah::PlasticityState* state) override;

    // Boilerplate methods used by the computational framework
    void addInitialComputesAndRequires(Uintah::Task* task, const Uintah::MPMMaterial* matl, const Uintah::PatchSet* patches) override;
    void addComputesAndRequires(Uintah::Task* task, const Uintah::MPMMaterial* matl, const Uintah::PatchSet* patches) override;
    void addComputesAndRequires(Uintah::Task* task, const Uintah::MPMMaterial* matl, const Uintah::PatchSet* patches, bool recurse, bool schedParent) override;
    void allocateCMDataAddRequires(Uintah::Task* task, const Uintah::MPMMaterial* matl, const Uintah::PatchSet* patch, Uintah::MPMLabel* lb) override;
    void allocateCMDataAdd(Uintah::DataWarehouse* new_dw, Uintah::ParticleSubset* addset, Uintah::ParticleLabelVariableMap* newState, Uintah::ParticleSubset* delset, Uintah::DataWarehouse* old_dw) override;
    void allocateAndPutRigid(Uintah::ParticleSubset* pset, Uintah::DataWarehouse* new_dw) override;

    // Methods needed if the model has its own internal variables
    void addParticleState(std::vector<const VarLabel*>& from, std::vector<const VarLabel*>& to) override;
    void initializeInternalVars(Uintah::ParticleSubset* pset, Uintah::DataWarehouse* new_dw) override;
    void getInternalVars(Uintah::ParticleSubset* pset, Uintah::DataWarehouse* old_dw) override;
    void allocateAndPutInternalVars(Uintah::ParticleSubset* pset, Uintah::DataWarehouse* new_dw) override;

  private:
    MyFlowParameters d_flow;
    void createInternalVarLabels();
  }
}
#endif
\end{lstlisting}

  \item In the implementation file \Textsfc{MyFlow.cc}, create the constructors and make
        sure that the model input parameters are copied to output files:
\begin{lstlisting}[language=Cpp]
#include <CCA/Components/MPM/ConstitutiveModels/PlasticityModels/FlowModel.h>
MyFlow::MyFlow(Uintah::ProblemSpecP& ps) 
{
  ps->require("my_flow_parameter1", d_flow.parameter1);
  ps->require("my_flow_parameter2", d_flow.parameter2);
  createInternalVarLabels();
}

MyFlow::MyFlow(const MyFlow* flow)
{
  d_flow.parameter1 = flow->d_flow.parameter1;
  d_flow.parameter2 = flow->d_flow.parameter2;
  createInternalVarLabels();
}

void
MyFlow::createInternalVarLabels()
{
  pInternalVar1Label = Uintah::VarLabel::create("p.my_flow_var1", Uintah::ParticleVariable<double>::getTypeDescription());
  pInternalVar2Label = Uintah::VarLabel::create("p.my_flow_var2", Uintah::ParticleVariable<double>::getTypeDescription());
  pInternalVar1Label_preReloc = Uintah::VarLabel::create("p.my_flow_var1+", Uintah::ParticleVariable<double>::getTypeDescription());
  pInternalVar2Label_preReloc = Uintah::VarLabel::create("p.my_flow_var2+", Uintah::ParticleVariable<double>::getTypeDescription());
}

void
MyFlow::outputProblemSpec(Uintah:;ProblemSpecP& ps)
{
  ProblemSpecP flow_ps = ps->appendChild("flow_model");
  flow_ps->setAttribute("type", "my_flow_model");
  flow_ps->appendElement("my_flow_parameter1", d_flow.parameter1);
  flow_ps->appendElement("my_flow_parameter2", d_flow.parameter2);
}
\end{lstlisting}

  \item We can now implement the flow stress computation code:
\begin{lstlisting}[language=Cpp]
double
MyFlow::computeFlowStress(const Uintah::PlasticityState* state, const double&, const double&, const Uintah::MPMMaterial*, const Uintah::particleIndex idx)
{
  // Get the internal variables
  double internalVar1 = pInternalVar1[idx];
  double internalVar2 = pInternalVar2[idx];

  // Compute the flow stress
  double flowStress = doSomeComputation(internalVar1, internalVar2);

  // Update the internal variables if needed
  pInternalVar1_new[idx] = internalVar1;
  pInternalVar2_new[idx] = internalVar2;

  return flowStress;
}
\end{lstlisting}

  \item Compute the plastic strain rate if possible:
\begin{lstlisting}[language=Cpp]
double
MyFlow::computeEpdot(const Uintah::PlasticityState* state, const double& delT, const double&, const Uintah::MPMMaterial*, const Uintah::particleIndex pidx)
{
  double epdot = doSomComputation();
  return epdot;
}
\end{lstlisting}

  \item Compute the derivative of the flow stress function with respect to the scalar plastic
        strain and strain rate:
\begin{lstlisting}[language=Cpp]
double
MyFlow::evalDerivativeWRTPlasticStrain(const Uintah::PlasticityState* state, const Uintah::particleIndex idx)
{
  // Do the calculation
  return derivative;
}
double
MyFlow::evalDerivativeWRTStrainRate(const Uintah::PlasticityState* state, const Uintah::particleIndex idx)
{
  // Do the calculation
  return derivative;
}
\end{lstlisting}

  \item Compute the derivative of the flow stress function with respect to some scalar
        quantities if needed:
\begin{lstlisting}[language=Cpp]
void
MyFlow::evalDerivativeWRTScalarVars(const Uintah::PlasticityState* state, const Uintah::particleIndex idx, Uintah::Vector& derivs)
{
  derivs[0] = evalDerivativeWRTStrainRate(state, idx);
  derivs[1] = evalDerivativeWRTTemperature(state, idx);
  derivs[2] = evalDerivativeWRTPlasticStrain(state, idx);
}
\end{lstlisting}

  \item Compute the tangent modulus for implicit calculations. This is typically quite 
        involved and usually avoided by the plasticity models in \Vaango.

  \item Compute the shear modulus and melting temperature if needed.  We usually use
        the separate shear modulus and melting temperature models to accomplish this task.

  \item At this stage we should set up the boilerplate code for communicating the 
        internal variables:
\begin{lstlisting}[language=Cpp]
void
MyFlow::addInitialComputesAndRequires(Uintah::Task* task, const Uintah::MPMMaterial* matl, const PatchSet*)
{
  const Uintah::MaterialSubset* matlset = matl->thisMaterial();
  task->computes(pInternalVar1Label, matlset);
  task->computes(pInternalVar2Label, matlset);
}

void
MyFlow::addComputesAndRequires(Uintah::Task* task, const Uintah::MPMMaterial* matl, const Uintah::PatchSet*)
{
  const Uintah::MaterialSubset* matlset = matl->thisMaterial();
  task->requires(Task::OldDW, pInternalVar1Label, matlset, Ghost::None);
  task->requires(Task::OldDW, pInternalVar2Label, matlset, Ghost::None);
  task->computes(pInternalVar1Label_preReloc, matlset);
  task->computes(pInternalVar2Label_preReloc, matlset);
}

void
MyFlow::addComputesAndRequires(Uintah::Task* task, const Uintah::MPMMaterial*, const Uintah::PatchSet*, bool, bool)
{
  const Uintah::MaterialSubset* matlset = matl->thisMaterial();
  task->requires(Task::ParentOldDW, pInternalVar1Label, matlset, Ghost::None);
  task->requires(Task::ParentOldDW, pInternalVar2Label, matlset, Ghost::None);
}

void
MyFlow::addParticleState(std::vector<const Uintah::VarLabel*>&, std::vector<const Uintah::VarLabel*>&)
{
  from.push_back(pInternalVar1Label);
  from.push_back(pInternalVar2Label);
  to.push_back(pInternalVar1Label_preReloc);
  to.push_back(pInternalVar2Label_preReloc);
}

void
MyFlow::initializeInternalVars(Uintah::ParticleSubset* pset, Uintah::DataWarehouse* new_dw)
{
  new_dw->allocateAndPut(pInternalVar1_new, pInternalVar1Label, pset);
  new_dw->allocateAndPut(pInternalVar2_new, pInternalVar2Label, pset);
  for (auto pidx : *pset) {
    pInternalVar1_new[pidx] = 0.0;
    pInternalVar2_new[pidx] = 0.0;
  }
}

void
MyFlow::getInternalVars(Uintah::ParticleSubset* pset, Uintah::DataWarehouse* old_dw)
{
  old_dw->get(pInternalVar1, pInternalVar1Label, pset);
  old_dw->get(pInternalVar2, pInternalVar2Label, pset);
}

void
MyFlow::allocateAndPutInternalVars(Uintah::ParticleSubset* pset, Uintah::DataWarehouse* new_dw)
{
  new_dw->allocateAndPut(pInternalVar1_new, pInternalVar1Label_preReloc, pset);
  new_dw->allocateAndPut(pInternalVar2_new, pInternalVar2Label_preReloc, pset);
}

void
MyFlow::allocateAndPutRigid(Uintah::ParticleSubset* pset, Uintah::DataWarehouse* new_dw)
{
  allocateAndPutInternalVars(pset, new_dw);
  for (auto pidx : *pset) {
    pInternalVar1_new[pidx] = 0.0;
    pInternalVar2_new[pidx] = 0.0;
  }
}

void
MyFlow::allocateCMDataAddRequires(Uintah::Task* task, const Uintah::MPMMaterial* matl, const Uintah::PatchSet*, Uintah::MPMLabel*)
{
  const Uintah::MaterialSubset* matlset = matl->thisMaterial();
  task->requires(Task::NewDW, pInternalVar1Label_preReloc, matlset, Ghost::None);
  task->requires(Task::NewDW, pInternalVar2Label_preReloc, matlset, Ghost::None);
}

void
MyFlow::allocateCMDataAdd(Uintah::DataWarehouse* new_dw, Uintah::ParticleSubset* addset, Uintah::ParticleLabelVariableMap* newState, Uintah::ParticleSubset* delset, Uintah::DataWarehouse* old_dw)
{
  Uintah::ParticleVariable<double> n_internalVar1, n_internalVar2;
  Uintah::constParticleVariable<double> o_internalVar1, o_internalVar2;

  new_dw->allocateTemporary(n_internalVar1, addset);
  new_dw->allocateTemporary(n_internalVar2, addset);

  new_dw->get(o_internalVar1 , pInternalVar1Label_preReloc, delset);
  new_dw->get(o_internalVar2 , pInternalVar2Label_preReloc, delset);

  ParticleSubset::iterator o,n = addset->begin();
  for (o = delset->begin(); o != delset->end(); o++, n++) {
    n_internalVar1[*n] = o_internalVar1[*o];
    n_internalVar2[*n] = o_internalVar2[*o];
  }

  (*newState)[pInternalVar1Label] = n_internalVar1.clone();
  (*newState)[pInternalVar2Label] = n_internalVar2.clone();
}

void
MyFlow::updateElastic(const Uintah::particleIndex pidx)
{
  pInternalVar1_new[idx] = pInternalVar1[idx];
  pInternalVar2_new[idx] = pInternalVar2[idx];
}

void
MyFlow::updatePlastic(const particleIndex pidx, const double& someValue)
{
  pInternalVar1_new[idx] = pInternalVar1_new[idx];
  pInternalVar2_new[idx] = pInternalVar2_new[idx] + someValue;
}
\end{lstlisting}

  \item We now add this model to the \Textbfc{FlowStressModelFactory} as follows:
\begin{lstlisting}[language=Cpp]
#include <CCA/Components/MPM/ConstitutiveModel/PlasticityModels/MyFlow.h>
using namespace Uintah;
FlowModel*
FlowStressModelFactory::create(ProblemSpecP& ps)
{
  ProblemSpecP child = ps->findBlock("flow_model");
  if (!child)
    throw ProblemSetupException("Cannot find flow_model tag", __FILE__,
                                __LINE__);
  string mat_type;
  if (!child->getAttribute("type", mat_type))
    throw ProblemSetupException("No type for flow_model", __FILE__, __LINE__);
  .....
  else if (mat_type == "my_flow_model") {
    return (scinew Vaango::MyFlow(child));
  }
  .....
}

FlowModel*
FlowStressModelFactory::createCopy(const FlowModel* pm)
{
  .....
  else if (dynamic_cast<const Vaango::MyFlow*>(pm))
    return (scinew Vaango::MyFlow(dynamic_cast<const Vaango::MyFlow*>(pm)));
  .....
}
\end{lstlisting}

  \item Next we add the new file to the compilation list in \Textsfc{CMakeLists.txt}
        in the \Textsfc{PlasticityModels} directory:
\begin{lstlisting}[language=sh, backgroundcolor=\color{background}]
SET(MPM_ConstitutiveModel_PlasticityModels_SRCS
   ................
   MyModel.cc
)
\end{lstlisting}

  \item Finally, we add the new model tags in the \Textsfc{constitutive\_models.xml} file 
        in the directory \Textsfc{src/StandAlone/inputs/UPS\_SPEC}:
\begin{lstlisting}[language=XML]
   <flow_model spec="OPTIONAL NO_DATA" attribute1="type REQUIRED STRING 'isotropic_hardening, johnson_cook, mts_model, .... , zerilli_armstrong_polymer, my_flow_model'" >
     ........
     <my_flow_parameter1 spec="REQUIRED DOUBLE" need_applies_to "type my_flow_model"/>
     <my_flow_parameter2 spec="REQUIRED DOUBLE" need_applies_to "type my_flow_model"/>
   </flow_model>
\end{lstlisting}

\end{enumerate}





