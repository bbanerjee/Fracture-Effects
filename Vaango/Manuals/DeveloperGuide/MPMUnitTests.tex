\chapter{Unit testing in Vaango}
Older versions of \Vaango used a limited form of CTest-based testing 
(\url{https://cmake.org/Wiki/CMake/Testing_With_CTest}) to unit test class  APIs
and utility functions.  Since version 17.0, we have moved to \Textsfc{googletest}
(\url{https://github.com/google/googletest}) and incorporated a basic version of
continuous integration into the \Vaango build process.

\section{Setting up googletest}
\Vaango now uses the combination of \Textsfc{CMake} and \Textsfc{googletest}
The reason for the shift is the convenience \Textsfc{googletest} provides.

\subsection{Installing googletest}
\begin{NoteBox}
To make sure that the \Textsfc{googletest} submodule is checked out when you check
out \Textsfc{ParSim}, you will have to use the \Textsfc{--recursive} flag:
\begin{lstlisting}[language=sh, backgroundcolor=\color{background}]
git clone --recursive ....
\end{lstlisting}
\end{NoteBox}

\Textsfc{googletest} is installed as a 
\Textsfc{git submodule}(\url{https://git-scm.com/docs/git-submodule})
in the \Vaango git repository.  The submodule is added using
\begin{lstlisting}[language=sh, backgroundcolor=\color{background}]
git submodule add git@github.com:google/googletest.git Vaango/src/googletest
\end{lstlisting}
Note that the typical user will not need to add a submodule to \Vaango.

\subsection{Making sure cmake finds and compiles googletest}
\begin{NoteBox}
For the typical developer, all that is needed for unit testing to be activated is
to run the following commands from a \Textsfc{dbg} or \Textsfc{opt} directory:
\begin{lstlisting}[language=sh, backgroundcolor=\color{background}]
  cmake -DUSE_CLANG=1 ../src
  make -j4
\end{lstlisting}
\end{NoteBox}

To make sure that \Textsfc{googletest} is found and built during the compile process,
we have added the following to root \Textsfc{CMakeLists.txt} file in the directory
\Textsfc{Vaango/src}:
\begin{lstlisting}[language=sh, backgroundcolor=\color{background}]
if (USE_CLANG)
  set(CMAKE_CXX_COMPILER "/usr/local/bin/clang++")
endif ()
set(gtest_force_shared_crt ON CACHE BOOL "" FORCE)
add_subdirectory(googletest)
\end{lstlisting}

The \Textsfc{add\_subdirectory}	command is all that is needed for \Textsfc{cmake} to 
compile \Textsfc{googletest} and produce static libraries.  However, the developer may
need to specify the absolute path to the \Textsfc{clang} compiler.

\begin{NoteBox}
At present the unit testing and continuous integration function is turned on by
default, but only when a \Textsfc{clang} compiler is used.  The absolute path to
the \Textsfc{clang} compiler is needed because I have not been successful in passing 
the \Textsfc{clang} compiler name to \Textsfc{googletest} without specifying the 
full path.  
\end{NoteBox}

\section{Adding local unit tests}
The \Vaango convention is to create a \Textsfc{UnitTests} directory in the directory 
of interest (e.g., \Textsfc{src/CCA/Components/MPM/ConstitutiveModel/Models}) and modify 
the local \Textsfc{CMakeLists.txt} (e.g., the \Textsfc{Models/CMakeLists.txt})
to be able to find the \Textsfc{UnitTests} directory.  For example,
\begin{lstlisting}[language=sh, backgroundcolor=\color{background}]
SET(MPM_ConstitutiveModel_SRCS
  ${MPM_ConstitutiveModel_SRCS}
  ${CMAKE_CURRENT_SOURCE_DIR}/ModelStateBase.cc
  .....
  PARENT_SCOPE
)
IF(NOT CMAKE_COMPILER_IS_GNUCXX)
  add_subdirectory(UnitTests)
ENDIF()
\end{lstlisting}

\subsection{The CMakeLists.txt file in UnitTests}
Now we are finally ready to add our unit tests to the build chain.  The 
\Textsfc{CMakeLists.txt} file in the \Textsfc{UnitTests} directory has the 
following components:
\begin{enumerate}
  \item The tests may need to read some input files.  Make sure that these files
        are copied to the build directory.  For example, if you need to copy
        \Textsfc{table\_elastic.json}, add the following near the top of
        the \Textsfc{CMakeLists.txt} file:
\begin{lstlisting}[language=sh, backgroundcolor=\color{background}]
configure_file(table_elastic.json table_elastic.json COPYONLY)
\end{lstlisting}

  \item The tests will probably need to be linked with all the \Vaango shared
        libraries.  Add these to the \Textsfc{CMakeLists.txt} file:
\begin{lstlisting}[language=sh, backgroundcolor=\color{background}]
SET(VAANGO_LIBS 
        Vaango_Core_Thread       
        Vaango_Core_DataArchive 
        Vaango_Core_Grid        
        Vaango_Core_Parallel    
        Vaango_Core_Labels      
        Vaango_Core_Util        
        Vaango_Core_Math        
        Vaango_Core_Disclosure  
        Vaango_Core_Exceptions  
        Vaango_Core_OS  
        Vaango_CCA_Ports        
        Vaango_CCA_Components_Parent 
        Vaango_CCA_Components_DataArchiver  
        Vaango_CCA_Components_LoadBalancers 
        Vaango_CCA_Components_Regridder     
        Vaango_Core_ProblemSpec             
        Vaango_CCA_Components_SimulationController 
        Vaango_CCA_Components_Schedulers           
        Vaango_CCA_Components_ProblemSpecification 
        Vaango_CCA_Components_Solvers
)
\end{lstlisting}

  \item Make sure that the \Textsfc{googletest} libraries can be found:
\begin{lstlisting}[language=sh, backgroundcolor=\color{background}]
set(GTEST_INCLUDE_DIR "${CMAKE_SOURCE_DIR}/submodules/googletest/googletest/include")
set(GTEST_LIB gtest_main gtest)
include_directories(${GTEST_INCLUDE_DIR})
\end{lstlisting}
        Note that the two \Textsfc{googletest} libraries that we use are 
        \Textsfc{gtest\_main} and \Textsfc{gtest}.

  \item  Add the actual test compilation and run procedure to the \Textsfc{CMakeLists.txt}
         file.  The test executable is built and also run during the build process.
  \item  For a typical unit test (e.g., \Textsfc{testTabularData.cc}) that does not 
         use \Textsfc{MPI}, the following will 
         have to be added to the \Textsfc{CMakeLists.txt} file:
\begin{lstlisting}[language=sh, backgroundcolor=\color{background}]
add_executable(testTabularData testTabularData.cc)
target_link_libraries(testTabularData
  ${GTEST_LIB}
  ${VAANGO_LIBS}
)
set(UNIT_TEST testTabularData)
add_custom_command(
  TARGET ${UNIT_TEST}
  POST_BUILD
  COMMAND ${UNIT_TEST}
)
\end{lstlisting}

  \item  For a unit test (e.g., \Textsfc{testMPITabularData.cc}) that uses
         \Textsfc{MPI}, the following will 
         have to be added to the \Textsfc{CMakeLists.txt} file:
\begin{lstlisting}[language=sh, backgroundcolor=\color{background}]
add_executable(testMPITabularData testMPITabularData.cpp)
target_link_libraries(testMPITabularData
  ${GTEST_LIB}
  ${VAANGO_LIBS}
  ${MPI_LIBRARY}
  ....
)
set(UNIT_TEST testMPITabularData)
set(MPI_COMMAND mpirun -np 2 ${UNIT_TEST})
add_custom_command(
  TARGET ${UNIT_TEST}
  POST_BUILD
  COMMAND ${MPI_COMMAND}
)
\end{lstlisting}
\end{enumerate}

\begin{NoteBox}
For the \Textsfc{MPI} tests, the \Textsfc{add\_executable} line identifies the unit test 
\Textsfc{testMPITabularData.cpp}. The \Textsfc{target\_link\_libraries} lists 
the libraries that are needed for this test, i.e.,  the \Textsfc{googletest}
libraries (\Textsfc{GTEST\_LIB}), the \Vaango libraries (\Textsfc{VAANGO\_LIBS}),
and the \Textsfc{MPI} libraries.
Then we set up a custom command (\Textsfc{MPI\_COMMAND}) that uses \Textsfc{mpirun}.
Finally we, add the \Textsfc{add\_custom\_command} line that tells \Textsfc{cmake} 
to run the \Textsfc{MPI\_COMMAND} after the build is complete (\Textsfc{POST\_BUILD}).
\end{NoteBox}

\subsection{A simple test}
Let us look at the test of the \Textbfc{reverse} function in the utility 
code \Textsfc{testYieldCondUtils.cc}.
The process is as follows:
\begin{enumerate}
  \item First, we include the required headers:
\begin{lstlisting}[language=Cpp]
#include <CCA/Components/MPM/ConstitutiveModel/Models/YieldCondUtils.h>
#include <gtest/gtest.h>
\end{lstlisting}

  \item Then we add the test
\begin{lstlisting}[language=Cpp]
TEST(YieldCondUtilsTest, reverse)
{
  std::vector<double> array = {{1,2,3,4,5,6,7}};
  std::size_t index = 0;
  for (auto i : Vaango::Util::reverse(array)) {
    if (index == 0) EXPECT_EQ(i, 7);
    if (index == array.size()-1) EXPECT_EQ(i, 1);
    index++;
  }
}
\end{lstlisting}

  \item When you run \Textsfc{make}, you will get an output of the form:
\begin{lstlisting}[language=sh, backgroundcolor=\color{background}]
[ 83%] Building CXX object CCA/Components/MPM/ConstitutiveModel/Models/UnitTests/CMakeFiles/testYieldCondUtils.dir/testYieldCondUtils.cc.o
[ 83%] Linking CXX executable testYieldCondUtils
Running main() from gtest_main.cc
[==========] Running 1 tests from 1 test case.
[----------] Global test environment set-up.
[----------] 1 tests from YieldCondUtilsTest
[ RUN      ] YieldCondUtilsTest.reverse
[       OK ] YieldCondUtilsTest.reverse (0 ms)
[----------] 1 tests from YieldCondUtilsTest (1 ms total)
[----------] Global test environment tear-down
[==========] 1 tests from 1 test case ran. (1 ms total)
[  PASSED  ] 1 tests.
[ 83%] Built target testYieldCondUtils
\end{lstlisting}
       This indicates that your unit test has passed the checks you have specified.
\end{enumerate}

\subsection{A  test suite with some initial setup}
For some unit tests, a lot of setup work is needed.  However, that work does not
need to be repeated in each test of a suite.  In those situations, we can 
create a custom \Textsfc{googletest} test procedure.  For example, in the 
tabular yield condition tests \Textsfc{testYieldCond\_Tabular.cc}, we do the following:
\begin{enumerate}
  \item Create a class that is derived from \Textbfc{::testing::Test}:
\begin{lstlisting}[language=Cpp]
class YieldCondTabularTest : public ::testing::Test {

protected:

  static void SetUpTestCase() {
    char currPath[2000];
    if (!getcwd(currPath, sizeof(currPath))) {
      std::cout << "Current path not found\n";
    }
    std::string json_file = std::string(currPath) + "/" + "table_yield.json";

    // Create a new document
    xmlDocPtr doc = xmlNewDoc(BAD_CAST "1.0");

    // Create root node
    xmlNodePtr rootNode = xmlNewNode(nullptr, BAD_CAST "plastic_yield_condition");
    xmlNewProp(rootNode, BAD_CAST "type", BAD_CAST "tabular");
    xmlDocSetRootElement(doc, rootNode);

    // Create a child node
    xmlNewChild(rootNode, nullptr, BAD_CAST "filename", 
                BAD_CAST "table_yield.json");
    xmlNewChild(rootNode, nullptr, BAD_CAST "independent_variables", 
                BAD_CAST "Pressure");
    xmlNewChild(rootNode, nullptr, BAD_CAST "dependent_variables", 
                BAD_CAST "SqrtJ2");
    auto interp = xmlNewChild(rootNode, nullptr, BAD_CAST "interpolation",
                              BAD_CAST "");
    xmlNewProp(interp, BAD_CAST "type", BAD_CAST "linear");

    // Print the document to stdout
    xmlSaveFormatFileEnc("-", doc, "ISO-8859-1", 1);

    // Create a ProblemSpec
    ps = scinew ProblemSpec(xmlDocGetRootElement(doc), false);
    if (!ps) {
      std::cout << "**Error** Could not create ProblemSpec." << std::endl;
      std::cout << __FILE__ << ":" << __LINE__ << std::endl;
      exit(-1);
    }

    // Update the json file and create a circular yield function
    auto doc1 = xmlCopyDoc(doc, 1);
    auto node = xmlDocGetRootElement(doc1);
    auto xpathCtx = xmlXPathNewContext(doc1);
    const xmlChar* xpathExpr = xmlStrncatNew(BAD_CAST ".//", BAD_CAST "filename", -1);
    auto xpathObj = xmlXPathNodeEval(node, xpathExpr, xpathCtx);
    xmlXPathFreeContext(xpathCtx);
    auto jsonNode = xpathObj->nodesetval->nodeTab[0];

    xmlNodeSetContent(jsonNode, BAD_CAST "table_yield_circle.json");
    xmlSaveFormatFileEnc("-", doc1, "ISO-8859-1", 1);

    ps_circle = scinew ProblemSpec(xmlDocGetRootElement(doc1), false);
    if (!ps_circle) {
      std::cout << "**Error** Could not create ProblemSpec." << std::endl;
      std::cout << __FILE__ << ":" << __LINE__ << std::endl;
      exit(-1);
    }
  }

  static void TearDownTestCase() {}

  static ProblemSpecP ps;
  static ProblemSpecP ps_circle;
};
\end{lstlisting}

  \item The variables that are used in the tests are declared \Textbfc{static} and
        initialized with \Textbfc{nullptr}:
\begin{lstlisting}[language=Cpp]
ProblemSpecP YieldCondTabularTest::ps = nullptr;
ProblemSpecP YieldCondTabularTest::ps_circle = nullptr;
\end{lstlisting}

  \item The actual tests are added using the \Textsfc{TEST\_F} tag:
\begin{lstlisting}[language=Cpp]
TEST_F(YieldCondTabularTest, constructorTest)
{
  // Create a model
  YieldCond_Tabular model(ps);

  // Copy
  YieldCond_Tabular modelCopy(&model);
  YieldCond_Tabular model_circle(ps_circle);

  // Check parameters
  auto params = model.getParameters();
  ASSERT_DOUBLE_EQ(params.at("I1_min"), -19200);
  ASSERT_DOUBLE_EQ(params.at("I1_max"), 30);
  ASSERT_DOUBLE_EQ(params.at("sqrtJ2_max"), 900);
}
\end{lstlisting}
\end{enumerate}

\subsection{A test suite with MPI}
Consider the unit test \Textsfc{testSPHParticleScatter.cpp} that tests whether the
scatter algorithm is working correctly.  In this case we need to set up the \Textsfc{MPI}
environment by deriving an class from the \Textbfc{::testing::Environment} class.
The procedure is similar to the previous examples (note that we use \Textsfc{Boost}
in this example).

\begin{enumerate}
  \item First, we include the required headers:
\begin{lstlisting}[language=Cpp]
#include <SmoothParticleHydro/SmoothParticleHydro.h>
#include <gtest/gtest.h>
#include <boost/mpi.hpp>
\end{lstlisting}

  \item Ideally we would like to return \Textsfc{RUN\_ALL\_TESTS} from our main
        function.   However, we cannot use the \Textsfc{boost::mpi::environment} call 
        to set up \Textsfc{MPI} (because the \Textsfc{environment} object is deleted 
        before \Textsfc{googletest}s are run).  Instead, we have to set up a custom 
        environment to run our \Textsfc{MPI} tests by creating a \Textsfc{MPIEnvironment} 
        class that extends the \Textsfc{::testing::Environment} class provided by 
        \Textsfc{googletest}.
\begin{lstlisting}[language=Cpp]
class MPIEnvironment : public ::testing::Environment
{
public:
  virtual void SetUp() {
    char** argv;
    int argc = 0;
    int mpiError = MPI_Init(&argc, &argv);
    ASSERT_FALSE(mpiError);
  }
  virtual void TearDown() {
    int mpiError = MPI_Finalize();
    ASSERT_FALSE(mpiError);
  }
  virtual ~MPIEnvironment() {}
};
\end{lstlisting}
       Here, the \Textsfc{SetUp} function calls \Textsfc{MPI\_Init} and sets up 
       the environment while the \Textsfc{TearDown} function calls 
       \Textsfc{MPI\_Finalize}.  All tests are performed when the environment is active.

  \item The \Textsfc{main} function in typically not needed in standard 
        \Textsfc{googletest} tests and is generated by some internal magic.  However, we 
         do need a \Textsfc{main} function in \Textsfc{testSPHParticleScatter.cpp}
         because we are using \Textsfc{mpirun} to run the test.
\begin{lstlisting}[language=Cpp]
int main(int argc, char* argv[])
{
  ::testing::InitGoogleTest(&argc, argv);
  ::testing::AddGlobalTestEnvironment(new MPIEnvironment);
  return RUN_ALL_TESTS();
}
\end{lstlisting}
         The MPI specific environment is created by the \Textsfc{AddGlobalTestEnvironment} 
         function to which we pass a \Textsfc{MPIEnvironment} object.  The object is 
         deleted internally by \Textsfc{googletest}.

         The tests in \Textsfc{testSPHParticleScatter.cpp} are then run using 
         \Textsfc{RUN\_ALL\_TESTS()}.  Note that this function returns from \Textsfc{main} 
         and, therefore, and non-googletest environment objects created in main (e.g., 
         boost::mpi::environment) are deleted before the tests are run.

  \item Finally, we add an actual test to \Textsfc{testSPHParticleScatter.cpp} as follows:
\begin{lstlisting}[language=Cpp]
TEST(SPHParticleScatterTest, scatter)
{
  // Set up communicator
  boost::mpi::communicator boostWorld;
  // Set up SPH object
  SmoothParticleHydro sph;
  sph.setCommunicator(boostWorld);
  // ... Some code to create particles
  // ....
  sph.setParticles(particles);
  sph.scatterSPHParticle(domain, ghostWidth, domainBuffer);
  if (boostWorld.rank() == 0) {
    EXPECT_EQ(sph.getSPHParticleVec().size(), 100);
  } else {
    EXPECT_EQ(sph.getSPHParticleVec().size(), 200);
  }
}
\end{lstlisting}
       \begin{NoteBox}
        The \Textsfc{Boost} MPI environment created by
        \Textsfc{boost::mpi::environment} allows MPI calls to
        fail without throwing an exception.  This does not create a problem if
        the \Textsfc{boost::mpi::environment} method is used to set up the
        MPI environment.  But when setting up the environment explicitly for 
        the unit tests, we should add
\begin{lstlisting}[language=Cpp]
MPI_Errhandler_set(cartComm, MPI_ERRORS_RETURN);
\end{lstlisting}
        before calls that can throw exceptions.
       \end{NoteBox}

  \item Now, if we run \Textsfc{make} the output has the following form:
\begin{lstlisting}[language=sh, backgroundcolor=\color{background}]
[ 69%] Built target ellip3D_lib
[ 71%] Built target paraEllip3D
[ 74%] Built target gtest
[ 76%] Built target gtest_main
.......
Scanning dependencies of target testSPHParticleScatter
[ 90%] Building CXX object SmoothParticleHydro/UnitTests/CMakeFiles/testSPHParticleScatter.dir/testSPHParticleScatter.cpp.o
[ 91%] Linking CXX executable testSPHParticleScatter
[==========] Running 1 test from 1 test case.
[----------] Global test environment set-up.
[==========] Running 1 test from 1 test case.
[----------] Global test environment set-up.
Set up environment
Set up environment
[----------] 1 test from SPHParticleScatterTest
[----------] 1 test from SPHParticleScatterTest
[ RUN      ] SPHParticleScatterTest.scatter
[ RUN      ] SPHParticleScatterTest.scatter
[       OK ] SPHParticleScatterTest.scatter (5 ms)
[----------] 1 test from SPHParticleScatterTest (5 ms total)

[----------] Global test environment tear-down
[       OK ] SPHParticleScatterTest.scatter (6 ms)
[----------] 1 test from SPHParticleScatterTest (6 ms total)

[----------] Global test environment tear-down
Tore down environment
[==========] 1 test from 1 test case ran. (289 ms total)
[  PASSED  ] 1 test.
Tore down environment
[==========] 1 test from 1 test case ran. (289 ms total)
[  PASSED  ] 1 test.
[ 91%] Built target testSPHParticleScatter
[ 95%] Built target gmock
[100%] Built target gmock_main
\end{lstlisting}
\end{enumerate}

\subsection{Unit testing a constitutive model}
Constitutive model testing involves a lot of set up work both for the 
environment and the computational framework.  This following example
shows how we test the \Textbfc{TabularPlasticity} model.
\begin{enumerate}
  \item First set up headers:
\begin{lstlisting}[language=Cpp]
#include <CCA/Components/MPM/ConstitutiveModel/Models/TabularData.h>

#include <CCA/Components/ProblemSpecification/ProblemSpecReader.h>
#include <CCA/Components/SimulationController/AMRSimulationController.h>
#include <CCA/Components/Regridder/RegridderCommon.h>
#include <CCA/Components/Solvers/SolverFactory.h>
#include <CCA/Components/Parent/ComponentFactory.h>
#include <CCA/Components/LoadBalancers/LoadBalancerCommon.h>
#include <CCA/Components/LoadBalancers/LoadBalancerFactory.h>
#include <CCA/Components/DataArchiver/DataArchiver.h>
#include <CCA/Components/Schedulers/SchedulerCommon.h>
#include <CCA/Components/Schedulers/SchedulerFactory.h>

#include <CCA/Ports/SolverInterface.h>
#include <CCA/Ports/SimulationInterface.h>
#include <CCA/Ports/Output.h>

#include <Core/Parallel/Parallel.h>
#include <Core/Parallel/UintahParallelComponent.h>
#include <Core/Malloc/Allocator.h>
#include <Core/ProblemSpec/ProblemSpec.h>
#include <Core/ProblemSpec/ProblemSpecP.h>
#include <Core/Exceptions/Exception.h>
#include <Core/Exceptions/ProblemSetupException.h>
#include <Core/Exceptions/InvalidValue.h>
#include <Core/Util/Environment.h>
#include <Core/OS/Dir.h>

#include <libxml/parser.h>
#include <libxml/tree.h>

#include <iostream>
#include <map>
#include <string>
#include <vector>

#include <gtest/gtest.h>
\end{lstlisting}

  \item Set up aliases:
\begin{lstlisting}[language=Cpp]
using namespace Vaango;
using nlohmann::json;
using Uintah::Dir;
using Uintah::ProblemSpec;
using Uintah::ProblemSpecP;
using Uintah::ProblemSpecReader;
using Uintah::Exception;
using Uintah::ProblemSetupException;
using Uintah::InvalidValue;
using Uintah::ProcessorGroup;
using Uintah::SimulationController;
using Uintah::AMRSimulationController;
using Uintah::RegridderCommon;
using Uintah::SolverInterface;
using Uintah::SolverFactory;
using Uintah::UintahParallelComponent;
using Uintah::ComponentFactory;
using Uintah::SimulationInterface;
using Uintah::LoadBalancerCommon;
using Uintah::LoadBalancerFactory;
using Uintah::DataArchiver;
using Uintah::Output;
using Uintah::SchedulerCommon;
using Uintah::SchedulerFactory;
\end{lstlisting}

  \item Set up a test environment class:
\begin{lstlisting}[language=Cpp]
class VaangoEnv : public ::testing::Environment {
public:

  int d_argc;
  char** d_argv;
  char** d_env;

  explicit VaangoEnv(int argc, char** argv, char* env[]) {
    d_argc = argc;
    d_argv = argv;
    d_env = env;
  }

  virtual ~VaangoEnv() {}

  virtual void SetUp() {
    Uintah::Parallel::determineIfRunningUnderMPI(d_argc, d_argv);
    Uintah::Parallel::initializeManager(d_argc, d_argv);
    Uintah::create_sci_environment(d_env, 0, true );
  }

  virtual void TearDown() {
    Uintah::Parallel::finalizeManager();
  }

  static ProblemSpecP createInput() {

    char currPath[2000];
    if (!getcwd(currPath, sizeof(currPath))) {
      std::cout << "Current path not found\n";
    }
    //std::cout << "Dir = " << currPath << std::endl;

    // Create a new document
    xmlDocPtr doc = xmlNewDoc(BAD_CAST "1.0");

    // Create root node
    xmlNodePtr rootNode = xmlNewNode(nullptr, BAD_CAST "Uintah_specification");
    xmlDocSetRootElement(doc, rootNode);

    // Meta
    auto meta = xmlNewChild(rootNode, nullptr, BAD_CAST "Meta", BAD_CAST "");
    xmlNewChild(meta, nullptr, BAD_CAST "title", BAD_CAST "Unit test Tabular Plasticity");

    // Simulation component 
    auto simComp = xmlNewChild(rootNode, nullptr, BAD_CAST "SimulationComponent",
                               BAD_CAST "");
    xmlNewProp(simComp, BAD_CAST "type", BAD_CAST "mpm");

    // Time
    auto time = xmlNewChild(rootNode, nullptr, BAD_CAST "Time", BAD_CAST "");
    xmlNewChild(time, nullptr, BAD_CAST "maxTime", BAD_CAST "1.0");
    xmlNewChild(time, nullptr, BAD_CAST "initTime", BAD_CAST "0.0");
    xmlNewChild(time, nullptr, BAD_CAST "delt_min", BAD_CAST "1.0e-6");
    xmlNewChild(time, nullptr, BAD_CAST "delt_max", BAD_CAST "0.04");
    xmlNewChild(time, nullptr, BAD_CAST "timestep_multiplier", BAD_CAST "0.3");
    xmlNewChild(time, nullptr, BAD_CAST "max_Timesteps", BAD_CAST "5");

    // DataArchiver
    auto da = xmlNewChild(rootNode, nullptr, BAD_CAST "DataArchiver", BAD_CAST "");
    xmlNewChild(da, nullptr, BAD_CAST "filebase", BAD_CAST "UniaxialStrainRotateTabularPlasticity.uda");
    xmlNewChild(da, nullptr, BAD_CAST "outputTimestepInterval", BAD_CAST "1");
    auto save = xmlNewChild(da, nullptr, BAD_CAST "save", BAD_CAST ""); 
    xmlNewProp(save, BAD_CAST "label", BAD_CAST "g.mass");
    save = xmlNewChild(da, nullptr, BAD_CAST "save", BAD_CAST ""); 
    xmlNewProp(save, BAD_CAST "label", BAD_CAST "p.x");
    save = xmlNewChild(da, nullptr, BAD_CAST "save", BAD_CAST ""); 
    xmlNewProp(save, BAD_CAST "label", BAD_CAST "p.color");
    save = xmlNewChild(da, nullptr, BAD_CAST "save", BAD_CAST ""); 
    xmlNewProp(save, BAD_CAST "label", BAD_CAST "p.temperature");
    save = xmlNewChild(da, nullptr, BAD_CAST "save", BAD_CAST ""); 
    xmlNewProp(save, BAD_CAST "label", BAD_CAST "p.velocity");
    save = xmlNewChild(da, nullptr, BAD_CAST "save", BAD_CAST ""); 
    xmlNewProp(save, BAD_CAST "label", BAD_CAST "p.particleID");
    save = xmlNewChild(da, nullptr, BAD_CAST "save", BAD_CAST ""); 
    xmlNewProp(save, BAD_CAST "label", BAD_CAST "p.stress");
    save = xmlNewChild(da, nullptr, BAD_CAST "save", BAD_CAST ""); 
    xmlNewProp(save, BAD_CAST "label", BAD_CAST "p.deformationGradient");
    save = xmlNewChild(da, nullptr, BAD_CAST "save", BAD_CAST ""); 
    xmlNewProp(save, BAD_CAST "label", BAD_CAST "g.acceleration");
    save = xmlNewChild(da, nullptr, BAD_CAST "checkpoint", BAD_CAST "");
    xmlNewProp(save, BAD_CAST "cycle", BAD_CAST "2");
    xmlNewProp(save, BAD_CAST "timestepInterval", BAD_CAST "4000");

    // MPM
    std::string prescribed = 
      std::string(currPath) + "/" + "UniaxialStrainRotate_PrescribedDeformation.inp";
    auto mpm = xmlNewChild(rootNode, nullptr, BAD_CAST "MPM", BAD_CAST "");
    xmlNewChild(mpm, nullptr, BAD_CAST "time_integrator", BAD_CAST "explicit");
    xmlNewChild(mpm, nullptr, BAD_CAST "interpolator", BAD_CAST "linear");
    xmlNewChild(mpm, nullptr, BAD_CAST "use_load_curves", BAD_CAST "false");
    xmlNewChild(mpm, nullptr, BAD_CAST "minimum_particle_mass", BAD_CAST "1.0e-15");
    xmlNewChild(mpm, nullptr, BAD_CAST "minimum_mass_for_acc", BAD_CAST "1.0e-15");
    xmlNewChild(mpm, nullptr, BAD_CAST "maximum_particle_velocity", BAD_CAST "1.0e5");
    xmlNewChild(mpm, nullptr, BAD_CAST "artificial_damping_coeff", BAD_CAST "0.0");
    xmlNewChild(mpm, nullptr, BAD_CAST "artificial_viscosity", BAD_CAST "true");
    xmlNewChild(mpm, nullptr, BAD_CAST "artificial_viscosity_heating", BAD_CAST "false");
    xmlNewChild(mpm, nullptr, BAD_CAST "do_contact_friction_heating", BAD_CAST "false");
    xmlNewChild(mpm, nullptr, BAD_CAST "create_new_particles", BAD_CAST "false");
    xmlNewChild(mpm, nullptr, BAD_CAST "use_momentum_form", BAD_CAST "false");
    xmlNewChild(mpm, nullptr, BAD_CAST "with_color", BAD_CAST "true");
    xmlNewChild(mpm, nullptr, BAD_CAST "use_prescribed_deformation", BAD_CAST "true");
    xmlNewChild(mpm, nullptr, BAD_CAST "prescribed_deformation_file", BAD_CAST prescribed.c_str());
    xmlNewChild(mpm, nullptr, BAD_CAST "minimum_subcycles_for_F", BAD_CAST "-2");
    auto ero = xmlNewChild(mpm, nullptr, BAD_CAST "erosion", BAD_CAST "");
    xmlNewProp(ero, BAD_CAST "algorithm", BAD_CAST "none");

    // Physical constants
    auto pc = xmlNewChild(rootNode, nullptr, BAD_CAST "PhysicalConstants", BAD_CAST "");
    xmlNewChild(pc, nullptr, BAD_CAST "gravity", BAD_CAST "[0,0,0]");

    // Material properties
    auto matProp = 
      xmlNewChild(rootNode, nullptr, BAD_CAST "MaterialProperties", BAD_CAST "");
    mpm = xmlNewChild(matProp, nullptr, BAD_CAST "MPM", BAD_CAST "");
    auto mat = xmlNewChild(mpm, nullptr, BAD_CAST "material", BAD_CAST "");
    xmlNewProp(mat, BAD_CAST "name", BAD_CAST "TabularPlastic");

    // General properties
    xmlNewChild(mat, nullptr, BAD_CAST "density", BAD_CAST "1050");
    xmlNewChild(mat, nullptr, BAD_CAST "melt_temp", BAD_CAST "3695.0");
    xmlNewChild(mat, nullptr, BAD_CAST "room_temp", BAD_CAST "294.0");
    xmlNewChild(mat, nullptr, BAD_CAST "thermal_conductivity", BAD_CAST "174.0e-7");
    xmlNewChild(mat, nullptr, BAD_CAST "specific_heat", BAD_CAST "134.0e-8");
    auto cm = xmlNewChild(mat, nullptr, BAD_CAST "constitutive_model", BAD_CAST "");
    xmlNewProp(cm, BAD_CAST "type", BAD_CAST "tabular_plasticity");

    // Elastic properties
    std::string table_elastic = 
      std::string(currPath) + "/" + "tabular_linear_elastic.json";
    auto elastic = xmlNewChild(cm, nullptr, BAD_CAST "elastic_moduli_model", 
                               BAD_CAST "");
    xmlNewProp(elastic, BAD_CAST "type", BAD_CAST "tabular");
    xmlNewChild(elastic, nullptr, BAD_CAST "filename", 
                BAD_CAST table_elastic.c_str());
    xmlNewChild(elastic, nullptr, BAD_CAST "independent_variables", 
                BAD_CAST "PlasticStrainVol, TotalStrainVol");
    xmlNewChild(elastic, nullptr, BAD_CAST "dependent_variables", 
                BAD_CAST "Pressure");
    auto interp_elastic = xmlNewChild(elastic, nullptr, BAD_CAST "interpolation",
                                      BAD_CAST "");
    xmlNewProp(interp_elastic, BAD_CAST "type", BAD_CAST "linear");
    xmlNewChild(elastic, nullptr, BAD_CAST "G0", 
                BAD_CAST "1.0e4");
    xmlNewChild(elastic, nullptr, BAD_CAST "nu", 
              BAD_CAST "0.2");

    // Yield criterion
    std::string table_yield = 
      std::string(currPath) + "/" + "tabular_von_mises.json";
    auto yield = xmlNewChild(cm, nullptr, BAD_CAST "plastic_yield_condition", 
                             BAD_CAST "");
    xmlNewProp(yield, BAD_CAST "type", BAD_CAST "tabular");
    xmlNewChild(yield, nullptr, BAD_CAST "filename", 
                BAD_CAST table_yield.c_str());
    xmlNewChild(yield, nullptr, BAD_CAST "independent_variables", 
                BAD_CAST "Pressure");
    xmlNewChild(yield, nullptr, BAD_CAST "dependent_variables", 
                BAD_CAST "SqrtJ2");
    auto yield_interp = xmlNewChild(yield, nullptr, BAD_CAST "interpolation",
                              BAD_CAST "");
    xmlNewProp(yield_interp, BAD_CAST "type", BAD_CAST "linear");

    // Geometry
    auto geom = xmlNewChild(mat, nullptr, BAD_CAST "geom_object", BAD_CAST "");
    auto box = xmlNewChild(geom, nullptr, BAD_CAST "box", BAD_CAST "");
    xmlNewProp(box, BAD_CAST "label", BAD_CAST "Plate1");
    xmlNewChild(box, nullptr, BAD_CAST "min", BAD_CAST "[0.0,0.0,0.0]");
    xmlNewChild(box, nullptr, BAD_CAST "max", BAD_CAST "[1.0,1.0,1.0]");
    xmlNewChild(geom, nullptr, BAD_CAST "res", BAD_CAST "[1,1,1]");
    xmlNewChild(geom, nullptr, BAD_CAST "velocity", BAD_CAST "[0,0,0]");
    xmlNewChild(geom, nullptr, BAD_CAST "temperature", BAD_CAST "294");
    xmlNewChild(geom, nullptr, BAD_CAST "color", BAD_CAST "0");

    // Contact
    auto contact = xmlNewChild(mpm, nullptr, BAD_CAST "contact", BAD_CAST "");
    xmlNewChild(contact, nullptr, BAD_CAST "type", BAD_CAST "null");
    xmlNewChild(contact, nullptr, BAD_CAST "materials", BAD_CAST "[0]");
    xmlNewChild(contact, nullptr, BAD_CAST "mu", BAD_CAST "0.1");

    // Grid
    auto grid = xmlNewChild(rootNode, nullptr, BAD_CAST "Grid", BAD_CAST "");
    xmlNewChild(grid, nullptr, BAD_CAST "BoundaryConditions", BAD_CAST "");
    auto level = xmlNewChild(grid, nullptr, BAD_CAST "Level", BAD_CAST "");
    box = xmlNewChild(level, nullptr, BAD_CAST "Box", BAD_CAST "");
    xmlNewProp(box, BAD_CAST "label", BAD_CAST "1");
    xmlNewChild(box, nullptr, BAD_CAST "lower", BAD_CAST "[-2,-2,-2]");
    xmlNewChild(box, nullptr, BAD_CAST "upper", BAD_CAST "[3,3,3]");
    xmlNewChild(box, nullptr, BAD_CAST "resolution", BAD_CAST "[5,5,5]");
    xmlNewChild(box, nullptr, BAD_CAST "extraCells", BAD_CAST "[0,0,0]");
    xmlNewChild(box, nullptr, BAD_CAST "patches", BAD_CAST "[1,1,1]");

    // Print the document to stdout
    //xmlSaveFormatFileEnc("-", doc, "ISO-8859-1", 1);
    std::string ups_file = std::string(currPath) + "/" + 
                           "UniaxialStrainRotateTabularPlasticity.ups";
    xmlSaveFormatFileEnc(ups_file.c_str(), doc, "ISO-8859-1", 1);

    // Create a ProblemSpec
    ProblemSpecP ps = scinew ProblemSpec(xmlDocGetRootElement(doc), false);
    if (!ps) {
      std::cout << "**Error** Could not create ProblemSpec." << std::endl;
      std::cout << __FILE__ << ":" << __LINE__ << std::endl;
      exit(-1);
    }

    return ps;
  }
};
\end{lstlisting}

  \item Set up the entry function for the test:
\begin{lstlisting}[language=Cpp]
int main(int argc, char** argv, char* env[]) {
  ::testing::InitGoogleTest(&argc, argv);
  ::testing::AddGlobalTestEnvironment(new VaangoEnv(argc, argv, env));
  return RUN_ALL_TESTS();
}
\end{lstlisting}

  \item Create the test:
\begin{lstlisting}[language=Cpp]
TEST(TabularPlasticityTest, singleParticleTest)
{
  char currPath[2000];
  if (!getcwd(currPath, sizeof(currPath))) {
    std::cout << "Current path not found\n";
  }
  std::string ups_file = std::string(currPath) + "/" + 
                         "UniaxialStrainRotateTabularPlasticity.ups";
  //std::cout << "Created Filename = " << ups_file << std::endl;

  // Remove existing uda
  std::string uda_file = std::string(currPath) + "/" + 
                         "UniaxialStrainRotateTabularPlasticity.uda.000";
  Dir::removeDir(uda_file.c_str());

  char * start_addr = (char*)sbrk(0);
  bool thrownException = false;

  try {
    ProblemSpecP ups = VaangoEnv::createInput();
    ups->getNode()->_private = (void *) ups_file.c_str();
    //std::cout << "Filename = " << static_cast<char*>(ups->getNode()->_private) << std::endl;

    const ProcessorGroup* world = Uintah::Parallel::getRootProcessorGroup();
    SimulationController* ctl = scinew AMRSimulationController(world, false, ups);
    
    RegridderCommon* reg = 0;
    SolverInterface* solve = SolverFactory::create(ups, world, "");

    UintahParallelComponent* comp = ComponentFactory::create(ups, world, false, "");
    SimulationInterface* sim = dynamic_cast<SimulationInterface*>(comp);
    ctl->attachPort("sim", sim);
    comp->attachPort("solver", solve);
    comp->attachPort("regridder", reg);

    LoadBalancerCommon* lbc = LoadBalancerFactory::create(ups, world);
    lbc->attachPort("sim", sim);

    DataArchiver* dataarchiver = scinew DataArchiver(world, -1);
    Output* output = dataarchiver;
    ctl->attachPort("output", dataarchiver);
    dataarchiver->attachPort("load balancer", lbc);
    comp->attachPort("output", dataarchiver);
    dataarchiver->attachPort("sim", sim);

    SchedulerCommon* sched = SchedulerFactory::create(ups, world, output);
    sched->attachPort("load balancer", lbc);
    ctl->attachPort("scheduler", sched);
    lbc->attachPort("scheduler", sched);
    comp->attachPort("scheduler", sched);
    sched->setStartAddr( start_addr );
    sched->addReference();

    ctl->run();
    delete ctl;

    sched->removeReference();
    delete sched;
    delete lbc;
    delete sim;
    delete solve;
    delete output; 

  } catch (ProblemSetupException& e) {
    std::cout << e.message() << std::endl;
    thrownException = true;
  } catch (Exception& e) {
    std::cout << e.message() << std::endl;
    thrownException = true;
  } catch (...) {
    std::cout << "**ERROR** Unknown exception" << std::endl;
    thrownException = true;
  }
}
\end{lstlisting}
\end{enumerate}




