\chapter{Basic Vaango usage} \label{Chapter:UCF}
Several executable programs have been developed using the \Vaango
Computational Framework (UCF).  The primary code that drives the
components implemented in \Vaango is called \Textsfc{vaango}.

Although \Uintah was developed for complex fire-explosion simulations,
the general nature of the algorithms and the framework has allowed 
researchers to use the code to investigate a
wide range of problems.  The \Vaango framework is general purpose enough to
allow for the implementation of a variety of implicit and explicit
algorithms on structured grids but focuses on particle based
algorithms.

\section{Installing the Vaango software}
For information on downloading the \Vaango software package 
and how to setup and build it, please refer to the \Vaango Installation Guide.

\section{Running Vaango}
For single processor simulations, the \Textsfc{vaango} executable
is run from the command line prompt like this:
\begin{lstlisting}[backgroundcolor=\color{background}]
  vaango input.ups
\end{lstlisting}
where \Textsfc{input.ups} is an XML formatted input file.  The
\Vaango software contains numerous example input files located
in the \Textsfc{src/StandAlone/inputs} directory.

For multiprocessor runs, the user generally uses \Textsfc{mpirun}
to launch the code.  Depending on the environment, batch
scheduler, launch scripts, etc, \Textsfc{mpirun} may or may not
be used.  However, in general, something like the following is used:
\begin{lstlisting}[backgroundcolor=\color{background}]
  mpirun -np num_processors vaango input.ups
\end{lstlisting}

\Textbfc{num\_processors} is the number of processors that will
be used.  The input file must contain a patch layout that has at least
the same number (or greater) of patches as processors specified by a
number following the \Textbfc{-np} option shown above.

In addition, the \Textsfc{-mpi} flag is optional but sometimes
necessary if the mpi environment is not automatically detected from
within the \Textsfc{vaango} executable.

\Vaango provides for restarting from checkpoint as well.  For information on
this, see Section~\ref{Sec:DataArchiver}, which describes how to create
checkpoint data, and how to restart from it.

\section{Vaango Problem Specification} \label{Sec:UPS}
The \Vaango framework uses XML like input files to specify the various
parameters required by simulation components.  These are called
\Textsfc{Uintah Problem Specification} files and have the extension \Textbfc{.ups}
because they are directly based on the \Uintah input file format.
The \Textbfc{.ups} files are validated based on the specification
found in \Textbfc{src/StandAlone/inputs/UPS\_SPEC/ups\_spec.xml}
and its sibling files.  

The application developer is free to use any of the specified tags to
specify the data needed by the simulation.  The essential tags that
are required by \Vaango include the following:
\begin{lstlisting}[language=XML]
  <Uintah_specification>
  <SimulationComponent>
  <Time>
  <DataArchiver>
  <Grid>
\end{lstlisting}

Individual components have additional tags that specify properties,
algorithms, materials, etc. that are unique to that individual
components.  Within the individual sections on MPM, ICE, and MPMICE
the individual tags will be explained more fully.

The \Textsfc{vaango} executable verifies that the input file adheres to a consistent
specification and that all necessary tags are specified.  However, it
is up to the individual creating or modifying the input file to put in
physically reasonable set of consistent parameters.

\section{Simulation Components} \label{Sec:SimulationComponent}
The input file tag for \Textsfc{SimulationComponent} has the \Textsfc{type}
attribute that must be specified with either \Textsfc{mpm, mpmice, ice, peridynamics}
as in:
\begin{lstlisting}[language=XML]
<SimulationComponent type = "mpm" />
\end{lstlisting}

\section{Time Related Variables} \label{Sec:TimeRelatedVariables}
\Vaango components are time-dependent codes.  As such, one of the first
entries in each input file describes the time-stepping parameters.  An
input file segment is given below that encompasses all of the possible
parameters.  The function of each of these parameters is described below.

\begin{lstlisting}[language=XML]
<Time>
    <maxTime>            1.0         </maxTime>
    <initTime>           0.0         </initTime>
    <delt_min>           0.0         </delt_min>
    <delt_max>           1.0         </delt_max>
    <delt_init>          1.0e-9      </delt_init>
    <max_delt_increase>  2.0         </max_delt_increase>
    <timestep_multiplier>1.0         </timestep_multiplier>
    <max_Timestep>       100         </max_Timestep>
    <end_on_max_time_exactly>true    </end_on_max_time_exactly>
</Time>
\end{lstlisting}

The following fields are required:
\begin{itemize}
\item \Textbfc{maxTime} - how long in physical time to run the simulation for
\item \Textbfc{initTime} - what time to begin the simulation at
\item \Textbfc{delt\_min} - the smallest timestep the simulation will take
\item \Textbfc{delt\_max} - the largest timestep the simulation will take
\item \Textbfc{timestep\_multiplier} - multiplies the timestep by this number (before adjusting to min or max timestep)
\end{itemize}

The following fields are optional:
\begin{itemize}
\item \Textbfc{delt\_init} - The timestep to take initially (assuming it's less than the one computed by the simulation)
\item \Textbfc{initial\_delt\_range} - The period of time to use the \Textbfc{delt\_init} (default = 0)
\item \Textbfc{max\_delt\_increase} - Maximum amount to multiply the previous delt by (if the newly computed delt is greater than the previous one)
\item \Textbfc{max\_iterations} - The number of timesteps to run the simulation for (even on a restart)
\item \Textbfc{max\_Timesteps} - The timestep number to end the simulation on (not usually used with \Textbfc{max\_iterations})
\item \Textbfc{override\_restart\_delt} - On a restart, use this delt instead of the most-recently-used delt.
\item \Textbfc{clamp\_timesteps\_to\_output} - Sync the delt with the DataArchiver - when an output timestep occurs, reduce the delt to have the time land on the timestep interval (default = false)
\item \Textbfc{end\_on\_max\_time\_exactly} - clamp the delt such that the last timesteps end on what was specified in \Textbfc{maxTime} (default = false)
\end{itemize}
\begin{NoteBox}
A word about timesteps: In general, the timestep (delt) is computed at various stages within a timestep, and the smallest one is used, unless it needs to raise the delt to the \Textbfc{delt\_min}.
\end{NoteBox}

\section{Data Archiver} \label{Sec:DataArchiver}
The Data Archiver section specifies the directory name where data will
be stored and what variables will be saved and how often data is saved
and how frequently the simulation is checkpointed.

The \Textsfc{\textless filebase\textgreater} tag is used to specify the directory
name and by convention, the \Textsfc{.uda} suffix is attached denoting the
``\Vaango Data Archive".

Data can be saved based on a frequency setting that is either based on integer time
intervals:
\begin{lstlisting}[language=XML]
  <outputTimestepInterval> 100 </outputTimestepInterval>
\end{lstlisting}
or real-valued timestep intervals:
\begin{lstlisting}[language=XML]
  <outputInterval> 1.0e-3 </outputInterval>}
\end{lstlisting}

Each simulation component specifies variables with label names that
can be specified for data output.  By convention, particle data are
denoted by \Textsfc{p.} followed by a particular variable name
such as mass, velocity, stress, etc.  Whereas for node based data, the
convention is to use the \Textsfc{g.} followed by the variable
name, such as mass, stress, velocity, etc.  Similarly, cell-centered
and face-centered data typically end with the a trailing \Textsfc{CC}
or \Textsfc{FC,}  respectively.  Within the DataArchiver
section, variables are specified with the following format:

\begin{lstlisting}[language=XML]
   <save label = "p.mass" />
   <save label = "g.mass" />
\end{lstlisting}

To see a list of
variables available for saving for a given component, execute the following
command from the \Textsfc{StandAlone} directory:

\begin{lstlisting}[backgroundcolor=\color{background}]
inputs/labelNames component
\end{lstlisting}
where \Textbfc{component} is, e.g., \Textbfc{mpm}, \Textbfc{ice}, etc.

Check-pointing information can be created that provides a mechanism for
restarting a simulation at a later point in time.  The \Textbfc{\textless checkpoint\textgreater}
tag with the \Textbfc{cycle} and \Textbfc{ interval} attributes describe how many
copies of checkpoint data is stored (cycle) and how often it is generated
(interval).  You may also use the \Textbfc{walltimeStart} and \Textbfc{walltimeInterval}
options for specifying when and how offen a checkpoint will be output based
on wall-clock time.

As an example of checkpoint data that has two timesteps worth of
che dckpoint data that is created every .01 seconds of simulation time
are shown below:

\begin{lstlisting}[language=XML]
<checkpoint cycle = "2" interval = "0.01"/>
\end{lstlisting}
% d

To restart from a checkpointed archive, simply put \Textsfc{-restart} in the
\Textsfc{vaango} command-line arguments and specify the .uda directory instead of
a ups file (\Textsfc{vaango} reads the copied \Textsfc{input.xml} from the
archive).  One can optionally specify a certain timestep to restart
from with \Textsfc{-t timestep} with multiple checkpoints, but the
last checkpointed timestep is the default.  When restarting, \Textsfc{vaango}
copies all of the appropriate information from the old uda directory to its
new uda directory.
% I'M NOT SURE ABOUT THIS, -nocopy REMOVES THE OLD UDA?  I THOUGHT IT
% JUST LEFT STUFF IN THE OLD UDA, WHERE AS -copy MADE A COPY OF OLD TIMESTEP
% DATA, AND -move MOVED THE OLD TIMESTEP DATA TO THE NEW UDA.  
%  If one doesn't want to keep the old uda directory
%around, they can specify \Textsfc{-nocopy} to have it be removed
%(e.g., if you are cramped for disk space).  Either way it creates a new uda
%directory for you as always.

Here are some examples:

\begin{lstlisting}[backgroundcolor=\color{background}]
./vaango -restart disks.uda.000 -nocopy
./vaango -restart disks.uda.000 -t 29
\end{lstlisting}
%
%__________________________________

\section{Simulation Options} \label{Sec:SimulationOptions}


There are many options available when running MPM simulations.  These
are generally specified in the \Textsfc{\textless MPM\textgreater} section of the input file.
A list of these options taken from 
\Textbfc{inputs/UPS\_SPEC/mpm\_spec.xml}  is given in the \Vaango
Developers Manual.

\section{Geometry objects} \label{Sec:GeometryObjects}

Within several of the components, the material is described by a
combination of physical parameters and the geometry.  Geometry objects
use the notion of constructive solid geometry operations to compose
the layout of the material from simple shapes such as boxes, spheres,
cylinders and cones, as well as operators which include the union,
intersections, differences of the simple shapes.  In addition to the
simple shapes, triangulated surfaces can be used in conjunction with
the simple shapes and the operations on these shapes.

Each geometry object has the following properties, label (string
name), type (box, cylinder, sphere, etc), resolution (vector
quantity), and any unique geometry parameters such as origin, corners,
triangulated data file, etc.  The operators which include, the union,
the difference, and intersection tags contain either lists of
additional operators or the primitives pieces.

As an example of a non-trivial geometry object is shown below:

\begin{lstlisting}[language=XML]
<geom_object>
     <intersection>
       <box label = "Domain">
          <min>[0.0,0.0,0.0]</min>
          <max>[0.1,0.1,0.1]</max>
       </box>
       <union>
         <sphere label = "First node">
            <origin>[0.022,0.028,0.1  ]</origin>
            <radius>0.01</radius>
         </sphere>
         <sphere label = "2nd node">
            <origin>[0.030,0.075,0.1  ]</origin>
            <radius>0.01</radius>
         </sphere>
       </union>
     </intersection>
     <res>[2,2,2]</res>
     <velocity>[0.,0.,0.]</velocity>
     <temperature>0 </temperature>
</geom_object>
\end{lstlisting}

The following geometry objects are given with their required tags:

\subsubsection{Box}
\Textsfc{box} has the following tags: min and max which are
vector quantities specified in the \Textsfc{[a, b, c]} format.

\subsubsection{Sphere}
\Textsfc{sphere} has an origin tag specified as a vector and the
radius tag specified as a float.

\subsubsection{Cone}
\Textsfc{cone} has a tag for the top and bottom origins (vector)
as well as tags for the top and bottom radius (float) to create a
right circular cone/frustum.

\subsubsection{Cylinder}
\Textsfc{cylinder} has a tag for the top and bottom origins
(vector) plus a tag for the radius (float).

\subsubsection{Smooth Cylinder}
\Textsfc{smoothcyl} is a geomtry object designed for use with
the cpdi algorithm, which uses a body fit particle spatial
distribution.  This eliminates ``stair-stepped'' boundaries typical of
the standard, grid-based, discretization scheme.  
\begin{NoteBox}
Thus it is
  important to note that the \Textsfc{smoothcyl} geometry is only designed to work with
  \Textsfc{\textless interpolator\textgreater cpdi \textless /interpolator\textgreater}. Other
  algorithms may give erroneous answers.
\end{NoteBox}

This geometry has the following tags:

\begin{lstlisting}[language=XML]
	   <smoothcyl label = "label name">
	     <discretization_scheme> string </discretization_scheme>
	     <bottom> vector </bottom>
	     <top> vector </top>
	     <outer_radius> float </outer_radius>
	     <inner_radius> float </inner_radius>
	     <num_radial> integer </num_radial>
	     <num_axial> integer </num_axial>
	     <num_angular> integer </num_angular>
	     <arc_start_angle> double (in degrees) </arc_start_angle>
	     <arc_angle> double (in degrees) </arc_angle>
	   </smoothcyl>
\end{lstlisting}

The complete or partial annulus or cylinder specified by inner
(optional, defaulting to zero) and outer radii, cylinder axis bottom
and top (all required) and start and final angle (both optional,
defaulting to 0 and 360) will be discretized using planes of
concentric rings of particles.  Particle density is specified by \Textbfc{num\_axial} and \Textbfc{num\_radial}, the number of
particles in the axial and radial dimensions, respectively.
  Note that these particle density specifications
supercede those specified in the \Textbfc{\textless geom\_object\textgreater \textless res\textgreater}
tag, which is ignored.

The required discretization scheme may be either \Textbfc{pie\_slices}
 or \Textbfc{constant\_particle\_volumes}.  For the
\Textbfc{pie\_slices} discretization, \Textbfc{num\_angular}  is required to specify the number of particles between \Textbfc{arc\_start} and \Textbfc{arc\_angle}.  For the \Textbfc{constant\_particle\_volumes} discretization, the number of
particles between \Textbfc{arc\_start} and \Textbfc{arc\_angle}
is determined individually for each ring of particles by
attempting to keep particle spacings approximately equal in the radial
and angular directions, and thus particle volumes approximately
constant.

End caps may be added to the smoothcyl using the optional \Textbfc{\textless endcap\_thickness\textgreater} tag, which specifies the axial
dimension of cylinders which are appended to each end of the specified
\Textsfc{smoothcyl} (the radii are the same as the \Textsfc{smoothcyl}
).  Presently, the end cap body fit discretization uses a
legacy scheme.

Note: At the time of writing, multiple \Textsfc{smoothcyl}
geometries within a \Textbfc{\textless geom\_object\textgreater} tag were not
discretized using a body fit particle distribution as described here
(rather the default discretization scheme is used).  This will be
fixed eventually, at which point it may be possible to create more
general endcaps using unions of \Textsfc{smoothcyl}.

\subsubsection{Ellipsoid}
\Textsfc{ellipsoid} has an origin tag specified as a vector.  There are 
two ways to assign axis lengths depending on the orientation of the ellipsoid.
If the axes are aligned with the Cartesian grid, they may be specified as 
floating point values with tagnames: rx, ry, rz.  For all other orientation, 
three vector quantities must be specified in the \Textsfc{[a,b,c]} format.
Vector quantity tag names are: v1, v2, v3.These vectors must be orthogonal 
to within 1e-12 after dot product or the simulation will throw an exception.  
Note, if both vector quantities and floating point tags are used, 
the vector quantity inputs will take precedence.  

\subsubsection{Parallelepiped}
\Textsfc{parallelepiped} requires that four points be specified as
illustrated by the ASCII art snippet taken from the source code:

\begin{lstlisting}[backgroundcolor=\color{background}]
//********************************************
//                                          //
//             *-------------*              //
//            / .           / \             //
//           /   .         /   \            //
//          P4-------------*    \           //
//           \    .         \    \          //
//            \   P2.........\....*         //
//             \ .            \  /          //
//             P1--------------P3           //
//
//  Returns true if the point is inside (or on) the parallelepiped.
//  (The order of p2, p3, and p4 doesn't really matter.)
\end{lstlisting}

\subsubsection{Triangulated surface}
\Textsfc{tri} is a tag for describing a triangulated surface.
The name tag specifies the file name to use for reading in the
triangulated surface description and the points file.  The
triangulated surface (\Textbfc{file\_name.tri}) contains a list of integers
describing the connectivity of points specified in \Textbfc{file\_name.pts}.
Here is an excerpt from a tri file and a points file:

\begin{lstlisting}[backgroundcolor=\color{background}]
Triangulated file

1 39 41
1 41 38
38 41 42
. . .

Points file

0 0.03863 -0.005
0.35227 0.13023 -0.005
0.00403479 0.0296797 -0.005
. . .
\end{lstlisting}
The Mach 2 Wedge example in Section~\ref{Sec:MPMICE_EXAMPLES} depicts usage of
this option.

\subsubsection{Boolean operations}
The boolean operators on the geometry pieces include \Textsfc{difference,
intersection,} and \Textsfc{union}.

The \Textsfc{difference} takes two geometry pieces and subtracts
the second geometry piece from the first geometry piece.  The \Textsfc{
intersection} operator requires at least two geometry
pieces in forming an intersection geometry piece.  Whereas the \Textsfc{
union} operator aggregates a collection of geometry pieces.
Multiple operators can be used to form very complex geometry pieces.

\subsubsection{Resolution}
An additional input in the \Textbfc{\textless geom\_object\textgreater} field is the
\Textsfc{\textless res\textgreater} tag.  In MPM, this simply refers to how many particles
are placed in each cell in each coordinate direction.  For multi-material ICE
simulations, the \Textsfc{\textless res\textgreater} serves a similar purpose in that one
can specify the subgrid resolution of the initial material distribution
of mixed cells at the interface of geometry objects.

\subsubsection{Specifying particle locations directly}
In addition to the above, it is also possible in MPM simulations to describe
geometry by providing a file containing a series of particle locations.  These
can be in either ASCII or binary format.  In addition, it is also possible to
provide initial data for certain variables on the particles, including
volume, temperature, external force, fiber direction (used in material models
with transverse isotropy) and velocity.  The following is an example in which
external force and fiber direction are specified:
\begin{lstlisting}[language=XML]
          <file>
              <name>LVcoarse.pts</name>
              <var>p.externalforce</var>
              <var>p.fiberdir</var>
          </file>
\end{lstlisting}
where the text file \Textsfc{LVcoarse.pts} looks like:
\begin{lstlisting}[backgroundcolor=\color{background}]
0.0385 0.0335 0.0015 0 0 0 0.248865 -0.0593421 -0.966718
0.0395 0.0335 0.0015 0 0 0 0.254892 -0.0220365 -0.966718
0.0405 0.0335 0.0015 0 0 0 0.267002 0.0197728 -0.963493
0.0415 0.0335 0.0015 0 0 0 0.261177 0.0588869 -0.963493
	.
	.
	.
\end{lstlisting}
Because these files can be arbitrarily large, an additional preprocessing step
must be taken before issuing the \Textsfc{vaango} command.
\Textsfc{pfs} for \Textsfc{Particle File Splitter} is a utility that splits the
data in the \Textsfc{.pts} file into a series of files
(\Textsfc{file.pts.0, file.pts.1,}, etc), one for each
patch.  By doing this, each processor needs only read in the data for the
patches that it contains, rather than each processor reading in the entire file,
which can be hard on the file system.  Note, that this step is required,
even if only using a single patch, and must be reissued any time the patch
configuration is changed.  Usage of this utility, which is compiled
into the \Textsfc{StandAlone/tools/pfs} directory, is:
\begin{lstlisting}[backgroundcolor=\color{background}]
   pfs input.ups
\end{lstlisting}

\subsubsection{Using image data to create particles}
One final option is available for initializing particle positions in MPM
simulations, and that is through the use of three dimensional image data,
such as might be collected via CT scans or confocal microscopy.  The image data are 
provided as 8-bit raw files, and usage in the input file is given as:
\begin{lstlisting}[language=XML]
        <image>
          <name>spheres.raw</name>
          <res>[1600, 1600, 1600]</res>
          <threshold>[1, 25]</threshold>
        </image>
        <file>
          <name>spheres.pts</name>
          <format>bin</format>
        </file>
\end{lstlisting}
The \Textsfc{\textless image\textgreater} section gives the name of the file, the resolution, in pixels,
in the various coordinate directions, and threshold range.  Particles will be
generated at voxels within the specified range.  The \Textsfc{\textless file\textgreater}
section is the same as that described above.  A different preprocessing utility
is provided when using image data (for the same reasons described previously).
Usage is as follows:
\begin{lstlisting}[backgroundcolor=\color{background}]
   pfs2 -b input.ups
\end{lstlisting}
The \Textbfc{-b} flag indicates that binary \Textsfc{spheres.pts.\#} files will be created, which
saves considerable disk space when performing large simulations.

\section{Boundary conditions}\label{sec:ucf_bc}

Boundary conditions are specified within the \Textsfc{\textless Grid\textgreater}
but are described separately for clarity.  The essential idea is that
boundary conditions are specified on the domain of the grid.  Values
can be assigned either on the entire face, or parts of the face.
Combinations of various geometric descriptions are used to aid in the
assignment of values over specific regions of the grid.  Each of the
six faces of the grid is denoted by either the minus or plus side of
the domain.

The XML description of a particular boundary condition includes which
side of the domain, the material id, what type of boundary condition
(Dirichlet or Neumann) and which variable and the value assigned.  The
following is a an MPM specification of a Dirichlet boundary condition
assigned to the velocity component on the x minus face (the entire
side) with a vector value of [0.0,0.0,0.0] applied to all of the materials.

\begin{lstlisting}[language=XML]
 <Grid>
       <BoundaryConditions>
         <Face side = "x-">
             <BCType id = "all" var = "Dirichlet" label = "Velocity">
                   <value> [0.0,0.0,0.0] </value>
             </BCType>
         </Face>
         <Face side = "x+">
            <BCType id = "all" var = "Dirichlet" label = "Velocity">
                 <value> [0.0,0.0,0.0] </value>
            </BCType>
         </Face>
        . . . .
        <BoundaryCondition>
   . . . .
  <Grid>
\end{lstlisting}

The notation \Textsfc{\textless Face side = "x-"\textgreater} indicates that the
entire x minus face of the boundary will have the boundary condition
applied.  The \Textsfc{id = "all"} means that all the
materials will have this value.  To specify the boundary condition for
a particular material, specify an integer number instead of the
"all".  The \Textsfc{var = "Dirichlet"} is used to specify
whether it is a Dirichlet or Neumann or symmetry boundary conditions.
Different components may use the \Textsfc{var} to include a
variety of different boundary conditions and are explained more fully
in the following component sections.  The \Textsfc{label = "Velocity"}
specifies which variable is being assigned and again is
component dependent.  The \Textsfc{\textless value\textgreater [0.0,0.0,0.0] \textless/value\textgreater}
specifies the value.

An example of a more complicated boundary condition demonstrating a
hot jet of fluid issued into the domain is described.  The jet is
described by a circle on one side of the domain with boundary
conditions that are different in the circular jet compared to the rest
of the side.
\begin{lstlisting}[language=XML]
 <Face circle = "y-" origin = "0.0 0.0 0.0" radius = ".5">
        <BCType id = "0"   label = "Pressure" var = "Neumann">
                              <value> 0.0   </value>
        </BCType>
        <BCType id = "0" label = "Velocity" var = "Dirichlet">
                              <value> [0.,1.,0.] </value>
        </BCType>
        <BCType id = "0" label = "Temperature" var = "Dirichlet">
                              <value> 1000.0  </value>
        </BCType>
        <BCType id = "0" label = "Density" var = "Dirichlet">
                              <value> .35379  </value>
        </BCType>
        <BCType id = "0" label = "SpecificVol"  var = "computeFromDensity">
                              <value> 0.0  </value>
        </BCType>
      </Face>
      <Face side = "y-">
        <BCType id = "0"   label = "Pressure"     var = "Neumann">
                              <value> 0.0   </value>
        </BCType>
        <BCType id = "0" label = "Velocity"     var = "Dirichlet">
                              <value> [0.,0.,0.] </value>
        </BCType>
        <BCType id = "0" label = "Temperature"  var = "Neumann">
                              <value> 0.0  </value>
        </BCType>
        <BCType id = "0" label = "Density"      var = "Neumann">
                              <value> 0.0  </value>
        </BCType>
        <BCType id = "0" label = "SpecificVol"  var = "computeFromDensity">
                              <value> 0.0  </value>
        </BCType>
      </Face>
\end{lstlisting}
The jet is described by the circle on the y minus face with the origin
at $(0,0,0)$ and a radius of $0.5$.  For the region outside of the circle,
the boundary conditions are different.  Each side must have at least
the \Textsfc{side} specified, but additional circles and
rectangles can be specified on a given face.

An example of the \Textsfc{rectangle} is specified as with the
lower corner at $(0,0.181,0)$ and upper corner at $(0,0.5,0)$.
\begin{lstlisting}[language=XML]
 <Face rectangle = "x-" lower = "0.0 0.181 0.0" upper = "0.0 0.5 0.0">
\end{lstlisting}

\section{Grid specification} \label{Sec:Grid}
The \Textsfc{\textless Grid\textgreater} section specifies the domain of the
structured grid and includes tags which indicate the lower and upper
corners, the number of extra cells which can be used by various
components for the application of boundary conditions or interpolation
schemes.  

The grid is decomposed into a number of patches.  For single processor
problems, usually one patch is used for the entire domain.  For
multiple processor simulations, there must be at least one patch per
processor.  Patches are specified along the x,y,z directions of the
grid using the \Textsfc{\textless patches\textgreater [2,5,3] \textless/patches\textgreater} which
specifies two patches along the x direction, five patches along the y
direction and 3 patches along the z direction.  The maximum number of
processors that \Textsfc{vaango} could use is $2\times 5\times 3 = 30$.
Attempting to use more processors than patches
will cause a run time error during initialization.

Finally, the grid spacing can specified using either a fixed number of
cells along each x,y,z direction or by the size of the grid cell in
each direction.  To specify a fixed number of grid cells, use the \Textsfc{
\textless resolution\textgreater [20,20,3] \textless/resolution\textgreater}.  This specifies 20
grid cells in the x direction, 20 in the y direction and 3 in the z
direction.  To specify the grid cell size use the \Textsfc{\textless spacing\textgreater
[0.5,0.5,0.3] \textless /spacing\textgreater}.  This specifies the a grid cell
size of .5 in the x and y directions and .3 in the z direction.  The
\Textsfc{\textless resolution\textgreater} and \Textsfc{\textless spacing\textgreater} cannot be
specified together.  The following two examples would generate
identical grids:
\begin{lstlisting}[language=XML]
<Level>
    <Box label="1">
       <lower>        [0,0,0]          </lower>
       <upper>        [5,5,5]          </upper>
       <extraCells>   [1,1,1]          </extraCells>
       <patches>      [1,1,1]          </patches>
    </Box>
    <spacing>         [0.5,0.5,0.5]    </spacing>
</Level>
\end{lstlisting}

\begin{lstlisting}[language=XML]
<Level>
    <Box label="1">
       <lower>        [0,0,0]          </lower>
       <upper>        [5,5,5]          </upper>
       <resolution>   [10,10,10]       </resolution>
       <extraCells>   [1,1,1]          </extraCells>
       <patches>      [1,1,1]          </patches>
    </Box>
</Level>
\end{lstlisting}

The above examples indicate that the grid domain has a lower corner at
$(0,0,0)$ and an upper corner at $(5,5,5)$ with one extra cell in each
direction.  The domain is broken down into one patch covering the
entire domain with a grid spacing of $(.5,.5,.5)$.  Along each dimension
there are ten cells in the interior of the grid and one layer of
\Textsfc{extraCells} outside of the domain.  \Textsfc{extraCells}
are the \Uintah nomenclature for what are frequently referred to 
as \Textsfc{ghost-cells}.
