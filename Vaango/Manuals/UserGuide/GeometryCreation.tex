\chapter{Geometry creation}\label{chap:GeometryCreation}
The creation of geometry objects within the computational domain is
a crucial part of the simulation process.  This chapter discusses the
options avialble for creating geometry in \Vaango.

\section{Basic geometry objects} \label{Sec:GeometryObjects}


Each geometry object has the following properties, label (string
name), type (box, cylinder, sphere, etc), resolution (vector
quantity), and any unique geometry parameters such as origin, corners,
triangulated data file, etc.  The operators which include, the union,
the difference, and intersection tags contain either lists of
additional operators or the primitives pieces.

As an example of a non-trivial geometry object is shown below:

\begin{lstlisting}[language=XML]
<geom_object>
     <intersection>
       <box label = "Domain">
          <min>[0.0,0.0,0.0]</min>
          <max>[0.1,0.1,0.1]</max>
       </box>
       <union>
         <sphere label = "First node">
            <origin>[0.022,0.028,0.1  ]</origin>
            <radius>0.01</radius>
         </sphere>
         <sphere label = "2nd node">
            <origin>[0.030,0.075,0.1  ]</origin>
            <radius>0.01</radius>
         </sphere>
       </union>
     </intersection>
     <res>[2,2,2]</res>
     <velocity>[0.,0.,0.]</velocity>
     <temperature>0 </temperature>
</geom_object>
\end{lstlisting}

The following geometry objects are given with their required tags:

\subsection{Box}
\Textsfc{box} has the following tags: min and max which are
vector quantities specified in the \Textsfc{[a, b, c]} format.

\subsection{Sphere}
\Textsfc{sphere} has an origin tag specified as a vector and the
radius tag specified as a float.

\subsection{Cone}
\Textsfc{cone} has a tag for the top and bottom origins (vector)
as well as tags for the top and bottom radius (float) to create a
right circular cone/frustum.

\subsection{Cylinder}
\Textsfc{cylinder} has a tag for the top and bottom origins
(vector) plus a tag for the radius (float).

\subsection{Ellipsoid}
\Textsfc{ellipsoid} has an origin tag specified as a vector.  There are 
two ways to assign axis lengths depending on the orientation of the ellipsoid.
If the axes are aligned with the Cartesian grid, they may be specified as 
floating point values with tagnames: rx, ry, rz.  For all other orientation, 
three vector quantities must be specified in the \Textsfc{[a,b,c]} format.
Vector quantity tag names are: v1, v2, v3.These vectors must be orthogonal 
to within 1e-12 after dot product or the simulation will throw an exception.  
Note, if both vector quantities and floating point tags are used, 
the vector quantity inputs will take precedence.  

\subsection{Parallelepiped}
\Textsfc{parallelepiped} requires that four points be specified as
illustrated by the ASCII art snippet taken from the source code:

\begin{lstlisting}[backgroundcolor=\color{background}]
//********************************************
//                                          //
//             *-------------*              //
//            / .           / \             //
//           /   .         /   \            //
//          P4-------------*    \           //
//           \    .         \    \          //
//            \   P2.........\....*         //
//             \ .            \  /          //
//             P1--------------P3           //
//
//  Returns true if the point is inside (or on) the parallelepiped.
//  (The order of p2, p3, and p4 doesn't really matter.)
\end{lstlisting}

\section{Special geometry objects} \label{Sec:SpecialGeometryObjects}
\subsection{Smooth Cylinder}
\Textsfc{smoothcyl} is a geomtry object designed for use with
the cpdi algorithm, which uses a body fit particle spatial
distribution.  This eliminates ``stair-stepped'' boundaries typical of
the standard, grid-based, discretization scheme.  
\begin{NoteBox}
Thus it is
  important to note that the \Textsfc{smoothcyl} geometry is only designed to work with
  \Textsfc{\textless interpolator\textgreater cpdi \textless /interpolator\textgreater}. Other
  algorithms may give erroneous answers.
\end{NoteBox}

This geometry has the following tags:

\begin{lstlisting}[language=XML]
	   <smoothcyl label = "label name">
	     <discretization_scheme> string </discretization_scheme>
	     <bottom> vector </bottom>
	     <top> vector </top>
	     <outer_radius> float </outer_radius>
	     <inner_radius> float </inner_radius>
	     <num_radial> integer </num_radial>
	     <num_axial> integer </num_axial>
	     <num_angular> integer </num_angular>
	     <arc_start_angle> double (in degrees) </arc_start_angle>
	     <arc_angle> double (in degrees) </arc_angle>
	   </smoothcyl>
\end{lstlisting}

The complete or partial annulus or cylinder specified by inner
(optional, defaulting to zero) and outer radii, cylinder axis bottom
and top (all required) and start and final angle (both optional,
defaulting to 0 and 360) will be discretized using planes of
concentric rings of particles.  Particle density is specified by \Textbfc{num\_axial} and \Textbfc{num\_radial}, the number of
particles in the axial and radial dimensions, respectively.
  Note that these particle density specifications
supercede those specified in the \Textbfc{\textless geom\_object\textgreater \textless res\textgreater}
tag, which is ignored.

The required discretization scheme may be either \Textbfc{pie\_slices}
 or \Textbfc{constant\_particle\_volumes}.  For the
\Textbfc{pie\_slices} discretization, \Textbfc{num\_angular}  is required to specify the number of particles between \Textbfc{arc\_start} and \Textbfc{arc\_angle}.  For the \Textbfc{constant\_particle\_volumes} discretization, the number of
particles between \Textbfc{arc\_start} and \Textbfc{arc\_angle}
is determined individually for each ring of particles by
attempting to keep particle spacings approximately equal in the radial
and angular directions, and thus particle volumes approximately
constant.

End caps may be added to the smoothcyl using the optional \Textbfc{\textless endcap\_thickness\textgreater} tag, which specifies the axial
dimension of cylinders which are appended to each end of the specified
\Textsfc{smoothcyl} (the radii are the same as the \Textsfc{smoothcyl}
).  Presently, the end cap body fit discretization uses a
legacy scheme.

Note: At the time of writing, multiple \Textsfc{smoothcyl}
geometries within a \Textbfc{\textless geom\_object\textgreater} tag were not
discretized using a body fit particle distribution as described here
(rather the default discretization scheme is used).  This will be
fixed eventually, at which point it may be possible to create more
general endcaps using unions of \Textsfc{smoothcyl}.

\subsection{Triangulated surface}
\Textsfc{tri} is a tag for describing a triangulated surface.
The name tag specifies the file name to use for reading in the
triangulated surface description and the points file.  The
triangulated surface (\Textbfc{file\_name.tri}) contains a list of integers
describing the connectivity of points specified in \Textbfc{file\_name.pts}.
Here is an excerpt from a tri file and a points file:

\begin{lstlisting}[backgroundcolor=\color{background}]
Triangulated file

1 39 41
1 41 38
38 41 42
. . .

Points file

0 0.03863 -0.005
0.35227 0.13023 -0.005
0.00403479 0.0296797 -0.005
. . .
\end{lstlisting}
The Mach 2 Wedge example in Section~\ref{Sec:MPMICE_EXAMPLES} depicts usage of
this option.

\subsection{Boolean operations}
The boolean operators on the geometry pieces include \Textsfc{difference,
intersection,} and \Textsfc{union}.

The \Textsfc{difference} takes two geometry pieces and subtracts
the second geometry piece from the first geometry piece.  The \Textsfc{
intersection} operator requires at least two geometry
pieces in forming an intersection geometry piece.  Whereas the \Textsfc{
union} operator aggregates a collection of geometry pieces.
Multiple operators can be used to form very complex geometry pieces.

\subsection{Specifying particle locations directly}
In addition to the above, it is also possible in MPM simulations to describe
geometry by providing a file containing a series of particle locations.  These
can be in either ASCII or binary format.  In addition, it is also possible to
provide initial data for certain variables on the particles, including
volume, temperature, external force, fiber direction (used in material models
with transverse isotropy) and velocity.  The following is an example in which
external force and fiber direction are specified:
\begin{lstlisting}[language=XML]
          <file>
              <name>LVcoarse.pts</name>
              <var>p.externalforce</var>
              <var>p.fiberdir</var>
          </file>
\end{lstlisting}
where the text file \Textsfc{LVcoarse.pts} looks like:
\begin{lstlisting}[backgroundcolor=\color{background}]
0.0385 0.0335 0.0015 0 0 0 0.248865 -0.0593421 -0.966718
0.0395 0.0335 0.0015 0 0 0 0.254892 -0.0220365 -0.966718
0.0405 0.0335 0.0015 0 0 0 0.267002 0.0197728 -0.963493
0.0415 0.0335 0.0015 0 0 0 0.261177 0.0588869 -0.963493
	.
	.
	.
\end{lstlisting}
Because these files can be arbitrarily large, an additional preprocessing step
must be taken before issuing the \Textsfc{vaango} command.
\Textsfc{pfs} for \Textsfc{Particle File Splitter} is a utility that splits the
data in the \Textsfc{.pts} file into a series of files
(\Textsfc{file.pts.0, file.pts.1,}, etc), one for each
patch.  By doing this, each processor needs only read in the data for the
patches that it contains, rather than each processor reading in the entire file,
which can be hard on the file system.  Note, that this step is required,
even if only using a single patch, and must be reissued any time the patch
configuration is changed.  Usage of this utility, which is compiled
into the \Textsfc{StandAlone/tools/pfs} directory, is:
\begin{lstlisting}[backgroundcolor=\color{background}]
   pfs input.ups
\end{lstlisting}

\subsection{Using image data to create particles}
One final option is available for initializing particle positions in MPM
simulations, and that is through the use of three dimensional image data,
such as might be collected via CT scans or confocal microscopy.  The image data are 
provided as 8-bit raw files, and usage in the input file is given as:
\begin{lstlisting}[language=XML]
        <image>
          <name>spheres.raw</name>
          <res>[1600, 1600, 1600]</res>
          <threshold>[1, 25]</threshold>
        </image>
        <file>
          <name>spheres.pts</name>
          <format>bin</format>
        </file>
\end{lstlisting}
The \Textsfc{\textless image\textgreater} section gives the name of the file, the resolution, in pixels,
in the various coordinate directions, and threshold range.  Particles will be
generated at voxels within the specified range.  The \Textsfc{\textless file\textgreater}
section is the same as that described above.  A different preprocessing utility
is provided when using image data (for the same reasons described previously).
Usage is as follows:
\begin{lstlisting}[backgroundcolor=\color{background}]
   pfs2 -b input.ups
\end{lstlisting}
The \Textbfc{-b} flag indicates that binary \Textsfc{spheres.pts.\#} files will be created, which
saves considerable disk space when performing large simulations.

