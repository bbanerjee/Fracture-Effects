\chapter{Data output files: UDA}
The \Textsfc{UDA} is a file/directory structure used to save \Vaango simulation
data.  For the most part, the user need not concern himself with the
UDA layout, but it is a good idea to have a general feeling for how
the data is stored on disk.

Every time a simulation (\Textsfc{vaango}) is run, a new UDA is created.  \Vaango uses
the \Textsfc{\textless filebase\textgreater} tag in the simulation input file to name the UDA
directory (appending a version number).  If an UDA of that name
already exists, the next version number is used.  Additionally, a
symbolic link named \Textsfc{<filename>.uda} is updated and will point to
the newest version of this simulations UDA.  For example,
\begin{lstlisting}[backgroundcolor=\color{background}]
disks.uda.000
disks.uda.001
disks.uda.001 <- disks.uda
\end{lstlisting}

Each UDA consists of a number of top level files, a checkpoints
subdirectory, and subdirectories for each saved timestep.  These files
include:
\begin{itemize}
\item \Textsfc{.dat} files contain global information about the
  simulation (each line in the .dat files contains: \Textbfc{simulation\_time}
  value).
\item \Textsfc{checkpoints} directory contains a limited
  number of time step data subdirectories that contain a complete
  snapshot of the simulation (allowing for the simulation to be
  restarted from that time).
\item \Textsfc{input.xml} contains the original problem
  specification (the .ups file).
\item \Textsfc{ index.xml} contains information on the actual
  simulation run.
\item \Textsfc{t0000\#} contains data saved for that specific
  time step.  The data saved is specified in .ups file and may be a
  very limited subset of the full simulation data.

\end{itemize}

The \Textsfc{validateUda} script in \Textsfc{src/Packages/Uintah/scripts/} can be used to
test the integrity of a UDA directory. It does not interrogate the data
for correctness, but performs 5 basic tests on each uda:
\begin{lstlisting}[backgroundcolor=\color{background}]
Usage validateUda  <udas>
Test 0:  Does index.xml exist?                                        true or false
Test 1:  Does each timestep in index.xml exist?                       true or false
Test 2:  Do all timesteps.xml files exist?                            true or false
Test 3:  Do all the level directories exist:                          true or false
Test 4:  Do all of the pxxxx.xml files exist and have size >0:        true or false
Test 5:  Do all of the pxxxx.data files exist and have size > 0:      true or false
\end{lstlisting}

If any of the tests fail then the corrupt output timestep should be removed from 
the \Textsfc{index.xml} file.

See Section~\ref{Sec:DataArchiver} for a description of how to specify
what data are saved and how frequently.


