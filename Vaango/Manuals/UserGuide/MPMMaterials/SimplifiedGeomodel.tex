\section{\bf Simplified Geomodel} This is a simplified model which has the basic features
needed for geomaterials. The yield function of this model is a two-surface plasticity model 
which includes a linear Drucker-Prager part and a cap yield function. The “cap” part reflects
the fact that plastic deformations can occur even under purely hydrostatic compression as
a consequence of void collapse. It means that the simplified geomodel considers the presence 
of both microscale flaws such as porosity and networks of microcracks. This mdel uses a 
multi-stage return algorithm proposed in \cite{Brannon10}. Usage is as follows:
\begin{lstlisting}[language=XML]
        <constitutive_model type="simplified_geo_model">
            <B0>10000</B0>
            <G0>3750</G0>
            <hardening_modulus>0.0</hardening_modulus>
            <FSLOPE> 0.057735026919 </FSLOPE>
            <FSLOPE_p> 0.057735026919 </FSLOPE_p>
            <PEAKI1> 612.3724356953976 </PEAKI1>
            <CR> 6.0 </CR>
            <p0_crush_curve> -1837.0724 </p0_crush_curve>
            <p1_crush_curve> 6.666666666666666e-4 </p1_crush_curve>
            <p3_crush_curve> 0.5 </p3_crush_curve>
            <p4_fluid_effect> 0.2 </p4_fluid_effect>
            <fluid_B0> 0.0 </fluid_B0>
            <fluid_pressur_initial> 0.0 </fluid_pressur_initial>
            <kinematic_hardening_constant> 0.0 </kinematic_hardening_constant>
        </constitutive_model>
\end{lstlisting}
where \tt <B0> \normalfont and \tt <G0> \normalfont are the bulk and 
shear moduli of the material, \tt <FSLOPE> \normalfont is the tangent of the friction angle 
of the Drucker-Prager part, \tt <FSLOPE\_p> \normalfont is the tangent of the dilation angle 
of the Drucker-Prager part, \tt <hardening\_modulus> \normalfont is the ensemble hardening 
modulus, and \tt <PEAKI1> \normalfont is the initial tensile limit of the 
first stress invariant, $I_1$. The Drucker-Prager yield criterion is given as
\begin{equation}
\sqrt{J_2}+FSLOPE \times (I_1-PEAKI1)=0,
\end{equation}
\tt <CR> \normalfont is a shape parameter that allows porosity to affect shear
strength which equals the eccentricity (width divided by
height) of the elliptical cap function, \tt <p0\_crush\_curve> 
\normalfont, \tt <p1\_crush\_curve> \normalfont, and \tt <p3\_crush\_curve> \normalfont are the
constants in the fitted post yielding part of the crush curve
\begin{equation}
p_3-\bar{\epsilon}_v^p=p_3\exp{-3p_1(\bar{p}-p_0)}
\end{equation}
in which $\bar{p}$ is the pressure.

In pure kinematic hardening the center of the yield surface changes with its size and
shape remaining unchanged.
Generally, kinematic hardening is modeled by introducing the back stress tensor, and defining
an appropriate evolution rule for it. In the simplified geomodel, linear Ziegler’s rule is used:
\begin{equation}
\dot{\alpha}_{ij}=\dot{\mu}(\Bsig_{ij}-\alpha_{ij})
\end{equation}
in which $\alpha_{ij}$ is the back stress tensor, $\dot{\alpha}_{ij}$ is time derivative of 
the back stress tensor, and
\begin{equation}
\dot{\mu}=c\dot{\xi}^p
\end{equation}
where $\dot{\xi}^p$ is the deviatoric invariant of the rate of plastic strain and $c$ is a 
constant defined yb the user as \tt <kinematic\_hardening\_constant> \normalfont.

Based on the research work done by M. Homel at the University of Utah, the following equations
are used in the simplified geomodel to consider the fluid-filled porous effects:
\begin{eqnarray}
\frac{\partial X}{\partial\varepsilon^p_v}=
&& \frac{1}{p_1p_3}\exp{(-p_1X-p_0)}
-\frac{3K_f\left(\exp(p_3+p_4)-1\right)\exp(p_3+p_4+\varepsilon^p_v)}
{\left(\exp(p_3+p_4+\varepsilon^p_v)-1\right)^2} \nonumber \\
&& +\frac{3K_f\left(\exp(p_3+p_4)-1\right)\exp(p_3+\varepsilon^p_v)}
{\left(\exp(p_3+\varepsilon^p_v)-1\right)^2}
\end{eqnarray}
in which $X$ is the value of the first stress invariant at the 
intersection of the cap yield surface and the mean pressure axis, 
$K_f$ is the fluid bulk modulus, which is defined by the user as \tt <fluid\_B0>\normalfont, 
$\varepsilon^p_v$ is the volumetric part 
of the plastic strain, and $p_4$ is a constant defined by the user as 
\tt <p4\_fluid\_effect>\normalfont. The isotropic part of the back stress tensor
is updated using the following equation
\begin{equation}
\alpha^{\mathrm{iso}}_{n+1}=\alpha^{\mathrm{iso}}_{n}+
\frac{3K_f\exp(p_3)\left(\exp(p_4)-\exp(\varepsilon^p_v)\right)}
{\left(\exp(p_3+\varepsilon^p_v)-1\right)}
\dot{\varepsilon}^p_v\Delta t ~\Bone
\end{equation}
in which $~\Bone$ is the second-order identity tensor. Also, the effective bulk modulus 
is calculated as
\begin{equation}
K_e=B0+
\frac{K_f\left(\exp(p_3+p_4)-1\right)\exp(p_3+p_4+\varepsilon^e_v+\varepsilon^p_v)}
{\left(\exp(p_3+p_4+\varepsilon^e_v+\varepsilon^p_v)-1\right)^2}
\end{equation}
in which $\varepsilon^e_v$ is the volumetric part of the elastic strain.
