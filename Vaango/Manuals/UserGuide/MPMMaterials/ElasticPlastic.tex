\section{ElasticPlastic} The \tt <elastic\_plastic> \normalfont model is a
general purpose model that was primarily implemented for the purpose of
modeling high strain rate metal plasticity.  Dr. Biswajit Banerjee has
written an extensive description of the theory, implementation and use of
this model.  Because of the amount of detail involved, and because these
subtopics are interwoven, this model is given its own section below.


There is a large number remaining models but these are not frequently utilized. 
This includes models for viscoelasticity, soil models, and transverse isotropic materials (i.e., fiber reinforced composites).
Examples of their use can be found in the \tt inputs \normalfont directory.
Input files can also be constructed by checking the source code to see
what parameters are required.

There are a few models whose use is explicitly not recommended.  In particular,
\tt HypoElasticPlastic, Membrane and SmallStrainPlastic. \normalfont  Input
files calling for the first of these should be switched to the
\tt ElasticPlastic \normalfont model instead.

  \paragraph{Example input file for the HypoElasticPlastic model}
  An example of the portion of an input file that specifies a copper body
  with a hypoelastic stress update, Johnson-Cook plasticity model,
  Johnson-Cook Damage Model and Mie-Gruneisen Equation of State is shown
  below.
  \begin{lstlisting}
  <material>

    <include href="inputs/MPM/MaterialData/MaterialConstAnnCopper.xml"/>
    <constitutive_model type="elastic_plastic">
      <tolerance>5.0e-10</tolerance>
      <include href="inputs/MPM/MaterialData/IsotropicElasticAnnCopper.xml"/>
      <include href="inputs/MPM/MaterialData/JohnsonCookPlasticAnnCopper.xml"/>
      <include href="inputs/MPM/MaterialData/JohnsonCookDamageAnnCopper.xml"/>
      <include href="inputs/MPM/MaterialData/MieGruneisenEOSAnnCopper.xml"/>
    </constitutive_model>

    <geom_object>
      <cylinder label = "Cylinder">
        <bottom>[0.0,0.0,0.0]</bottom>
        <top>[0.0,2.54e-2,0.0]</top>
        <radius>0.762e-2</radius>
      </cylinder>
      <res>[3,3,3]</res>
      <velocity>[0.0,-208.0,0.0]</velocity>
      <temperature>294</temperature>
    </geom_object>

  </material>
  \end{lstlisting}

  The general material constants for copper are in the file
  \TT{MaterialConstAnnCopper.xml}.  The contents are shown below
  \begin{lstlisting}
  <?xml version='1.0' encoding='ISO-8859-1' ?>
  <Uintah_Include>
    <density>8930.0</density>
    <toughness>10.e6</toughness>
    <thermal_conductivity>1.0</thermal_conductivity>
    <specific_heat>383</specific_heat>
    <room_temp>294.0</room_temp>
    <melt_temp>1356.0</melt_temp>
  </Uintah_Include>
  \end{lstlisting}

  The elastic properties are in the file \TT{IsotropicElasticAnnCopper.xml}.
  The contents of this file are shown below.
  \begin{lstlisting}
  <?xml version='1.0' encoding='ISO-8859-1' ?>
  <Uintah_Include>
    <shear_modulus>45.45e9</shear_modulus>
    <bulk_modulus>136.35e9</bulk_modulus>
  </Uintah_Include>
  \end{lstlisting}

  The constants for the Johnson-Cook plasticity model are in the file
  \TT{JohnsonCookPlasticAnnCopper.xml}.  The contents of this file are
  shown below.
  \begin{lstlisting}
  <?xml version='1.0' encoding='ISO-8859-1' ?>
  <Uintah_Include>
    <flow_model type="johnson_cook">
      <A>89.6e6</A>
      <B>292.0e6</B>
      <C>0.025</C>
      <n>0.31</n>
      <m>1.09</m>
    </flow_model>
  </Uintah_Include>
  \end{lstlisting}

  The constants for the Johnson-Cook damage model are in the file
  \TT{JohnsonCookDamageAnnCopper.xml}.  The contents of this file are
  shown below.
  \begin{lstlisting}
  <?xml version='1.0' encoding='ISO-8859-1' ?>
  <Uintah_Include>
    <damage_model type="johnson_cook">
      <D1>0.54</D1>
      <D2>4.89</D2>
      <D3>-3.03</D3>
      <D4>0.014</D4>
      <D5>1.12</D5>
    </damage_model>
  </Uintah_Include>
  \end{lstlisting}

  The constants for the Mie-Gruneisen model (as implemented in the
  Uintah Computational Framework) are in the file
  \TT{MieGruneisenEOSAnnCopper.xml}.  The contents of this file are
  shown below.
  \begin{lstlisting}
  <?xml version='1.0' encoding='ISO-8859-1' ?>
  <Uintah_Include>
    <equation_of_state type="mie_gruneisen">
      <C_0>3940</C_0>
      <Gamma_0>2.02</Gamma_0>
      <S_alpha>1.489</S_alpha>
    </equation_of_state>
  </Uintah_Include>
  \end{lstlisting}

  As can be seen from the input file, any other plasticity model, damage
  model and equation of state can be used to replace the Johnson-Cook
  and Mie-Gruneisen models without any extra effort (provided the models
  have been implemented and the data exist).

  The material data can easily be taken from a material database or specified
  for a new material in an input file kept at a centralized location.  At this
  stage material data for a range of materials is kept in the directory
  \TT{.../Uintah/StandAlone/inputs/MPM/MaterialData}.

