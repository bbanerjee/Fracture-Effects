  \subsection{Shear Modulus} \label{sec:ModelShear}
  Three models for the shear modulus ($\mu$) have been tested in our
  simulations.  The first has been associated with the Mechanical
  Threshold Stress (MTS) model and we call it the MTS shear model.
  The second is the model used by Steinberg-Cochran-Guinan and we call
  it the SCG shear model while the third is a model developed by
  Nadal and Le Poac that we call the NP shear model.

  \subsubsection{Default model}
  The default model gives a constant shear modulus.  The model is
  invoked using
  \begin{lstlisting}
    <shear_modulus_model type="constant_shear">
    </shear_modulus_model>
  \end{lstlisting}

  \subsubsection{MTS Shear Modulus Model}
  The simplest model is of the form suggested by \cite{Varshni70}
  (\cite{Chen96})
  \begin{equation} \label{eq:MTSShear}
    \mu(T) = \mu_0 - \frac{D}{exp(T_0/T) - 1}
  \end{equation}
  where $\mu_0$ is the shear modulus at 0K, and $D, T_0$ are material
  constants.

  The model is invoked using
  \begin{lstlisting}
    <shear_modulus_model type="mts_shear">
      <mu_0>28.0e9</mu_0>
      <D>4.50e9</D>
      <T_0>294</T_0>
    </shear_modulus_model>
  \end{lstlisting}

  \subsubsection{SCG Shear Modulus Model}
  The Steinberg-Cochran-Guinan (SCG) shear modulus
  model (\cite{Steinberg80,Zocher00}) is pressure dependent and
  has the form
  \begin{equation} \label{eq:SCGShear}
    \mu(p,T) = \mu_0 + \Partial{\mu}{p} \frac{p}{\eta^{1/3}} +
         \Partial{\mu}{T}(T - 300) ; \quad
    \eta = \rho/\rho_0
  \end{equation}
  where, $\mu_0$ is the shear modulus at the reference state($T$ = 300 K,
  $p$ = 0, $\eta$ = 1), $p$ is the pressure, and $T$ is the temperature.
  When the temperature is above $T_m$, the shear modulus is instantaneously
  set to zero in this model.

  The model is invoked using
  \begin{lstlisting}
  <shear_modulus_model type="scg_shear">
    <mu_0> 81.8e9 </mu_0>
    <A> 20.6e-12 </A>
    <B> 0.16e-3 </B>
  </shear_modulus_model>
  \end{lstlisting}

  \subsubsection{NP Shear Modulus Model}
  A modified version of the SCG model has been developed by
  \cite{Nadal03} that attempts to capture the sudden drop in the
  shear modulus close to the melting temperature in a smooth manner.
  The Nadal-LePoac (NP) shear modulus model has the form
  \begin{equation} \label{eq:NPShear}
    \mu(p,T) = \frac{1}{\mathcal{J}(\That)}
      \left[
        \left(\mu_0 + \Partial{\mu}{p} \cfrac{p}{\eta^{1/3}} \right)
        (1 - \That) + \frac{\rho}{Cm}~k_b~T\right]; \quad
    C := \cfrac{(6\pi^2)^{2/3}}{3} f^2
  \end{equation}
  where
  \begin{equation}
    \mathcal{J}(\That) := 1 + \exp\left[-\cfrac{1+1/\zeta}
        {1+\zeta/(1-\That)}\right] \quad
       \text{for} \quad \That:=\frac{T}{T_m}\in[0,1+\zeta],
  \end{equation}
  $\mu_0$ is the shear modulus at 0 K and ambient pressure, $\zeta$ is
  a material parameter, $k_b$ is the Boltzmann constant, $m$ is the atomic
  mass, and $f$ is the Lindemann constant.

  The model is invoked using
  \begin{lstlisting}
    <shear_modulus_model type="np_shear">
      <mu_0>26.5e9</mu_0>
      <zeta>0.04</zeta>
      <slope_mu_p_over_mu0>65.0e-12</slope_mu_p_over_mu0>
      <C> 0.047 </C>
      <m> 26.98 </m>
    </shear_modulus_model>
  \end{lstlisting}

  \subsubsection{PTW Shear model}
  The PTW shear model is a simplified version of the SCG shear model.
  The inputs can be found in \TT{.../MPM/ConstitutiveModel/PlasticityModel/PTWShear.h}.

