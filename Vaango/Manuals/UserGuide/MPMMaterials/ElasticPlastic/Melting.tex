\subsection{Melting Temperature}
  \subsubsection{Default model}
  The default model is to use a constant melting temperature.  This model
  is invoked using
  \begin{lstlisting}
      <melting_temp_model type="constant_Tm">
      </melting_temp_model>
  \end{lstlisting}

  \subsubsection{Linear melt model}
  A melting temperature model designed to be linear in pressure can be invoked using
  \begin{lstlisting}
      <melting_temp_model type="linear_Tm">
        <T_m0> </T_m0>
        <a>   </a>
        <b> </b>
        <Gamma_0> </Gamma_0>
        <K_T> </K_T>
      </melting_temp_model>
  \end{lstlisting}

  where T\_m0 is required, and either a or Gamma\_0 along with K\_T, or b.  
  T\_m0 is the initial melting temperature in Kelvin, a is the Kraut-Kennedy coefficient,
  Gamma\_0 is the Gruniesen Gamma, b is the pressure coefficient in Kelvin per Pascal, and K\_T
  is the isothermal bulk modulus in Pascals.
  The pressure is calculated using either
  \begin{equation}
    T_{m}=T_{m0}+bP
  \end{equation}
  or
  \begin{equation}
    T_{m}=T_{m0}\left(1+a\frac{\rho_0}{\rho}\right)
  \end{equation}
  The constants in these equations are linked together via the following equation
  \begin{equation}
   b=\frac{aT_{m0}}{K_T}
  \end{equation}
  and a is related to the Gruneisen Gamma by the Lindemann Law \cite{Poirier91}:
  \begin{equation}
   a=2\left(\Gamma_0-\frac{1}{3}\right)
  \end{equation}

  

  \subsubsection{SCG melt model}
  We use a pressure dependent relation to determine the melting
  temperature ($T_m$).  The Steinberg-Cochran-Guinan (SCG) melt model
  (\cite{Steinberg80}) has been used for our simulations of copper.
  This model is based on a modified Lindemann law and has the form
  \begin{equation} \label{eq:TmSCG}
    T_m(\rho) = T_{m0} \exp\left[2a\left(1-\frac{1}{\eta}\right)\right]
              \eta^{2(\Gamma_0-a-1/3)}; \quad
    \eta = \frac{\rho}{\rho_0}
  \end{equation}
  where $T_{m0}$ is the melt temperature at $\eta = 1$,
  $a$ is the coefficient of the first order volume correction to
  Gr{\"u}neisen's gamma ($\Gamma_0$).

  This model is invoked with
  \begin{lstlisting}
    <melting_temp_model type="scg_Tm">
      <T_m0> 2310.0 </T_m0>
      <Gamma_0> 3.0 </Gamma_0>
      <a> 1.67 </a>
    </melting_temp_model>
  \end{lstlisting}

  \subsubsection{BPS melt model}
  An alternative melting relation that is based on dislocation-mediated
  phase transitions - the Burakovsky-Preston-Silbar (BPS) model
  (\cite{Burakovsky00}) can also be used.  This model has been used to
  determine the melt temperature for 4340 steel.  The BPS model has the form
  \begin{align}
    T_m(p) & = T_m(0)
      \left[\cfrac{1}{\eta} + 
            \cfrac{1}{\eta^{4/3}}~\cfrac{\mu_0^{'}}{\mu_0}~p\right]~; 
    \quad
    \eta = \left(1 + \cfrac{K_0^{'}}{K_0}~p\right)^{1/K_0^{'}} 
    \label{eq:TmBPS}\\
    T_m(0) & = \cfrac{\kappa\lambda\mu_0~v_{WS}}{8\pi\ln(z-1)~k_b}
               \ln\left(\cfrac{\alpha^2}{4~b^2\rho_c(T_m)}\right)
  \end{align}
  where $p$ is the pressure, $\eta = \rho/\rho_0$ is the compression,
  $\mu_0$ is the shear modulus at room temperature and zero pressure,
  $\mu_0^{'} = \partial\mu/\partial p$ is the derivative of the shear modulus
  at zero pressure, $K_0$ is the bulk modulus at room temperature and
  zero pressure, $K_0^{'} = \partial K/\partial p$ is the derivative of the
  bulk modulus at zero pressure, $\kappa$ is a constant, $\lambda = b^3/v_{WS}$
  where $b$ is the magnitude of the Burgers' vector, $v_{WS}$ is the
  Wigner-Seitz volume, $z$ is the coordination number, $\alpha$ is a
  constant, $\rho_c(T_m)$ is the critical density of dislocations, and
  $k_b$ is the Boltzmann constant.

  This model is invoked with
  \begin{lstlisting}
    <melting_temp_model type="bps_Tm">
      <B0> 137e9 </B0>
      <dB_dp0> 5.48 <dB_dp0>
      <G0> 47.7e9 <G0>
      <dG_dp0> 1.4 <dG_dp0>
      <kappa> 1.25 <kappa>
      <z> 12 <z>
      <b2rhoTm> 0.64 <b2rhoTm>
      <alpha> 2.9 <alpha>
      <lambda> 1.41 <lambda>
      <a> 3.6147e-9<a>
      <v_ws_a3_factor> 1/4 <v_ws_a3_factor>
      <Boltzmann_Constant> <Boltzmann_Constant>
    </melting_temp_model>
  \end{lstlisting}
