\subsection{Equation of State Models}
The elastic-plastic stress update assumes that the volumetric part of
the Cauchy stress can be calculated using an equation of state.  There
are three equations of state that are implemented in Uintah.  These
are
\begin{enumerate}
    \item A default hypoelastic equation of state.
    \item A neo-Hookean equation of state.
    \item A Mie-Gruneisen type equation of state.
\end{enumerate}

\subsubsection{Default hypoelastic equation of state}
In this case we assume that the stress rate is given by
\Beq
    \dot{\Bsig} = \lambda~\Tr(\Bd^e)~\Bone + 2~\mu~\Bd^e
\Eeq
where $\Bsig$ is the Cauchy stress, $\Bd^e$ is the elastic part of
the rate of deformation, and $\lambda, \mu$ are constants.

If $\Beta^e$ is the deviatoric part of $\Bd^e$ then we can write
\Beq
    \dot{\Bsig} = \left(\lambda + \frac{2}{3}~\mu\right)~\Tr(\Bd^e)~\Bone +
        2~\mu~\Beta^e = \kappa~\Tr(\Bd^e)~\Bone + 2~\mu~\Beta^e ~.
\Eeq
If we split $\Bsig$ into a volumetric and a deviatoric part, i.e.,
$\Bsig = p~\Bone + \Bs$ and take the time derivative to get
$\dot{\Bsig} = \dot{p}~\Bone + \dot{\Bs}$ then
\Beq
    \dot{p} = \kappa~\Tr(\Bd^e) ~.
\Eeq
In addition we assume that $\Bd = \Bd^e + \Bd^p$.  If we also assume that
the plastic volume change is negligible, we can then write that
\Beq
    \dot{p} = \kappa~\Tr(\Bd) ~.
\Eeq
This is the equation that is used to calculate the pressure $p$ in the
default hypoelastic equation of state, i.e.,
\Beq
    \boxed{
    p_{n+1} = p_n + \kappa~\Tr(\Bd_{n+1})~\Delta t ~.
    }
\Eeq
To get the derivative of $p$ with respect to $J$, where $J = \det(\BF)$,
we note that
\Beq
    \dot{p} = \Partial{p}{J}~\dot{J} = \Partial{p}{J}~J~\Tr(\Bd) ~.
\Eeq
Therefore,
\Beq
    \boxed{
    \Partial{p}{J} = \cfrac{\kappa}{J} ~.
    }
\Eeq

This model is invoked in Uintah using
\begin{lstlisting}[language=XML]
  <equation_of_state type="default_hypo">
  </equation_of_state>
\end{lstlisting}
The code is in \tt.../MPM/ConstitutiveModel/PlasticityModels/DefaultHypoElasticEOS.cc \normalfont
If an EOS is not specified then this model is the {\bf default}.

\subsubsection{Default hyperelastic equation of state}
In this model the pressure is computed using the relation
\begin{equation}
  p = \Half~\kappa~\left(J^e - \cfrac{1}{J^e}\right)
\end{equation}
where $\kappa$ is the bulk modulus and $J^e$ is determinant of the elastic
part of the deformation gradient.

We can also compute
\begin{equation}
  \Deriv{p}{J} = \Half~\kappa~\left(1 + \cfrac{1}{(J^e)^2}\right) ~.
\end{equation}

This model is invoked in Uintah using
\begin{lstlisting}[language=XML]
  <equation_of_state type="default_hyper">
  </equation_of_state>
\end{lstlisting}
The code is in \TT{.../MPM/ConstitutiveModel/PlasticityModels/HyperElasticEOS.cc}.

\subsubsection{Mie-Gruneisen equation of state}
The pressure ($p$) is calculated using a Mie-Gr{\"u}neisen equation of state
\begin{equation} \label{eq:MG}
  p = p_{ref} +\rho \Gamma (e - e_{ref})
\end{equation}
where $\rho$ is the mass density, $\Gamma$ the Gr{\"u}neisen parameter (unitless) and $p_{ref}$
and $e_{ref}$ are known pressure and internal specific energy on a reference curve {\it and are a function
of volume only}.  As the form can be formally viewed as an expansion valid near the reference curve,
ideally the reference curve prescribes states near those of interest.  The reference curve could be
the shock Hugoniot, the standard adiabat (through the initial state), the 0 K isotherm, the isobar $p=0$,
the curve $e=0$, or some composite of these curves to cover the range of interest.  

For shock calculations it makes sense to use the Hugoniot as a reference curve.  We Assume the following
relationship between shock wave velocity, $U_s$ and particle velocity,  $U_p$,
\begin{equation} \label{eq:SteinbergUsUp}
  U_s = C_0 + s_\alpha U_p + s_2 \frac{U_p^2}{U_s} + s_3 \frac{U_p^3}{U_s^2}
\end{equation}
where $C_0$ is the bulk speed of sound, and the $s$'s are dimensionless coefficients.
This form is due to Steinberg (\cite{Steinberg1991}), and is a straight--forward extension to a nonlinear 
shock velocity, particle velocity relationship.  It reduces to the linear relationship most frequently 
used, e.g. (\cite{Wilkins1999,Zocher2000}), with $s_2 = s_3 = 0$.  Using the steady shock jump conditions for
conservation of mass, momentum and energy, the Hugoniot reference pressure, $p_H$ and specific energy,
$e_H$ may be determined
\begin{equation} \label{eq:pH}
  p_H = \frac{\rho_0 C_0^2 \eta}{1 - s_\alpha \eta - s_2 \eta^2 - s_3 \eta^3}
\end{equation}
\begin{equation} \label{eq:eH}
  e_H = \frac{p_H \eta}{2 \rho_0}
\end{equation}
where $\rho_0$ is the initial density (pre-shock) and
\begin{equation} \label{eq:eta}
  \eta = 1-\frac{\rho_0}{\rho}
\end{equation}
is a measure of volumetric deformation.  Using these relationships and the additional assumption that
$\rho\Gamma=\rho_0\Gamma_0$ the Mie-Gr{\"u}neisen equation of state may be written
\begin{equation} \label{eq:EOSMG_upd}
  \boxed{
  p =  \frac{\rho_0~C_0^2~\eta(1-\frac{\Gamma_0\eta}{2})}{(1 - s_\alpha \eta - s_2 \eta^2 - s_3 \eta^3)^2}
           + \rho_0\Gamma_0~e
  }
\end{equation}
To extend the Mie Gr{\"u}neisen EOS into tensile stress regimes, for $\eta<0$ the pressure is evaluated as
\begin{equation} \label{eq:EOSMG_upd_1}
  \boxed{
  p =  \rho_0~C_0^2~\eta + \rho_0\Gamma_0~e
  }
\end{equation}
This equation is integrated explicitly, using beginning timestep values for energy and the current value
of density to update the pressure.  For isochoric plasticity,
\begin{equation*}
  J^e = J = \det(\BF) = \cfrac{\rho_0}{\rho} ~.
\end{equation*}
where $\BF^e$ is the elastic part of the deformation gradient.  
The increment in specific internal energy is computed using
  \begin{equation}
    \rho^* \Delta e = (\sigma^*_{ij} D_{ij} - q D_{kk}) \Delta t
  \end{equation}
where $\sigma^*_{ij}$ and $\rho^*$ are the average stress and density over the time step, 
$D_{ij}$ is the rate of deformation tensor, and $q$ is
the artificial viscosity.  Note that the artificial velocity term must be included explicitly
since it is not accumulated in the total stress.

The temperature, $T$, is calculated using the thermodynamic relationship
  \begin{equation}\label{eq:dTthermo}
    dT = -\rho\Gamma T dv + \frac{TdS}{C_v}
  \end{equation}
where $v$ is the specific volume, $S$ the entropy and $C_v$ the specific heat at constant volume.
Entropy change is associated with irreversible, or dissipative, processes.  Equating $TdS$ to the
dissipated work terms, those components of temperature change are computed in the appropriate routines,
such as plasticity or artificial viscosity.  In fact, not all of the dissipated energy needs be
converted to heat, as allowed for by using the Taylor--Quinney coefficient (see below).  The first term
in \label{eq:dT} may be integrated to give the isentropic temperature change
  \begin{equation}\label{eq:dTisentropic}
    \Delta T_{isentropic} = -T\Gamma_0\frac{\rho_0}{\rho}D_{kk}\Delta t
  \end{equation}
where the same assumption, $\rho\Gamma=\rho_0\Gamma_0$ is used.  The isentropic temperature change
is computed as part of the EOS response.

Should an implicit integration scheme be used the tangent moduli are needed, which in turn require calculation of
\Beq
  \boxed{
  \Partial{p}{J^e} =
  \cfrac{\rho_0~C_0^2~[1 + (S_{\alpha}-\Gamma_0)~(1-J^e)]}
        {[1-S_{\alpha}~(1-J^e)]^3} - \Gamma_0~\Partial{e}{J^e}.
  }
\Eeq
We neglect the $\Partial{e}{J^e}$ term in our calculations.  {\it Note: this calculation hasn't been updated for
the Steinberg nonlinear shock velocity, particle velocity relationship.}

This model is invoked in Uintah using
\begin{lstlisting}[language=XML]
  <equation_of_state type="mie_gruneisen">
    <C_0>5386</C_0>
    <Gamma_0>1.99</Gamma_0>
    <S_alpha>1.339</S_alpha>
    <S_2>1.339</S_2>
    <S_3>1.339</S_3>
  </equation_of_state>
\end{lstlisting}
The code is in \TT{.../MPM/ConstitutiveModel/PlasticityModels/MieGruneisenEOS.cc}.

It is worth noting that this approach to calculating energy and temperature is not necessarily consistent
with existing implementations.  In fact, it does not appear that there is a standard approach, many shock
codes have unique implementations.  In particular, it appears that the elastic stored energy term is 
often neglected, as well as the isentropic temperature change.

It is worth noting that the approach outlined above is consistent with that taken fairly
recently by Wilkins (\cite{Wilkins1999}).  Wilkins expands the pressure and energy as polynomials in $\eta$
and uses Hugoniot data (and a linear $U_s$, $U_p$ relationship) to determine the coefficients.  Using the
additional thermodynamic relationship
  \begin{equation}\label{eq:de}
    de = -p dv + TdS
  \end{equation}
and substituting for $TdS$ from \ref{eq:dTthermo}, assuming $C_v$ constant and 
$\rho\Gamma=\rho_0\Gamma_0$ (as in Wilkins), the following relationship may be derived
  \begin{equation}\label{eq:e}
    e = -\int_{v_0}^v (p - \rho_0 \Gamma_0 C_v T) dv + C_v(T-T_0)
  \end{equation}
Since the integral is for $dS=0$, it may be integrated to give an alternate form of \ref{eq:dTisentropic},
  \begin{equation}\label{eq:dTisentropic2}
    T_{isentropic} = T_0\exp(\Gamma_0(1-\frac{\rho_0}{\rho}))
  \end{equation}
which may be substituted into \ref{eq:e} along with the expansion for $p(\eta)$ (here for $s_2=s_3=0$)
  \begin{equation}\label{eq:pofeta}
    p = \rho_0\Gamma_0 e + \rho_0 C_0^2(\eta + (2s_\alpha-\frac{\Gamma_0}{2})\eta^2 + s_\alpha(3s_\alpha-\Gamma_0)\eta^3) + O(\eta^4)
  \end{equation}
to give the equation
  \begin{equation}\label{eq:e2}
    e+\int_{v_0}^v\rho_0\Gamma_0 e dv = C_v(T-T_0) + \rho_0 \Gamma_0C_vT_0\int_{v_0}^v \exp(\Gamma_0\eta)dv 
       -\rho_0\Gamma_0\int_{v_0}^v(\eta + (2s_\alpha-\frac{\Gamma_0}{2})\eta^2 + s_\alpha(3s_\alpha-\Gamma_0)\eta^3)dv
  \end{equation}
Finally, using 
  \begin{equation}
    e = e_0(\eta) + \int_{T_0}^T C_vdT
  \end{equation}
with a polynomial expansion for $e(\eta)$ in powers of $\eta$, \ref{eq:e2} can be integrated 
to determine the coefficients in the expansion.  Determining the coefficients this way gives 
exactly the same expansion for energy as derived in
Wilkins, using a different approach.  The advantages of the approach outlined above are it's relative simplicity,
and generality -- no assumption of constant specific heat is needed.
