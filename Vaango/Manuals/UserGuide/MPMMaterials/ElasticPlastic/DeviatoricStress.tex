\subsection{Deviatoric Stress Models}
The elastic-plastic stress update assumes that the deviatoric part of
the Cauchy stress can be calculated independently of the equation of
state.  There are two deviatoric stress models that are implemented in
Uintah.  These are
\begin{enumerate}
    \item A default hypoelastic deviatoric stress.
    \item A linear hypoviscooelastic deviatoric stress.
\end{enumerate}

\subsubsection{Default Hypoelastic Deviatoric Stress}
  In this case the stress rate is given by
  \begin{equation}
    \dot{\Bs} = 2\mu(\Beta - \Beta^p)
  \end{equation}
  where $\mu$ is the shear modulus.  This model is invoked using
  \begin{lstlisting}  
      <deviatoric_stress_model type="hypoElastic">
      </deviatoric_stress_model>
  \end{lstlisting}
  If a deviatoric stress model is not specified then this model is the
  {\bf default}.

\subsubsection{Linear Hypoviscoelastic Deviatoric Stress}
  This model is a three--dimensional version of a Generalized Maxwell
  model, as presented in \cite{Zerilli07}.  It is specifically
  implemented to be combined with the ZA for Polymers Flow Stress
  Model described previously.  Together these models combine into a
  hypoviscoplastic model.  The stress update is given by
  \begin{equation}
    \dot{\Bs} = 2\mu(\Beta - \Beta^p) - \sum_{i=1}^N {\Bs_i\over \tau_i}
  \end{equation}
  where $\mu$ is the shear modulus and $\Bs_i$ are maxwell element
  stresses which are tracked internally.  Also
  \begin{equation}
    \Bs = \sum_{i=1}^N \Bs_i \quad\quad \mu = \sum_{i=1}^N \mu_i
  \end{equation}
  This model is invoked using
  \begin{lstlisting}  
      <deviatoric_stress_model type="hypoViscoElastic">
        <mu> [3.0, 5.0, 7.0] </mu>
        <tau> [1.67, 10.7, 107.0] </tau>
      </deviatoric_stress_model>
  \end{lstlisting}
where the number of elements in arrays mu and tau must be the same.
