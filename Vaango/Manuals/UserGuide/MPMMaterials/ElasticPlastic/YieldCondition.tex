\subsection{Yield conditions}
When failure is to be simulated we can use the
Gurson-Tvergaard-Needleman yield condition instead of the von Mises
condition.

  \subsubsection{The von Mises yield condition}
  The von Mises yield condition is the default.  It specifies a yield condition of
  the form
  \begin{equation}
    \Phi = \sqrt{3J_2} - \sigma_y
  \end{equation}
  where $J_2$ is the second invariant of the deviatoric stress tensor
  ($J_2={1\over2}\Bs:\Bs$) and $\sigma_y$ is the flow stress. Currently
  the return algorithms are restricted to plastic flow in the
  direction of the deviatoric stress
  (Eqn. \ref{eq:associated_flow}). See the discussion in the Radial
  Return algorithm description for details.  The von Mises yield
  condition is invoked using the tags
  \begin{lstlisting}[language=XML]
    <yield_condition type="vonMises">
    </yield_condition>
  \end{lstlisting}

  \subsubsection{The Gurson-Tvergaard-Needleman (GTN) yield condition}
  The Gurson-Tvergaard-Needleman (GTN) yield
  condition~\cite{Gurson77,Tver84} depends on porosity.  {\it This
  model is for experts only!!!}  Here are some caveats: Formally,
  you can replace the flow stress in Gurson’s model with the flow
  stresses of Johnson-Cook, ZA, etc., but the internal variable
  updates would have to be modified extensively.  For example, the JC
  yield stress depends on the equivalent plastic strain, but this
  needs to be the equivalent plastic strain of the matrix material,
  which is very different from the equivalent plastic strain of the
  porous composite. For example, under pure hydrostatic compression at
  the macroscale, the matrix material will suffer massive amounts of
  plastic SHEAR strains at the microscale (even though, for
  hydrostatic loading, it has zero plastic shear strain at the
  macroscale) and thus would need to harden. While the models should
  run, they are unlikely to give realistic results.

  The GTN yield condition is a fairly good bound in compression but a
  TERRIBLE bound in tension (in fact using it in tension can produce
  non-physical predictions of negative plastic work tantamount to
  tension causing pore COLLAPSE; very few Gurson implementations catch
  this problem because very few of them include run-time checks of
  solution quality, including this one).

  The GTN yield condition will not work with Radial Return.  An error
  will be generated in this case.  Presently it only runs with the
  modified Nemat-Nasser/Maudlin return algorithm.  Plastic flow is
  assumed to be in the direction of deviatoric stress.  Hence this
  is nonassociated flow for this pressure dependent yield condition.

  The GTN yield condition can be written as
  \begin{equation}
    \Phi = \left(\frac{\sigma_{eq}}{\sigma_f}\right)^2 +
    2 q_1 f_* \cosh \left(q_2 \frac{Tr(\sigma)}{2\sigma_f}\right) -
    (1+q_3 f_*^2) = 0
  \end{equation}
  where $q_1,q_2,q_3$ are material constants and $f_*$ is the porosity
  (damage) function given by
  \begin{equation}
    f* = 
    \begin{cases}
      f & \text{for}~~ f \le f_c,\\ 
      f_c + k (f - f_c) & \text{for}~~ f > f_c 
    \end{cases}
  \end{equation}
  where $k$ is a constant and $f$ is the porosity (void volume fraction).  The
  flow stress in the matrix material is computed using either of the two
  plasticity models discussed earlier.  Note that the flow stress in the matrix
  material also remains on the undamaged matrix yield surface and uses an
  associated flow rule.

  This yield condition is invoked using
  \begin{lstlisting}[language=XML]
    <yield_condition type="gurson">
      <q1> 1.5 </q1>
      <q2> 1.0 </q2>
      <q3> 2.25 </q3>
      <k> 4.0 </k>
      <f_c> 0.05 </f_c>
    </yield_condition>
  \end{lstlisting}

  \subsubsection{Porosity model}
  The evolution of porosity is calculated as the sum of the rate of growth
  and the rate of nucleation~\cite{Ramaswamy98a}.  The rate of growth of
  porosity and the void nucleation rate are given by the following equations
  ~\cite{Chu80}
  \begin{align}
    \dot{f} &= \dot{f}_{\text{nucl}} + \dot{f}_{\text{grow}} \\
    \dot{f}_{\text{grow}} & = (1-f) \text{Tr}(\BD_p) \\
    \dot{f}_{\text{nucl}} & = \cfrac{f_n}{(s_n \sqrt{2\pi})}
            \exp\left[-\Half \cfrac{(\epsilon_p - \epsilon_n)^2}{s_n^2}\right]
            \dot{\epsilon}_p
  \end{align}
  where $\BD_p$ is the rate of plastic deformation tensor, $f_n$ is the volume
  fraction of void nucleating particles , $\epsilon_n$ is the mean of the
  distribution of nucleation strains, and $s_n$ is the standard
  deviation of the distribution.

  The inputs tags for porosity are of the form
  \begin{lstlisting}[language=XML]
    <evolve_porosity> true </evolve_porosity>
    <initial_mean_porosity>         0.005 </initial_mean_porosity>
    <initial_std_porosity>          0.001 </initial_std_porosity>
    <critical_porosity>             0.3   </critical_porosity>
    <frac_nucleation>               0.1   </frac_nucleation>
    <meanstrain_nucleation>         0.3   </meanstrain_nucleation>
    <stddevstrain_nucleation>       0.1   </stddevstrain_nucleation>
    <initial_porosity_distrib>      gauss </initial_porosity_distrib>
  \end{lstlisting}


