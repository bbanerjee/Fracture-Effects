  \subsection{Adiabatic Heating and Specific Heat}
  A part of the plastic work done is converted into heat and used to update the
  temperature of a particle.  The increase in temperature ($\Delta T$) due to
  an increment in plastic strain ($\Delta\epsilon_p$) is given by the equation
  \begin{equation}
    \Delta T = \cfrac{\chi\sigma_y}{\rho C_p} \Delta \epsilon_p
  \end{equation}
  where $\chi$ is the Taylor-Quinney coefficient, and $C_p$ is the specific
  heat.  The value of the Taylor-Quinney coefficient is taken to be 0.9
  in all our simulations (see \cite{Ravi01} for more details on the
  variation of $\chi$ with strain and strain rate).

  The Taylor-Quinney coefficient is taken as input in the ElasticPlastic model
  using the tags
  \begin{lstlisting}
    <taylor_quinney_coeff> 0.9 </taylor_quinney_coeff>
  \end{lstlisting}

  \subsubsection{Default specific heat model}
  The default model returns a constant specific heat and is invoked using
  \begin{lstlisting}
    <specific_heat_model type="constant_Cp">
    </specific_heat_model>
  \end{lstlisting}

  \subsubsection{Cubic specific heat model}
  The specific heat model is of the form \cite{Menikoff2003}:
  \Beq
    C_v = \frac{\tilde{T}^3}{c_3\tilde{T}^3+c_2\tilde{T}^2+c_1\tilde{T}+c_0} 
  \Eeq
  where $\tilde{T}$ is the reduced temperature, and $c_0$-$c_3$ are fit parameters.
  The reduced temperature is calculated using $\tilde{T}=T/\theta (V)$ and the
  Debye temperature is:
  \Beq
    \theta (V)=\theta_0\left(\frac{V_0}{V}\right)^ae^{b\left(V_0-V\right)/V}
  \Eeq
  where $V$ is the specific volume and $\theta_0$ is the reference Debye temperature and
  a and b are fit parameters.
  The constant pressure specific heat is calculated via:
  \Beq
    C_p = C_v+\beta^2TVK_T
  \Eeq
  where $\beta$ and $K_T$ are the volumetric expansion coefficeint and 
  isothermal bulk modulus respectively.
  
  The model is invoked using:
  \begin{lstlisting}
    <specific_heat_model type="cubic_Cp">
      <a> 1.0 </a>
      <b> 1.0 </b>
      <beta> 1.0 </beta>
      <c0> 1.0 </c0>
      <c1> 1.0 </c1>
      <c2> 1.0 </c2>
      <c3> 1.0 </c3>
    </specific_heat_model>
  \end{lstlisting}
  where all parameters but $\beta$, $a$ and $b$ are required.


  \subsubsection{Specific heat model for copper}
  The specific heat model for copper is of the form
  \Beq
    C_p =
    \begin{cases}
      A_0~T^3 - B_0~T^2 + C_0~T - D_0 & \text{if} ~~T < T_0 \\
      A_1~T + B_1 & \text{if} ~~T \ge T_0 ~.
    \end{cases}
  \Eeq
  The model is invoked using
  \begin{lstlisting}
  <specific_heat_model type = "copper_Cp"> </specific_heat_model>
  \end{lstlisting}

  \subsubsection{Specific heat model for steel}
  A relation for the dependence of $C_p$ upon temperature is
  used for the steel (\cite{Lederman74}).
  \begin{align}
    C_p & = \begin{cases}
            A_1 + B_1~t + C_1~|t|^{-\alpha} & \text{if}~~ T < T_c \\
            A_2 + B_2~t + C_2~t^{-\alpha^{'}} & \text{if}~~ T > T_c 
          \end{cases} \label{eq:CpSteel}\\
    t & = \cfrac{T}{T_c} - 1 
  \end{align}
  where $T_c$ is the critical temperature at which the phase transformation
  from the $\alpha$ to the $\gamma$ phase takes place, and $A_1, A_2, B_1, B_2,
  \alpha, \alpha^{'}$ are constants.

  The model is invoked using
  \begin{lstlisting}
  <specific_heat_model type = "steel_Cp"> </specific_heat_model>
  \end{lstlisting}

  The heat generated at a material point is conducted away at the end of a
  time step using the transient heat equation.  The effect of conduction on
  material point temperature is negligible (but non-zero) for the high
  strain-rate problems simulated using Uintah.
