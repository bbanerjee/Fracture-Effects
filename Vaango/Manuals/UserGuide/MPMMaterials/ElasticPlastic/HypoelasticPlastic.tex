\subsection{Hypo-Elastic Plasticity in Uintah}  \label{Sec:ElasticPlastic}

The hypoelastic-plastic stress update is a mix and match combination
of isotropic models.  The equation of state may be varied
independently of the deviatoric response.  For elastic deviatoric
response the shear moduli may be taken to be functions of temperature
and pressure.  Plasticity is based on an additive decomposition of the
rate of deformation tensor into elastic and plastic parts.
Incompressibility is assumed for plastic deformations, i.e. the
plastic strain rate is proportional to the deviatoric stress.  Various
yield conditions and flow stresses may be mixed and matched.  There
are also options for damage/melting modeling.  {\it Note that there
  are few checks to prevent users from mixing and matching
  inappropriate models}.

Additional models can be added to this framework.  Presently, the
material models available are:
\begin{enumerate}
    \item Adiabatic Heating and Specific Heat:
      \begin{itemize}
        \item Taylor-Quinney coefficient.
        \item Constant Specific Heat ({\bf default}).
        \item Cubic Specific Heat.
        \item Specific Heat for Copper.
        \item Specific Heat for Steel.
      \end{itemize}
    \item The equation of state (pressure/volume response):
      \begin{itemize}
        \item Hypoelastic ({\bf default}).
        \item Neo-Hookean.
        \item Mie-Gruneisen.
      \end{itemize}
    \item The deviatoric stress model:
      \begin{itemize}
        \item Linear hypoelasticity ({\bf default}).
        \item Linear hypoviscoelasticity.
      \end{itemize}
    \item The melting model:
      \begin{itemize}
        \item Constant melt temperature ({\bf default}).
        \item Linear melt temperature.
        \item Steinberg-Cochran-Guinan (SCG) melt.
        \item Burakovsky-Preston-Silbar (BPS) melt.
      \end{itemize}
    \item Temperature and pressure dependent shear moduli (only works
      with linear hypoelastic deviatoric stress model):
      \begin{itemize}
        \item Constant shear modulus ({\bf default}).
        \item Mechanical Threshold Stress (MTS) model.
        \item Steinberg-Cochran-Guinan (SCG) model.
        \item Nadal-LePoac (NP) model.
        \item Preston-Tonks-Wallace (PTW) model.
      \end{itemize}
    \item The yield condition:
      \begin{itemize}
        \item von Mises.
        \item Gurson-Tvergaard-Needleman (GTN).
      \end{itemize}
    \item The flow stress:
      \begin{itemize}
        \item the Isotropic Hardening model
        \item the Johnson-Cook (JC) model
        \item the Steinberg-Cochran-Guinan-Lund (SCG) model.
        \item the Zerilli-Armstrong (ZA) model.
        \item the Zerilli-Armstrong for polymers model.
        \item the Mechanical Threshold Stress (MTS) model.
        \item the Preston-Tonks-Wallace (PTW) model.
      \end{itemize}
    \item The plastic return algorithm:
      \begin{itemize}
        \item Radial Return ({\bf default}).
        \item Modified Nemat-Nasser/Maudlin.
      \end{itemize}
    \item The damage model:
      \begin{itemize}
        \item Johnson-Cook damage model.
      \end{itemize}
\end{enumerate}

This model is invoked using
  \begin{lstlisting}
      <constitutive_model="elastic_plastic_hp">
        <shear_modulus>0.2845e4</shear_modulus>
        <bulk_modulus>1.41e4</bulk_modulus>
        <initial_material_temperature>298</initial_material_temperature>
        <plastic_convergence_algo>radialReturn</plastic_convergence_algo>
        <taylor_quinney_coeff> 0.9 </taylor_quinney_coeff>

        submodels

      </constitutive_model>
  \end{lstlisting}
  where ``submodels'' indicates subsets of tags corresponding to the
  listed above (and detailed below).  {\it Note that the specified
    bulk and shear moduli are used to calcuate a stable time step size
    for the first time step (hence it is important that they be
    consistent with EOS and deviatoric stress submodel material
    constants).  However, if the default EOS and/or deviatoric stress
    models are used, then these material constants are sufficient for
    bulk and/or deviatoric stress response, and are automatically used
    in those models.}  The bulk modulus is also used to determine
  artificial viscosity parameters (throughout the simulation).

