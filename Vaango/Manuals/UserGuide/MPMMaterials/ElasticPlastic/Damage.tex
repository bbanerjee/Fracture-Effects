\subsection{Damage Models and Failure}
Only the Johnson-Cook damage evolution rule has been added to the
DamageModelFactory so far.  The damage model framework is designed
to be similar to the plasticity model framework.  New models can
be added using the approach described later in this section.

  A particle is tagged as ``failed'' when its temperature is greater than the
  melting point of the material at the applied pressure.  An additional
  condition for failure is when the porosity of a particle increases beyond a
  critical limit and the strain exceeds the fracture strain of the material.
  Another condition for failure is when a material bifurcation
  condition such as the Drucker stability postulate is satisfied.  Upon failure,
  a particle is either removed from the computation by setting the stress to
  zero or is converted into a material with a different velocity field
  which interacts with the remaining particles via contact.  Either approach
  leads to the simulation of a newly created surface.  More details of the
  approach can be found in ~\cite{Banerjee2004a,Banerjee2004c,Banerjee2005}.

  \subsubsection{Damage model}
  After the stress state has been determined on the basis of the yield condition
  and the associated flow rule, a scalar damage state in each material point can
  be calculated using the Johnson-Cook model ~\cite{Johnson1985}.
  The Johnson-Cook model has an explicit dependence on temperature, plastic
  strain, ans strain rate.

  The damage evolution rule for the Johnson-Cook damage model can be written as
  \begin{equation}
    \dot{D} = \cfrac{\dot{\epsilon_p}}{\epsilon_p^f} ~;~~
    \epsilon_p^f = 
      \left[D_1 + D_2 \exp \left(\cfrac{D_3}{3} \sigma^*\right)\right]
      \left[1+ D_4 \ln(\dot{\epsilon_p}^*)\right]
      \left[1+D_5 T^*\right]~;~~
    \sigma^*= \cfrac{\text{Tr}(\Bsig)}{\sigma_{eq}}~;~~
  \end{equation}
  where $D$ is the damage variable which has a value of 0 for virgin material
  and a value of 1 at fracture, $\epsilon_p^f$ is the fracture strain,
  $D_1, D_2, D_3, D_4, D_5$ are constants, $\Bsig$ is the Cauchy stress, and
  $T^*$ is the scaled temperature as in the Johnson-Cook plasticity model.

  The input tags for the damage model are :
  \begin{lstlisting}[language=XML]
    <damage_model type="johnson_cook">
      <D1>0.05</D1>
      <D2>3.44</D2>
      <D3>-2.12</D3>
      <D4>0.002</D4>
      <D5>0.61</D5>
    </damage_model>
  \end{lstlisting}

  An initial damage distribution can be created using the following tags
  \begin{lstlisting}[language=XML]
    <evolve_damage>                 true  </evolve_damage>
    <initial_mean_scalar_damage>    0.005  </initial_mean_scalar_damage>
    <initial_std_scalar_damage>     0.001 </initial_std_scalar_damage>
    <critical_scalar_damage>        1.0   </critical_scalar_damage>
    <initial_scalar_damage_distrib> gauss </initial_scalar_damage_distrib>
  \end{lstlisting}

  \subsubsection{Erosion algorithm}
  Under normal conditions, the heat generated at a material point is conducted
  away at the end of a time step using the heat equation.  If special adiabatic
  conditions apply (such as in impact problems), the heat is accumulated at a
  material point and is not conducted to the surrounding particles.  This
  localized heating can be used to determine whether a material point has
  melted.

  The determination of whether a particle has failed can be made on the
  basis of either or all of the following conditions:
  \begin{itemize}
    \item The particle temperature exceeds the melting temperature.
    \item The TEPLA-F fracture condition~\cite{Johnson1988} is satisfied.
       This condition can be written as
       \begin{equation}
         (f/f_c)^2 + (\epsilon_p/\epsilon_p^f)^2 = 1
       \end{equation}
       where $f$ is the current porosity, $f_c$ is the maximum
       allowable porosity, $\epsilon_p$ is the current plastic strain, and
       $\epsilon_p^f$ is the plastic strain at fracture.
    \item An alternative to ad-hoc damage criteria is to use the concept of
       bifurcation to determine whether a particle has failed or not.  Two
       stability criteria have been explored in this paper - the Drucker
       stability postulate~\cite{Drucker1959} and the loss of hyperbolicity
       criterion (using the determinant of the acoustic tensor)
       \cite{Rudnicki1975,Perzyna1998}.
  \end{itemize}

  The simplest criterion that can be used is the Drucker stability postulate
  \cite{Drucker1959} which states that time rate of change of the rate of
  work done by a material cannot be negative.  Therefore, the material is
  assumed to become unstable (and a particle fails) when
  \begin{equation}
    \dot\Bsig:\BD^p \le 0
  \end{equation}
  Another stability criterion that is less restrictive is the acoustic
  tensor criterion which states that the material loses stability if the
  determinant of the acoustic tensor changes sign~\cite{Rudnicki1975,Perzyna1998}.
  Determination of the acoustic tensor requires a search for a normal vector
  around the material point and is therefore computationally expensive.  A
  simplification of this criterion is a check which assumes that the direction
  of instability lies in the plane of the maximum and minimum principal
  stress~\cite{Becker2002}.  In this approach, we assume that the strain is
  localized in a band with normal $\Bn$, and the magnitude of the velocity
  difference across the band is $\Bg$.  Then the bifurcation condition
  leads to the relation
  \begin{equation} 
    R_{ij} g_{j} = 0 ~;~~~
    R_{ij} = M_{ikjl} n_k n_l + M_{ilkj} n_k n_l - \sigma_{ik} n_j n_k
  \end{equation}
  where $M_{ijkl}$ are the components of the co-rotational tangent
  modulus tensor and $\sigma_{ij}$ are the components of the co-rotational
  stress tensor.  If $\det(R_{ij}) \le 0 $, then $g_j$ can be arbitrary and
  there is a possibility of strain localization.  If this condition for
  loss of hyperbolicity is met,  then a particle deforms in an unstable
  manner and failure can be assumed to have occurred at that particle.
  We use a combination of these criteria to simulate failure.

  Since the material in the container may unload locally after fracture, the
  hypoelastic-plastic stress update may not work accurately under certain
  circumstances.  An improvement would be to use a hyperelastic-plastic stress
  update algorithm.  Also, the plasticity models are temperature dependent.
  Hence there is the issue of severe mesh dependence due to change of the
  governing equations from hyperbolic to elliptic in the softening regime
  ~\cite{Hill1975,Bazant1985,Tver1990}.  Viscoplastic stress update models or
  nonlocal/gradient plasticity models~\cite{Ramaswamy1998,Hao2000} can be used
  to eliminate some of these effects and are currently under investigation.

  The tags used to control the erosion algorithm are in two places.
  In the \TT{<MPM> </MPM>} section the following flags can be set
  \begin{lstlisting}[language=XML]
    <erosion algorithm = "ZeroStress"/>
    <create_new_particles>           false      </create_new_particles>
    <manual_new_material>            false      </manual_new_material>
  \end{lstlisting}
  If the erosion algorithm is \TT{"none"} then no particle failure is done.

  In the \TT{<constitutive_model type="elastic_plastic">} section, the
  following flags can be set
  \begin{lstlisting}[language=XML]
    <evolve_porosity>               true  </evolve_porosity>
    <evolve_damage>                 true  </evolve_damage>
    <do_melting>                    true  </do_melting>
    <useModifiedEOS>                true  </useModifiedEOS>
    <check_TEPLA_failure_criterion> true  </check_TEPLA_failure_criterion>
    <check_max_stress_failure>      false </check_max_stress_failure>
    <critical_stress>              12.0e9 </critical_stress>
  \end{lstlisting}
